\section{文字列}\label{s1:文字列} %{
	この節で用いる約束を挙げておく。この節では、これらの約束を断りなく使う。
	\section{文字列の基本事項}\label{s1:文字列の基本事項} %{
	$R=(R,+,0,\myspace,1)$を自明でない可換半環とする。

	任意の集合$A$に対して$WA=(WA,m_*,1_W)$を$A$から生成された自由モノイド
	とする。ここで、積$m_*$は文字列の連結で定義された積とし、連結積と書き、
	中置記法で$*$とも書く。また、$1_W$は文字数$0$の単語とする。
	$WA$の元を$A$の元を並べたものを括弧でくくって表すことにする。
	例えば、$a_1,a_2,\dots, a_n\in A$を並べた$WA$の元を$[a_1a_2\cdots a_n]$
	と書く。
	任意の$n\in\mybf{N}$に対して$W_nA\subseteq WA$を文字数$n$の単語の集合と
	する。つまり、$WA=\oplus_{n\in N}W_nA$となる。$RW_nA$を$W_nA$を基底とする
	$R$係数自由半モジュールとする。やはり、$RWA=\oplus_{n\in N}RW_nA$となる。
	積$m_*$は双線形写像$m_*:RW_mA\otimes RW_nA\to RW_{m+n}A$として
	みることもできる。

	任意の集合$A$に対して$RA$を$A$を基底とする$R$係数自由半モジュールとする。

	任意の集合$A$に対して、次の写像$i_W$を文字列への
	標準埋め込み(canonical injection)という。
	\begin{equation*}\begin{split} %{
		i_W: A\to WA,\quad a\mapsto [a] \quad\text{for all }a\in A
	\end{split}\end{equation*} %}
	標準埋め込みを線形写像に拡張したものも$i_W:RA\to RWA$と書く。

	任意の集合$A$に対して、次の写像$i_R$を自由半モジュールへの
	標準埋め込み(canonical injection)という。
	\begin{equation*}\begin{split} %{
		i_R: A\to RA,\quad a\mapsto a \quad\text{for all }a\in A
	\end{split}\end{equation*} %}
	標準埋め込みについて次の命題が成り立つ。

	\begin{proposition}[自由半代数の普遍性]\label{prop:自由半代数の普遍性} %{
		$A$を集合、$V$を$R$半代数とする。
		\begin{itemize}\setlength{\itemsep}{-1mm} %{
			\item $RA$から$V$への線形写像全体$\homset(RA,V)$と、
			\item $RWA=(RWA,m_*,1_W)$から$V$への$R$半代数射全体$\homset(RWA,V)$
		\end{itemize} %}
		は集合同型となり、その同型$\phi:\homset(RA,V)\to\homset(RWA,V)$は次の
		可換図によって定めることができる。
		\begin{equation*}\xymatrix{
			RA \ar[r]^{i_W} \ar[rd]_{f} & RWA \ar@{.>}[d]^{\phi f} \\
			& V \\
		}\end{equation*}
	\end{proposition} %prop:自由半代数の普遍性}
	\begin{proof} %{
		$V$の積を$m_\myspace$、単位元を$1_V$とする。
		任意の$f\in\homset(RA,V)$に対して$(\phi f):RWA\to V$を次のように
		定義すると、
		\begin{equation*}\begin{split} %{
			(\phi f)1_W &= 1_V \\
			(\phi f)[a_1a_2\cdots a_m]_W &= (fa_1)(fa_2)\cdots(fa_m)
			\quad\text{for all }a_1,a_2,\dots,a_m\in A
		\end{split}\end{equation*} %}
		$\phi f$は$R$半代数射になり命題の可換図を満たす。
		逆に、任意の$R$半代数射$g\in\homset(RWA,V)$に対して$\phi^{-1}g$を
		$(\phi^{-1}g)a=g[a]$と定義すると、命題の可換図を満たす。
	\end{proof} %}

	テンソル積同士の積を$R$半モジュールに対して定義する。
	$V$を$R$半モジュールとする。
	$V$の自己線形写像全体を$\myop{end}V$または$\homset(V,V)$と書く。
	任意の$\phi\in\myop{end}V$は次の性質を満たす。
	\begin{equation*}\begin{array}{rcll} %{
		\phi(v_1+f_2) &=& (\phi v_1) + (\phi f_2)
			&\text{for all } v_1,v_2\in V \\
		\phi(rv) &=& r(\phi v) &\text{for all }r\in R,\;f\in V \\
	\end{array}\end{equation*} %}
	$V$に双線形二項演算$\beta_\square$が定義された時、
	線形写像$\myhere\square:R\to\myop{end}V$が次のように定義できる。
	\begin{equation*}\begin{split} %{
		(v_1\square)v_2 = v_1\square v_2 \quad\text{for all } v_1,v_2\in V
	\end{split}\end{equation*} %}
	$\beta_\square$が$1_\square$を単位元とする積$m_\square$であれば、
	写像$\myhere\square$は$R$半代数射
	$(V,m_\square,1_\square)\to(\myop{end}V,m_\myspace,\myid)$となる。
	ここで、$\myop{end}V$の積$m_\myspace$は写像の合成である。
	このとき、$V\square\subseteq\myop{end}V$の計算は、
	$(V,m_\square,1_\square)$の計算に翻訳できる。このことが、積を考える
	動機の一つである。

	$\beta_\square$を$V$の双線形二項演算とする。
	テンソル積に対する双線形二項演$\beta_\square$を、
	中置記法$\myhere\square\myhere$で次のように定義する。
	\begin{equation}\begin{split} %{
		&(v_{11}\otimes v_{12}\otimes\cdots\otimes v_{1m})
			\square(v_{21}\otimes v_{22}\otimes\cdots\otimes v_{2m}) \\
		&= (v_{11}\square v_{21})\otimes (v_{12}\square v_{22})\otimes\cdots
			\otimes (v_{1m}\square v_{2m}) \\
		&\quad\text{for all }v_{11},v_{12},\dots,v_{1m},v_{21},v_{22},\dots
			,v_{2m}\in V
	\end{split}\end{equation} %}
	テンソル積をベクトルのように縦に並べて書くと、$\myhere\square\myhere$は
	次のようになる。
	\begin{equation*}\begin{split} %{
		\begin{pmatrix}
			v_{11}\\ v_{12}\\ \vdots\\ v_{1m}
		\end{pmatrix}\square \begin{pmatrix}
			v_{21}\\ v_{22}\\ \vdots\\ v_{2m}
		\end{pmatrix} = \begin{pmatrix}
			v_{11}\square v_{21}\\ v_{12}\square v_{22}\\
			\vdots\\ v_{1m}\square v_{2m}
		\end{pmatrix}
	\end{split}\end{equation*} %}
	この定義は、$V$から$\myop{end}V$への線形写像$\myhere\square$を
	次のように定義したとき、
	\begin{equation*}\begin{split} %{
		(v_1\square)v_2 = v_1\square v_2 \quad\text{for all } v_1,v_2\in V
	\end{split}\end{equation*} %}
	積$m_\square$をとることと、線形写像$\myhere\square$を作用させることが、
	次の式のように同じ形で書けるようにしている。
	\begin{equation}\begin{split} %{
		&\bigl((v_{11}\square)\otimes(v_{12}\square)\otimes\cdots
			\otimes(v_{1m}\square)\bigr)
			(v_{21}\otimes v_{22}\otimes\cdots\otimes v_{2m}) \\
		&= (v_{11}\square v_{21})\otimes (v_{12}\square v_{22})\otimes\cdots
			\otimes (v_{1m}\square v_{2m}) \\
		&=(v_{11}\otimes v_{12}\otimes\cdots\otimes v_{1m})
			\square(v_{21}\otimes v_{22}\otimes\cdots\otimes v_{2m}) \\
		&\quad\text{for all }v_{11},v_{12},\dots,v_{1m},v_{21},v_{22},\dots
			,v_{2m}\in V
	\end{split}\end{equation} %}

	'ケース文'を簡潔に書くために、デルタ関数'$\jump{\mathchar`-}$を定義して
	おく。論理値$\mybf{B}=\set{0_{\mybf{B}},1_{\mybf{B}}}$から半環
	$R=(R,+,0_R,\myspace,1_R)$への写像$\jump{\mathchar`-}:\mybf{B}\mapsto R$
	を次のように定義する。
	\begin{equation*}\begin{split} %{
		0_{\mybf{B}}\mapsto 0_R,\quad 1_{\mybf{B}}\mapsto 1_R
	\end{split}\end{equation*} %}
%s1:文字列の基本事項}

\section{内積}\label{s1:内積} %{
	この節では、係数を複素数$\mybf{C}=(\mybf{C},m_+,0,m_\myspace,1)$に
	固定する。複素共役を$c\in\mybf{C}$に対して$c^\dag$と書く。

	$A$を集合とする。$\mybf{C}A$から$\mybf{C}$への線形写像全体の
	作る空間を$(\mybf{C}A)^\dag$と書く。
	$\mybf{C}A$の基底$a\in A$に双対な$(\mybf{C}A)^\dag$の基底を$a^\dag$と
	書く。
	\begin{equation*}\begin{split} %{
		a_1^\dag a_2 = \jump{a_1=a_2} \quad\text{for all }a_1,a_2\in A
	\end{split}\end{equation*} %}
	写像$\myhere^\dag:\mybf{C}A\to(\mybf{C}A)^\dag$を次のように定義する。
	\begin{equation*}\begin{array}{rcll} %{
		(f_1+f_2)^\dag &=& f_1^\dag + f_2^\dag
			& \quad\text{for all }f_1,f_2\in \mybf{C}A \\
		(cf)^\dag &=& c^\dag f^\dag
			& \quad\text{for all }c\in \mybf{C},\;f\in \mybf{C}A \\
		a^\dag &=& a^\dag & \quad\text{for all }a\in A \\
	\end{array}\end{equation*} %}
	さらに、線形写像$\myhere^\dag:\mybf{C}A^\dag\to\mybf{C}A$を
	同様に定義して、$\myhere^\dag\myhere^\dag=\myid$となるようにする。

	内積は複素共役に対して対称になる。
	\begin{equation*}\begin{split} %{
		(f_1^\dag f_2)^\dag = f_2^\dag f_1 
		\quad\text{for all }f_1,f_2\in\mybf{C}A
	\end{split}\end{equation*} %}
	\begin{proof} %{
		$i=1,2$に対して$f_i=\sum_{a\in A}f_{ia}a\in\mybf{C}A
		,\;f_{ia}\in\mybf{C}\quad\text{for all }a\in A$とする。
		すると、$(f_1^\dag f_2)^\dag=\sum_{a\in A}(f_{1a}^\dag f_{2a})^\dag
		=f_2^\dag f_1=f_2^\dag f_1$となる。
	\end{proof} %}

	$\mybf{C}A\otimes \mybf{C}A$から$\mybf{C}$への線形写像全体の
	作る空間を$(\mybf{C}A\otimes\mybf{C}A)^\dag$と書く。
	$\mybf{C}A\otimes \mybf{C}A$の基底全体の集合を$A\otimes A$と書く。
	任意の$a_1\otimes a_2\in A\otimes A$に双対な元を
	$(a_1\otimes a_2)^\dag\in(A\otimes A)^\dag$と書く。
	\begin{equation*}\begin{split} %{
		(a_1\otimes a_2)^\dag(a_3\otimes a_4) = (a_1^\dag a_3)(a_2^\dag a_4)
		\quad\text{for all }a_1,a_2,a_3,a_4\in A
	\end{split}\end{equation*} %}
	写像$\myhere^\dag:
	\mybf{C}A\times\mybf{C}A\to(\mybf{C}A\otimes\mybf{C}A)^\dag$を
	次のように定義する。
	\begin{equation*}\begin{array}{rcll} %{
		(f_1+f_2)^\dag &=& f_1^\dag + f_2^\dag
			& \quad\text{for all }f_1,f_2\in \mybf{C}A\otimes\mybf{C}A \\
		(cf)^\dag &=& c^\dag f^\dag
			& \quad\text{for all }c\in \mybf{C},\;f\in\mybf{C}A\otimes\mybf{C}A \\
		a^\dag &=& a^\dag & \quad\text{for all }a\in A\otimes A \\
	\end{array}\end{equation*} %}
	さらに、線形写像$\myhere^\dag:
	(\mybf{C}A\times\mybf{C}A)^\dag\to\mybf{C}A\otimes\mybf{C}A$を同様に
	定義して、$\myhere^\dag\myhere^\dag=\myid$となるようにする。
	二次以上のテンソル積についても同様に定義する。

	$(\mybf{C}A\times\mybf{C}A)^\dag$と
	$(\mybf{C}A)^\dag\otimes(\mybf{C}A)^\dag$は、
	$\mybf{C}\simeq\mybf{C}\otimes\mybf{C}$という同一視で、
	次のような同一視をする。
	\begin{equation*}\begin{array}{cccl} %{
		(a_1\otimes a_2)^\dag(a_3\otimes a_4) 
			&\simeq& (a_1^\dag\otimes a_2^\dag)(a_3\otimes a_4) \\
		|| && || & \text{for all }a_1,a_2\in A \\
		(a_1^\dag a_3)(a_2^\dag a_4) 
			&\simeq& (a_1^\dag a_3)\otimes (a_2^\dag a_4) \\
	\end{array}\end{equation*} %}

	\begin{note}[無限の場合]\label{note:無限の場合} %{
		集合$A$が無限集合の場合、$\mybf{C}A$と$\mybf{C}A^\dag$が集合同型
		でなくなるそうだ。$\mybf{C}A$と$\mybf{C}A^\dag$が集合同型でないと、
		$1:1$かつ$\myop{onto}$の関係を使って使って転置を定義することができなく
		なる。ここではまず、有限の場合に成り立つことをナイーブに無限に拡張して、
		成り立つ関係式を列挙する。
	\end{note} %note:無限の場合}

	\begin{example}[無限次元で双対空間との同型対応が破綻する例]\label{eg:無限次元で双対空間との同型対応が破綻する例} %{
		有理数を係数とする多項式全体の集合を$\mybf{Q}[x]$とする。
		$\mybf{Q}[x]$の元$f=\sum_{n\in\mybf{N}}\frac{1}{n!}x^n\in\mybf{Q}[x]$
		に対して、その転置と$\mybf{Q}[x]$の元
		$\frac{1}{1-x}=\sum_{n\in\mybf{N}}x^n\in\mybf{Q}[x]$との内積を
		とると、$f^\dag\frac{1}{1-x}=\exp1\not\in\mybf{Q}$となってしまう。
		こうした事態を防ぐために、$\mybf{Q}[x]$を有限化した部分集合
		$\mybf{Q}_0[x]\subset\mybf{Q}[x]$
		\begin{equation*}\begin{split} %{
			\mybf{Q}_0[x] = \set{f\in\mybf{Q}[x]
				\bou (x^n)^\dag f\neq0 \text{ for only finitely many }n\in\mybf{N}
			}
		\end{split}\end{equation*} %}
		のみを考えるという処置がとられる。定数写像が
		$\left(\frac{1}{1-x}\right)^\dag x^n=1\text{ for all }n\in\mybf{N}$
		で与えられるように、無限和を許さないと応用上有効な理論が構築できなく
		なるので、$\mybf{Q}[x]^\dag$はそのまま使って、$\mybf{Q}[x]^\dag$と
		$\mybf{Q}_0[x]$という組で理論を組み立てることが多いようだ。
		このとき、$\myhere^\dag:\mybf{Q}_0[x]\to\mybf{Q}[x]^\dag$は
		定義できるが、$\myhere^\dag:\mybf{Q}[x]^\dag\to\mybf{Q}_0[x]$は単純には
		定義できなくなってしまう。
		\begin{equation*}\begin{split} %{
			\mybf{Q}_0[x]\xrightarrow[1:1\text{ but not }\myop{onto}]
				{\myhere^\dag}\mybf{Q}[x]^\dag
		\end{split}\end{equation*} %}
		その場合でも、次の図を可換にする埋め込み$i_0$が唯一つ存在するから、
		\begin{equation*}\xymatrix{
			\mybf{Q}[x] \ar@{<->}[r]^{\myhere^\dag} & \mybf{Q}[x]^\dag \\
			\mybf{Q}_0[x] \ar[ur]_{\myhere^\dag} \ar@{.>}[u]^{i_0} \\
		}\end{equation*}
		$\myhere^\dag\myhere^\dag=i_0:\mybf{Q}_0[x]\to\mybf{Q}[x]$という形で
		一方通行の冪等性は成り立つ。
	\end{example} %eg:無限次元で双対空間との同型対応が破綻する例}

	$\mybf{C}A=(\mybf{C}A,m,1_m)$を代数とする。積$m$の中置記号は省略する。
	次の畳み込みによって、余積$\Delta:\mybf{C}A\to \mybf{C}A\otimes \mybf{C}A$
	を定義する。
	\begin{equation*}\xymatrix{
		\mybf{C}A\otimes\mybf{C}A \ar[r]^{m} \ar@{.>}[rd]_{(\Delta f)^\dag} 
			& \mybf{C}A \ar[d]^{f^\dag} \\
		& \mybf{C}
	} \quad\text{for all } f\in\mybf{C}A
	\end{equation*}
	式で書くと次のようになる。
	\begin{equation*}\begin{split} %{
		(\Delta f)^\dag = f^\dag m \quad\text{for all }f\in\mybf{C}A
	\end{split}\end{equation*} %}
	$\Delta$が余積になることは、次の式からわかる。
	\begin{equation*}\begin{array}{cccccc} %{
		\bigl((\Delta\otimes \myid)\Delta f\bigr)^\dag
		&=& (\Delta f)^\dag(m\otimes \myid) &=& f^\dag m(m\otimes \myid) \\
		&&&& ||\quad\quad & \quad\text{for all }f\in\mybf{C}A \\
		\bigl((\myid\otimes \Delta)\Delta f\bigr)^\dag
		&=& (\Delta f)^\dag(\myid\otimes m) &=& f^\dag m(\myid\otimes m) \\
	\end{array}\end{equation*} %}

	余積$\Delta$は次のように積$m$の転置となっているから、余積$\Delta$を
	$m^\dag$とも書く。
	\begin{equation*}\begin{split} %{
		(\Delta f)^\dag(f_1\otimes f_2) = f^\dag m(f_1\otimes f_2)
		\quad\text{for all }f,f_1,f_2\in\mybf{C}A
	\end{split}\end{equation*} %}
	$\Delta$は次の式を満たす必要がある。
	\begin{equation*}\begin{split} %{
		f_1^\dag f_2 = f_1^\dag m(1_m\otimes f_2) 
		= (\Delta f_1)^\dag(1_m\otimes f_2)
		\quad\text{for all }f_1,f_2\in\mybf{C}A
	\end{split}\end{equation*} %}
	したがって、$\mybf{C}\otimes\mybf{C} \simeq \mybf{C}$の同一視で、
	次の式が成り立つ必要がある。
	\begin{equation*}\begin{split} %{
		f_2^\dag f_1 \simeq (1_m^\dag\otimes f_2^\dag)\Delta f_1
		\quad\text{for all }f_1,f_2\in\mybf{C}A
	\end{split}\end{equation*} %}
	この式は余積$\Delta$に対する余単位射$1_m^\dag$の定義に他ならない。
	したがって、積$m$から転置によって余積$\Delta$を定義したときには、
	その余単位射は単位元$1_m$の転置$1_m^\dag$になる。

	以上を定義と命題の形にまとめておく。

	\begin{definition}[積の転置]\label{def:積の転置} %{
		$\mybf{C}A=(\mybf{C}A,m,1_m)$を代数とする。
		次の式で定義された余積$\Delta$を積$m$の転置という。
		\begin{equation*}\begin{split} %{
			(\Delta f)^\dag = f^\dag m \quad\text{for all }f\in\mybf{C}A
		\end{split}\end{equation*} %}
	\end{definition} %def:積の転置}

	\begin{proposition}[単位元の転置]\label{prop:単位元の転置} %{
		$\mybf{C}A=(\mybf{C}A,m,1_m)$を代数とする。
		積$m$の転置による余積の余単位射は単位元$1_m$の転置$1_m^\dag$となる。
	\end{proposition} %prop:単位元の転置}

	基底を使って転置による余積を書き表してみる。
	$\Delta$を積$m$の転置による余積とする。
	\begin{equation*}\begin{split} %{
		a^\dag(a_1a_2) = (\Delta a)^\dag(a_1\otimes a_2)
		\quad\text{for all }a,a_1,a_2\in A
	\end{split}\end{equation*} %}
	$\Delta$が具体的に求まり、次のようになる。
	\begin{equation*}\begin{split} %{
		\Delta = \sum_{a_1,a_2\in A}(a_1\otimes a_2)(a_1a_2)^\dag
	\end{split}\end{equation*} %}

	$m\Delta$はエルミートなので対角化可能で、$\Delta$の二乗
	$m\Delta=\Delta^\dag \Delta$なので非負の固有値を持つ。
	$\Delta m$も同様である。
	$m\Delta$と$\Delta m$の固有値について次の命題が成り立つ。

	\begin{proposition}[非ゼロ固有値の対応]\label{prop:非ゼロ固有値の対応} %{
		$0$でない$f_\lambda\in\mybf{C}A$を$m\Delta$の固有値$\lambda$を持つ
		固有ベクトルとする。$\Delta f_\lambda\neq0$ならば、
		\begin{itemize}\setlength{\itemsep}{-1mm} %{
			\item $\Delta f_\lambda$は$\Delta m$の固有値$\lambda$を持ち、
			\item $0<\lambda$となる。
		\end{itemize} %}
	\end{proposition} %prop:非ゼロ固有値の対応}
	\begin{proof} %{
		任意の$f\in\mybf{C}A$に対して二乗ノルムを$\zettai{f}^2=f^\dag f$と書く。

		$\Delta m\Delta f_\lambda=\lambda\Delta f_\lambda$より、
		$\Delta f_\lambda$が$\Delta m$の固有値$\lambda$を持つことがわかる。

		$m\Delta f_\lambda=\lambda f_\lambda$より、
		$\lambda\zettai{f_\lambda}^2=\zettai{\Delta f_\lambda}^2$となるから、
		$0<\zettai{\Delta f_\lambda}^2$より、$0<\lambda$となる。
	\end{proof} %}

	\begin{proposition}[ゼロ固有値の非対応]\label{prop:ゼロ固有値の非対応} %{
		任意の$f_0\in\mybf{C}A$に対して$m\Delta f_0=0\implies\Delta f_0=0$が
		成り立つ。
	\end{proposition} %prop:ゼロ固有値の非対応}
	\begin{proof} %{
		$f_0=0$ならば命題が成り立つことはすぐわかるから、$f_0\neq0$とする。
		$\Delta f_0\neq0$ならば、命題\ref{prop:非ゼロ固有値の対応}より、
		$f_0$は$0<\lambda$となる$m\Delta$の固有値$\lambda$を持つ必要があり、
		仮定に矛盾する。
	\end{proof} %}

	$\Delta m$に対しても同様の命題が成り立つ。

	\begin{proposition}[テンソル積での固有値の対応]\label{prop:テンソル積での固有値の対応} %{
		$0$でない$f_\lambda\in\mybf{C}A\otimes\mybf{C}A$を$\Delta m$の固有値
		$\lambda$を持つ固有ベクトルとする。$\Delta f_\lambda\neq0$ならば、
		\begin{itemize}\setlength{\itemsep}{-1mm} %{
			\item $\Delta f_\lambda$は$\Delta m$の固有値$\lambda$を持ち、
			\item $0<\lambda$となる。
		\end{itemize} %}
		逆に、固有値$\lambda$が$0$ならば、$mf_\lambda=0$となる。
	\end{proposition} %prop:テンソル積での固有値の対応}
	\begin{proof} %{
		命題\ref{prop:非ゼロ固有値の対応}と命題\ref{prop:ゼロ固有値の非対応}
		と同じようにする。
	\end{proof} %}

	命題\ref{prop:非ゼロ固有値の対応}、命題\ref{prop:ゼロ固有値の非対応}
	、命題\ref{prop:テンソル積での固有値の対応}により、$m\Delta$の$0$より
	大きい固有値の分布と$\Delta m$の$0$のより大きい固有値の分布は等しくなる。
	$m\Delta$と$\Delta m$の$0$より大きい固有値の集合を$\Lambda_+$とする。
	$m\Delta$の固有値$\lambda$の固有空間を
	$(\mybf{C}A)_\lambda\subseteq\mybf{C}A$、
	$\Delta m$の固有値$\lambda$の固有空間を
	$(\mybf{C}A\otimes\mybf{C}A)_\lambda\subseteq\mybf{C}A\otimes\mybf{C}A$
	と書く。$\mybf{C}A$と$\mybf{C}A\otimes\mybf{C}A$は、固有空間で
	直和分解されて次のようになる。
	\begin{equation*}\begin{split} %{
		\mybf{C}A 
		&= \oplus_{\lambda\in\set{0}\cup\Lambda_+}(\mybf{C}A)_\lambda \\
		\mybf{C}A\otimes\mybf{C}A
		&= \oplus_{\lambda\in\set{0}\cup\Lambda_+}
			(\mybf{C}A\otimes\mybf{C}A)_\lambda
	\end{split}\end{equation*} %}
	$\lambda\in\Lambda_+$に対して、$\Delta$の定義域を$(\mybf{C}A)_\lambda$に
	制限したものを$\Delta_\lambda$と書く。同様に、$\lambda\in\Lambda_+$
	に対して、$m$の定義域を$(\mybf{C}A\times\mybf{C}A)_\lambda$に制限したもの
	を$m_\lambda$と書く。
	\begin{equation*}\begin{split} %{
		(\mybf{C}A)_\lambda 
		\overset{\Delta_\lambda}{\underset{m_\lambda}{\rightleftarrows}}
		(\mybf{C}A\otimes\mybf{C}A)_\lambda
		\quad\text{for all }\lambda\in\Lambda_+
	\end{split}\end{equation*} %}
	固有値$0$の固有空間の間に$\Delta$と$m$による対応関係はない。

	任意の$\lambda\in\Lambda_+$、$f_1,f_2\in(\mybf{C}A)_\lambda$に対して
	$\Delta f_1=\Delta f_2$ならば、$\Delta(f_1-f_2)=0$となり、
	命題\ref{prop:非ゼロ固有値の対応}より、$f_1=f_2$となる。
	したがって、$\Delta_\lambda$は$1:1$となる。また、
	任意の$\lambda\in\Lambda_+$、
	$f\neq0\in(\mybf{C}A\otimes\mybf{C}A)_\lambda$に対して
	$\Delta mf=\lambda f$となるが、$f=\Delta\left(\frac{1}{\lambda}mf\right)$
	となり、$\Delta_\lambda$は$\myop{onto}$となる。
	以上より、任意の$\lambda\in\Lambda_+$に対して$\Delta_\lambda$は
	$(\mybf{C}A)_\lambda$と$(\mybf{C}A\otimes\mybf{C}A)_\lambda$の
	ベクトル同型を与える。$m_\lambda$がベクトル同型を与えることも同様に
	示される。
	
	以上を命題のかたちでまとめる。

	\begin{proposition}[非ゼロ固有空間の同型射]\label{prop:非ゼロ固有空間の同型射} %{
		任意の$\lambda\in\Lambda_+$に対して、
		$\Delta_\lambda$は次のベクトル同型を与え、
		\begin{equation*}\begin{split} %{
			\Delta_\lambda: (\mybf{C}A)_\lambda
			\xrightarrow{\simeq} (\mybf{C}A\otimes\mybf{C}A)_\lambda
		\end{split}\end{equation*} %}
		$m_\lambda$は次のベクトル同型を与える。
		\begin{equation*}\begin{split} %{
			m_\lambda: (\mybf{C}A\otimes\mybf{C}A)_\lambda
			\xrightarrow{\simeq} (\mybf{C}A)_\lambda
		\end{split}\end{equation*} %}
	\end{proposition} %prop:非ゼロ固有空間の同型射}

	積とその転置の関係の例を見てみる。

	\begin{example}[モノイドの積の転置]\label{eg:モノイドの積の転置} %{
		$G=(G,m,1_G)$をモノイド、$\mybf{C}G=(RG,m,1_G)$を$G$を基底とし、
		$G$の積$m$を線形に拡張した自由$\mybf{C}$代数とする。
		$\Delta$を積$m$の転置とする。
		このとき、$g_1g_2\in G$となり、$\Delta$は次のようになる。
		\begin{equation*}\begin{split} %{
			\Delta g = \sum_{g_1,g_2\in G}\jump{g=g_1g_2}(g_1\otimes g_2)
			\quad\text{for all }g\in G
		\end{split}\end{equation*} %}
		また、$m\Delta$は基底$G$で対角化されていて、次のようになる。
		\begin{equation*}\begin{split} %{
			m\Delta g = (\sharp_m g)g \quad\text{for all }g\in G
		\end{split}\end{equation*} %}
		ここで、線形写像$\sharp_m:G\to \mybf{N}$を次のように定義した。
		\begin{equation*}\begin{split} %{
			\sharp_m g = \sum_{g_1,g_2\in G}\jump{g=g_1g_2}
			\quad\text{for all }g\in G
		\end{split}\end{equation*} %}

		ある$g_0\in G$が$\Delta g_0$となれば、すべての$g_1,g_2\in G$に対して
		$g_0^\dag(g_1g_2)=0$となるから、$g_0$はすべての$G$の元と直交すること
		になり、転置の定義と矛盾する。したがって、$\Delta$は固有値$0$を
		持たない。一方、$m$は次のベクトル$(g_1\otimes g_2)_0\in G\otimes G$を
		$0$固有ベクトルとして持つ。
		\begin{equation*}\begin{split} %{
			(g_1\otimes g_2)_0 = \bigl((\sharp_m g_1g_2)- \Delta m\bigr)
			(g_1\otimes g_2)\quad\text{for all }g_1,g_2\in G
		\end{split}\end{equation*} %}
		ただし、次のようになるから、
		\begin{equation*}\begin{split} %{
			\sum_{g_1,g_2\in G}\jump{g_1g_2=g}(g_1\otimes g_2)_0=0
			\quad\text{for all }g\in G
		\end{split}\end{equation*} %}
		$(g_1\otimes g_2)_0$は$\sharp_m g$個すべてが一次独立ではなく、
		$(\sharp_m g)-1$個のみが一次独立になる。力学での重心$\Delta g$
		とその周りの相対座標$(g_1\otimes g_2)_0$と対応している。
	\end{example} %eg:モノイドの積の転置}

	\begin{example}[群の積の転置]\label{eg:群の積の転置} %{
		$G=(G,m,1_G)$を群、$\mybf{C}G=(RG,m,1_G)$を$G$を基底とし、
		$G$の積$m$を線形に拡張した自由$\mybf{C}$代数とする。
		$\Delta$を積$m$の転置とする。
		このとき、$\Delta$は次のようになる。
		\begin{equation*}\begin{split} %{
			\Delta = \sum_{g_1,g_2\in G}(g_1\otimes g_1^{-1}g_2)g_2^\dag
		\end{split}\end{equation*} %}
		また、例\ref{eg:モノイドの積の転置}の写像$\sharp_m$は次のように、
		$G$の元の数$\zettai{G}$を与える定数になる。
		\begin{equation*}\begin{split} %{
			\sharp_m g = \sum_{g_1,g_2\in G}\jump{g=g_1g_2} = \zettai{G}
			\quad\text{for all }g\in G
		\end{split}\end{equation*} %}
		したがって、$m\Delta$は定数$m\Delta=\zettai{G}$になる。
		一方、$\Delta m$は次のようになる。
		\begin{equation*}\begin{split} %{
			\Delta m(g_1\otimes g_2) = \sum_{h\in G}(g_1h\otimes h^{-1}g_2)
			\quad\text{for all }g_1,g_2\in G
		\end{split}\end{equation*} %}
	\end{example} %eg:群の積の転置}

	\begin{example}[自然数の乗法の転置]\label{eg:自然数の乗法の転置} %{
		$\mybf{N}_+$を基底とする自由半モジュールで、自然数の乗法を積$m$とし、
		その転置を$\Delta=m^\dag$とする。
		\begin{equation*}\begin{split} %{
			\Delta\ket{m} = \sum_{m_1,m_2\in\mybf{N}_+}
			\jump{m=m_1\cdot m_2}\ket{m_1}\otimes\ket{m_2}
			\quad\text{for all }m\in \mybf{N}_+
		\end{split}\end{equation*} %}
		特に、素数に対しては次のようになる。
		\begin{equation*}\begin{split} %{
			\Delta\ket{p} = \ket{1}\otimes\ket{p}+\ket{p}\otimes\ket{1}
			\quad\text{for all }p\in '1\text{でない素数}'
		\end{split}\end{equation*} %}
		リー環論で、余積が$v\otimes 1+1\otimes v$となるリー環の元$v$を素な元
		というが、この'素'という形容詞は自然数の乗法とのアナロジーなのだろう。
		さらに、素数に対しては次の式が成り立つ。
		\begin{equation*}\begin{split} %{
			\Delta\ket{p^m} = \sum_{0\le k\le m}\ket{p^k}\otimes\ket{p^{m-k}}
			= (\Delta\ket{p})^{m}
			\quad\text{for all }p\in '1\text{でない素数}'
		\end{split}\end{equation*} %}
		また、次の式も成り立つ。
		\begin{equation*}\begin{split} %{
			\Delta\ket{p_1^{m_1}p_2^{m_2}\cdots p_n^{m_n}} 
			= (\Delta\ket{p_1})^{m_1}(\Delta\ket{p_2})^{m_2}
				\cdots(\Delta\ket{p_n})^{m_n} \\
			\quad\text{for all }
			p_1,p_2,\dots,p_n\in '1\text{でない互いに異なる素数}'
		\end{split}\end{equation*} %}
		よって、次の式が成り立つ。
		\begin{equation*}\begin{split} %{
			m\Delta\ket{p_1^{m_1}p_2^{m_2}\cdots p_n^{m_n}}
			= 2^{m_1+m_2+\cdots+m_n}
			\ket{p_1^{m_1}p_2^{m_2}\cdots p_n^{m_n}} \\
			\quad\text{for all }
			p_1,p_2,\dots,p_n\in '1\text{でない互いに異なる素数}'
		\end{split}\end{equation*} %}
		したがって、$m\Delta\ket{m}$は$m\in\myop{N}$に含まれる$1$でない素数
		の重複を含めた数の和を($2$の冪のかたちで)与える。
	\end{example} %eg:自然数の乗法の転置}

	\begin{example}[文字列の連結の転置]\label{eg:文字列の連結の転置} %{
		$WA$の積$m_*$の転置を$\Delta_*$する。
		例\ref{eg:モノイドの積の転置}により、$\Delta_*$は次のようになり、
		\begin{equation*}\begin{split} %{
			\Delta_*1_W &= 1_W\otimes 1_W \\
			\Delta_*[a_1a_2\cdots a_m] &= 1_W\otimes [a_1a_2\cdots a_m] \\
			&\;+ [a_1]\otimes [a_2\cdots a_m] \\
			&\;+ [a_1a_2]\otimes [a_3\cdots a_m] \\
			&\;+ \cdots \\
			&\;+ [a_1a_2\cdots a_m]\otimes 1_W \\
			&\quad\text{for all }a_1,a_2,\dots, a_m\in A
		\end{split}\end{equation*} %}
		$m_*\Delta_*$は次のようになる。
		\begin{equation*}\begin{split} %{
			m_*\Delta_*w = (\zettai{w}+1)w \quad\text{for all }w\in WA
		\end{split}\end{equation*} %}
		$\Delta_*m_*$は次の式が成り立つ。
		\begin{equation*}\begin{split} %{
			\Delta_*m_* + \myid
			= (\myid\otimes m_*)(\Delta_*\otimes \myid)
			+ (m_*\otimes \myid)(\myid\otimes \Delta_*)
		\end{split}\end{equation*} %}
	\end{example} %eg:文字列の連結の転置}
	\begin{proof} %{
		例\ref{eg:文字列の連結の転置}の最後の式を証明する。
		例\ref{eg:モノイドの積の転置}より、$\Delta_*$は次のようになる。
		\begin{equation*}\begin{split} %{
			\Delta m(w_1\otimes w_2)
			= \sum_{x_1,x_2\in WA}\jump{w_1*w_2=x_1*x_2}(x_1\otimes x_2) \\
			\quad\text{for all }w_1,w_2\in WA
		\end{split}\end{equation*} %}
		文字列の場合、任意の$w\in WA$に対して、
		\begin{itemize}\setlength{\itemsep}{-1mm} %{
			\item 文字数$p$の元$x_1\in WA$と、
			\item 文字数$\zettai{w}-p$の元$x_2\in WA$で、
		\end{itemize} %}
		$w=x_1*x_2$となる分割$x_1\otimes x_2$は、$0\le p\le \zettai{w}$
		のとき一意に定まる。この分割を$w$の$p$分割と書く。
		そして、$\Delta_*w$は、任意の$w\in WA$に対して、
		$0\le p\le \zettai{w}$となる$w$の$p$分割をすべてただ一度
		だけ列挙する。したがって、任意の$w\in WA$に対して、
		\begin{itemize}\setlength{\itemsep}{-1mm} %{
			\item $(\Delta_*w_1)*(1_W\otimes w_2)$は、$0\le p\le \zettai{w_1}$
			となる$w_1*w_2$の$p$分割を、すべてただ一度だけ列挙し、
			\item $(1_W\otimes w_1)*(\Delta_*w_2)$は、
			$\zettai{w_1}\le p\le \zettai{w_1}+\zettai{w_2}$となる$w_1*w_2$の
			$p$分割を、すべてただ一度だけ列挙する。
		\end{itemize} %}
		したがって、任意の$w_1,w_2\in WA$に対して次の式が成り立つ。
		\begin{equation*}\begin{split} %{
			&(\Delta_*w_1)*(1_W\otimes w_2)+(1_W\otimes w_1)*(\Delta_*w_2) \\
			&= \left(\sum_{0\le p\le \zettai{w_1}+\zettai{w_2}}
				w_1*w_2\text{の}p\text{分割}\right) + w_1\otimes w_2 \\
			&= \Delta_*(w_1*w_2) + w_1\otimes w_2 \\
		\end{split}\end{equation*} %}
	\end{proof} %}

	\begin{example}[写像の合成の転置]\label{eg:写像の合成の転置} %{
		$\myop{end}(\mybf{C}A)$の基底は
		$AA^\dag=\set{a_1a_2^\dag}_{a_1,a_2\in A}$ととれる。
		この基底を使って写像の合成の転置を計算すると次のようになる。
		\begin{equation*}\begin{split} %{
			\Delta(a_1a_2^\dag)
			= \sum_{a_3,a_4,a_5,a_6\in A}
				\jump{a_1a_2^\dag=a_3a_4^\dag a_5a_6^\dag}
				a_3a_4^\dag\otimes a_5a_6^\dag
			= \sum_{a\in A}a_1a^\dag\otimes aa_2^\dag \\
			\quad\text{for all }a_1,a_2\in A
		\end{split}\end{equation*} %}
		この式は、群の例\ref{eg:群の積の転置}によく似ている。
		線形写像$\mu:RA\otimes RA\to \myop{end}(RWA)$を次のように定義すると、
		\begin{equation*}\begin{split} %{
			\mu(a_1\otimes a_2) = a_1a_2^\dag
			\quad\text{for all }a_1,a_2\in A
		\end{split}\end{equation*} %} 
		群と写像の合成は次のように対比できる。
		\begin{equation*}\begin{array}{rrcl} %{
			\text{群}&
			\Delta m(g_1\otimes g_2)
				&=& \sum_{g\in G} m(g_1\otimes g)\otimes m(g^{-1}\otimes g_2) \\
			\text{写像の合成}&
			\Delta \mu(a_1\otimes a_2) 
				&=& \sum_{a\in A}\mu(a_1\otimes a)\otimes \mu(a\otimes a_2) \\
		\end{array}\end{equation*} %}
		モノイドを群になりかけているものと思えば、この類似は'そういうものか'
		と思える。
	\end{example} %eg:写像の合成の転置}
%s1:内積}

\section{積による作用素}\label{s1:積による作用素} %{
	この節では、係数を複素数$\mybf{C}=(\mybf{C},m_+,0,m_\myspace,1)$に
	固定する。

	$A$を集合、$\myop{C}A=(\myop{C}A,m_\square,1_\square)$を自由代数とする。
	$\Delta_\square$を積$m_\square$の転置とする。

	積$m_\square$によって、線形写像
	$\myhere\square:\myop{C}A\to\myop{end}(\myop{C}A)$が次のように定義できる。
	\begin{equation*}\begin{split} %{
		(a_1\square)a_2 = a_1\square a_2 \quad\text{for all } a_1,a_2\in A
	\end{split}\end{equation*} %}
	定義から写像$\myhere\square$は代数射$(\myop{C}A,m_\square,1_\square)\to
	(\myop{end}(\myop{C}A),m_\myspace,\myid)$となる。
	ここで、$\myop{end}(\myop{C}A)$の積$m_\myspace$は写像の合成とする。

	作用素$a\square$の転置$(a\square)^\dag$は次のように定義される。
	\begin{equation*}\begin{split} %{
		\bigl((a\square)a_1\bigr)a_2 = a_1^\dag\bigl((a\square)^\dag a_2\bigr)
		\quad\text{for all } a,a_1,a_2\in A
	\end{split}\end{equation*} %}
	写像$(\myhere\square)^\dag$は、積の転置$\Delta_\square=m_\square^\dag$
	を使って次のように書ける。
	\begin{equation*}\begin{split} %{
		(a\square)^\dag= m_\square(1_\square a^\dag\otimes \myid)\Delta_\square
		\quad\text{for all } a\in A
	\end{split}\end{equation*} %}
	大雑把には、$(\myhere\square)^\dag$は余積$\Delta_\square$の第二成分を
	取り出す操作になる。
	一般に、転置$\myhere^\dag:\myop{end}(\myop{C}A)\to\myop{end}(\myop{C}A)$
	は逆順代数同型となる。
	\begin{equation*}\begin{split} %{
		(\phi_1\phi_2\cdots \phi_m a_1)^\dag a_2
		= (\phi_2\cdots \phi_m a_1)^\dag\phi_1^\dag a_2
		= \cdots
		= a_1^\dag \phi_m^\dag\cdots \phi_2^\dag\phi_1^\dag a_2 \\
		\quad\text{for all }a_1,a_2\in A
			,\;\phi_1,\phi_2,\dots,\phi_m\in\myop{end}(\myop{C}A)
	\end{split}\end{equation*} %}
	したがって、写像$(\myhere\square)^\dag$は、積$m_\square$について
	逆順代数射になることがわかる。
	ここでは、写像$(\myhere\square)^\dag$の定義から計算して、
	$(\myhere\square)^\dag$が逆順代数射になっていることを確かめる。
	$\myop{C}\otimes\myop{C}A\simeq\myop{C}A\simeq\myop{C}A\otimes\myop{C}$
	の同一視を使うと、作用素の積$(\myhere\square)^\dag(\myhere\square)^\dag$
	は次のようになる。
	\begin{equation*}\begin{array}{rll} %{
		(a_1\square)^\dag(a_2\square)^\dag
		&= m_\square(1_\square a_1^\dag \otimes \myid)\Delta_\square
		 m_\square(1_\square a_2^\dag \otimes \myid)\Delta_\square \\
		&\simeq (a_2^\dag\otimes a_1^\dag \otimes \myid)
		 (\myid \otimes \Delta_\square)\Delta_\square 
		 &\quad\lcomment{$\myop{C}A\to \myop{C}\otimes\myop{C}\otimes\myop{C}A$} \\
		&\simeq \bigl((a_2\otimes a_1)^\dag \otimes \myid\bigr)
		 (\myid \otimes \Delta_\square)\Delta_\square 
		 &\quad\lcomment{$\myop{C}A\to \myop{C}\otimes\myop{C}A$} \\
		&= \bigl((a_2\otimes a_1)^\dag \otimes \myid\bigr)
		 (\Delta_\square \otimes \myid)\Delta_\square 
		 &\quad\lcomment{余積の余結合性} \\
		&= \bigl((a_2\square a_1)^\dag \otimes \myid\bigr)\Delta_\square 
		 &\quad\lcomment{$\Delta_\square$は$m_\square$の転置} \\
		&\simeq m_\square\bigl(1_W(a_2\square a_1)^\dag \otimes \myid\bigr)
			\Delta_\square 
			&\quad\lcomment{$V\to V$} \\
		&\quad\text{for all }a_1,a_2\in A
	\end{array}\end{equation*} %}
	よって、写像$(\myhere\square)^\dag$は積$m_\square$について逆順代数射となる
	ことがわかる。

	線形写像$\square\myhere:\mybf{C}A\to\myop{end}(\myop{C}A)$を次のように
	定義する。
	\begin{equation*}\begin{split} %{
		(\square a_1)a_2 = a_2\square a_1 \quad\text{for all } a_1,a_2\in A
	\end{split}\end{equation*} %}
	次の式が成り立ち、写像$\square\myhere$は同型な逆順代数射になることが
	わかる。
	\begin{equation*}\begin{split} %{
		(\square a_1)(\square a_2)a = a\square a_2\square a_1 
		=  \bigl((a_2\square a_1)\square\bigr)a
		\quad\text{for all } a,a_1,a_2\in A
	\end{split}\end{equation*} %}
	積$m_\square$の結合性により、$\mybf{C}A\square$と$\square\mybf{C}A$は
	可換になる。
	\begin{equation*}\begin{split} %{
		(a_1\square)(\square a_2)a
		= a_1\square v\square a_2
		= (a_2\square)(\square a_1)a 
		\quad\text{for all } a,a_1,a_2\in A
	\end{split}\end{equation*} %}
	また、写像$(\square\myhere)^\dag$は、積の転置
	$\Delta_\square=m_\square^\dag$を使って次のように書ける。
	\begin{equation*}\begin{split} %{
		(\square a)^\dag= m_\square(\myid\otimes 1_\square a^\dag)\Delta_\square
		\quad\text{for all } a\in A
	\end{split}\end{equation*} %}
	写像$(\square\myhere)^\dag$は積$m_\square$について正順代数射となる。

	以上を定義と命題のかたちでまとめる。

	\begin{definition}[積による作用素]\label{def:積による作用素} %{
		$\myop{C}A=(\myop{C}A,m_\square,1_\square)$を自由代数とする。
		次のよう定義された線形写像$\myhere\square$を積$m_\square$による
		左から積をとる作用素ということにする。
		\begin{equation*}\begin{split} %{
			\myhere\square:\mybf{C}A &\to \myop{end}(\myop{C}A) \\
			(a_1\square)a_2 &= a_1\square a_2 
			\quad\text{for all }a_1,a_2\in A
		\end{split}\end{equation*} %}
		次のよう定義された線形写像$\square\myhere$を積$m_\square$による
		右から積をとる作用素ということにする。
		\begin{equation*}\begin{split} %{
			\square\myhere:\mybf{C}A &\to \myop{end}(\mybf{C}A) \\
			(\square a_1)a_2 &= a_2\square a_1
			\quad\text{for all }a_1,a_2\in A
		\end{split}\end{equation*} %}
	\end{definition} %def:積による作用素}

	\begin{proposition}[積による作用素の性質]\label{prop:積による作用素の性質} %{
		$\myop{C}A=(\myop{C}A,m_\square,1_\square)$を自由代数とする。
		$\Delta_\square$を積$m_\square$の転置とする。
		このとき、積$m_\square$による作用素に関して次の事柄が成り立つ。
		\begin{itemize}\setlength{\itemsep}{-1mm} %{
			%
			\item 転置への写像$(\myhere\square)^\dag$は次のようになり、
			\begin{equation*}\begin{split} %{
				(a\square)^\dag 
				= m_\square(1_\square a^\dag\otimes \myid)\Delta_\square
				\quad\text{for all }a\in A
			\end{split}\end{equation*} %}
			転置への写像$(\square\myhere)^\dag$は次のようになる。
			\begin{equation*}\begin{split} %{
				(\square a)^\dag 
				= m_\square(\myid\otimes 1_\square a^\dag)\Delta_\square
				\quad\text{for all }a\in A
			\end{split}\end{equation*} %}
			%
			\item $\myhere\square$と$(\square\myhere)^\dag$は代数射になり、
			$\square\myhere$と$(\myhere\square)^\dag$は逆順代数射となる。
			%
			\item 像$\myop{C}A\square\subseteq\myop{end}(\myop{C}A)$と
			像$\square\myop{C}A\subseteq\myop{end}(\myop{C}A)$は可換、
			像$(\myop{C}A\square)^\dag\subseteq\myop{end}(\myop{C}A)$と
			像$(\square\myop{C}A)^\dag\subseteq\myop{end}(\myop{C}A)$は可換
			になる。
		\end{itemize} %}
	\end{proposition} %prop:積による作用素の性質}

	\begin{example}[自然数]\label{eg:自然数} %{
		自然数を基底とする自由半モジュールで、
		自然数の加法$m_+$による作用素は次のようになる。
		\begin{equation*}\begin{split} %{
			\begin{array}{rl}
				(\ket{m}+)\ket{n} &= \ket{n+m} \\
				(\ket{m}+)^\dag\ket{n} &= \jump{0\le n-m}\ket{n-m} \\
			\end{array}
			\quad\text{for all }m,n\in\mybf{N}
		\end{split}\end{equation*} %}
		$(m+)^\dag n$は$\myop{max}(m\times n)$や$\zettai{m-n}$ではなく、
		$\jump{0\le n-m}(n-m)$となる。
		自然数の乗法$m_\cdot$による作用素は次のようになる。
		\begin{equation*}\begin{split} %{
			\begin{array}{rl}
				(\ket{m}\cdot)\ket{n} &= \ket{m\cdot n} \\
				(\ket{m}\cdot)^\dag\ket{n} 
				&= \jump{\frac{n}{m}\in\mybf{N}}\ket{\frac{n}{m}} \\
			\end{array}
			\quad\text{for all }m,n\in\mybf{N}
		\end{split}\end{equation*} %}
	\end{example} %eg:自然数}
%s1:積による作用素}

\section{正規言語}\label{s1:正規言語} %{
	\begin{note}[半モジュールの係数]\label{note:半モジュールの係数} %{
		自然数$\mybf{N}$からor-andブーリアン$\mybf{B}$への写像$f$を次のように
		定義する。
		\begin{equation*}\begin{split} %{
			fm = \jump{m\neq0} \quad\text{for all }m\in\mybf{N}
		\end{split}\end{equation*} %}
		すると、次の可換図が成り立ち、$f$は半環準同型になるこことがわかる。
		\begin{equation*}\xymatrix@C+2pc{
			m_1\times m_2 \ar[r]^{m_+} \ar[d]^{f\times f} 
			& m_1+m_2 \ar[d]^{f} \\
			\jump{m_1\neq0}\times \jump{m_2\neq0} \ar[r]^{m_\lor}
			& \jump{m_1\neq0}\lor\jump{m_2\neq0}=\jump{m_1+m_2\neq0}
		}\end{equation*}
		\begin{equation*}\xymatrix@C+2pc{
			m_1\times m_2 \ar[r]^{m_\myspace} \ar[d]^{f\times f} 
			& m_1m_2 \ar[d]^{f} \\
			\jump{m_1\neq0}\times \jump{m_2\neq0} \ar[r]^{m_\land}
			& \jump{m_1\neq0}\land\jump{m_2\neq0}=\jump{m_1m_2\neq0}
		}\end{equation*}
		したがって、自然数を係数とする半モジュールで作られた理論はそのまま
		or-andブーリアンを係数とする理論に移行できると思われる。
		逆の写像$\mybf{B}\to\mybf{N}$では半環準同型は多分存在しない。
		したがって、or-andブーリアンを係数とする半モジュールで作られた理論
		は自然数を係数とする理論へ移行することはできないと思われる。
	\end{note} %note:半モジュールの係数}

	\subsection{Brzozowsky微分}\label{s2:Brzozowsky微分} %{
		文字列に対する作用素$(\myhere*)^\dag$のライプニッツ規則に相当する式を
		導く。

		$R$を標数$0$の環、$A$を集合、$WA=(WA,m_*,1_W)$を$A$から生成された
		自由モノイド、$RWA$を自由モノイド環とする。

		余積$\Delta_*$を積$m_*$の転置とすると、作用素$(WA*)^\dag$は次のように
		書ける。
		\begin{equation*}\begin{split} %{
			(w*)^\dag = m_*(1_Ww^\dag\otimes \myid)\Delta_*
			\quad\text{for all }w\in WA
		\end{split}\end{equation*} %}
		例\ref{eg:文字列の連結の転置}から、次の式が成り立つので、
		\begin{equation*}\begin{split} %{
			\Delta_*m_* = (\myid\otimes m_*)(\Delta_*\otimes \myid)
			+ (m_*\otimes \myid)(\myid\otimes \Delta_*) - \myid
		\end{split}\end{equation*} %}
		次の式が成り立つ。
		\begin{equation*}\begin{split} %{
			(w*)^\dag m_* &= m_*(1_Ww^\dag\otimes \myid)\Delta_*m_* \\
			&= m_*(1_Ww^\dag\otimes m_*)(\Delta_*\otimes \myid)
				+ m_*(1_Ww^\dag m_*\otimes \myid)(\myid\otimes \Delta_*) \\
				&\;- m_*(1_Ww^\dag\otimes \myid) \\
			&\quad\text{for all }w\in WA
		\end{split}\end{equation*} %}
		一項目は次のようになり、
		\begin{equation*}\begin{array}{ll} %{
			m_*(1_Ww^\dag\otimes m_*)(\Delta_*\otimes \myid) \\
			= m_*(m_*\otimes \myid)(1_Ww^\dag\otimes \myid\otimes \myid)
				(\Delta_*\otimes \myid) & \quad\lcomment{$m_*$の結合性} \\
			= m_*((w*)^\dag\otimes \myid) 
				& \quad\lcomment{$(\myhere_*)^\dag$の定義} \\
		\end{array}\end{equation*} %}
		二項目は次のようになる。
		\begin{equation*}\begin{array}{ll} %{
			m_*(1_Ww^\dag m_*\otimes \myid)(\myid\otimes \Delta_*) \\
			= m_*\bigl(1_W(\Delta_*w)^\dag\otimes \myid\bigr)
				(\myid\otimes \Delta_*) & \quad\lcomment{$\Delta_*$の定義} \\
			= m_*(m_*\otimes \myid)\bigl(1_W(\Delta_{*(1)}w)^\dag
				\otimes 1_W(\Delta_{*(2)}w)^\dag\otimes \myid\bigr)
				(\myid\otimes \Delta_*) \\
			= m_*(\myid\otimes m_*)\bigl(1_W(\Delta_{*(1)}w)^\dag
				\otimes 1_W(\Delta_{*(2)}w)^\dag\otimes \myid\bigr)
				(\myid\otimes \Delta_*) & \quad\lcomment{$m_*$の結合性} \\
			= m_*\biggl(1_W(\Delta_{*(1)}w)^\dag
				\otimes \bigl((\Delta_{*(2)}w)*\bigr)^\dag\biggr)
				& \quad\lcomment{$(\myhere_*)^\dag$の定義} \\
		\end{array}\end{equation*} %}
		したがって、ライプニッツ規則に似た次の式が成り立つ。
		\begin{equation*}\begin{split} %{
			(w*)^\dag m_* &= m_*(\Gamma w) \quad\text{for all }w\in WA
		\end{split}\end{equation*} %}
		ここで、$\Gamma$は線形写像$\Gamma:RWA\to
		\myop{end}(RWA)\otimes \myop{end}(RWA)$で次のように定義される。
		\begin{equation*}\begin{split} %{
			\Gamma w &= (w*)^\dag\otimes \myid
			+ 1_W(\Delta_{*(1)}w)^\dag\otimes \bigl((\Delta_{*(2)}w)*\bigr)^\dag
			- 1_Ww^\dag\otimes \myid \\
			&\quad\text{for all }w\in WA
		\end{split}\end{equation*} %}
		特に、文字$A$に対しては次のようになる。
		\begin{equation*} %{
			\begin{split}
				(a*)^\dag m_* &= m_*(\Gamma a) \\
				\Gamma a &= (a*)^\dag\otimes \myid + 1_W1_W^\dag\otimes (a*)^\dag
			\end{split} \quad\text{for all }a\in A
		\end{equation*} %}
		コンパイラ理論では、$(RA*)^\dag$のことをBrzozowsky微分(derivative)
		と言う。少なくとも係数$R$がand-orブーリアンの時はBrzozowsky微分と言う。
		係数が一般の半環の時もBrzozowsky微分というかは不明であるが、上記の
		結果は引き算を含まないので、内積が定義できれば半環に対しても成り立つ。

		以上を定義のかたちでまとめておく。

		\begin{definition}[Brzozowsky微分]\label{def:Brzozowsky微分} %{
			$R$を半環、$A$を集合、$WA=(WA,m_*,1_W)$を$A$から生成された
			自由モノイド、$RWA$を$R$係数の自由モノイド環とする。
			次のように定義された線形写像$(\myhere*)^\dag:RA\to \myop{end}(RWA)$を
			Brzozowsky微分という。
			\begin{equation*}\begin{split} %{
				(a*)^\dag &= m_*(1_W[a]^\dag\otimes \myid)\Delta_*
				\quad\text{for all }a\in A
			\end{split}\end{equation*} %}
			ここで、$\Delta_*$は積$m_*$の転置である。
		\end{definition} %def:Brzozowsky微分}

		\begin{proposition}[Brzozowsky微分の積との交換関係]\label{prop:Brzozowsky微分の積との交換関係} %{
			$R$を半環、$A$を集合、$WA=(WA,m_*,1_W)$を$A$から生成された
			自由モノイド、$RWA$を$R$係数の自由モノイド環とする。
			Brzozowsky微分$(\myhere*)^\dag$は次の式を満たす。
			\begin{equation*} %{
				\begin{split}
					(a*)^\dag m_* &= m_*(\Gamma a) \\
					\Gamma a &= (a*)^\dag\otimes \myid + 1_W1_W^\dag
						\otimes (a*)^\dag
				\end{split} \quad\text{for all }a\in A
			\end{equation*} %}
			同じことは次の式でも表される。
			\begin{equation*}\begin{split} %{
				(a*)^\dag(f_1*f_2) &= \bigl((a*)^\dag f_1\bigr)*f_2 
					+ (1_W^\dag f_1)\bigl((a*)^\dag f_2\bigr) \\
				&\quad\text{for all }a\in A,\;f_1,f_2\in RWA
			\end{split}\end{equation*} %}
		\end{proposition} %prop:Brzozowsky微分の積との交換関係}

		生成元同士、$a*$と$(a*)^\dag$の交換関係を求めておく。
		$\alpha_1=1_W1_W^\dag$、$\alpha_2=\myid$とすると、
		命題\ref{prop:Brzozowsky微分の積との交換関係}から次の式が成り立つ。
		\begin{equation*}\begin{split} %{
			(a*)^\dag m_* 
			= m_*\bigl(\alpha_1\otimes (a*)^\dag+(a*)^\dag\otimes \alpha_2\bigr)
			\quad\text{for all }a\in A
		\end{split}\end{equation*} %}
		生成元同士の交換関係は、任意の$\alpha_1,\alpha_2\in\myop{end}(RWA)$
		に対して次のようになることがわかる。
		\begin{equation*}\begin{split} %{
			(a_1*)^\dag(a_2*)w &= (a_1*)^\dag m_*([a_2]\otimes w) \\
			&= m_*\biggl((\alpha_1[a_2])\otimes \bigl((a_1*)^\dag w\bigr)
			+ 1_W(a_1^\dag a_2)\otimes (\alpha_2w)\biggr) \\
			&= (\alpha_1[a_2])*\bigl((a_1*)^\dag w\bigr)
			+ (a_1^\dag a_2)(\alpha_2w) \\
			&\quad\text{for all }a_1,a_2\in A,\;w\in WA
		\end{split}\end{equation*} %}
		$\alpha_1$と$\alpha_2$の組み合わせで次のようになる。
		\begin{itemize}\setlength{\itemsep}{-1mm} %{
			\item $\alpha_1=1_W1_W^\dag$、$\alpha_2=\myid$の時は、Brzozowsky微分
			になって、次のような交換関係となる。
			\begin{equation*}\begin{split} %{
				(a_1*)^\dag(a_2*) = (a_1^\dag a_2)
				\quad\text{for all }a_1,a_2\in A
			\end{split}\end{equation*} %}
			\item $\alpha_1=\alpha_2=\myid$の時は、正準交換関係になる。
			\begin{equation*}\begin{split} %{
				(a_1*)^\dag(a_2*) = (a_2*)(a_1*)^\dag + (a_1^\dag a_2)
				\quad\text{for all }a_1,a_2\in A
			\end{split}\end{equation*} %}
		\end{itemize} %}

		\begin{todo}[正準交換関係]\label{todo:正準交換関係} %{
			正準交換関係
			$(a_1\square)^\dag(a_2*) = (a_2*)(a_1\square)^\dag+(a_1^\dag a_2)$
			を満たす作用素$(a\square)^\dag$の元になる積$m_\square$はシャッフル積
			になると予想する。
		\end{todo} %todo:正準交換関係}
	%s2:Brzozowsky微分}
%s1:正規言語}

\section{この後}\label{s1:この後} %{
	\begin{itemize}\setlength{\itemsep}{-1mm} %{
		\item 表現
	\end{itemize} %}


	\subsection{微分形式}\label{s2:微分形式} %{
		微分幾何において微分形式とは微分の転置として定義される。
		微分とは多様体から複素数への写像空間のライプニッツ規則を満たす
		線形作用として定義される。
		多様体を$M$、$M$から$\mybf{C}$への写像全体を$M^\dag$、
		その微分全体を$TM$とすると、$TM\subseteq \myop{end}M^\dag$となる。
		ここで、$M^\dag$は$\mybf{R}^{\myop{dim}M}$を基底、$\mybf{C}$を係数
		とする自由ベクトル空間と局所的に同型となるので、$\myop{end}M^\dag$
		は局所的には$\mybf{C}\mybf{R}^{\myop{dim}M}$の線形写像全体である。
		微分形式$\omega\in T_*M$は$w:TM\to\mybf{C}$として定義される。

		\begin{table}[htbp] %{
			\begin{center}\begin{tabular}{cc} \hline
				微分幾何 & 文字列 \\ \hline
				$M$ & $RA$ \\ \hline
				$TM$ & $\myop{end}(RA)$ \\ \hline
				$T_*M$ & $(\myop{end}(RA))^\dag$ \\ \hline
			\end{tabular}\end{center}
			\caption{対応}
		\end{table} %}
	%s2:微分形式}
%s1:この後}


	\subsection{積の摂動計算の例}\label{s2:積の摂動計算の例} %{
		$R$を半環、$m$を積とする。積$m$の定義域はある$R$係数半モジュールとする。
		積$m_t$がパラメータ$t\in R$で次のように与えられたとする。
		\begin{equation*}\begin{split} %{
			m_t &= \beta_0 + t\beta_1 + t^2\beta_2 + \cdots \\
			\beta_0 &= m \\
		\end{split}\end{equation*} %}
		このとき、$m_t$の結合性$m_t(m_t\otimes \myid)=m_t(\myid\otimes m_t)$
		を$t$のべきについて展開すると、任意の$t$に対しての結合性が成り立つための
		必要十分条件(少なくとも十分条件)が次の式が与えられる。
		\begin{equation*}\begin{split} %{
			\beta_1(\beta_0\otimes \myid) + \beta_0(\beta_1\otimes \myid)
			&= \beta_1(\myid\otimes \beta_0) + \beta_0(\myid\otimes \beta_1)
			\\
			\beta_2(\beta_0\otimes \myid) + \beta_1(\beta_1\otimes \myid)
			+ \beta_0(\beta_2\otimes \myid)
			&= \beta_2(\myid\otimes \beta_0) + \beta_1(\myid\otimes \beta_1)
			+ \beta_0(\myid\otimes \beta_2)
			\\
			\cdots \\
			\sum_{0\le i\le n}\beta_{n-i}(\beta_i\otimes \myid)
			&= \sum_{0\le i\le n}\beta_{n-i}(\myid\otimes \beta_i)
			\cdots \\
		\end{split}\end{equation*} %}

		集合$A$から生成された自由モノイド$WA$で、文字列の連結$m_*$からの変形
		\begin{equation}\label{eq:積の変形その一}\begin{split} %{
			m_t([a_1]\otimes[a_2]) &= [a_1a_2] + t[a_2a_1] \\
		\end{split}\end{equation} %}
		を計算してみる。$m_t$の結合性の条件は、$1$以上の自然数$n$に対して
		\begin{equation*}\begin{split} %{
			\beta_L^n = \sum_{0\le i\le n}\beta_{n-i}(\beta_i\otimes \myid)
			,\quad
			\beta_R^n = \sum_{0\le i\le n}\beta_{n-i}(\myid\otimes \beta_i)
		\end{split}\end{equation*} %}
		とおくと任意の$w_1,w_2,w_3\in WA$に対して
		\begin{equation*}\begin{split} %{
			\beta_L^n(w_1\otimes w_2\otimes w_3)
			=\beta_R^n(w_1\otimes w_2\otimes w_3)
		\end{split}\end{equation*} %}
		となる。文字列の並べ替えを計算するだけなので、
		$[a_1a_2\cdots a_m]=(12\cdots m)$と並びの順序を括弧$()$でくくって書き、
		テンソル積の記号は省略する。例えば、$[a_1a_2]\otimes[a_3a_4]$は$(12)(34)$
		と書く。
		
		$(1)(2)(3)$に対する結合性の条件をの$t^n$の項ごとに計算する。
		$0=\beta_2(1)(2)=\beta_3(1)(2)=\cdots$だから、$(1)(2)(3)$に対する計算
		では、任意の$n\in\mybf{N}+$で、次のように$\beta_L^n$と$\beta_R^n$は
		二項だけの和になる。
		\begin{equation*}\begin{split} %{
			\beta_L^n(1)(2)(3) &= \Bigl(\beta_n(\beta_0\otimes \myid)
			+\beta_{n-1}(\beta_1\otimes \myid)\Bigr)(1)(2)(3) \\
			&= \beta_n(12)(3) + \beta_{n-1}(21)(3) \\
			\beta_R^n(1)(2)(3) &= \Bigl(\beta_n(\myid\otimes \beta_0)
			+\beta_{n-1}(\myid\otimes \beta_1)\Bigr)(1)(2)(3) \\
			&= \beta_n(1)(23) + \beta_{n-1}(1)(32) \\
		\end{split}\end{equation*} %}
		となる。つまり、$t^n$の項で$m_t(1)(2)(3)$の結合性が成り立つための条件が
		\begin{equation*}\begin{split} %{
			\beta_n(12)(3) + \beta_{n-1}(21)(3)
			=\beta_n(1)(23) + \beta_{n-1}(1)(32)
		\end{split}\end{equation*} %}
		となる。したがって、ある$n$で$\beta_n(12)(3)=\beta_n(1)(23)=0$とできれば、
		$n$以上のすべての$p$で$\beta_p(12)(3)=\beta_p(1)(23)=0$とできる。
		以上を注意して計算する。

		$(1)(2)(3)$に対する一次の条件は
		$\beta_1(12)(3) + (213)=\beta_1(1)(23) + (132)$となり、
		$\beta_1(12)(3) = (132),\quad \beta_1(1)(23) = (213)$とおける。 
		二次の条件は
		$\beta_2(12)(3)+(231)=\beta_2(1)(23)+(312)$となり、
		$\beta_2(12)(3)=(312),\quad\beta_2(1)(23)=(231)$とおける。
		三次の条件は$\beta_3(12)(3)+(321)= \beta_3(1)(23)+(321)$となり、
		$\beta_3(12)(3)=\beta_3(1)(23)=0$とおける。したがって、
		\begin{equation*}\begin{split} %{
			m_t(12)(3) &= (123) + t(132) + t^2(312) \\
			m_t(1)(23) &= (123) + t(213) + t^2(231) \\
		\end{split}\end{equation*} %}
		となることがわかる。

		$(12)(3)(4)$に対する$m_t$の結合性のための条件は
		\begin{equation*}\begin{split} %{
			\beta_1(123)(4)+(1324)=\beta_1(12)(34)+(1243)
		\end{split}\end{equation*} %}
		および、$n\in\mybf{N}$に対して
		\begin{equation*}\begin{split} %{
			&\beta_{n+2}(123)(4) +\beta_{n+1}(132)(4) +\beta_{n}(312)(4) \\
			&= \beta_{n+2}(12)(34)+\beta_{n+1}(12)(43)
		\end{split}\end{equation*} %}
		となり、$(12)(3)$に対する場合と同様に、$\beta_n$の次数$n$について順に
		計算していくと、
		\begin{equation*}\begin{split} %{
			m_t(123)(4) &= (1234)+ t(1243)+ t^2(1423)+ t^3(4123) \\
			m_t(12)(34) &= (1234)+ t(1324)+ t^2\Bigl((1342)+ t^2(3124)\Bigr)
			+ t^3(3142)+ t^4(3412)
		\end{split}\end{equation*} %}
		となることがわかる。

		$\beta_n$の次数$n$について低次の項を二つ計算してみたが、$m_t$の規則の
		予想がつく。$m_t([a_1a_2\cdots a_m]\otimes[b_1b_2\cdots b_n])$に対して、
		一項目の文字の集合を$(a)=\set{a_1,a_2,\dots,a_n}$、
		二項目の文字の集合を$(b)=\set{b_1,b_2}$とすると、
		\begin{itemize} %{
			\item 文字列$[a_1a_2\cdots a_mb_1b_2\cdots b_n]$から始めて、
			\item 一度に一組だけ隣り合った$(a)$と$(b)$の文字の順序を入れ替えて
			新たな文字列を作り出す。
			ただし、$(a)$を右へ$(b)$を左へ動かす入れ替えしか許さないものと
			する。例えば、$[a_1a_2b_1a_3b_2]$という文字列から$[a_1b_1a_2a_3b_2]$
			または$[a_1a_2b_1b_2a_3]$へ文字の順序を入れ替える。
			\item 文字を入れ替えた時、因子$t$を文字列に掛ける。
			\item 文字を入れ替えたとき、既に同一の文字があったらそれ以上の
			文字の入れ替えを中止する。
			\item 文字列$[b_1b_2\cdots b_na_1a_2\cdots a_m]$に到達したとき
			終了する。
		\end{itemize} %}
		例えば、$m_t([a_1a_2]\otimes[b_1b_2])$の計算では、次のような規則で
		項を列挙していく。
		\begin{equation*}\xymatrix{
			& [a_1a_2b_1b_2] \ar[d] \\
			& [a_1b_1a_2b_2] \ar[dl] \ar[dr] \\
			[b_1a_1a_2b_2] \ar[dr] && [a_1b_1b_2a_2] \ar[dl] \\
			& [b_1a_1b_2a_2] \ar[d] \\
			& [b_1b_2a_1a_2] \\
		}\end{equation*}
		\begin{equation*}\begin{split} %{
		\end{split}\end{equation*} %}
		始点と終点を唯一つだけもつDAG(分岐と合流だけでループを持たないグラフ)
		で表現される。始点からの経路の長さが$t$のべきになる。
		また、次のような木の操作で表現してもよいだろう。
		\begin{equation*}\begin{split} %{
			\mytree{
			& \bullet \ar@{-}[dl] \ar@{-}[dr] \\
			a_1 && a_2 \\
			} \lhd [b_1b_2] &= \mytree{
			&& \bullet \ar@{-}[dll] \ar@{-}[dl]\ar@{-}[dr] \ar@{-}[drr]\\
			a_1 & a_2 && b_1 & b_2 \\
			}+ \mytree{
			& \bullet \ar@{-}[dl] \ar@{-}[dr] \\
			a_1 && a_2 \ar@{-}[dl] \ar@{-}[dr] \\
			& b_1 && b_2 \\
			}+ \mytree{
			& \bullet \ar@{-}[dl] \ar@{-}[d] \ar@{-}[dr]\\
			a_1 & a_2 \ar@{-}[d]  & b_2\\
			& b_1 \\
			} \\
			&\;+ \mytree{
			&& \bullet \ar@{-}[dl] \ar@{-}[dr] \\
			& a_1 \ar@{-}[dl] \ar@{-}[dr] && a_2 \\
			b_1 && b_2 \\
			} + \mytree{
			& \bullet \ar@{-}[dl] \ar@{-}[dr] \\
			a_1 \ar@{-}[d] && a_2 \ar@{-}[d] \\
			b_1 && b_2 \\
			} + \mytree{
			& \bullet \ar@{-}[dl] \ar@{-}[d] \ar@{-}[dr] \\
			a_1 \ar@{-}[d] & a_2 & b_2\\
			b_1 \\
			} \\
			&= \sum_{1\le i_1\le i_2\le 3}\left(\mytree{
			& \bullet:3 \ar@{-}[dl] \ar@{-}[dr] \\
			a_1:1 && a_2:2 \\
			}\lhd_{i_1}b_1\right)\lhd_{i_2}b_2
		\end{split}\end{equation*} %}
		ここで、$x\lhd_i b$は葉$b$を木$x$の$i$番目の頂点の最右の子供として
		付け加えるという操作である。頂点の番号は帰りがけ順につけられる。
		こうして作られた木を帰りがけ順に並べて単語にして、根$\bullet$を取り除く
		と、積$m_t([a_1a_2]\otimes[b_1b_2])$の項の和が列挙される。この方法では
		$t$のべきが明確でないが、DAGを用いた計算で必要となる重複のチェックを
		必要としないことが利点となる。

		\begin{todo}[接木によるシャッフル積の導出]\label{todo:接木によるシャッフル積の導出} %{
			$\beta_\lhd:RT_+A\otimes RWA\to RT_+A$を、任意の木$t\in T_+A$に対して
			\begin{equation*}\begin{split} %{
				t\lhd 1_W = t
			\end{split}\end{equation*} %}
			、任意の木$t\in T_+A,\;a_1,a_2,\dots,a_m\in A$に対して
			\begin{equation*}\begin{split} %{
				t\lhd [a_1a_2\cdots a_m] = 
				\sum_{i_1\le i_2\le \cdots\le i_m\in \myop{post}t}
				\Bigl(\cdots\bigl((t\lhd_{i_1}a_1)\lhd_{i_2}a_2\bigr)\cdots\Bigr)\lhd_{i_m}a_m
			\end{split}\end{equation*} %}
			と定義する。すると、
			\begin{equation*}\begin{split} %{
				t\lhd[a_1]\lhd[a_2]
				&= (\sum_{i_1<i_2\in \myop{post}t}+\sum_{i_2<i_1\in \myop{post}t})
				t\lhd_{i_1}a_1\lhd_{i_2}a_2 \\
				&\; + \sum_{i\in \myop{post}t}(t\lhd_{i}a_1)\lhd_{i}a_2
			\end{split}\end{equation*} %}
			となるが、二項目は$a_1$と$a_2$が同じの頂点$i$の部分木になる場合で、
			次の二通りの場合がある。
			\begin{equation*}\begin{split} %{
				\mytree{
					& b:i \ar@{-}[dl]\ar@{-}[d]\ar@{-}[dr] \\
					*+[F]{u} & a_1 & a_2 \\
				},\quad \mytree{
					& b:i \ar@{-}[dl]\ar@{-}[dr] \\
					*+[F]{u} && a_1 \ar@{-}[d] \\
					&& a_2 \\
				}
			\end{split}\end{equation*} %}
			ここで、$u$は操作前に$t$に存在する$i$の部分木で、$b\in A$は頂点$i$
			のラベルとする。二つ目の場合は、帰りがけ順で単語に射影
			$\pi_{\myop{post}}$すると次の木と同一の単語
			$[\pi_{\myop{post}}u]*[a_2a_1b]$を与える。
			\begin{equation*}\begin{split} %{
				\mytree{
					& b:i \ar@{-}[dl]\ar@{-}[d]\ar@{-}[dr] \\
					*+[F]{u} & a_2 & a_1 \\
				}
			\end{split}\end{equation*} %}
			したがって、$
				\pi_{\myop{post}}\Bigl(t\lhd[a_1]\lhd[a_2]\Bigr) 
				= \pi_{\myop{post}}\Bigl(t\lhd\bigl([a_1a_2] + [a_2a_1]\bigr)\Bigr)
			$となって、シャッフル積$[a_1]\sqcup[a_2]=[a_1a_2]+[a_2a_1]$を与える。
			三文字以上の場合は、場合分けが複雑になるので工夫が必要になるが、
			木の操作からシャッフル積を導出できるかもしれない。
			そもそもシャッフル積そのものが複雑である。
		\end{todo} %todo:接木によるシャッフル積の導出}

		\begin{todo}[余積からシャッフル積を導出]\label{todo:余積からシャッフル積を導出} %{
			シャッフル積を計算する手立てを余積で与えることを考える。
			余積$\Delta_\sqcup:RWA\to RWA\otimes RWA$を、
			\begin{equation*}\begin{split} %{
				\Delta_\sqcup1_W = 1_W\otimes 1_W
			\end{split}\end{equation*} %}
			任意の$a_1,a_2,\dots,a_m\in A$に対して
			\begin{equation*}\begin{split} %{
				\Delta_\sqcup [a_1a_2a_3\cdots a_m]
				& = 1_W\otimes [a_1a_2a_3\cdots a_m] \\
				&\; + [a_1]\otimes [a_2a_3\cdots a_m] \\
				&\; + [a_1a_2]\otimes [\cdots a_m] \\
				&\; + \cdots \\
				&\; + [a_1a_2a_3\cdots a_m]\otimes 1_W
			\end{split}\end{equation*} %}
			と定義する。次の可換図を満たす$R$双線形二項写像$\beta$を考える。
			\begin{equation*}\xymatrix@C+2pc{
				RWA^{\otimes 2} 
				\ar[r]^{(\Delta_\sqcup\otimes \Delta_\sqcup)\sigma_{23}}
				\ar[d]^{\beta}
				& RWA^{\otimes 4} \ar[d] 
				\ar[d]^{\beta\otimes \beta}
				\\
				RWA 
				& RWA^{\otimes 2} \ar[l]_{m_*} \\
			}\end{equation*}
		\end{todo} %todo:余積からシャッフル積を導出}

		\begin{todo}[課題]\label{todo:課題} %{
			\begin{itemize} %{
				\item 任意の元に対する積$m_t$を結合性の条件のみから定めることが
				できるか?
				\item できるとするならばなぜ?
				\item できるとするならば、線形代数で任意の元に対する積$m_t$を
				を求められないか?
				\item 双対な余積の変形
				\item 余積$
				[a_1a_2\cdots a_m]\mapsto
				1_W\otimes[a_1a_2a_3\cdots a_m]+[a_1]\otimes[a_2a_3\cdots a_m]
				+[a_1a_2]\otimes[\cdots a_m]+\cdots+[a_1a_2a_3\cdots a_m]\otimes1_W
				$に双対で$m([a_1]\otimes[a_2])=[a_1a_2]+[a_2a_1]$となる積は
				唯一定まるか?
				\item
			\end{itemize} %}
		\end{todo} %todo:課題}
	%s2:積の摂動計算の例}
%s1:文字列}

\section{リスト}\label{s1:リスト} %{
	$R=(R,+,0,m_R,1)$を可換半環、$A$を有限集合、$WA=(WA,m_*,1_*)$を$A$から
	生成された自由モノイドとする。ここで、積$m_*$は文字列の連結で定義される。
	積$m_*$を中置記法で$*$とも書くことにする。$WA$の元を$A$の元を並べたものを
	括弧でくくって表すことにする。例えば、$a_1,a_2,\dots, a_n\in A$を並べた
	$WA$の元を$[a_1a_2\cdots a_n]$と書く。

	$RWA$を$WA$を基底とする$R$係数半モジュールとする。$WA$の積$m_*$を$R$線形
	に拡張して$RWA$の積としたものを同じ記号$m_*$で書き、中置記法で$*$とも書く。
	さらに、中置記法$*$をテンソル積に対して次のように定義する。
	\begin{equation}\begin{split} %{
		&(w_{11}\otimes w_{12}\otimes\cdots\otimes w_{1m})
		*(w_{21}\otimes w_{22}\otimes\cdots\otimes w_{2m}) \\
		&\quad= (w_{11}*w_{21})\otimes (w_{12}*w_{22})\otimes\cdots\otimes (w_{1m}*w_{2m}) \\
		&\quad\text{for all }w_{11},w_{12},\dots,w_{1m},w_{21},w_{22},\dots,w_{2m}\in WA
	\end{split}\end{equation} %}
	積$m_*$に双対で、任意の$a\in A$に対して
	$\Delta_*[a]=[a]\otimes 1_*+1_*\otimes [a]$となる余積$\Delta_*$を求める。
	任意の$a_1,a_2,\dots,a_m\in A$に対して次の式が成り立つ必要がある。
	\begin{equation}\begin{split} %{
		\Delta_*[a_1a_2\cdots a_m] &= (\Delta_*[a_1])*(\Delta_*[a_2])*\cdots*(\Delta_*[a_m]) \\
		&= [a_1a_2\cdots a_m]\otimes 1_* \\
		&\; + \sum_{1\le i\le n}[a_1a_2\cdots a_m]_{\neg{\set{i}}}\otimes [a_i] \\
		&\; + \sum_{1\le i<j\le n}[a_1a_2\cdots a_m]_{\neg{\set{i,j}}}\otimes [a_ia_j] \\
		&\; + \cdots \\
		&\; + 1_*\otimes [a_1a_2\cdots a_m] \\
	\end{split}\end{equation} %}
	ここで、任意の$1\le i_1<i_2<i_n\le m$に対して
	$[a_1a_2\cdots a_m]_{\neg\set{i_1,i_2,\dots,i_n}}$を$[a_1a_2\cdots a_m]$
	から$i_1$番目と$i_2$番目と...と$i_n$番目の文字を取り除いた文字列とした。
	例えば、$[abc]_{\neg\set{1}}=[bc]$、$[abc]_{\neg\set{2}}=[ac]$、
	$[abc]_{\neg\set{1,3}}=[b]$となる。更に、余単位射を
	$\epsilon_*:w\mapsto \jump{w=1_*}$で定めると、単位元$1_*$に対する余積が
	$\Delta_*1_*=1_*\otimes 1_*+\cdots$という形になる必要がある。
	一方、双対性$\Delta_*[a]=(\Delta_*1_*)*(\Delta_*[a])$を満たすためには、
	$\Delta_*1_*=1_*\otimes 1_*$となる必要があることがわかる。まとめると、
	次のようになる。

	\begin{definition}[文字列の連結に双対な余積]\label{def:文字列の連結に双対な余積} %{
		次の余積$\Delta_*$は積$m_*$に双対になる。
		\begin{equation}\begin{split} %{
			\Delta_*: RWA\otimes RWA &\to RWA \\
			1_* &\mapsto 1_*\otimes 1_* \\
			[a_1a_2\cdots a_m] &\mapsto (\Delta_*[a_1])*(\Delta_*[a_2])*\cdots*(\Delta_*[a_m]) \\
			&= [a_1a_2\cdots a_m]\otimes 1_* \\
			&\; + \sum_{1\le i\le n}[a_1a_2\cdots a_m]_{\neg{\set{i}}}\otimes [a_i] \\
			&\; + \sum_{1\le i<j\le n}[a_1a_2\cdots a_m]_{\neg{\set{i,j}}}\otimes [a_ia_j] \\
			&\; + \cdots \\
			&\; + 1_*\otimes [a_1a_2\cdots a_m] \\
		\end{split}\end{equation} %}
		次の線形写像$\epsilon_*$は余積$\Delta_*$の余単位射となる。
		\begin{equation}\begin{split} %{
			\epsilon_*: RWA &\to R \\
				w &\mapsto \jump{w=1_*} \\
		\end{split}\end{equation} %}
	\end{definition} %def:文字列の連結に双対な余積}

	\begin{proposition}[$\Delta_*$は余可換]\label{prop:Delta_*は余可換} %{
		$\Delta_*$は余可換である。
	\end{proposition} %prop:Delta_*は余可換}
	\begin{proof} %{
		文字数についての帰納法で証明する。
		$\Delta_*11_*=1_*\otimes 1_*$だから、文字数が$0$の場合は余可換となる
		ことがわかる。
		任意の$a\in A$に対して$\Delta_*1[a]=[a]\otimes 1_*+[a]\otimes 1_*$
		だから、文字数が$1$の場合も余可換となることがわかる。
		文字数が$n\in\mybf{N}\bou 1\le n$以下の任意の単語に対して$\Delta\*$が
		余可換だとする。
		\begin{equation*}\begin{split} %{
			\Delta_*^{(1)}w\otimes\Delta_*^{(2)}w
			=\Delta_*^{(2)}w\otimes\Delta_*^{(1)}w
			\quad\text{for all }w\in WA\bou \zettai{w}\le n
		\end{split}\end{equation*} %}
		$w_1,w_2$を文字数が$n$以下の単語とする。次の式から単語$w_1*w_2$は
		余可換になることがわかる。
		\begin{equation*}\begin{split} %{
			\Delta_*(w_1*w_2) &= (\Delta_*w_1)*(\Delta_*w_2) \\
			&= \left((\Delta_*^{(1)}w_1)*(\Delta_*^{(1)}w_2)\right)
			\otimes \left((\Delta_*^{(2)}w_1)*(\Delta_*^{(2)}w_2)\right) \\
			&= \left((\Delta_*^{(2)}w_1)*(\Delta_*^{(2)}w_2)\right)
			\otimes \left((\Delta_*^{(1)}w_1)*(\Delta_*^{(1)}w_2)\right) \\
			& = \left(\Delta_*^{(2)}(w_1*w_2)\right)
			\otimes \left(\Delta_*^{(1)}(w_1*w_2)\right) \\
		\end{split}\end{equation*} %}
		任意の$n+1$文字の単語$w$は、ある$1$文字の単語$x$とある$n$文字の単語$y$
		の積$w=x*y$で書くことができるので、任意の$n+1$文字の単語に対する
		余積$\Delta_*$は余可換となることがわかる。
	\end{proof} %}

	この証明で用いた事柄は次のものである。
	\begin{itemize} %{
		\item 任意の$m+n$文字の単語は、ある$m$文字の単語とある$n$文字の単語の積
		で書くことができる。
		\item 積$m_*$と余積$\Delta_*$が双対である。
		\item $0$文字の単語に対する余積$\Delta_*$が余可換である。
		\item $1$文字の単語に対する余積$\Delta_*$が余可換である。
	\end{itemize} %}
	したがって、
	\begin{itemize} %{
		\item 積$m_*$と双対になり、
		\item $0$文字と$1$文字の単語に対する余積$\Delta_*$が余可換
	\end{itemize} %}
	となる任意の余積は余可換となる。

	ここで、余可換な余積について調べる。$\Delta$を一般の余積とすると、
	余結合性より、$\Delta^2$を次のように定義することができる。
	\begin{equation*}\begin{split} %{
		\Delta^2w=(\Delta\otimes\myid)\Delta w=(\myid\otimes\Delta)\Delta w
	\end{split}\end{equation*} %}
	同様にして、$\Delta^2$を次のように定義することができる。
	\begin{equation*}\begin{split} %{
		\Delta^3w
		&=(\Delta\otimes\myid\otimes\myid)(\Delta\otimes\myid)\Delta w \\
		&=(\myid\otimes\Delta\otimes\myid)(\Delta\otimes\myid)\Delta w \\
		&=(\myid\otimes\myid\otimes\Delta)(\Delta\otimes\myid)\Delta w \\
		&=(\Delta\otimes\myid\otimes\myid)(\myid\otimes\Delta)\Delta w \\
		&=(\myid\otimes\Delta\otimes\myid)(\myid\otimes\Delta)\Delta w \\
		&=(\myid\otimes\myid\otimes\Delta)(\myid\otimes\Delta)\Delta w \\
	\end{split}\end{equation*} %}
	任意の$n\in\mybf{N}\bou 1\le n$に対して、$\Delta^n$を次のように定義する。
	\begin{equation*}\begin{split} %{
		\Delta^nw = \Delta^{(1)}w\otimes \Delta^{(1)}\Delta^{(2)}w\otimes 
		\cdots\otimes \Delta^{(1)}\Delta^{(2)(n-1)}w\otimes \Delta^{(2)n}w
	\end{split}\end{equation*} %}
	この式を次の二分木で表すことにする。
	\begin{equation}\label{eq:余積の二分木}
		\Delta^nw = \xymatrix@R=1pc@C=1pc{
			w \ar[r]\ar[d] & \Delta^{(2)}w \ar[r]\ar[d] 
			& \cdots \ar[r] & \Delta^{(2)(n-1)} \ar[r]\ar[d] & \Delta^{(2)n}w \\
			\Delta^{(1)}w & \Delta^{(1)}\Delta^{(2)}w 
			& \cdots & \Delta^{(1)}\Delta^{(2)(n-1)}w \\
		}
	\end{equation}
	この木のどの葉に対して余積$\Delta$をとっても、余結合性により次の図のよう
	になり、最終的に図\eqref{eq:余積の二分木}の二分木の形になる。
	\begin{equation*}\begin{split} %{
		&\xymatrix@R=1pc@C=1pc{
			\cdots \ar[r] & \Delta^{(2)k}w \ar[r]\ar[d] & \Delta^{(2)(k+1)} \ar[r]\ar[d] & \cdots \\
			& \Delta^{(1)}\Delta^{(2)k}w \ar[ld]\ar[rd] & \Delta^{(1)}\Delta^{(2)(k+1)}w \\
			\Delta^{(1)}\Delta^{(1)}\Delta^{(2)(k+1)}w && \Delta^{(2)}\Delta^{(1)}\Delta^{(2)(k+1)}w \\
		} \\
		\\
		&= \xymatrix@R=1pc@C=1pc{
			\cdots \ar[r] & \Delta^{(2)k}w \ar[r]\ar[d] & \Delta^{(2)(k+1)} \ar[r]\ar[d] & \cdots \\
			& \Delta^{(1)}\Delta^{(2)k}w & \Delta^{(1)}\Delta^{(2)(k+1)}w \ar[ld]\ar[rd] \\
			& \Delta^{(1)}\Delta^{(1)}\Delta^{(2)k}w && \Delta^{(2)}\Delta^{(1)}\Delta^{(2)k}w \\
		} \\
	\end{split}\end{equation*} %}
	$WA$の基底を$\set{e_0,e_1,\cdots}$とおき、$R$値行列$\Delta_i^{jk}$を
	用いて$\Delta e_i=\Delta_i^{jk}e_j\otimes e_k$とする。
	余積$\Delta$の余結合性から、
	$\Delta_i^{ja}\Delta_a^{kl}=\Delta_a^{jk}\Delta_i^{al}$という式を満たす。
	図\eqref{eq:余積の二分木}による$\Delta^n$の表し方は、
	$(\Delta^3)_i^{jklm}=\Delta_i^{ja}\Delta_a^{kb}\Delta_b^{lm}$
	という縮約のとり方を指定していることに対応する。
	
	余積$\Delta$が余可換であった場合、$\Delta_i^{jk}=\Delta_i^{kj}$となる。
	$\Delta^2$についてみてみる。
	$(\Delta^2)_i^{jkl}=\Delta_i^{ja}\Delta_a^{kl}$となるが、
	$(\Delta^2)_i^{jkl}$の添え字$(kl)$について対称なことは、
	余積$\Delta$が余可換であることからわかる。また、余結合性
	$\Delta_i^{ja}\Delta_a^{kl}=\Delta_a^{jk}\Delta_i^{al}$を使うと、
	$(\Delta^2)_i^{jkl}$の添え字$(jk)$について対称なこともわかる。
	したがって、$(\Delta^2)_i^{jkl}$の添え字は$(jkl)$の任意の置換によって
	不変になっていることがわかる。テンソル積$\otimes$の記法を用いて
	$\Delta^nw=w_{(1)}\otimes w_{(2)}\otimes\cdots\otimes w_{(n+1)}$
	と書くと、$n+1$次の任意の置換$\sigma$に対して
	$\Delta^nw=w_{(\sigma1)}\otimes w_{(\sigma2)}\otimes\cdots\otimes w_{\left(\sigma(n+1)\right)}$
	となる。

	\begin{todo}[シャッフル積]\label{todo:シャッフル積} %{
		次の可換図で定義された$R$双線形二項演算$\beta_\sqcup$は積になるか?
		\begin{equation}\xymatrix{
			RWA\otimes RWA \ar[r]^{m_*} \ar@{.>}[d]^{\beta_\sqcup} 
			& RWA \ar[d]^{\Delta_*} \\
			RWA & RWA\otimes RWA \ar[l]_{m_*} \\
		}\end{equation}
	\end{todo} %todo:シャッフル積}

	\begin{todo}[ここまで]\label{todo:ここまで} %{
	\end{todo} %todo:ここまで}

	次の$R$線形写像$\Delta_\amalg$は余積になる。
	\begin{equation}\begin{split} %{
		\Delta_\amalg: RWA &\to RWA\otimes RWA \\
			[a_1a_2\cdots a_{m-1}a_m] 
				&\mapsto [a_1a_2\cdots a_{m-1}a_m]\otimes 1_* \\
				&\quad + [a_1a_2\cdots a_{m-1}]\otimes [a_m] \\
				&\quad + \cdots \\
				&\quad + [a_1]\otimes [a_2\cdots a_{m-1}a_m] \\
				&\quad + 1_*\otimes [a_1a_2\cdots a_{m-1}a_m] \\
	\end{split}\end{equation} %}
	余積$\Delta_\amalg$に対する余単位射は$\epsilon_*$となる。

	\begin{todo}[余積から積の導出]\label{todo:余積から積の導出} %{
		与えられた余積と双対になる積を導出する方法を考える。
		逆の場合の、与えられた積に双対になる余積の導出は、文字数の小さいもの
		から大きなものを順の求めていけばよい。
		一般に、積$m_\odot$に双対な余積$\Delta_\odot$は次のようになる。
		\begin{equation}\begin{split} %{
			\Delta_\odot(w_1\odot w_2) &= (\Delta_\odot w_1)\odot(\Delta_\odot w_2) \\
		\end{split}\end{equation} %}
		したがって、積$m_\odot$が文字数を保存する場合には、文字数の小さいもの
		から大きなものへと余積$\Delta_\odot$が順に求まる。
	\end{todo} %todo:余積から積の導出}
%s1:リスト}
