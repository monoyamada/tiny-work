\begingroup %{
{\setlength\arraycolsep{2pt}
%
\section{二分木}\label{s1:二分木} %{
	\begin{description}\setlength{\itemsep}{-1mm} %{
		\item[列挙$\lambda$] $A$を集合、$\clP A$を$A$の冪集合とする。
		元の列挙を表す写像$\lambda_R:\clP A\to\sizen A$を次のように定義する。
		\begin{equation*}\begin{split}
			\lambda B = \sum_{b\in B}b \quad\text{for all } B\subseteq A
		\end{split}\end{equation*}
	\end{description} %}

	$\clT_*$を二分木のつくる集合、$\clT_n\subset\clT_*$を葉でない頂点が$n$個
	ある二分木の集合とする。$\clT_*=\cup_{n\in\sizen}\clT_n$
	木の頂点を$\bullet$で表すと、$\clT_n$は次のようになる。
	\begin{equation*}\begin{split}
		\lambda\clT_0 &= \bullet \\
		\lambda\clT_1 &= \xymatrix@R=4pt@C=4pt{
			& \bullet \hen[dl] \hen[dr] \\
			\bullet & & \bullet
		} \\
		\lambda\clT_2 &= \xymatrix@R=4pt@C=4pt{
			& & \bullet \hen[dl] \hen[dr] \\
			& \bullet \hen[dl] \hen[dr] & & \bullet \\
			\bullet & & \bullet
		} + \xymatrix@R=4pt@C=4pt{
			& \bullet \hen[dl] \hen[dr] \\
			\bullet & & \bullet \hen[dl] \hen[dr] \\
			& \bullet & & \bullet
		}
	\end{split}\end{equation*}
	二分木の大きさ$|-|:\clT_*\to\sizen$を次のように定義する。
	\begin{equation*}\begin{split}
		|t| = n \xiff{\dfn} t\in \clT_n
	\end{split}\end{equation*}
	$\clT_*$の二項演算$\beta$を次のように定義する。
	\begin{equation*}\begin{split}
		\beta(t_1\times t_2) := \xymatrix@R=4pt@C=4pt{
			& \bullet \hen[dl] \hen[dr] \\
			t_1 & & t_2
		} \quad\text{for all } t_1,t_2\in\clT_*
	\end{split}\end{equation*}
	$\beta$を線形射に拡張すると、次の式が成り立つ。
	\begin{equation}\label{eq:ベータによる二分木の列挙}\begin{split}
		\lambda\clT_{n+1} = \sum_{k=0}^n\beta\plr{\lambda\clT_k\otimes\lambda\clT_{n-k}}
		\quad\text{for all } n\in\sizen
	\end{split}\end{equation}

	二分木の左右反転を写像$\rho:\clT_*\to\clT_*$で表す。$\rho$は次の式を満たす。
	\begin{equation*}\begin{split}
		\rho\bullet = \bullet,\quad \rho\beta = \beta\sigma_{1,2}\plr{\rho\times\rho}
	\end{split}\end{equation*}
	ここで、$\sigma_{i,j}$は直積の$i$番目と$j$番目の要素を入れ替える操作と
	する。

	$\clT_*$の二項演算$m_0$を次のように定義する。
	\begin{equation*}\begin{split}
		m_0\plr{\bullet\times t} &:= t \quad\text{for all } t\in\clT_* \\
		m_0\plr{\beta\times\id} &:= \beta\plr{\id\times m_0}
	\end{split}\end{equation*}
	$m_0$は積になり、単位元$\bullet$を持つ。

	線形射$\epsilon:\sizen\clT_*\to\sizen$を次のように定義する。
	\begin{equation*}\begin{split}
		\epsilon t = \is{|t|=0} \quad\text{for all } t\in\clT_*
	\end{split}\end{equation*}
	線形射$\alpha:\sizen\clT_*\to\clT_*$を次のように定義する。
	\begin{equation*}\begin{split}
		\alpha\bullet = \beta(\bullet\otimes\bullet),\quad 
		\alpha\beta = \beta\plr{\alpha\otimes\id + \epsilon\otimes\alpha}
	\end{split}\end{equation*}
	$\alpha$を二分木の左成長ということにする。次の式を証明しよう。
	\begin{equation}\label{eq:左成長は列挙になる}\begin{split}
		\lambda\clT_{n+1}=\alpha\lambda\clT_n \quad\text{for all } n\in\sizen
	\end{split}\end{equation}

	左端の葉(行きがけ順に見て最初の葉)と頂点を結ぶ辺の数に変更する。
	\begin{equation*}\begin{split}
		\kappa_0\bullet = 0,\quad 
		\kappa_0\beta\plr{t_1\times t_2} = \plr{\kappa_0t_1} + 1
		\quad\text{for all } t_1,t_2\in\clT_*
	\end{split}\end{equation*}
	$\kappa_0$は根と左端の葉を結ぶ辺の数を表す。
	\begin{equation*}\begin{split}
		\kappa\xymatrix@R=4pt@C=4pt{
			& \bullet \hen[dl] \hen[dr] \\
			\bullet & & \bullet
		} = \kappa\xymatrix@R=4pt@C=4pt{
			& \bullet \hen[dl] \hen[dr] \\
			\bullet & & \bullet \hen[dl] \hen[dr] \\
			& \bullet & & \bullet
		} = 1,\quad \kappa\xymatrix@R=4pt@C=4pt{
			& & \bullet \hen[dl] \hen[dr] \\
			& \bullet \hen[dl] \hen[dr] & & \bullet \\
			\bullet & & \bullet
		} = 2
	\end{split}\end{equation*}
	$\clT^n_k\subseteq\clT^n$を次のように定義すると、
	\begin{equation*}\begin{split}
		\clT^n_k := \set{t\in\clT^n\mid \kappa_0t=k}
	\end{split}\end{equation*}
	次の性質が成り立つ。
	\begin{itemize}\setlength{\itemsep}{-1mm} %{
		\item $\clT_n=\oplus_{k=0}^n\clT^n_k$と直和分解される。
		%
		\item 任意の$n\in\sizen$に対して$\clT^{n+1}_0=\emptyset$となる。
		%
		\item 写像$\beta_-:\clT_*\to\op{Set}(\clT_*,\clT_*)$を次のように定義すると、
		\begin{equation*}\begin{split}
			\beta_t u = \beta(t\times u) \quad\text{for all } t,u\in\clT_*
		\end{split}\end{equation*}
		任意の$n\in\sizen$に対して$\clT^n_n=\set{\rho\beta_\bullet^n\bullet}$となる。
		%
		\item 式\eqref{eq:ベータによる二分木の列挙}から、次の漸化式が成り立つ。
		\begin{equation}\label{eq:ベータによる二分木の列挙その二}\begin{split}
			\lambda\clT^{n+1}_{k+1} 
			= \sum_{r=k}^n\beta\plr{\lambda\clT^r_k\otimes\lambda\clT_{n-r}}
			\quad\text{for all } n\in\sizen,; k\in0..n
		\end{split}\end{equation}
	\end{itemize} %}
	$\clT^n_k$は次のようになる。
	\begin{equation*}\begin{split}
		\lambda\clT^0_0 = \bullet,\quad
		\lambda\clT^1_1 = \xymatrix@R=4pt@C=4pt{
			& \bullet \hen[dl] \hen[dr] \\
			\bullet & & \bullet
		},\quad 
		\lambda\clT^2_1 = \xymatrix@R=4pt@C=4pt{
			& \bullet \hen[dl] \hen[dr] \\
			\bullet & & \bullet \hen[dl] \hen[dr] \\
			& \bullet & & \bullet
		},\quad
		\lambda\clT^2_2 = \xymatrix@R=4pt@C=4pt{
			& & \bullet \hen[dl] \hen[dr] \\
			& \bullet \hen[dl] \hen[dr] & & \bullet \\
			\bullet & & \bullet
		} \\
		\lambda\clT^3_1 = \xymatrix@R=4pt@C=4pt{
			& \bullet \hen[dl] \hen[dr] \\
			\bullet & & \bullet \hen[dl] \hen[dr] \\
			& \bullet \hen[dl] \hen[dr] & & \bullet \\
			\bullet & & \bullet
		}	+ \xymatrix@R=4pt@C=4pt{
			& \bullet \hen[dl] \hen[dr] \\
			\bullet & & \bullet \hen[dl] \hen[dr] \\
			& \bullet & & \bullet \hen[dl] \hen[dr] \\
			& & \bullet & & \bullet \\
		},\quad \lambda\clT^3_2 = \xymatrix@R=4pt@C=4pt{
			& & \bullet \hen[dl] \hen[dr] \\
			& \bullet \hen[dl] \hen[d] & & \bullet \hen[d] \hen[dr] \\
			\bullet & \bullet & & \bullet & \bullet \\
		} + \xymatrix@R=4pt@C=4pt{
			& & \bullet \hen[dl] \hen[dr] \\
			& \bullet \hen[dl] \hen[dr] & & \bullet \\
			\bullet & & \bullet \hen[dl] \hen[dr] \\
			& \bullet & & \bullet \\
		} \\
		\lambda\clT^3_3 = \xymatrix@R=4pt@C=4pt{
			& & & \bullet \hen[dl] \hen[dr] \\
			& & \bullet \hen[dl] \hen[dr] & & \bullet \\
			& \bullet \hen[dl] \hen[dr] & & \bullet \\
			\bullet & & \bullet \\
		}  
	\end{split}\end{equation*}

	\begin{todo}[左右反転]\label{todo:左右反転} %{
	\end{todo} %todo:左右反転}
	
	右端の葉(行きがけ順に見て最後の葉)に着目して$\clT_n$を分解する。
	写像$\kappa:\clT_*\to\sizen$を次のように定義する。
	\begin{equation*}\begin{split}
		\kappa\bullet = 0,\quad 
		\kappa\beta\plr{t_1\times t_2} = \plr{\kappa t_2} + 1
		\quad\text{for all } t_1,t_2\in\clT_*
	\end{split}\end{equation*}
	$\kappa$は根と右端の葉を結ぶ辺の数を表す。
	\begin{equation*}\begin{split}
		\kappa\xymatrix@R=4pt@C=4pt{
			& \bullet \hen[dl] \hen[dr] \\
			\bullet & & \bullet
		} = \kappa\xymatrix@R=4pt@C=4pt{
			& & \bullet \hen[dl] \hen[dr] \\
			& \bullet \hen[dl] \hen[dr] & & \bullet \\
			\bullet & & \bullet
		} = 1,\quad \kappa\xymatrix@R=4pt@C=4pt{
			& \bullet \hen[dl] \hen[dr] \\
			\bullet & & \bullet \hen[dl] \hen[dr] \\
			& \bullet & & \bullet
		} = 2
	\end{split}\end{equation*}
	$\clT^n_k\subseteq\clT^n$を次のように定義すると、
	\begin{equation*}\begin{split}
		\clT^n_k := \set{t\in\clT^n\mid \kappa t=k}
	\end{split}\end{equation*}
	$\clT_n=\oplus_{k=0}^n\clT^n_k$と直和分解される。
	任意の$n\in\sizen$に対して$\clT^{n+1}_0=\emptyset$かつ
	$\clT^n_n=\set{\beta_\bullet^n\bullet}$となる。
	\begin{equation*}\begin{split}
		\lambda\clT^0_0 = \bullet,\quad \lambda\clT^1_1 = \xymatrix@R=4pt@C=4pt{
			& \bullet \hen[dl] \hen[dr] \\
			\bullet & & \bullet
		},\quad \lambda\clT^2_1 = \xymatrix@R=4pt@C=4pt{
			& & \bullet \hen[dl] \hen[dr] \\
			& \bullet \hen[dl] \hen[dr] & & \bullet \\
			\bullet & & \bullet
		},\quad \lambda\clT^2_2 = \xymatrix@R=4pt@C=4pt{
			& \bullet \hen[dl] \hen[dr] \\
			\bullet & & \bullet \hen[dl] \hen[dr] \\
			& \bullet & & \bullet
		} \\
		\lambda\clT^3_1 = \xymatrix@R=4pt@C=4pt{
			& & & \bullet \hen[dl] \hen[dr] \\
			& & \bullet \hen[dl] \hen[dr] & & \bullet \\
			& \bullet \hen[dl] \hen[dr] & & \bullet \\
			\bullet & & \bullet \\
		} + \xymatrix@R=4pt@C=4pt{
			& & \bullet \hen[dl] \hen[dr] \\
			& \bullet \hen[dl] \hen[dr] & & \bullet \\
			\bullet & & \bullet \hen[dl] \hen[dr] \\
			& \bullet & & \bullet \\
		},\quad \lambda\clT^3_2 = \xymatrix@R=4pt@C=4pt{
			& & \bullet \hen[dl] \hen[dr] \\
			& \bullet \hen[dl] \hen[d] & & \bullet \hen[d] \hen[dr] \\
			\bullet & \bullet & & \bullet & \bullet \\
		} + \xymatrix@R=4pt@C=4pt{
			& \bullet \hen[dl] \hen[dr] \\
			\bullet & & \bullet \hen[dl] \hen[dr] \\
			& \bullet \hen[dl] \hen[dr] & & \bullet \\
			\bullet & & \bullet
		} \\
		\lambda\clT^3_3 = \xymatrix@R=4pt@C=4pt{
			& \bullet \hen[dl] \hen[dr] \\
			\bullet & & \bullet \hen[dl] \hen[dr] \\
			& \bullet & & \bullet \hen[dl] \hen[dr] \\
			& & \bullet & & \bullet \\
		} \\
	\end{split}\end{equation*}
	式\eqref{eq:ベータによる二分木の列挙}から、次の漸化式が成り立つ。
	\begin{equation}\label{eq:ベータによる二分木の列挙その二}\begin{split}
		\lambda\clT^{n+1}_{k+1} 
		= \sum_{r=0}^{n-k}\beta\plr{\lambda\clT_r\otimes\lambda\clT^{n-r}_k}
		\quad\text{for all } n\in\sizen,; k\in0..n
	\end{split}\end{equation}
	二分木の左成長は$\clT^n_k$の$k$をほぼ保存する。
	\begin{equation*}\begin{split}
		\xymatrix@R=2ex@C=2ex{
			\ar[dddd]_\alpha & \clT^0_0 \ar[rd] \\
			& & \clT^1_1 \ar[d] \ar[rd] \\
			& & \clT^2_1 \ar[d] & \clT^2_2 \ar[d] \ar[rd] \\
			& & \clT^3_1 \ar[d] & \clT^3_2 \ar[d] & \clT^3_3 \ar[d] \ar[rd] \\
			& & \vdots & \vdots & \vdots & \cdots \\
		}
	\end{split}\end{equation*}
	したがって、証明したい式\eqref{eq:左成長は列挙になる}は次の式に帰着する。
	\begin{equation}\label{eq:アルファによる二分木の列挙}\begin{split}
		\alpha\lambda\clT^n_k = \lambda\clT^{n+1}_k + \is{k=n}\beta_\bullet^{n+1}\bullet
		\quad\text{for all } n\in\sizen,\; k\in0..n
	\end{split}\end{equation}
	この式が成り立つことは帰納法により簡単に証明できる。
	\begin{proof} %{
		まず、$\alpha\clT^0_0=\clT^1_1$となる。そして、ある$N\in\sizen$があって、
		$N$以下の自然数$n$でこの式が成り立つとすると、
		式\eqref{eq:ベータによる二分木の列挙その二}から、
		\begin{equation*}\begin{split}
			\alpha\lambda\clT^{N+1}_{k+1}
			&= \sum_{r=0}^{N-k}\alpha\beta\plr{\lambda\clT_r\otimes\lambda\clT^{N-r}_k} \\
			&= \sum_{r=0}^{N-k}\beta\plr{\alpha\lambda\clT_r\otimes\lambda\clT^{N-r}_k}
				+ \beta\plr{\lambda\clT_0\otimes\alpha\lambda\clT^N_k} \\
			&= \sum_{r=0}^{N-k}\beta\plr{\lambda\clT_{r+1}\otimes\lambda\clT^{N-r}_k}
				+ \beta\plr{\lambda\clT_0\otimes\alpha\lambda\clT^{N+1}_k}
				+ \is{k=N}\beta\plr{\lambda\clT_0\otimes\beta^{N+1}\bullet} \\
			&= \lambda\clT^{N+2}_{k+1} + \is{k=N}\beta^{N+2}\bullet
		\end{split}\end{equation*}
		となって、$n$が$N+1$でも成り立つことがわかる。
	\end{proof} %}
%s1:二分木}
%
}\endgroup %}
