\begingroup %{
\newcommand{\bou}{{\,|\,}}
\newcommand{\boug}{{\,\big|\,}}
\newcommand{\bougg}{{\,\bigg|\,}}
\newcommand{\qa}[1]{{\blra{#1}}}
\newcommand{\q}[1]{{\blr{#1}}}
\newcommand{\qg}[1]{{\blrg{#1}}}
\newcommand{\qgg}[1]{{\blrgg{#1}}}
\newcommand{\qggg}[1]{{\blrggg{#1}}}
\newcommand{\qgggg}[1]{{\blrgggg{#1}}}
\newcommand{\opN}{{\op{N}}}
\newcommand{\opL}{{\op{L}}}
\newcommand{\opR}{{\op{R}}}
\newcommand{\grow}{{\op{grow}}}
\newcommand{\code}{{\op{code}}}
\newcommand{\word}[1]{{|\!\lfloor{#1}\rfloor\!|}}
\newcommand{\hf}{{\what{f}}}
{\setlength\arraycolsep{2pt}
%
\section{微分方程式と二分木}\label{s1:微分方程式と二分木} %{
\subsection{通常の微分方程式}\label{s2:通常の微分方程式} %{
	微分のチェイン則が簡単な$q=1$で考える。

	ある与えられた初期値$x_0\in R$と形式級数$\plr{f\bou x}\in R\bblr{x}$
	に対して次の常微分方程式を考える。
	\begin{equation}\label{eq:通常の微分方程式}\begin{split}
		(x\bou t) = x_0 + \int_0^t\plr{f\circ x\bou s}ds
	\end{split}\end{equation}
	ここで、$(f\circ x\bou t)$は形式級数の合成
	$(f\circ x\bou t):=\plrgg{f\boug(x\bou t)}$を表すとする。

	微分$\plr{\partial^nf\bou x}$と二分木の成長が次のように対応し、
	\begin{equation*}\begin{split}
		\plr{f\bou x} \xmapsto{\partial_x} \xymatrix@R=2pt@C=2pt{
			\bullet \hen[d]\hen[r] & \plr{\partial f\bou x} \\
			\plr{f\bou x} \\
		} \xmapsto{\partial_x} \xymatrix@R=2pt@C=2pt{
			\bullet \hen[d]\hen[r] & \plr{\partial f\bou x} \\
			\circ \hen[d]\hen[r] & \plr{\partial f\bou x} \\
			\plr{f\bou x} \\
		} + \xymatrix@R=2pt@C=2pt{
			\bullet \hen[d]\hen[r] & \circ \hen[d]\hen[r] 
			& \plr{\partial^2 f\bou x} \\
			\plr{f\bou x} & \plr{f\bou x} \\
		}
	\end{split}\end{equation*}
	次のような対応を見ることができる。
	\begin{equation*}\begin{split}
		\plr{\partial^{n+1}x\bou t} \sim \grow_1^n\bullet
	\end{split}\end{equation*}
	そこで、任意の$g\in R\bblr{x}$に対して代数射
	$\q{\phi_xg}_1:R\sizen^*\to R\bblr{x}$を次のように定義すると、
	\begin{equation*}\begin{split}
		\q{\phi_xg}_1\word{n} := \plr{\partial^ng\bou x}
		\quad\text{for all } n\in\sizen\end{split}\end{equation*}
	微分方程式の解$x_t$は次のように書くことがきる。
	\begin{equation*}\begin{split}
		x_t &= x_0 + \lim_{x\to x_0} \q{\phi_xf}_1\code_\clD\sum_{n\in\sizen}
			\frac{t^{n+1}}{\plr{n+1}!}\q{\grow}_1^n\bullet \\
		&= x_0 + \lim_{x\to x_0} \q{\phi_xf}_1\sum_{n\in\sizen}
			\frac{t^{n+1}}{\plr{n+1}!}\q{\grow_\sizen}_1^n\word{0} \\
		&= x_0 + \lim_{x\to x_0} \q{\phi_xf}_1\int_t
			\qgg{t\q{\grow_\sizen}_1}_1^*\word{0} \\
	\end{split}\end{equation*}
	一方、次の式が既知だから、
	\begin{equation}\label{eq:通常の微分方程式の解}\begin{split}
		x_t = x_0 + \lim_{x\to x_0}\int_t
		\qgg{t\plr{f\bou x}\partial_x}^*_1\plr{f\bou x}
	\end{split}\end{equation}
	次の式が成り立つはずである。
	\begin{equation}\label{eq:通常の微分方程式と二分木}\begin{split}
		\q{\phi_xf}_1\qgg{t\q{\grow_\sizen}_1}_1^*\word{0}
		= \qgg{t\plr{f\bou x}\partial_x}^*_1\plr{f\bou x}
	\end{split}\end{equation}
	次の命題によって示すことができる。

	\begin{proposition}[通常の微分と二分木]\label{prop:通常の微分と二分木} %{
		任意の$g\in R\bblr{x}$に対して次の式が成り立つ。
		\begin{equation*}\begin{split}
			\q{\phi_xg}_1\q{\grow_\sizen}_1^n\word{k}
			= \plrgg{\plr{g\bou x}\partial_x}^n\plr{\partial^kg\bou x}
			\quad\text{for all } k,n\in\sizen
		\end{split}\end{equation*}
	\end{proposition} %prop:通常の微分と二分木}
	\begin{proof} %{
		$n$についての帰納法で証明する。$n=0$の時は、$\phi_x$の定義より、命題が
		成り立つ。
		\begin{equation*}\begin{split}
			\q{\phi_xg}_1\word{k} = \plr{\partial^kg\bou x} 
			\quad\text{for all } k\in\sizen
		\end{split}\end{equation*}
		ある$n\in\sizen$以下で命題が成り立つとする。一般の$q$で次の式が成り立つから、
		\begin{equation*}\begin{split}
			\grow_\sizen\word{k} &= \word{0,k+1}
			\quad\text{for all } k\in\sizen \\
			\grow_\sizen m_0 &= m_0\plr{\grow_\sizen\otimes1
				+ q^{\opN_\sizen}\otimes\grow_\sizen} \\
		\end{split}\end{equation*}
		任意の$k\in\sizen$に対して次の式が成り立つ。
		\begin{equation*}\begin{split}
			\grow_\sizen^{n+1}\word{k}
			&= \grow_\sizen^nm_0\plrgg{\word{0}\otimes\word{k+1}} \\
			&= \sum_{r=0}^n\qbinom{n}{r}m_0\plrgg{
				\grow_\sizen^r\word{0}\otimes\grow_\sizen^{n-r}\word{k+1}} \\
		\end{split}\end{equation*}
		したがって、変数$x$を省略して、次のように書くと、
		\begin{equation*}\begin{split}
			g := \plr{g\bou x},\quad \partial := \partial_x
		\end{split}\end{equation*}
		帰納法の仮定から、$\phi_xf$が代数射であることに注意すると、次の式が成り立つ。
		\begin{equation*}\begin{split}
			\q{\phi_xg}_1\q{\grow_\sizen}_1^{n+1}\word{k}
			= \sum_{r=0}^n\binom{n}{r}m_0\plrgg{
				\plr{g\partial}^rg\otimes\plr{g\partial}^{n-r}\partial^{k+1}g} \\
		\end{split}\end{equation*}
		一方、次の式が成り立つから、
		\begin{equation*}\begin{split}
			\plr{g\partial}^{n+1}\partial^kg
			&= \plr{g\partial}^nm_0\plrgg{
				\plr{g\partial}g\otimes\plr{g\partial}\partial^kg} \\
			&= \sum_{r=0}^n\binom{n}{r}m_0\plrgg{
				\plr{g\partial}^rg\otimes\plr{g\partial}^{n-r}\partial^{k+1}g} \\
		\end{split}\end{equation*}
		次の式が成り立ち、$n+1$でも命題が成り立つことがわかる。
		\begin{equation*}\begin{split}
			\q{\phi_xg}_1\q{\grow_\sizen}_1^{n+1}\word{k}
			= \plr{g\partial}^{n+1}\partial^kg
		\end{split}\end{equation*}
	\end{proof} %}

	この命題から次の式が成り立つことがわかり、
	\begin{equation}\label{eq:通常の微分方程式と二分木その二}\begin{split}
		\q{\phi_xf}_1\qgg{t\q{\grow_\sizen}_1}_1^*\word{k} 
		= \qgg{t\plr{f\bou x}\partial_x}^*_1\plr{\partial^kf\bou x}
		\quad\text{for all } k\in\sizen
	\end{split}\end{equation}
	この式で$k=0$としたものが\eqref{eq:通常の微分方程式と二分木}になる。
%s2:通常の微分方程式}
\subsection{q-微分方程式}\label{s2:q-微分方程式} %{
%s2:q-微分方程式}
	$q=1$の場合の微分方程式での際立った特徴として次の式がある。
	\begin{equation*}\begin{split}
		x_t = x_0 + \int_t\plr{f\bou x_t} \implies \plr{f\bou x_t}
		= \lim_{x\to x_0}\q{t\plr{f\bou x}\partial_x}_1^*\plr{f\bou x}
	\end{split}\end{equation*}
	前節では、この式を利用して、微分方程式の解を二分木の成長によって表した。
	\begin{equation*}\begin{split}
		x_t = x_0 + \q{\phi_xf}_1\int_t\q{t\grow_\sizen}_1^*\word{0}
	\end{split}\end{equation*}
%s1:微分方程式と二分木}
%
}\endgroup %}
