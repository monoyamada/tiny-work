\begingroup %{
	\newcommand{\W}{\mycal{W}}
	\newcommand{\T}{\mycal{T}}
	\newcommand{\End}{\myop{End}}
	\newcommand{\Map}{\myop{Map}}
	\newcommand{\Lin}{\mathcal{L}}
	\newcommand{\Hol}{\mathcal{H}}
	%
	\newcommand{\id}{\myop{id}}
	\newcommand{\tran}{\mathbf{t}}
	\newcommand{\dfn}{\,\myop{def}\,}
	\newcommand{\xiff}[2][]{\xLongleftrightarrow[#1]{#2}}
	%
	\newcommand{\bvec}[2]{\begin{bmatrix}{#1}\\{#2}\end{bmatrix}}
	%
	{\setlength\arraycolsep{2pt}
	%
\section{二次方程式の摂動}\label{s1:二次方程式の摂動} %{
	この節では次のような便宜を用いる。
	\begin{description}\setlength{\itemsep}{-1mm} %{
		\item[係数]
		$R=(R,+,0,\myspace,1)$を可換半体とする。ここで、半体とは次の集合とする。
		\begin{itemize}\setlength{\itemsep}{-1mm} %{
			\item 分配則を満たす可換な加法$+$と乗法$\myspace$が定義されている。
			\item 加法と乗法はそれぞれ単位元$0$と$1$を持つ。
			\item $0$以外の元は乗法の逆元を持つ。
		\end{itemize} %}
		加算集合$A$に対して$RA$を自由半ベクトル空間とする。
		\begin{equation*}\begin{split}
			RA := \set{f:A\to R\bou fa \neq 0 \quad\text{for only finitely many } a\in A}
		\end{split}\end{equation*}
		%
		\item[フォック空間]
		$H:=\set{\eta_n\bou n\in\sizen_+}$を可算集合、
		$H^\tran=\set{\eta_{-n}\bou n\in\sizen_+}$を$RH$の双対空間の基底とする。
		\begin{equation*}\begin{split}
			\eta_{-m}\eta_n = \jump{m=n} \quad\text{for all } m,n\in\sizen_+
		\end{split}\end{equation*}
		$R\W H$の単位射を$\ket{0}$、余単位射を$\bra{0}$として、$R\W A$をフォック空間
		で表す。
		\begin{equation*}\begin{split}
			\eta_{-n}\ket{0} = 0 = \bra{0}\eta_n \quad\text{for all } n\in\sizen_+
		\end{split}\end{equation*}
	\end{description} %}

	早速命題を述べる。

	\begin{proposition}[二次方程式の摂動展開]\label{prop:二次方程式の摂動展開} %{
		$A$を集合とする。任意の$a,b,c,d\in R\W A$に対して次の式が成り立つ。
		\begin{equation*}\begin{split}
			x = a + bxcxd \iff x = \bra{0}\begin{pmatrix}
				1 & 0
			\end{pmatrix}\begin{pmatrix}
				b\eta_{-1} & a \\ \eta_{1}c\eta_{-2} & \eta_2d
			\end{pmatrix}^*\begin{pmatrix}
				0 \\ 1
			\end{pmatrix}\ket{0}
		\end{split}\end{equation*}
	\end{proposition} %prop:二次方程式の摂動展開}
	\begin{proof} %{
		$2\times 2$行列$[]$と$\sigma$を次のように定義する。
		\begin{equation*}\begin{split}
			\begin{bmatrix}
				\alpha \\ \beta
			\end{bmatrix} := \begin{pmatrix}
				\alpha & 0 \\ 0 & \beta
			\end{pmatrix},\quad \sigma := \begin{pmatrix}
				0 & 1 \\ 1 & 0
			\end{pmatrix}
		\end{split}\end{equation*}
		すると、命題の行列$M$は次のように書くことができる。
		\begin{equation*}\begin{split}
			M := \begin{pmatrix}
				b\eta_{-1} & a \\ \eta_{1}c\eta_{-2} & \eta_2d
			\end{pmatrix} 
			= \bvec{b\eta_{-1}}{eta_2d} + \bvec{a}{\eta_{1}c\eta_{-2}}\sigma
		\end{split}\end{equation*}
		行列を対角成分と非対角成分に分けてKleeneスターを展開すると次のようになる。
		\begin{equation*}\begin{split}
			M^* &= \left(
				\bvec{b\eta_{-1}}{\eta_2d}^*\bvec{a}{\eta_{1}c\eta_{-2}}\sigma
			\right)^*\bvec{b\eta_{-1}}{\eta_2d}^*
		\end{split}\end{equation*}
		この行列を二次元ベクトル$(1\;0)$と$(0\;1)^\tran$で挟み込むと次のようになる。
		\begin{equation*}\begin{split}
			\begin{pmatrix}
				1 & 0
			\end{pmatrix}M^*\begin{pmatrix}
				0 \\ 1
			\end{pmatrix} = (b\eta_{-1})^*a\biggl(
					(\eta_2d)^*\eta_1c\eta_{-2}(b\eta_{-1})^*a
				\biggr)^*(\eta_2d)^*
		\end{split}\end{equation*}
		さらに、任意の半代数$V$に対して成り立つ次の式を使って、
		\begin{equation*}\begin{split}
			(fg)^* = 1 + f(gf)^*g \quad\text{for all } f,g\in V
		\end{split}\end{equation*}
		$\beta$と$\delta$を次のようにおいて、
		\begin{equation*}\begin{split}
			\beta := \eta_{-2}(b\eta_{-1})^*,\quad \delta := (\eta_2d)^*\eta_1
		\end{split}\end{equation*}
		ブラとケットを対称化すると次のようになる。
		\begin{equation}\label{eq:二次元部分の遷移}\begin{split}
			\begin{pmatrix}
				1 & 0
			\end{pmatrix}M^*\begin{pmatrix}
				0 \\ 1
			\end{pmatrix} &= (b\eta_{-1})^*a(\eta_2d)^*
				+ (b\eta_{-1})^*a\delta c(\beta a\delta c)^* \beta a(\eta_2d)^*
		\end{split}\end{equation}
		ブラ$\bra{b:1}$とケット$\ket{d:2}$を次のように定義して、
		\begin{equation*}\begin{split}
			\bra{b:1} := \bra{0}(b\eta_{-1})^*,\quad \ket{d:2} := (\eta_2d)^*\ket{0}
		\end{split}\end{equation*}
		式\eqref{eq:二次元部分の遷移}の真空期待値をとると次のようになる。
		\begin{equation}\label{eq:真空期待値その一}\begin{split}
			\bra{0}\begin{pmatrix}
				1 & 0
			\end{pmatrix}M^*\begin{pmatrix}
				0 \\ 1
			\end{pmatrix}\ket{0} &= a + b\bra{b:1}ac(\beta a\delta c)^*a\ket{d:2}d
		\end{split}\end{equation}
		この式の二項目の$\bra{b:1}ac(\beta a\delta c)^*a\ket{d:2}$を因子化することを
		考える。Kleeneスターの部分を正規積で表すと次のようになる。
		\begin{equation*}\begin{split}
			(\beta a\delta c)^1 &= ad\delta c + \beta b)c \\
			(\beta a\delta c)^2 &= \bigl((ad\delta c)^2 + bacad^2\delta c\bigr)
			+ (ad\delta c)(\beta bac)
			+ \bigl((\beta bac)^2 + \beta b^2bacad\bigr)
		\end{split}\end{equation*}
		そこで、点列$\set{x_n\in R\W A\bou n\in\sizen}$を次のように、
		\begin{equation}\label{eq:摂動係数の漸化式}\begin{split}
			x_0 &= a \\
			x_{n+1} &= \sum_{p=0}^n bx_{n-p}cx_pd \quad\text{for all } n\in\sizen
		\end{split}\end{equation}
		線形写像$\xi_\pm:R\sizen_+\to V$を次のように、
		\begin{equation*}\begin{split}
			\xi_{+n} = x_{n-1}d\delta,\quad \xi_{-n} = \beta bx_{n-1}
			\quad\text{for all } n\in\sizen_+
		\end{split}\end{equation*}
		線形写像$\xi_{\pm}^c:R\W\sizen_+\to V$を次のように定義すると、
		\begin{equation*}\begin{split}
			\xi_\pm^c1_\W &= 1 \\
			\xi_\pm^c[n_1\cdots n_p] &= (\xi_\pm n_1)c\cdots(\xi_\pm n_p)c
			\quad\text{for all } n_1,\dots,n_p\in \sizen_+
		\end{split}\end{equation*}
		$\beta a\delta=(\xi_++\xi_-)1$と書け次のようになる。
		\begin{equation*}\begin{split}
			(\beta a\delta c)^0 &= \bigl(\xi_+^c1_\W\bigr)\bigl((\xi_-^c1_\W\bigr) \\
			(\beta a\delta c)^1 &= \bigl(\xi_+^c[1]\bigr)\bigl(\xi_-^c1_\W\bigr) 
				+ \bigl(\xi_+^c1_\W\bigr)\bigl(\xi_-^c[1]\bigr) \\
			(\beta a\delta c)^2 &= \bigl(\xi_+^c[2] 
				+ \xi_+^c[11]\bigr)\bigl(\xi_-^c1_\W\bigr) 
				+ \bigl(\xi_+^c[1]\bigr)\bigl(\xi_-^c[1]\bigr)
				+ \bigl(\xi_+^c1_\W\bigr)\bigl(\xi_-^c[2] + \xi_-^c[11]\bigr) \\
		\end{split}\end{equation*}
		したがって、数の合成$C_n\subset \W\sizen_+$を次のようにおき、
		\begin{equation*}\begin{split}
			C_0 &:= \Set{1_\W} \\
			C_1 &:= \Set{[1]} \\
			C_2 &:= \Set{[2],[11]} \\
			C_3 &:= \Set{[3],[21],[12],[111]} \\
			\cdots
		\end{split}\end{equation*}
		ある$n\in\sizen_+$で次の式が成り立つと仮定する。
		\begin{equation}\label{eq:帰納法の仮定その一}\begin{split}
			(\beta a\delta c)^n &= \sum_{p=0}^n \sum_{\gamma_+\in C_{n-p}}
				\sum_{\gamma_-\in C_p} (\xi_+^c\gamma_+)(\xi_-^c\gamma_-)
		\end{split}\end{equation}
		すると、任意の$p,q\in\sizen$に対して$x_{pq}$を次のようにおき、
		\begin{equation*}\begin{split}
			x_{pq} := bx_pcx_qd
		\end{split}\end{equation*}
		任意の$n\in\sizen$に対して$\Gamma_{\pm n}$を次のようにおき、
		\begin{equation*}\begin{split}
			\Gamma_{\pm n} := \sum_{\gamma\in C_n}(\xi_\pm^c\gamma)
		\end{split}\end{equation*}
		任意の$n\in\sizen_+$に対して成り立つ次の式を使うと、
		\begin{equation*}\begin{split}
			\Gamma_n = \sum_{p=1}^n (\xi_+^c[p])\Gamma_{n-p}
			= \sum_{p=1}^n (\xi_+p)c \Gamma_{n-p}
		\end{split}\end{equation*}
		任意の$n\in\sizen_+$に対して次の式が成り立つことがわかり、
		\begin{equation*}\begin{split}
			&(\xi_-1)c\Gamma_n \\
			&= \sum_{p=1}^n \bigl(x_{0(p-1)}d\delta + \beta bx_{0(p-1)}\bigr)c \Gamma_{n-p} \\
			&= (\xi_+2)c\Gamma_{n-1} + (\xi_-2)c\Gamma_{n-1}
				+ \sum_{p=2}^n \bigl(x_{0(p-1)}d\delta + \beta bx_{0(p-1)}\bigr)c \Gamma_{n-p} \\
			&= (\xi_+2)c\Gamma_{n-1}
				+ \sum_{p=2}^n \bigl(
					x_{0(p-1)}d\delta + \beta bx_{0(p-1)} + x_{1(p-1)}d\delta + \beta bx_{1(p-1)}
				\bigr)c \Gamma_{n-p} \\
			&= (\xi_+2)c\Gamma_{n-1} + (\xi_+3)c\Gamma_{n-2} + (\xi_-3)c\Gamma_{n-2}
				+ \sum_{p=3}^n \bigl(x_{0(p-1)}d\delta + \beta bx_{0(p-1)}\bigr)c \Gamma_{n-p} \\
			&= \cdots \\
			&= (\xi_+2)c\Gamma_{n-1} +\cdots+ (\xi_+n)c\Gamma_1 + \sum_{q=0}^{n-1}
				\bigl(x_{q(n-1-q)}d\delta + \beta bx_{q(n-1-q)}\bigr)c \Gamma_0 \\
			&= (\xi_+2)c\Gamma_{n-1} +\cdots+ (\xi_+(n+1))c\Gamma_0 
				+ (\xi_-(n+1))c\Gamma_0 \\
		\end{split}\end{equation*}
		この式を使うと次の式が成り立つことがわかり、
		\begin{equation*}\begin{split}
			(\beta a\delta c)^{n+1} 
			&= (\xi_+1+\xi_-1)c \sum_{p=0}^n \Gamma_{(n-p)}\Gamma_{-p} \\
			&=  \sum_{p=0}^n \biggl(
				(\xi_+1)c\Gamma_{(n-p)} + (\xi_-1)c\Gamma_{n-p}\biggr)\Gamma_{-p} \\
			&=  \sum_{p=0}^n \biggl(
				\Gamma_{(n-p+1)} + (\xi_-(n-p+1))c\biggr)\Gamma_{-p} \\
			&= \sum_{p=0}^{n+1} \Gamma_{(n-p)}\Gamma_{-p} \\
			&= \sum_{p=0}^{n+1} \sum_{\gamma_+\in C_{n-p}}
				\sum_{\gamma_-\in C_p} (\xi_+^c\gamma_+)(\xi_-^c\gamma_-)
		\end{split}\end{equation*}
		帰納法の仮定\eqref{eq:帰納法の仮定その一}が$n+1$でも成り立つことがわかる。
		したがって、次の式が成り立つが、
		\begin{equation*}\begin{split}
			(\beta a\delta c)^* 
			&= \sum_{n\in\sizen} \sum_{p=0}^n \Gamma_{n-p}\Gamma_{-p}
			= \left\{\begin{array}{clclclcl}
				& \Gamma_0\Gamma_0 \\
				+ & \Gamma_0\Gamma_{-1} &+& \Gamma_1\Gamma_0 \\
				+ & \Gamma_0\Gamma_{-2} &+& \Gamma_1\Gamma_{-1} &+& \Gamma_2\Gamma_{0} \\
				+ & \cdots
			\end{array}\right. \\
			&= (\sum_{n\in\sizen}\Gamma_n)(\sum_{n\in\sizen}\Gamma_{-n}) \\
		\end{split}\end{equation*}
		次の集合同型が成り立つから、
		\begin{equation*}\begin{split}
			\cup_{n\in\sizen}C_n \simeq \W\sizen_+ \quad\text{as set}
		\end{split}\end{equation*}
		次の式が成り立ち、
		\begin{equation*}\begin{split}
			\sum_{n\in\sizen}\Gamma_n
			= \sum_{n\in\sizen}\sum_{\gamma\in C_n} \xi_+^c\gamma
			= \bigl(\sum_{m\in\sizen_+}(\xi_+m)c\bigr)^*
			= \bigl(\sum_{n\in\sizen}x_nd\delta c\bigr)^*
		\end{split}\end{equation*}
		次の式が成り立つ。
		\begin{equation*}\begin{split}
			(\beta a\delta c)^* = \bigl(\sum_{n\in\sizen}x_nd\delta c\bigr)^*
				\bigl(\sum_{n\in\sizen}\beta bx_nc\bigr)^*
		\end{split}\end{equation*}
		したがって、$x_*\in R\W A$を次のようにおくと、
		\begin{equation*}\begin{split}
			x_* := \sum_{n\in\sizen} x_n
		\end{split}\end{equation*}
		真空期待値\eqref{eq:真空期待値その一}は次のようになる。
		\begin{equation*}\begin{split}
			\bra{0}\begin{pmatrix}
				1 & 0
			\end{pmatrix}M^*\begin{pmatrix}
				0 \\ 1
			\end{pmatrix}\ket{0} &= a + b\bra{b:1}ac(\beta a\delta c)^*a\ket{d:2}d \\
			&= a + b\bra{b:1}a\bigl(cx_*d\delta c\bigr)^*
				c\bigl(\beta bx_*c\bigr)^*a\ket{d:2}d \\
			&= a + b\bra{b:1}a\bigl(cx_*d\eta_1 c\bigr)^*
				c\bigl(\eta_{-2} bx_*c\bigr)^*a\ket{d:2}d \\
		\end{split}\end{equation*}
		そして、$\ket{d:2}$は$\eta_{-2} bx_*c$の固有ベクトルとなっていて、
		\begin{equation*}\begin{split}
			(\eta_{-2} bx_*c)f\ket{d:2} = bx_*cfd \quad\text{for all } f\in R\W A
		\end{split}\end{equation*}
		$\bra{b:1}$と$\ket{d:2}$は次の式を満たすから、
		\begin{equation*}\begin{split}
			\bra{b:1}\ket{0} = \bra{b:1}\ket{d:2} = \bra{0}\ket{d:2}
		\end{split}\end{equation*}
		次の因子化が成り立つことがわかる。
		\begin{equation*}\begin{split}
			&\bra{b:1}a\bigl(cx_*d\eta_1 c\bigr)^*
				c\bigl(\eta_{-2} bx_*c\bigr)^*a\ket{d:2} \\
			&= \bra{b:1}a\bigl(cx_*d\eta_1 c\bigr)^*\ket{0}
				c\bra{0}\bigl(\eta_{-2} bx_*c\bigr)^*a\ket{d:2} \\
			&= \bra{b}a\ket{cx_*d}c\bra{bx_*c}a\ket{d} \\
		\end{split}\end{equation*}
		そして、命題の式をパラメーター$t\in R$を用いて拡張した次の式
		\begin{equation*}\begin{split}
			y_t = a + tby_tcy_td
		\end{split}\end{equation*}
		の正則解$y_t=\sum_{n\in\sizen}t^ny_n$の摂動係数$y_n$が満たす次の漸化式は、
		\begin{equation*}\begin{split}
			y_0 &= a \\
			y_{n+1} &= \sum_{p=0}^n by_{n-p}cy_pd \quad\text{for all } n\in\sizen
		\end{split}\end{equation*}
		$x_*$を定義する漸化式\eqref{eq:摂動係数の漸化式}に他ならないので、
		$x_*$は命題の式の解となることがわかる。
		\begin{equation*}\begin{split}
			x_* = a + bx_*cx_*d
		\end{split}\end{equation*}
		そして、次の式より、
		\begin{equation*}\begin{split}
			\bra{b}a\ket{cx_*d} = \sum_{n\in\sizen}b^na(cx_*d)^n 
			= x_* = \sum_{n\in\sizen}(bx_*c)^nad^n = \bra{bx_*c}a\ket{d}
		\end{split}\end{equation*}
		真空期待値\eqref{eq:真空期待値その一}は次のようになり、命題が成り立つこと
		がわかる。
		\begin{equation*}\begin{split}
			\bra{0}\begin{pmatrix}
				1 & 0
			\end{pmatrix}M^*\begin{pmatrix}
				0 \\ 1
			\end{pmatrix}\ket{0} &= a + bx_*ax_*d = x_*
		\end{split}\end{equation*}
	\end{proof} %}
	\begin{todo}[保留]\label{todo:保留} %{
	任意の$f,g\in R\W A$と$n\in\sizen$に対して次の式が成り立つ。
	\begin{equation*}\begin{split}
		(\beta f)(g\delta)^* &= \beta h_n + h_ngd\delta(g\delta)^*
			+ \beta b^{n+1}xg^{n+1}(g\delta)^* \\
		h_n &= \sum_{p=0}^nb^nfg^n \\
	\end{split}\end{equation*}
	\end{todo} %todo:保留}
%s1:二次方程式の摂動}
	%
}\endgroup %}
