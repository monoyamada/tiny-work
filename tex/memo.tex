\section{モノイド半環の自己線形写像}\label{s1:モノイド半環の自己線形写像} %{
	$R=(R,+,0,m_R,1)$を可換半環、$R_x=R[[x]]$を$\set{x^n}_{n=0}^\infty$を基底
	とする$R$係数半モジュールとする。
	$R_x$の基底を任意の$n\in \mybf{N}$に対して括弧つきで$[n]=x^n$のように書く。

	$R_x$に次の積$m_x$と余積$\Delta_x$を定義する。
	\begin{equation}\label{eq:多項式の積と余積}\begin{split} %{
		m_x([m]\otimes[n]) &= [m+n] \\
		\Delta_x[m] &= \sum_{k=0}^m\binom{m}{k}[k]\otimes[m-k] \\
	\end{split}\end{equation} %}
	$m_x$と$\Delta_x$は双対となる。$m_x$に関する単位射は$u_x1=[0]$、
	$\Delta_x$に関する余単位射は$\epsilon_x[m]=\jump{m=0}$となる。

	$R_x$から$R$への線形写像全体を$R_x^t$とおく。任意の$p,m\in\mybf{N}$に
	対して$[p]^t_*\in R_x^t$を$[p]^t_*[m]=\jump{p=m}$で定義する。すると、
	$R_x^t$は$\set{[n]^t_*}_{n\in\mybf{N}}$を基底とする$R$係数半モジュールとなる。
	$R_x^t$には畳み込みにより次の積$m_x^t$と余積$\Delta_x^t$が定義される。
	\begin{equation}\begin{split} %{
		m_x^t([p]^t_*\otimes [q]^t_*) &= \binom{p+q}{p}[p+q]^t_* \\
		\Delta_x^t[p]^t_* &= \sum_{k=0}^p[k]^t_*\otimes[p-k]^t_* \\
	\end{split}\end{equation} %}
	この積$m_x^t$と余積$\Delta_x^t$の定義は、積$m_x$と余積$\Delta_x$の定義
	\eqref{eq:多項式の積と余積}をスケール変換したものと同じ形になっている。
	$[p]^t=p![p]^t_*$とすると、畳み込みによる積$m_x^t$と余積$\Delta_x^t$は
	次のようになる。
	\begin{equation}\begin{split} %{
		m_x^t([p]^t\otimes [q]^t) &= [p+q]^t \\
		\Delta_x^t[p]^t &= \sum_{k=0}^p\binom{p}{k}[k]^t\otimes[p-k]^t \\
	\end{split}\end{equation} %}
	$m_x^t$に関する単位射は$u_x^t1=[0]^t$、$\Delta_x^t$に関する余単位射は
	$\epsilon_x^t[p]^t=\jump{p=0}$となる。$m_x^t$と$\Delta_x^t$は双対となる。

	$MR_x=\homset(R_x,R_x)$を自己線形写像全体のつくる空間とする。
	\begin{equation}\begin{split} %{
		f(y_1+y_2) &= (fy_1) + (fy_2)\quad\text{for all }f\in MR_x,\;y_1,y_2\in R_x \\
		f(ry) &= r(fy)\quad\text{for all }f\in MR_x,\;y\in R_x,\;r\in R \\
	\end{split}\end{equation} %}
	$MR_x$には次のように加法とスカラー積を定義することができる。
	\begin{equation}\begin{split} %{
		+: MR_x\times MR_x &\to MR_x \\
			f\times g&\mapsto (f+g) \quad\text{such that }\\
			&\quad (f+g)y=(fy)+(gy)\quad\text{for all }y\in R_x \\
		\rhd_R: R\times MR_x &\to MR_x \\
			r\times f &\mapsto (r\rhd_Rf) \quad\text{such that } \\
			&\quad(r\rhd_Rf)y=r(fy)\quad\text{for all }y\in R_x \\
		\lhd_R: MR_x\times R &\to MR_x \\
			f\times r &\mapsto (f\rhd_Rr) \quad\text{such that } \\
			&\quad(f\rhd_Rr)y=(fy)r\quad\text{for all }y\in R_x \\
	\end{split}\end{equation} %}
	したがって、$MR_x$は$R$を係数とする半モジュールとなる。
	以下では、スカラー積$\rhd_R$と$\lhd_R$の記号は省略して書く。
	テンソル積$R_x\otimes R_x^t$から$MR_x$への線形写像を$\mu$を次のように
	定義する。
	\begin{equation}\begin{split} %{
		\mu: R_x\otimes R_x^t &\to MR_x \\
			y_1\otimes y_2^t &\mapsto \mu(y_1\otimes y_2^t)\quad\text{such that } \\
			&\quad \mu(y_1\otimes y_2^t)z = y_1(y_2^t z)\quad\text{for all }z\in R_x \\
	\end{split}\end{equation} %}
	以下では、任意の$f,g\in A$に対して、$fg^t=\mu(f\otimes g^t)$と書く。
	任意の$f\in MR_x$は次のように書くことができる。
	\begin{equation*}\begin{split} %{
		f = \sum_{p=0}^\infty\frac{1}{p!}(f[p])[p]^t
	\end{split}\end{equation*} %}
	したがって、$\mu$は$R$半モジュール同型となる。
	$MR_x$の基底として$\set{[p][q]^t}_{p,q\in\mybf{N}}$をとることができる。
	見やすいように$\mybinom{p}{q}=[p][q]^t$と書くことにする。
	任意の$p,q,m\in \mybf{N}$に対して次のようになる。
	\begin{equation}\begin{split} %{
		\mybinom{p}{q}[m] &= \jump{m=q}q![p] \\
		\mybinom{p}{q} &= q![p][q]^t_* \\
	\end{split}\end{equation} %}

	$MR_x$の合成$\circ$もまた$MR_x$の元になる。
	\begin{equation}\begin{split} %{
		f\circ g\in MR_x \quad\text{for all }f,g\in MR_x \\
	\end{split}\end{equation} %}
	したがって、写像の合成$\circ$は恒等写像$1_\circ=\sum_{p\in \mybf{N}}\mybinom{p}{p}$を
	単位元とする積となる。写像の合成を前置記号で$m_\circ$と書くことにする。
	積$m_\circ$は、任意の$p_1,p_2,q_1,q_2\in \mybf{N}$に対して次のようになる。
	\begin{equation}\begin{split} %{
		m_\circ(\mybinom{p_1}{q_1}\otimes \mybinom{p_2}{q_2}) 
		&= ([q_1]^t[p_2])\mybinom{p_1}{q_2} \\
		&= \jump{q_1=p_2}q_2!\mybinom{p_1}{q_2} 
	\end{split}\end{equation} %}
	積$m_\circ$に関する生成系は次のものをとることができる。
	\begin{equation}\begin{split} %{
		\set{\mybinom{0}{1},\mybinom{0}{2},\dots,\mybinom{1}{0},\mybinom{2}{0},\dots}
	\end{split}\end{equation} %}

	写像の合成以外による積の他に、次の可換図で積$m_\mu:MR_x\otimes MR_x\to MR_x$
	を定義することができる。
	\begin{equation}\xymatrix{
		R_x\otimes R_x^t\otimes R_x\otimes R_x^t \ar[r]^{\mu\times \mu} \ar[d]^{(m_x\otimes m_x^t)\sigma_{23}}
		& MR_x\otimes MR_x \ar@{.>}[d]^{m_\mu} \\
		R_x\otimes R_x^t \ar[r]^{\mu} & MR_x \\
	}\end{equation}
	積$m_\mu$は、任意の$p_1,p_2,q_1,q_2\in\mybf{N}$に対して次のようになる。
	\begin{equation}\begin{split} %{
		m_\mu(\mybinom{p_1}{q_1}\otimes\mybinom{p_2}{q_2}) = \mybinom{p_1+p_2}{q_1+q_2}
	\end{split}\end{equation} %}
	任意の$p_1,p_2,p_3,q_1,q_2,q_3\in\mybf{N}$に対して次のようになるから、
	積$m_x$と積$m_x^t$の結合性から、写像$m_\mu$が結合的になることは確かめられる。
	\begin{equation}\begin{split} %{
		&(\myid\otimes m_\mu)m_\mu(\mybinom{p_1}{q_1}\otimes\mybinom{p_2}{q_2}\otimes\mybinom{p_3}{q_3}) \\
		&\quad= ((\myid\otimes m_x)m_x[p_1p_2p_3]_\otimes)((\myid\otimes m_x^t)m_x^t[q_1q_2q_3]^t_\otimes) \\
		&(m_\mu\otimes \myid)m_\mu(\mybinom{p_1}{q_1}\otimes\mybinom{p_2}{q_2}\otimes\mybinom{p_3}{q_3}) \\
		&\quad= ((m_x\otimes \myid)m_x[p_1p_2p_3]_\otimes)((m_x^t\otimes \myid)m_x^t[q_1q_2q_3]^t_\otimes) \\
	\end{split}\end{equation} %}
	ここで、表記を簡略化するために、任意の$n\in\mybf{N}$および
	$p_1,p_2,\dots,p_n\in\mybf{N}$次のようにおいた。
	\begin{equation}\begin{split} %{
		[p_1p_2\cdots p_n]_\otimes &= [p_1]\otimes[p_2]\otimes\cdots\otimes[p_n] \\
		[p_1p_2\cdots p_n]^t_\otimes &= [p_1]^t\otimes[p_2]^t\otimes\cdots\otimes[p_n]^t \\
	\end{split}\end{equation} %}
	積$m_\mu$に関する単位射は$u_\mu1=\mybinom{0}{0}$となる。
	積$m_\mu$に関する単位元を$1_\mu=\mybinom{0}{0}$ とおく。
	積$m_\mu$に関する生成系は次のものをとることができる。
	\begin{equation}\begin{split} %{
		\set{\mybinom{0}{1},\mybinom{1}{0}}
	\end{split}\end{equation} %}
	余積$\Delta_\mu$を次の可換図で定義する。
	\begin{equation}\xymatrix{
		R_x\otimes R_x^t\otimes R_x\otimes R_x^t \ar[r]^{\mu\times \mu}
		& MR_x\otimes MR_x \\
		R_x\otimes R_x^t \ar[r]^{\mu} \ar[u]_{\sigma_{23}(\Delta_x\otimes \Delta_x^t)} & MR_x \ar@{.>}[u]_{\Delta_\mu} \\
	}\end{equation}
	余積$\Delta_\mu$は、任意の$p,q\in\mybf{N}$に対して次のようになる。
	\begin{equation}\begin{split} %{
		\Delta_\mu\mybinom{p}{q} = \sum_{k=0}^p\sum_{l=0}^q\binom{p}{k}\binom{q}{l}\mybinom{k}{l}\otimes \mybinom{p-k}{q-l} \\
	\end{split}\end{equation} %}
	任意の$p,q\in\mybf{N}$に対して次のようになるから、余積$\Delta_x$と
	余積$\Delta_x^t$の余結合性から、写像$\Delta_\mu$が余結合的になることは
	確かめられる。
	\begin{equation}\begin{split} %{
		(\myid\otimes \Delta_\mu)\Delta_\mu\mybinom{p}{q}
		&= ((\myid\otimes \Delta_x)\Delta_x[p])((\myid\otimes \Delta_x^t)\Delta_x^t[q]^t_*) \\
		(\Delta_\mu\otimes \myid)\Delta_\mu\mybinom{p}{q}
		&= ((\Delta_x\otimes \myid)\Delta_x[p])((\Delta_x^t\otimes \myid)\Delta_x^t[q]^t_*) \\
	\end{split}\end{equation} %}
	余積$\Delta_\mu$に関する余単位射$\epsilon_\mu$は次のようになる。
	\begin{equation}\begin{split} %{
		\epsilon_\mu\mybinom{p}{q} = \jump{p=q=0} \quad\text{for all }p,q\in\mybf{N}
	\end{split}\end{equation} %}
	任意の$p_1,p_2,q_1,q_2\in\mybf{N}$に対して次の式が成り立つから、
	積$m_x$と余積$\Delta_x$の双対性、積$m_x^t$と余積$\Delta_x^t$の双対性から、
	積$m_\mu$と余積$\Delta_\mu$は互いに双対になる。
	\begin{equation}\begin{split} %{
		&\Delta_\mu m_\mu\kakko{\mybinom{p_1}{q_1}\otimes\mybinom{p_2}{q_2}} \\
		&\quad = (\Delta_x m_x[p_1p_2])(\Delta_x^t m_x^t[q_1q_2]^t_*) \\
		&(m_\mu\otimes m_\mu)\sigma_{23}(\Delta_\mu\otimes \Delta_\mu)\kakko{\mybinom{p_1}{q_1}\otimes\mybinom{p_2}{q_2}} \\
		&\quad = ((m_x\otimes m_x)\sigma_{23}(\Delta_x\otimes \Delta_x)[p_1p_2])
		((m_x^t\otimes m_x^t)\sigma_{23}(\Delta_x^t\otimes \Delta_x^t)[q_1q_2]^t_*)
	\end{split}\end{equation} %}
	任意の$p_1,p_2,q_1,q_2,m\in\mybf{N}$に対して次の可換図が成り立つ。
	\begin{equation}\xymatrix{
		\displaystyle\sum_{k=0}^m\binom{m}{k}[k]\otimes[m-k] 
		\ar[d]^{\mybinom{p_1}{q_1}\otimes \mybinom{p_2}{q_2}}
		& [m] \ar[l]_{\Delta_x} 
		\ar[d]^{m_\mu(\mybinom{p_1}{q_1}\otimes \mybinom{p_2}{q_2})}
		\\
		\jump{m=q_1+q_2}m![p_1]\otimes[p_2] \ar[r]^{m_x}
		& \jump{m=q_1+q_2}m![p_1+p_2]
		\\
	}\end{equation}
	また、任意の$p,q,m,n\in\mybf{N}$に対して次の可換図が成り立つ。
	\begin{equation}\xymatrix@C+3pc{
		[m]\otimes[n] \ar[r]^{m_x} \ar[d]^{\Delta_\mu\mybinom{p}{q}}
		& [m+n] \ar[d]^{\mybinom{p}{q}}
		\\
		\jump{q=m+n}q!\sum_{k=0}^p\binom{p}{k}[k]\otimes[p-k] 
		& \jump{q=m+n}q![p] \ar[l]_{\Delta_x}
		\\
	}\end{equation}
	したがって、双半代数$(MR_x,m_\mu,u_\mu,\Delta_\mu,\epsilon_\mu)$は
	双半代数$(R_x,m_x,u_x,\Delta_x,\epsilon_x)$の畳み込みからも得ることが
	できる。

	ここで、半代数準同型写像$(R_x,m_x,u_x)\to (R_x,m_x,u_x)$を考える。
	半代数準同型写像$(R_x,m_x,u_x)\to (R_x,m_x,u_x)$全体を
	$\myop{end}R_x=(R_x,m_x,u_x)\to (R_x,m_x,u_x)$と書く。
	任意の$y\in R_x$に対して$x\mapsto y$となる$y_*\in\myop{end}R_x$
	は次のように書ける。
	\begin{equation}\begin{split} %{
		y_* &= 1_\mu + y[1] + \frac{1}{2!}m_x(y\otimes y)[2]^t + \cdots \\
		&= 1_\mu + y[1]^t + \frac{1}{2!}m_\mu(y[1]^t\otimes y[1]^t) + \cdots \\
		&= \exp_\mu(y[1]^t) \\
	\end{split}\end{equation} %}
	ここで、$\exp_\mu:MR_x\to MR_x$を、任意の$f\in MR_x$に対して次のように
	おいた。
	\begin{equation}\begin{split} %{
		\exp_\mu f &= 1_\mu + f + \frac{1}{2!}m_\mu(f\otimes f) + \cdots \\
	\end{split}\end{equation} %}
	特に、合成の単位元$1_\circ$は$\exp_\mu$を用いて次のように書ける。
	\begin{equation}\begin{split} %{
		\exp_\mu \mybinom{1}{1} &= 1_\mu + \mybinom{1}{1} + \frac{1}{2!}m_\mu(\mybinom{1}{1}\otimes \mybinom{1}{1}) + \cdots \\
	\end{split}\end{equation} %}
	写像$\exp_\mu$は線形写像ではないことに注意する。

\begin{todo}[until]\label{todo:until} %{
	自己半代数準同型の合成は自己半代数準同型になるから、
	$(\myop{end}R_x,m_\circ,u_\circ)$は$(MR_x,m_\circ,u_\circ)$の部分半代数
	となる。次の写像$\phi$は集合同型となる。
	\begin{equation}\begin{split} %{
		\phi: R_x &\to \myop{end}R_x \\ 
			y &\mapsto \exp_\mu(y[1]^t) \\
	\end{split}\end{equation} %}

	任意の$f,g\in MR_x$に対して$m_\mu(f\otimes g)=fg$とし、
	$y=y_0+y_1[1]+y_2[2]+\cdots\in R_x$を
	$y=y_0\mybinom{0}{0}+y_1\mybinom{1}{0}+y_2\mybinom{2}{0}+\cdots\in MR_x$
	と同一視する。$y\in R_x$として微分作用素$y\partial$は次の式で得られる。
	\begin{equation}\label{eq:微分変換群の一次近似}\begin{split} %{
		\phi([1]+y) &= \exp_\mu(\mybinom{1}{1}+y\mybinom{0}{1}) \\
			&= (\exp_\mu\mybinom{1}{1})(\exp_\mu y\mybinom{0}{1}) \\
			&= 1_\circ(\exp_\mu y\mybinom{0}{1}) \\
			&= 1_\circ + 1_\circ(y\mybinom{0}{1}) + \frac{1}{2!}1_\circ(y\mybinom{0}{1})^2 + \cdots \\
			&= 1_\circ + y(1_\circ\mybinom{0}{1}) + \frac{1}{2!}y^2(1_\circ\mybinom{0}{2}) + \cdots \\
			&= 1_\circ + y\partial + \frac{1}{2!}y^2(\partial\circ\partial) + \cdots \\
	\end{split}\end{equation} %}
	ここで、$\partial$を次のようにおき、
	\begin{equation}\begin{split} %{
		\partial = 1_\circ*\mybinom{0}{1} = \sum_{k=0}^\infty \frac{1}{k!}\mybinom{k}{k+1} \\
	\end{split}\end{equation} %}
	次の式が成り立つことを使った。
	\begin{equation}\begin{split} %{
		1_\circ\mybinom{0}{2} &= \partial\circ\partial \\
		1_\circ\mybinom{0}{3} &= \partial\circ\partial\circ\partial \\
		\cdots &= \cdots \\
	\end{split}\end{equation} %}
	式\eqref{eq:微分変換群の一次近似}は、微分変換$\exp_\circ(y\partial)$の
	一次近似となっている。
\end{todo} %todo:until}
%s1:モノイド半環の自己線形写像}

\section{The Explicit Computation of Integration Algorithms and First Integrals for Ordinary Differential Equations With Polynomial Coefficients Using Trees}\label{s1:The Explicit Computation of Integration Algorithms and First Integrals for Ordinary Differential Equations With Polynomial Coefficients Using Trees} %{
	論文\cite{url:grossman:explicit}に関するメモを書いておく。

	簡単のために一次元で考える。体$k$を実数または複素数とする。
	$k$上の常微分方程式を考える。与えられた$v\in(k[x]\to k[x])$に対して、
	次の式を満たす写像$x:k\to k$を求めることを考える。
	\begin{equation}\label{eq:お題の常微分方程式}\begin{split} %{
		\partial_tx = v(x),\quad x0=x_0 \\
	\end{split}\end{equation} %}
	この式の形式解$x$は次のように書ける。
	\begin{equation}\begin{split} %{
		x = e^{tv(x)\partial_x}x\bou_{x=x_0}
	\end{split}\end{equation} %}
	したがって、次の式を処理することで\eqref{eq:お題の常微分方程式}の解
	の持つ性質が導かれる。
	\begin{equation}\label{eq:常微分方程式の形式解}\begin{split} %{
		e^{tv(x)\partial_x}x
	\end{split}\end{equation} %}
	この形式解の中で写像$e^{tv(x)\partial_x}$は$k[x]\to k[x]$の線形写像
	となっている。$k[x]\to k[x]$の線形写像全体を$\myop{end}k[x]$と書く。
	写像$e^{tv(x)\partial_x}$を$t$についてべき展開すると次のようになる。
	\begin{equation}\begin{split} %{
		e^{tv(x)\partial_x} &= \sum_{k\in\mybf{N}}\frac{t^k}{k!}(v\partial_x)^n \\
	\end{split}\end{equation} %}
	$(v\partial_x)^n$は$n=0,1,2,3$に対して次のようになる。
	\begin{equation}\begin{split} %{
		(v\partial_x)^0 &= 1 \\
		(v\partial_x)^1 &= v\partial_x \\
		(v\partial_x)^2 &= v^2\partial_x^2 + v(\partial_xv)\partial_x \\
		(v\partial_x)^3 &= v^3\partial_x^3 + 3v^2(\partial_xv)\partial_x^2 
			+ v^2(\partial_x^2v)\partial_x + v(\partial_xv)^2\partial_x \\
	\end{split}\end{equation} %}
	この$t$に関するべき展開は、根を黒丸、根以外の頂点を$v$とする木で
	書くことができる。
	\begin{equation}\begin{split} %{
		(v\partial_x)^0 &\sim \mytree {
			\bullet
		} \\
		(v\partial_x)^1 &\sim \mytree {
			\bullet \ar@{-}[d] \\
			v
		} \\
		(v\partial_x)^2 &\sim \mytree {
			& \bullet \ar@{-}[dl] \ar@{-}[dr] \\
			v && v \\
		} + \mytree {
			\bullet \ar@{-}[d] \\
			v \ar@{-}[d] \\
			v \\
		} \\
		(v\partial_x)^3 &\sim \mytree {
			& \bullet \ar@{-}[dl] \ar@{-}[d] \ar@{-}[dr] \\
			v & v & v \\
		} + 3 \mytree {
			& \bullet \ar@{-}[dl] \ar@{-}[dr] \\
			v \ar@{-}[d] && v \\
			v \\
		} + \mytree {
			& \bullet \ar@{-}[d] \\
			& v \ar@{-}[dl] \ar@{-}[dr] \\
			v && v \\
		} + \mytree {
			\bullet \ar@{-}[d] \\
			v \ar@{-}[d] \\
			v \ar@{-}[d] \\
			v \\
		}  \\
	\end{split}\end{equation} %}
	$(v\partial_x)^{n+1}$を表す木は、$(v\partial_x)^{n}$を表す木の根を含む
	各頂点の最後の子供に頂点$v$を付け足すと得られれる。
	わかりやすくするために頂点の種類を$\set{v_1,v_2,\dots}$と複数に増やして、
	頂点$v_i$を付け足す操作を$\lhd$で表すと次のような図になる。
	\begin{equation}\begin{split} %{
		\mytree{
			\bullet
		} \lhd v_1 &= \mytree{
			\bullet \ar@{-}[d] \\
			v_1
		} \\
		\mytree{
			\bullet \ar@{-}[d] \\
			v_1
		} \lhd v_2 &= \mytree{
			& \bullet \ar@{-}[ld] \ar@{-}[rd] \\
			v_1 && v_2
		} + \mytree{
			\bullet \ar@{-}[d] \\
			v_1 \ar@{-}[d] \\
			v_2
		} \\
		\mytree{
			& \bullet \ar@{-}[ld] \ar@{-}[rd] \\
			v_1 && v_2
		} \lhd v_3 &= \mytree{
			& \bullet \ar@{-}[ld] \ar@{-}[d] \ar@{-}[rd] \\
			v_1 & v_2 & v_3
		} + \mytree{
			& \bullet \ar@{-}[ld] \ar@{-}[rd] \\
			v_1 \ar@{-}[d] && v_2 \\
			v_3
		} + \mytree{
			& \bullet \ar@{-}[ld] \ar@{-}[rd] \\
			v_1 && v_2 \ar@{-}[d] \\
			&& v_3
		} \\
		\mytree{
			\bullet \ar@{-}[d] \\
			v_1 \ar@{-}[d] \\
			v_2
		} \lhd v_3 &= \mytree{
			& \bullet \ar@{-}[ld] \ar@{-}[rd] \\
			v_1 \ar@{-}[d] && v_3 \\
			v_2
		} + \mytree{
			& \bullet \ar@{-}[d] \\
			& v_1 \ar@{-}[ld] \ar@{-}[rd] \\
			v_2 && v_3
		} + \mytree{
			\bullet \ar@{-}[d] \\
			v_1 \ar@{-}[d] \\
			v_2 \ar@{-}[d] \\
			v_3
		}
	\end{split}\end{equation} %}
	木の子供の並びの順序の違いを無視すると次のようになる。
	\begin{equation}\begin{split} %{
		(v\partial_x)^0 &\sim \bullet \\
		(v\partial_x)^1 &\sim \bullet\lhd v \\
		(v\partial_x)^2 &\sim (\bullet\lhd v)\lhd v \\
		(v\partial_x)^3 &\sim ((\bullet\lhd v)\lhd v)\lhd v \\
	\end{split}\end{equation} %}

	\begin{todo}[次の目標]\label{todo:次の目標} %{
		\begin{itemize} %{
			\item 頂点を付け足す操作を木の演算で書き直す。
		\end{itemize} %}
	\end{todo} %todo:次の目標}

	\subsection{保留}\label{s2:保留} %{
	微分係数が木で表すことをもう少し詳細に説明する。

	\begin{definition}[根付きの木]\label{def:根付きの木} %{
		根が一つ指定された木を根付きの木という。
	\end{definition} %def:根付きの木}

	組み合わせ論などでは、根が固定されていない木が用いられることが多いために、
	根付きの木という特別な名前が用いられる。根付きの木と根が固定されていない木
	の違いは同値関係の違いである。根付きの木に次のような同値関係をとったもの
	が根が固定されていない木となる。
	\begin{equation}\begin{split} %{
		\text{根付きでない木}\quad \mytree{
			&\bullet \ar@{-}[ld] \ar@{-}[rd] \\
			s_1 \ar@{-}[d] && s_2 \\
			s_3 \\
		} \sim \mytree{
			& s_1 \ar@{-}[ld] \ar@{-}[rd] \\
			s_3 && \bullet \ar@{-}[d] \\
			&& s_2 \\
		} \sim \mytree{
			s_3 \ar@{-}[d] \\
			s_1 \ar@{-}[d] \\
			\bullet \ar@{-}[d] \\
			s_2 \\
		} \sim \mytree{
			s_2 \ar@{-}[d] \\
			\bullet \ar@{-}[d] \\
			s_1 \ar@{-}[d] \\
			s_3 \\
		}
	\end{split}\end{equation} %}

	\begin{definition}[対称な木]\label{def:対称な木} %{
		子供の頂点の並びの順序を入れ替えたものを同値とみなす木を対称な木という。
	\end{definition} %def:対称な木}

	対称でない木に次のような同値関係をとったものが対称な木になる。
	\begin{equation}\begin{split} %{
		\text{対称な木}\quad \mytree{
			& \bullet \ar@{-}[dl] \ar@{-}[dr] \\
			*+[F]{t_1} && *+[F]{t_2} \\
		} \sim \mytree{
			& \bullet \ar@{-}[dl] \ar@{-}[dr] \\
			*+[F]{t_2} && *+[F]{t_1} \\
		} 
	\end{split}\end{equation} %}

	$S$を有限集合とする。すべての頂点を$S$の元とする根付きの木の集合を$T_+S$
	とする。根を$\bullet$、根以外の頂点を$S$の元とする根付きの木の集合を
	$T_\bullet S$とする。$T_+S$と$T_\bullet S$には空の木を含めないものとする。
	$T_\bullet S$の元を子供の頂点を括弧でくくって次のように書くことにする。
	\begin{equation}\begin{split} %{
		\bullet &= \bullet[] = \bullet = 1_\bullet \\
		\mytree {
			& \bullet \ar@{-}[dl]\ar@{-}[d]\ar@{-}[drr] \\
			*+[F]{t_1} & *+[F]{t_2} & \cdots & *+[F]{t_n} \\
		} &= \bullet[t_1t_2\cdots t_n] \quad\text{for all }t_1,t_2,\dots,t_n\in T_+S \\
	\end{split}\end{equation} %}
	$T_+S$の元を次のように書くことにする。
	\begin{equation}\begin{split} %{
		s &= s[] = s \quad\text{for all }s\in S \\
		\mytree {
			& s \ar@{-}[dl]\ar@{-}[d]\ar@{-}[drr] \\
			*+[F]{t_1} & *+[F]{t_2} & \cdots & *+[F]{t_n} \\
		} &= s[t_1t_2\cdots t_n] \quad\text{for all }s\in S,\;t_1,t_2,\dots,t_n\in T_+S \\
	\end{split}\end{equation} %}
	$T_\bullet S$は$T_+S$から生成された自由モノイドとしてみることができる。
	$T_\bullet S$に写像$m_*$を次のように定義する。
	\begin{equation}\begin{split} %{
		m_*: T_\bullet S\times  T_\bullet S &\to T_\bullet S \\
			\bullet[a_1a_2\cdots a_m]\times \bullet[b_1b_2\cdots b_n]
			&\mapsto \bullet[a_1a_2\cdots a_mb_1b_2\cdots b_n] \\
			&\text{for all }a_1,a_2,\dots,a_m,b_1,b_2,\dots,b_n\in T_+S \\
	\end{split}\end{equation} %}
	$m_*$は結合的になり、$1_\bullet$を単位元とする。$m_*$を中置記法で$*$と
	書くことにする。

	$R$を半環とする。$T_\bullet$を基底とする自由$R$半モジュールを
	$RT_\bullet S$とする。$T_\bullet S$の積$m_*$を$R$線形に拡張し、同じ記号
	$m_*:RT_\bullet S\otimes RT_\bullet S\to RT_\bullet S$を用いることにする。
	積$m_*$に双対となる余積$\Delta_*$を次のように定義する。
	\begin{equation}\begin{split} %{
		\Delta_*: RT_\bullet S &\to  RT_\bullet S\otimes RT_\bullet S \\
			\bullet[t] &\mapsto \bullet[t]\otimes 1_\bullet + 1_\bullet\otimes \bullet[t] \quad\text{for all }t\in T_+S \\
	\end{split}\end{equation} %}
	余積$\Delta_*$は積$m_*$と双対になるから、任意の$t_1,t_2,\dots,t_n\in T_+S$
	に対して次のようになる。
	\begin{equation}\begin{split} %{
		&\Delta_*[t_1t_2\cdots t_n]
			= (\Delta_*[t_1])*(\Delta_*[t_2])*\cdots*(\Delta_*[t_n]) \\
			&= 1_\bullet\otimes \bullet[t_1t_2\cdots t_n] \\
			&+ \bullet[t_1]\otimes \bullet[t_2t_3\cdots t_n] 
			+ \bullet[t_2]\otimes \bullet[t_1t_3\cdots t_n]
			+ \cdots
			+ \bullet[t_n]\otimes \bullet[t_1t_2t_3 \cdots t_{n-1}] \\
			&+ \bullet[t_1t_2]\otimes \bullet[t_3\cdots t_n] 
			+ \bullet[t_1t_3]\otimes \bullet[t_2\cdots t_n]
			+ \cdots
			+ \bullet[t_{n-1}t_n]\otimes \bullet[t_1t_2t_3 \cdots t_{n-2}] \\
			&+\cdots \\
			&+ \bullet[t_1t_2\cdots t_n]\otimes 1_\bullet \\
	\end{split}\end{equation} %}
	余積$\Delta_*$に対する余単位射$\epsilon_*$は次のようになる。
	\begin{equation}\begin{split} %{
		\epsilon_*: RT_\bullet &\to R \\
			t &\mapsto \begin{cases} %{
				1, &\text{ iff }t = 1_\bullet \\
				0, &\text{ otherwise } \\
			\end{cases} %}
	\end{split}\end{equation} %}

	$m_*$と$\Delta_*$のみでは、$T_+S$から生成された自由モノイド半双代数で
	しかない。$m_*$を深さ方向に拡張することを考える。$R$双線形写像$m_\odot$
	を次のように定義する。
	\begin{equation}\begin{split} %{
		m_\odot: RT_\bullet S\otimes RT_\bullet S &\to RT_\bullet S \\
			t_1\otimes t_2 &\mapsto m_*(t_1\otimes t_2) + m_{\widebar{\odot}}(t_1\otimes t_2) \quad\text{for all }t_1,t_2\in T_\bullet S \\
		m_{\widebar{\odot}}: RT_\bullet S\otimes RT_\bullet S &\to RT_\bullet S \\
			\bullet[t_1t_2\cdots t_n]\otimes t 
				&\mapsto m_*(t_1\otimes t_2) + m_{\widebar{\odot}}(t_1\otimes t_2) \quad\text{for all }t_1,t_2,\dots,t_n\in T_+S,\;t\in T_\bullet S \\
	\end{split}\end{equation} %}

	\begin{equation}\begin{split} %{
			[t_1t_2\cdots t_n] 
				&\mapsto t*[t_1t_2\cdots t_n] \\
				&\quad + [(t\odot t_1)t_2\cdots t_n] + [t_1(t\odot t_2)\cdots t_n] \\
				&\quad + \cdots + [t_1t_2\cdots(t\odot t_n)] \\
				&\quad\text{for all }t\in T_\bullet S,\; t_1,t_2,\dots,t_n\in T_+S \\
		m_\odot: RT_\bullet S\otimes T_+S &\to T_+S \\
			1_\bullet\otimes u &\mapsto u \quad\text{for all }u\in T_+S \\
			[t_1t_2\cdots t_{n+1}]\otimes u &\mapsto u \quad\text{for all }u\in T_+S \\
	\end{split}\end{equation} %}
	ここで、$m_\odot$を中置記法で$\odot$と書いた。

	すべての頂点を$S$の元とする対称な根付きの木の集合を
	$T_+S$と書く。$T_+S$には空の木を含めない。$T_+S$に空の木を含めたものを
	$TS$と書く。$TS$の元を次のように、子供の頂点を括弧でくくって書くことに
	する。
	\begin{equation}\begin{split} %{
		\text{空の木} &= 1_{TS}[] = 1_{TS} \\
		s &= s[] = s \quad\text{for all }s\in S \\
		\mytree {
			& s \ar@{-}[dl]\ar@{-}[d]\ar@{-}[drr] \\
			*+[F]{t_1} & *+[F]{t_2} & \cdots & *+[F]{t_n} \\
		} &= s[t_1t_2\cdots t_n] \quad\text{for all }s\in S,\; t_1,t_2,\dots,t_n\in T_+S \\
	\end{split}\end{equation} %}
	$R=(R,+,0,,\myspace,1)$を半環とする。$TS$を基底とする$R$係数半モジュール
	を$RTS$とする。$RTS$に双線形写像$m_\cdot:RT_+S\otimes RT_+S\to RT_+S$を
	次のように定義する。
	\begin{itemize} %{
		\item $m_\cdot$を中置記法で$\cdot$と書き、
		\item 任意の$t\in T_+S,\;s\in S$に対して
		\begin{equation}\begin{split} %{
			t\cdot s = s[t] \\
		\end{split}\end{equation} %}
		とし、
		\item 任意の$t,t_1,t_2,\dots,t_n\in T_+S,\;s\in S$に対して
		\begin{equation}\begin{split} %{
			t\cdot s[t_1t_2\cdots t_n]
				&= s[tt_1t_2\cdots t_n] + s[t_1tt_2\cdots t_n] 
				%&\quad + \cdots + s[t_1t_2\cdots tt_n] \\
				+ \cdots + s[t_1t_2\cdots tt_n] \\
		\end{split}\end{equation} %}
		とする。
	\end{itemize} %}

	根を$\bullet$、根以外の頂点を$S$の元とする対称な根付きの木の集合を$T_\bullet S$、
	また、$S_\bullet=\set{\bullet}\cup S$とする。

	\begin{definition}[ラベル付き根付き木 (labeled rooted tree)]\label{def:ラベル付き根付き木} %{
		$S$を集合とする。$S$の元を根以外の頂点とする木を$S$ラベル根付き木という。
	\end{definition} %def:ラベル付き根付き木}

	$S$ラベル根付き木の集合$TS$は次のようになる。
	\begin{equation}\begin{split} %{
		TS &= \Set{\mytree{
			\bullet
		}}\cup \Set{\mytree{
			\bullet \ar@{-}[d] \\
			s
		}}_{s\in S} \\
		&\quad\cup \Set{\mytree{
			&\bullet \ar@{-}[ld] \ar@{-}[rd] \\
			s_1 && s_2
		}}_{s_1,s_2\in S}\cup \Set{\mytree{
			\bullet \ar@{-}[d] \\
			s_1 \ar@{-}[d] \\
			s_2
		}}_{s_1,s_2\in S}\cup \cdots \\
	\end{split}\end{equation} %}
	根付き木というのは、根を固定しない木との対比で用いられる。
	根を固定しない木は、根付き木に対して次のような同値関係で剰余をとったもの
	である。
	木の絵を書くのはしんどいので、次のように行きがけ順に括弧を使って木を
	表現する。
	\begin{equation}\begin{split} %{
		\bullet = [],\; \mytree{
			\bullet \ar@{-}[d] \\
			s_1
		} = [s_1],\; \mytree{
			&\bullet \ar@{-}[ld] \ar@{-}[rd] \\
			s_1 && s_2
		} = [s_1s_2],\; \mytree{
			\bullet \ar@{-}[d] \\
			s_1 \ar@{-}[d] \\
			s_2
		} = [s_1[s_2]] \\
		\mytree{
			&\bullet \ar@{-}[ld] \ar@{-}[d] \ar@{-}[rd] \\
			s_1 & s_2 & s_3 \\
		} = [s_1s_2s_3],\; \mytree{
			&\bullet \ar@{-}[ld] \ar@{-}[rd] \\
			s_1 \ar@{-}[d] && s_2 \\
			s_3
		} = [s_1[s_3]s_2] \\
		\mytree{
			&\bullet \ar@{-}[ld] \ar@{-}[rd] \\
			s_1 && s_2 \ar@{-}[d] \\
			&& s_3
		} = [s_1s_2[s_3]] 
	\end{split}\end{equation} %}
	$S$を集合、$R$を半環とする。$S$ラベル付き根付き木$TS$を基底とする
	$R$係数半モジュールを$RTS$と書く。頂点数が$n+1$の$TS$の部分集合を
	$T_nS$と書き、$T_nS$で張られる$RTS$の部分空間を$RT_nS$と書く。
	線形写像$\phi$を次のように定義する。
	\begin{equation}\begin{split} %{
		\phi:RT_1S\otimes RT_1S &\to RT_2S \\
		[s_1]\otimes [s_2] &\mapsto [s_1s_2] + [s_1[s_2]] \quad\text{for all }s_1,s_2\in S
	\end{split}\end{equation} %}

	$WS$をから$S$から生成された自由半群とする。
	$WS$の元を$s_1,s_2,\dots,s_n\in S$に対して括弧でくくって
	$[s_1s_2\cdots s_n]$と書き、単語の連結を中置記法で$*$と書く。
	例えば次のようになる。
	\begin{equation*}\begin{split} %{
		[s_1s_2\cdots s_n] &= [s_1]*[s_2]*\cdots*[s_n]
	\end{split}\end{equation*} %}
	任意の$n\in \mybf{N}_+$に対して$W^{n+1}S$を$W^nS$から生成された自由半群
	とする。例えば、$WS,W^2S$は$S$の元を用いて次のように書くことができる。
	\begin{equation}\begin{split} %{
		WS &= \set{[s]}_{s\in S}\cup\set{[s_1s_2]}_{s_1,s_2\in S}\cup\cdots \\
		W^2S &= \set{[[s]]}_{s\in S}\cup\set{[[s_1][s_2]]}_{s_1,s_2\in S}\cup\set{[[s_1s_2]]}_{s_1,s_2\in S}\cup\cdots \\
	\end{split}\end{equation} %}
	$W^+S=\cup_{n=1}^\infty W^nS$とする。$W^+S$から生成された自由半群を
	$TS$と書く。$TS$に対しても$WS$と同様の表記をする。
	$TS$の元を$w_1,w_2,\dots,w_n\in W^+S$に対して括弧でくくって
	$[w_1w_2\cdots w_n]$と書き、単語の連結を中置記法で$*$と書く。
	例えば次のようになる。
	\begin{equation*}\begin{split} %{
		[w_1w_2\cdots w_n] &= [w_1]*[w_2]*\cdots*[w_n]
	\end{split}\end{equation*} %}
	$TS$に単位元$[]$を加えたものを$\mycal{T}S$と書く。

	$W^2S$を$WS$から生成された自由モノイドを同値関係$[[]]\sim[]$で剰余した
	ものとする。$W^2S$の単位元$1_*$以外の任意の元は、単位元以外の元
	$w_1,w_2,\dots,w_n\in WS$を用いて$[w_1w_2\cdots w_n]$と一意に書くことが
	できる。
	同様に$W^{n+1}S$を$W^nS$から生成された自由モノイドを同値関係$[[]]\sim[]$で
	剰余したものとする。
	\begin{formula}[Picardの再帰法]\label{formula:Picardの再帰法} %{
		式変形
		\begin{equation}\begin{split} %{
			\exp tx &= 1 + tx + \frac{t^2}{2!}x^2 + \cdots \\
				&= 1 + x(t + \frac{t^2}{2!}x + \cdots) \\
				&= 1 + x\int_0^t ds (\exp sx) \\
		\end{split}\end{equation} %}
		を用いて\eqref{eq:常微分方程式の形式解}を変形すると次のようになる。
		\begin{equation}\begin{split} %{
			e^{tv(x)\partial_x}x
			&= x + (vx)\partial_x\int_{0}^t dse^{sv(x)\partial_x}x \\
			&= x + \int_{0}^t dse^{sv(x)\partial_x}v(x) \\
			&= x + \int_{0}^t dsv(e^{sv(x)\partial_x}x) \\
		\end{split}\end{equation} %}
		$x_t=e^{tv(x)\partial_x}x$とおくと、更に次のように変形される。
		\begin{equation}\begin{split} %{
			x_t &= x + \int_0^t dsv(x_s) \\
		\end{split}\end{equation} %}
	\end{formula} %formula:Picardの再帰法}
	%s2:保留}
%s1:The Explicit Computation of Integration Algorithms and First Integrals for Ordinary Differential Equations With Polynomial Coefficients Using Trees}

\section{Bialgebra Deformations of Certain Universal Enveloping Algebras}\label{s1:Bialgebra Deformations of Certain Universal Enveloping Algebras} %{
	論文\cite{url:grossman:deform}に関するメモを書いておく。
	すべてを理解していないので、断片的なメモになっている。

	論文\cite{url:grossman:deform}の動機は常微分方程式の摂動による解法である。
	$x:\mybf{C}\to \mybf{C}^n,\;F_x:\mybf{C}^n\to \mybf{C}^n$として、常微分方程式
	\begin{equation}\begin{split} %{
		\partial_tx = F_xx,\quad x0 = x_0 \\
	\end{split}\end{equation} %}
	の形式解は$\exp(tF_x)x|_{x=x_0}$と書ける。次の級数を計算できれば、任意の初期条件
	に対して解を時刻の摂動として求めることができる。
	\begin{equation}\begin{split} %{
		\exp(tF_x)x &= x + tF_xx + \frac{t^2}{2}F_xF_xx + \cdots \\
	\end{split}\end{equation} %}

	\begin{itemize} %{
		\item 数値計算の方法を見つける事は、代数の変形を見つけることである。
		\item 時刻の導入 \\
		体$k$係数モジュール$A$が与えられたとき、$A$を係数とする文字$t$の多項式
		$A_t=A[t]$は$k$係数モジュールとなる。$A$が半環であれば、$A_t$は$A$係数
		半モジュールとしてみることができるが、ここではそうみない。
		\item テンソル積の完備化 \\
		テンソル積$A_t\otimes_kA_t$に次の同値関係を入れたものを完備化された
		テンソル積$A_t\widehat{\otimes}_kA_t$と定義している。
		\begin{equation}\begin{split} %{
			a_1t^{m+1}\otimes_ka_2^n &\sim a_1t^m\otimes_ka_2^{n+1} \\
			a_1t^m\otimes_ka_2^{n+1} &\sim a_1t^(m+1)\otimes_ka_2^n \\
			&\quad\text{for all }m,n\in\mybf{N},\;a_1,a_2\in A \\
		\end{split}\end{equation} %}
		\item べき乗系列 \\
		$C_0=\set{c_0,c_1,c_2,\dots}$を余代数$C=(C,\Delta,\epsilon)$の部分集合
		とする。$\Delta c_m=\sum_{p=0}^mc_p\otimes c_{m-p}$となるとき、$C_0$を
		べき乗系列ということにする。
		\item 素な元 \\
		$A=(A,m,u,\Delta,\epsilon)$を双代数とする。元$a\in A$が
		\begin{itemize} %{
			\item $a=m(a_1\otimes  a_2)$となる二つの元は
			$a_1\otimes a_2=a\otimes 1_A$ または$a_1\otimes a_2=1_A\otimes a$
			に限られ、
			\item $\Delta a=a\otimes 1_A+1\otimes a$となる
		\end{itemize} %}
		とき、$a$を素な元ということにする。
		\item 群的な元 \\
		$A=(A,m,u,\Delta,\epsilon)$を双代数とする。元$a\in A$が
		$\Delta a=a\otimes a$かつ$\epsilon a=1$となるとき、$a$を群的な元と
		言うことにする。
		\item ほぼ素な元 \\
		$A=(A,m,u,\Delta,\epsilon)$を双代数とする。$K,H\in A$を群的な元とする。
		元$a\in A$が
		\begin{itemize} %{
			\item $a=m(a_1\otimes  a_2)$となる二つの元は
			$a_1\otimes a_2=a\otimes 1_A$ または$a_1\otimes a_2=1_A\otimes a$
			に限られ、
			\item $\Delta a=a\otimes K+H\otimes a$となる
		\end{itemize} %}
		とき、$a$をほぼ素な元ということにする。素な元$\implies$ほぼ素な元になる。
		\item 体の標数が$0$であれば、任意の$a\in A$に対して$\exp(a)$が定義できて、
		$\exp(at)$は群的な元になる。
	\end{itemize} %}

	双代数における素な元は、Lie環論で使われる概念らしい。

	\begin{definition}[双半代数における素な元]\label{def:双半代数における素な元} %{
		$R$を半環、$A=(A,m,1_A,\Delta,\epsilon)$を$R$係数の双半代数とする。
		元$a\in A$が$\Delta a=a\otimes 1_A+1_A\otimes a$となるとき、元$a$を
		素な元という。
	\end{definition} %def:双半代数における素な元}

	$R$を半環、$A=(A,m,1_A,\Delta,\epsilon)$を$R$係数の双半代数とする。
	積$m(a_1\otimes a_2)$を単に$a_1a_2$と書く。
	$a\in A$を素な元とすると、$m$と$\Delta$は双対だから次のようになる。
	\begin{equation}\begin{split} %{
		\Delta a^m &= (a\otimes 1_A+1_A\otimes a)^m \\
			&= \sum_{p=0}^m\binom{n}{p}a^p\otimes a^{m-p} \\
	\end{split}\end{equation} %}
	したがって、$\set{\frac{a^m}{m!}}_{m=0}^\infty$は$A$のべき乗系列となる。

	素な元がLie環論で使われる理由は、素な元$a_1,a_2\in A$のLie括弧
	$[a_1,a_2]\in A$もまた素な元になるためである。
	\begin{equation}\begin{split} %{
		\Delta[a_1,a_2] &= \Delta(a_1a_2-a_2a_1) \\
			&= (\Delta a_1)(\Delta a_2) - (\Delta a_2)(\Delta a_1) \\
			&= (a_1a_2\otimes 1_A + a_1\otimes a_2 + a_2\otimes a_1 + 1_A\otimes a_1a_2 ) \\
			&\quad - (a_2a_1\otimes 1_A + a_1\otimes a_2 + a_2\otimes a_1 + 1_A\otimes a_2a_1 ) \\
			&= [a_1,a_2]\otimes 1_A + 1_A\otimes [a_1,a_2] \\
	\end{split}\end{equation} %}
%s1:Bialgebra Deformations of Certain Universal Enveloping Algebras}

\section{立体視}\label{s1:立体視} %{
	立体視を実現させるための仕組みを列挙する。ここで、立体視とは、人間の左目と右目
	に異なる絵を受像させることである。一番簡単な方法は、ヘッドマウントディスプレイ
	を使って、右目用と左目用の絵を映し出すことである。シャッターグラスを用いた
	方法は、ヘッドマウントディスプレイに近い方法である。
	\begin{equation}\xymatrix{
		\txt{左目用の絵} \ar[rrr] &&& \txt{左目用の絵} \\
		\txt{右目用の絵} \ar[rrr] &&& \txt{右目用の絵} \\
	}\end{equation}
	その他の方法として、左目用と右目用の絵を合成して送信し、それを受信する側で
	分離する方法がある。
	\begin{equation}\xymatrix{
		\txt{左目用の絵} \ar[rd] &&& \txt{左目用の絵} \\
		& \txt{合成} \ar[r] & \txt{分離} \ar[ru] \ar[rd] \\
		\txt{右目用の絵} \ar[ru] &&& \txt{右目用の絵} \\
	}\end{equation}
	絵を合成/分離する方法としては、光の偏向を利用した方法と光の波長を利用した方法
	がある。どちらの方法も、絵を分離するために特殊なメガネを利用する。
	光の偏向を利用する方法では、鉛直方向の偏向のみを通す物質を左目のレンズに、
	水平方向の偏向のみを通す物質を右目のレンズに使うメガネを用いる。
	光の波長を利用する方法では、赤色の光のみを通す物質を左目のレンズに、
	青色の光のみを通す物質を右目のレンズに使うメガネを用いる。
%s1:立体視}

\section{DFAとキャンセル可能性}\label{s1:DFAとキャンセル可能性} %{
	次のDFAを考える。
	\begin{equation}\xymatrix{
		& c \ar[rd]^c \\
		ac+bc \ar[ru]^a \ar[rd]_b & & 1 \\
		& c \ar[ru]_c \\
	}\end{equation}
	これを半群の言葉でみると、左キャンセル可能な表現と思える。
	このDFAを最小化すると次のようになる。
	\begin{equation}\xymatrix{
		ac+bc \ar[r]^{a+b} & c \ar[r]^c & 1 \\
	}\end{equation}
	これを半群の言葉でみると、左右キャンセル可能な表現と思える。
	最小化したDFAを次のように書き直してみる。
	\begin{equation}\xymatrix{
		(a+b)c \ar[r]^{a+b} & c \ar[r]^c & 1 \ar@(u,u)[ll]_{(a+b)c} \\
	}\end{equation}
	これは、有限なキャンセル可能な半群は群になることを示していると思える。
	オートマトンの言葉と半群の言葉を表の形で対応させると次のようになりそうだ。
	\begin{tabular}{cc}
		オートマトン & 半群 \\ \hline
		有限オートマトン & 半群の有限次元表現 \\
		左DFA & 左キャンセル可能 $\implies$ 右単位元が存在 \\
		左右DFA & 左右キャンセル可能 $\implies$ 左右単位元が存在 \\ \hline
	\end{tabular}

	この対応表が正しいならば、正規表現は有限オートマトンによる表現を持つから、
	正規表現は群となる表現をもつことになる。
	ただし、半群ではなく加法をもったモノイド半環の線形結合で表される。
	図でいうと、$\set{1,a+b),c}$である。
%s1:DFAとキャンセル可能性}

\section{自然数から01へ}\label{s1:自然数から01へ} %{
	自然数から$\mybf{2}$への加法を保つ射影を考える。
	次の二つの射影が簡単に思いつく。
	\begin{itemize}
		\item OR-射影 \\
		$0$を、$1$以上の自然数を$1$に射影する。
		射影から誘導される加法はブーリアンのORとなる。
		\item XOR-射影 \\
		偶数を$0$、奇数を$1$に射影する。
		射影から誘導される加法はブーリアンのXORとなる。
	\end{itemize}
%s1:自然数から01へ}

\section{ゲーデル関数の例}\label{s1:ゲーデル関数の例} %{
	例として、自然数$\mybf{N}$とその直積$\mybf{N}^2$との次の対応を考える。
	\begin{equation*}\begin{split} %{
		f:\mybf{N}^2 &\to \mybf{N} \\
			m\times n &\mapsto \binom{m+n+1}{2}+n = \frac{(m+n+1)(m+n)}{2}+n \\
	\end{split}\end{equation*} %}
	\begin{equation*}\begin{split} %{
		\bordermatrix {
			  & 0 & 1 & 2 & 3 & \cdots \\
			0 & 0 & 2 & 5 & 9 & \\
			1 & 1 & 4 & 8 & 13 & \\
			2 & 3 & 7 & 12 & 18 & \\
			3 & 6 & 11 & 17 & 24 & \\
			\vdots & & & & & \\
		}
	\end{split}\end{equation*} %}
	\begin{cprog}
		f<-function(m,n) (m+n)*(m+n+1)/2+n
		x<-sapply(0:3,f,n=0:3);
		t(x);
	\end{cprog}
	$f$は集合同型となる。
%s1:ゲーデル関数の例}

\section{双半群の使い方}\label{s1:双半群の使い方} %{
	数学の重要な道具として準同型がある。準同型は、
	\begin{itemize}
		\item 準同型によって演算結果の対応関係がつけられて、
		\item 準同型の合成は準同型になる
	\end{itemize}
	というものである。
	\begin{equation*}\xymatrix{
		A\times A \ar[d]^{m_A} \ar[r]^{f\times f} & B\times B \ar[d]^{m_B} \ar[r]^{g\times g} & C\times C \ar[d]^{m_C} \\
		A \ar[r]^{f} & B \ar[r]^{g} & C \\
	}\end{equation*}
	この性質を満たすように準同型を拡張してみる。対角的な直積$f\times f$の
	代わりに余積を用いて次のような可換図を考える。
	\begin{equation*}\xymatrix{
		A\times A \ar[d]^{m_A} \ar[r]^{\Delta f} & B\times B \ar[d]^{m_B} \ar[r]^{\Delta g} & C\times C \ar[d]^{m_C} \\
		A \ar[r]^{f} & B \ar[r]^{g} & C \\
	}\end{equation*}
	写像の合成を$\circ$と書くと、合成が成り立つための条件は次のようになる。
	\begin{equation*}\begin{split} %{
		\Delta(g\circ f) &= (\Delta g)\circ(\Delta f) \\
	\end{split}\end{equation*} %}
	この式は、$\circ$を積、$\Delta$を余積とする双対の関係に他ならない。
	$\Delta$を群的な余積とした場合が準同型となる。
%s1:双半群の使い方}

\section{パーサーの方針}\label{s1:パーサーの方針} %{ 
	最初の問題設定として、文字列が与えられたパターンに合致するかどうかを
	判定する問題を考える。文字列の集合として、有限集合$A$から生成された
	自由モノイド$A^*$を考える。$A^*$からブーリアン、自然数、複素数などの
	可換半環$B$への写像全体$B^A$を考える。$B^A$には$B$の代数構造を反映した
	代数構造を定めることができて、$B^A$は半環となる。
	パターンを与えるということは$B^A$の元を一つ定めることになるだろう。
	そして、任意の単語$w_1,w_2$に対して、$f(w_1*w_2)=(\Delta f)(w_1\times w_2)$
	で$B^A$に余積$\Delta$を定めることができれば、最終的には一文字のマッチング
	にまで帰着させることができる。
	そして、$a\rhd f=(\Delta f)(a\times-)$で$A^*$の$B^A$への作用を定義すると、
	モノイド$A$の$A\rhd f\subseteq A^B$への表現を得ることができる。
	この表現が言語理論でのオートマトンと呼ばれているものになると思われる。
	また、正規表現に対するBrzozowski微分$D$は、$\Delta f=1\otimes f + Df$
	になるのではないかと思っている。

	半単純リー環の場合、表現を用いた分類(A,B,...)がされている。
	オートマトンでも半単純に相当する性質が定義できて、その分類ができればうれしい。
%s1:パーサーの方針}

\section{分配性}\label{s2:分配性} %{
	分配性が成り立つような乗法の定義の仕方を考える。$A=(A,+,0)$を可換モノイドとする。
	$A$から$A$への写像全体を$MA$と書く。$MA$は、写像の合成を積、恒等写像$1_{MA}$
	を単位元とするモノイドになる。写像の合成の記号は省略する。さらに、次のようにして
	$MA$に積$+$を定義することができる。
	\begin{equation*}\begin{split} %{
		(f+g)a &= fa + ga \\
	\end{split}\end{equation*} %}
	$a\in A$への恒等写像を$a^*$とする。特に、$0^*$は積$+$の単位元となる。
	$MA$の中で半群準同型を満たす元の集まりを$HA$と書く。
	\begin{equation*}\begin{split} %{
		HA &= \set{f\in MA\bou f \text{ satisfies the below condition}} \\
			& f(a_1+a_2) = fa_1 + fa_2 \text{ for all }a_1,a_2\in A \\
	\end{split}\end{equation*} %}
	$HA$は写像の合成と$+$で閉じている。
	\begin{equation*}\begin{split} %{
		fg(a_1+a_2) &= f(ga_1+ga_2) \\
			&= fga_1 + fga_2 \\
		(f+g)(a_1+a_2) &= f(a_1+a_2) + g(a_1+a_2)  = fa_1 + fa_2 + ga_1 + ga_2 \\
			& = (f+g)a_1 + (f+g)a_2 \\
	\end{split}\end{equation*} %}
	さらに、$HA$は写像の合成と$+$で分配性を持つ。
	\begin{equation*}\begin{split} %{
		f(g+h)a &= fga + fha = (fg+fh)a \\
		(f+g)ha &= fha + gha = (fh+gh)a \\
	\end{split}\end{equation*} %}

	以上の事柄をブーリアン$\mybf{B}=(\set{0,1},+,0)$で確かめてみる。
	可換な積$+$は論理和で定義する。
	\begin{equation*}\begin{split} %{
		+: \mybf{B}\times \mybf{B} &\to \mybf{B} \\
			b_1\times b_2 &\mapsto \begin{cases}
				0, &\text{ iff } b_1 = b_2 = 0 \\
				1, &\text{ otherwise } \\
			\end{cases}
	\end{split}\end{equation*} %}
	からへの写像全体$M\mybf{B}$は$M\mybf{B}=\set{0^*,1^*,\myid,\neg}$となる。$0^*,1^*$
	はそれぞれ$0,1$への定数写像、$\myid$は恒等写像、$\neg$は次のように定義する。
	\begin{equation*}\begin{split} %{
		\neg \begin{pmatrix}
			0 \\
			1 \\
		\end{pmatrix} &= \begin{pmatrix}
			1 \\
			0 \\
		\end{pmatrix} \\
	\end{split}\end{equation*} %}
	群表は次のようになる。
	\begin{equation*}\begin{split} %{
		\bordermatrix {
			+ & 0^* & 1^* & \neg & \myid \\
			0^* & 0^* & 1^* & \neg & \myid \\
			1^* & 1^* & 1^* & 1^* & 1^* \\
			\neg & \neg & 1^* & \neg & 1^* \\
			\myid & \myid & 1^* & 1^* & \myid \\
		} & \quad \bordermatrix {
			\text{合成} & 0^* & 1^* & \neg & \myid \\
			0^* & 0^* & 0^* & 0^* & 0^* \\
			1^* & 1^* & 1^* & 1^* & 1^* \\
			\neg & 1^* & 0^* & \myid & \neg \\
			\myid & 0^* & 1^* & \neg & \myid \\
		} \\
	\end{split}\end{equation*} %}
	$\neg$は次のとおり準同型とならない。
	\begin{equation*}\begin{split} %{
		\neg(0+1) = 0 \neq 1 = \neg0 + \neg1 \\
	\end{split}\end{equation*} %}
	$\neg$以外は準同型となるから、$M\mybf{B}$の中で準同型となるものの部分集合
	$H\mybf{B}$は$H\mybf{B}=\set{0^*,1^*,\myid}$となる。$H\mybf{B}$の群表は次の
	ようになる。
	\begin{equation}\label{eq:論理和の分配的な双対空間}\begin{split} %{
		\bordermatrix {
			+ & 0^* & 1^* & \myid \\
			0^* & 0^* & 1^* & \myid \\
			1^* & 1^* & 1^* & 1^* \\
			\myid & \myid & 1^* & \myid \\
		} & \quad \bordermatrix {
			\text{合成} & 0^* & 1^* & \myid \\
			0^* & 0^* & 0^* & 0^* \\
			1^* & 1^* & 1^* & 1^* \\
			\myid & 0^* & 1^* & \myid \\
		} \\
	\end{split}\end{equation} %}
	計算してみると、$H\mybf{B}$は分配性を持つことがわかる。以上より、上で述べた
	ことが確かめられる。

	式を見ると、部分集合$H_0\mybf{B}=\set{0^*,\myid}$は写像の合成と$+$について
	閉じているのがわかる。さらに、$H_0\mybf{B}$では次の事柄が成り立っている。
	\begin{itemize}
		\item 写像の合成が可換になる。
		\item 任意の$f\in H_0\mybf{B}$に対して、$0^*f=0^*=f0^*$が成り立つ。
		\item $+$に関して、$\mybf{B}$と同型$\varphi$になる。
		\begin{equation*}\begin{split} %{
			\varphi: \mybf{B} &\to H_0\mybf{B} \\
			\begin{pmatrix} 
				0 \\
				1 \\
			\end{pmatrix}
			&\mapsto \begin{pmatrix}
				0^* \\
				\myid \\
			\end{pmatrix}
		\end{split}\end{equation*} %}
	\end{itemize}
	$H\mybf{B}$は、$1^*0^*\neq0^*$なので、半環ではないが、$H_0\mybf{B}$は半環となる。
	したがって、$+$に関する$\mybf{B}$と$H_0\mybf{B}$のモノイド同型によって、
	$\mybf{B}$に乗法が定義される。

	\begin{problem}[分配的になる積に対する条件]\label{prob:分配的になる積に対する条件} %{ 
	$H\mybf{B}$にどのような条件を課すと$H_0\mybf{B}$が得られるのだろうか。
	また、ここで観察した$\mybf{2}$の論理和の場合は、任意の可換半群から半環を
	得ることに一般化できるだろうか。
	\end{problem} %prob:分配的になる積に対する条件}

	一般の場合に戻って問題を考えてみる。次の方法が考えられる。
	\begin{itemize}
		\item 逐次的な方法 \\
		$\set{0^*,1_M}\subseteq HA$の群表は次のようになる。
		\begin{equation}\begin{split} %{
			\bordermatrix {
				+ & 0^* & 1_M \\
				0^* & 0^* & 1_M \\
				1_M & 1_M & ? \\
			} & \quad \bordermatrix {
				\text{合成} & 0^* & 1_M \\
				0^* & 0^* & 0^* \\
				1_M & 0^* & 1_M \\
			} \\
		\end{split}\end{equation} %}
		$1_M+1_M$のところが未定である。任意の$a\in A$に対して、$(1_M+1_M)a=a+a$である。
		したがって、次の場合があり得る。
		\begin{itemize}
			\item $1_M+1_M=0^*$の場合は、任意の$a\in A$に対して、$0=a+a$となる。
			ブール代数でのXORなどが相当する。
			\item $1_M+1_M=1_M$の場合は、任意の$a\in A$に対して、$a=a+a$となる。
			ブール代数でのORなどが相当する。
			\item それ以外の場合は、$f_2=1_M+1_M\in HA$とおく。$f_2$は準同型となり、
			$0$を固定点に持つ。$\set{0^*,1_M,f_2}\subseteq HA$の群表は次のようになる。
			\begin{equation}\begin{split} %{
				\bordermatrix {
					+ & 0^* & 1_M & f_2 \\
					0^* & 0^* & 1_M & 0^* \\
					1_M & 1_M & f_2 & ? \\
					f_2 & 0^* & ? & 1_M + ? \\
				} & \quad \bordermatrix {
					\text{合成} & 0^* & 1_M & f_2 \\
					0^* & 0^* & 0^* & 0^* \\
					1_M & 0^* & 1_M & f_2 \\
					f_2 & 0^* & f_2 & 1_M + ? \\
				} \\
			\end{split}\end{equation} %}
			$f_2+1_M$と$f_2f_2=f_2+f_2$のところが未定である。
			こうやって、群表を埋めていった結果と$A$がモノイド同型になることを示せればよい。
			この方法で、$A$が可算集合の場合には構成的に乗法が定義できるように思える。
			\item 天下り的な方法 \\
			$HA$のなかで$0$を固定点に持つ元全体$H_0A$は$+$と写像の合成に関して閉じているので、
			半環となる。$A$と$H_0A$の$+$に関するモノイド同型が得られれば話が早い。
		\end{itemize}
	\end{itemize}
%s2:分配性}
