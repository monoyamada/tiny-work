\section{スターリング数}\label{s1:スターリング数} %{
	数演算子$a_+a_-$のべき乗を正規積の和に書き直すときに、第二種スターリング数
	$S_k^n$が現れる。
	\begin{equation}\label{eq:第二種スターリング数}\begin{split} %{
		(a_+a_-)^n = \sum_{k\in1..n} S_k^n a_+^ka_-^k
	\end{split}\end{equation} %}
	逆に、正規積をべき乗の和に書き直すときに、第一種スターリング数$s_k^n$が
	現れる。
	\begin{equation*}\begin{split} %{
		a_+^na_-^n = \sum_{k\in1..n} s_k^n (a_+a_-)^k
	\end{split}\end{equation*} %}
	また、第二種スターリング数を拡張した数列$S_p(n,k)$を次のように定めることが
	できる。
	\begin{equation}\label{eq:第二種拡張スターリング数}\begin{split} %{
		(a_+^pa_-)^n = \sum_{k\in1..n} S_p^(n,k) a_+^{k+(p-1)}a_-^k
	\end{split}\end{equation} %}
	同様に、第一種スターリング数を拡張した数列$s_p(n,k)$を次のように定めること
	ができる。
	\begin{equation*}\begin{split} %{
		a_+^{pn}a_-^n = \sum_{k\in1..n} s_p^(n,k) a_+^{(p-1)(n-k)}(a_+^pa_-)^k
	\end{split}\end{equation*} %}
	このように拡張したスターリング数$S_p(n,k),\;s_p(n,k)$の性質を求めることが
	この節の目標である。

	スターリング数は様々な一般化がある\cite{hsu.stirling}。
	したがって、第二種スターリング数の拡張\eqref{eq:第二種拡張スターリング数}
	を'一般化スターリング数'という文献があるが、この節では
	'第二種スターリング数$S_p$'等と書き、特に名前をつけないことにする。

	この節では次のような記号を用いることにする。
	\begin{description}\setlength{\itemsep}{-1mm} %{
		\item[昇降演算子] 消滅演算子を$a_-$、生成演算子を$a_+$と書く。
		\begin{equation*}\begin{split} %{
			[a_-,a_+] = 1
		\end{split}\end{equation*} %}
		また、共形場理論にならって$l_m$と書くことにする。
		\begin{equation*}\begin{split} %{
			l_m &= a_+^{m+1}a_- \\
		\end{split}\end{equation*} %}
		$l_m$の積は次のようになる。
		\begin{equation*}\begin{split} %{
			l_ml_n &= nl_{m+n} + l_m*l_n
		\end{split}\end{equation*} %}
		エネルギーテンソル$t_z$も同様に定義しておく。
		\begin{equation*}\begin{split} %{
			t_z &= z^{-2}\sum_{m\in\sizen}l_mz^{-m} \\
		\end{split}\end{equation*} %}
		ここで定義した$t_z$は共形場と異なり、振動モードを独立に扱っていない
		ので、$l_m$の添字の範囲は自然数となり、中心拡大も存在しない。
		\item[正規積] 生成消滅演算子の通常の積を記号なしまたは$m_\myspace$と
		書き、正規積を$*$または$m_*$と書く。これらの記号を用いると、
		第二種スターリング数の現れる式\ref{eq:第二種拡張スターリング数}は
		次のようになる。
		\begin{equation*}\begin{split} %{
			l_m^n = \sum_{k\in1..n} S_{m+1}(n,k) a_+^ml_0^{*k}
			\quad\text{for all }m,n\in\sizen
		\end{split}\end{equation*} %}
	\end{description} %}
%s1:スターリング数}
