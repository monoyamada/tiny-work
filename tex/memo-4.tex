\begingroup %{
	\newcommand{\sosei}{\ensuremath{\mycal{C}}}
	\newcommand{\bunkatu}{\ensuremath{\mycal{P}}}
	\newcommand{\grow}{\ensuremath{\gamma}}
	%
\section{スターリング数}\label{s1:スターリング数} %{
	数演算子$a_+a_-$のべき乗を正規積の和に書き直すときに、第二種スターリング数
	$S_k^n$が現れる。
	\begin{equation}\label{eq:第二種スターリング数}\begin{split} %{
		(a_+a_-)^n = \sum_{k\in1..n} S_k^n a_+^ka_-^k
	\end{split}\end{equation} %}
	逆に、正規積をべき乗の和に書き直すときに、第一種スターリング数$s_k^n$が
	現れる。
	\begin{equation*}\begin{split} %{
		a_+^na_-^n = \sum_{k\in1..n} s_k^n (a_+a_-)^k
	\end{split}\end{equation*} %}
	また、第二種スターリング数を拡張した数列$S_p(n,k)$を次のように定めることが
	できる。
	\begin{equation}\label{eq:第二種拡張スターリング数}\begin{split} %{
		(a_+^pa_-)^n = \sum_{k\in1..n} S_p(n,k) a_+^{k+(p-1)}a_-^k
	\end{split}\end{equation} %}
	同様に、第一種スターリング数を拡張した数列$s_p(n,k)$を次のように定めること
	ができる。
	\begin{equation*}\begin{split} %{
		a_+^{pn}a_-^n = \sum_{k\in1..n} s_p^(n,k) a_+^{(p-1)(n-k)}(a_+^pa_-)^k
	\end{split}\end{equation*} %}
	このように拡張したスターリング数$S_p(n,k),\;s_p(n,k)$の性質を求めることが
	この節の目標である。

	スターリング数は様々な一般化\cite{hsu.stirling}がある。
	したがって、第二種スターリング数の拡張\eqref{eq:第二種拡張スターリング数}
	を'一般化スターリング数'という文献があるが、この節では
	'第二種スターリング数$S_p$'等と書き、特に名前をつけないことにする。

	この節では次のような記号を用いることにする。
	\begin{description}\setlength{\itemsep}{-1mm} %{
		\item[フォック空間] 消滅演算子を$a_-$、生成演算子を$a_+$と書く。
		\begin{equation*}\begin{split} %{
			[a_-,a_+] = 1
		\end{split}\end{equation*} %}
		また、共形場理論にならって$l_m$と書くことにする。
		\begin{equation*}\begin{split} %{
			l_m &= a_+^{m+1}a_- \\
		\end{split}\end{equation*} %}
		$l_0$を粒子数作用素といい、$n$粒子数状態を$\ket{n}$と書く。
		\begin{equation*}\begin{split} %{
			l_0\ket{n} &= n\ket{n}\quad\text{for all }n\in\sizen
		\end{split}\end{equation*} %}
		$\set{\ket{n}}_{n\in\sizen}$で張られる複素ベクトル空間を$\mycal{H}$、
		$\mycal{H}$の内積に関する双対を$\mycal{H}^\dag$、
		$\mycal{H}$の自己線形写像全体を$L\mycal{H}$と書く。
		$L\mycal{H}$の部分空間で$\set{a_+^ma_-^n}_{m,n\in\sizen}$で張られる
		ベクトル空間を$A\mycal{H}$と書く\footnote{
			$L\mycal{H}=A\mycal{H}$となると思えるが証明できない。
			少なくともフォック空間が有限次元の場合
			\begin{equation*}\begin{split} %{
				a_+\ket{N} = 0 \quad\text{for some }N\in\sizen
			\end{split}\end{equation*} %}
			$L\mycal{H}=A\mycal{H}$が成り立つ。
		}	。
		\item[正規積] 生成消滅演算子の通常の積を$m_\myspace$と書き、正規積を
		$m_*$と書く。$l_m$の積は次のようになる。
		\begin{equation}\label{eq:lの積}\begin{split} %{
			l_ml_n &= (n+1)l_{m+n} + l_m*l_n
		\end{split}\end{equation} %}
		共形場理論にならってこの関係をVirasoro代数ということにする。
		$l_m$の通常の積と正規積の間には次の擬似Leibniz則が成り立つ。
		\begin{equation}\label{eq:lのライプニッツ則}\begin{split} %{
			l_p(l_m*l_n) &= (m+1)l_{m+p}*l_n + (n+1)l_m*l_{n+p} + l_p*l_m*l_n
		\end{split}\end{equation} %}
		正規積の部分が通常のLeibniz則からのズレになっている。
		\begin{equation*}\begin{split} %{
			l_p(l_m*l_n) &= (l_pl_m)*l_n + l_m*(l_pl_n) - l_p*l_m*l_n
		\end{split}\end{equation*} %}
		%
		\item[数の組成] 正の自然数$n\in\sizen_+$を$n_1+n_2+\cdots+n_k$となる
		$k$個の正の自然数$n_1,n_2,\dots,n_k\in\sizen_+$の組で表すこと
		を数$n$の$k$組成といい、数$n$の$k$組成の集合を$\sosei_k^n$と書く。
		また、数$n$のすべての組成を$\sosei_+^n=\oplus_{k=1}^n\sosei_k^n$
		と書き、数のすべての組成を
		$\sosei_+^+=\oplus_{n\in\sizen_+}\sosei_+^n$と書く。
		%
		\item[数の分割] 正の自然数$n\in\sizen_+$を$n_1+n_2+\cdots+n_k$となる
		$k$個の正の自然数$n_1\ge n_2\ge\cdots\ge n_k\in\sizen_+$の組で表すこと
		を数$n$の$k$分割といい、数$n$の$k$分割全体の集合を$\bunkatu_k^n$と
		書く。
		また、数$n$のすべての分割を$\bunkatu_+^n=\oplus_{k=1}^n\bunkatu_k^n$
		と書き、数のすべての分割を
		$\bunkatu_+^+=\oplus_{n\in\sizen_+}\bunkatu_+^n$と書く。
		組成と分割の違いは、組成は$(n_1,n_2,\dots,n_k)$の数字の並びの順序を
		考慮したもので、分割は数字の並びの順序を考慮しないものである。
	\end{description} %}

	第二種スターリング$S_p(n,k)$\eqref{eq:第二種スターリング数}が満たす
	漸化式を求める。低次の項を見ると次のようになっている。
	\begin{equation*}\begin{split} %{
		l_p &= a_+^{p+1}a_- \\
		l_p^2 &= a_+^{2p+2}a_-^2 + (p+1)a_+^{2p+1}a_- \\
		l_p^3 &= a_+^{3p+3}a_-^3 + (3p+3)a_+^{3p+2}a_-^2 
			+ (p+1)(2p+1)a_+^{3p+1}a_- \\
	\end{split}\end{equation*} %}
	したがって、自然数$\sigma_p(n,k)$を用いて次のように書けると仮定する。
	\begin{equation}\label{eq:シグマの定義}\begin{split} %{
		l_p^n=\sum_{k=1}^n\sigma_p(n,k)a_+^{np+k}a_-^k
	\end{split}\end{equation} %}
	すると、次のようになり、
	\begin{equation*}\begin{split} %{
		l_p^{n+1} &= \sum_{k=1}^n\sigma_p(n,k)l_pa_+^{np+k}a_-^k \\
		&= \sum_{k=1}^n\sigma_p(n,k)\bigl(a_+^{(n+1)p+k+1}a_-^{k+1}
			+ (np+k)a_+^{(n+1)p+k}a_-^k\bigr) \\
		&= \sum_{k=2}^{n+1}\sigma_p(n,k-1)a_+^{(n+1)p+k}a_-^k
			+ \sum_{k=1}^n(np+k)\sigma_p(n,k)a_+^{(n+1)p+k}a_-^k \\
	\end{split}\end{equation*} %}
	仮定が成り立つことがわかり、$\sigma_p(n,k)$に対する次の漸化式が導かれる。
	\begin{equation}\label{eq:シグマの漸化式}\begin{split} %{
		\sigma_p(n+1,k) = \sigma_p(n, k-1) + (np+k)\sigma_p(n,k)
		\quad\text{for all }n,k\in\sizen
	\end{split}\end{equation} %}
	この漸化式の境界条件は次のようになっている。
	\begin{equation}\label{eq:シグマの境界条件}\begin{split} %{
		\sigma_p(n,n+1) = 0,\quad \sigma_p(n,0) = 0
		\quad\text{for all }n\in\sizen
	\end{split}\end{equation} %}
	式\eqref{eq:シグマの定義}から$\sigma_p(n,k)$は次のように与えられる。
	\begin{equation*}\begin{split} %{
		\sigma_p(n,k) &= \frac{1}{k!}\bra{np+k}l_p^n\ket{k}
		\quad\text{for all }k\le n\in\sizen
	\end{split}\end{equation*} %}
	ここで、任意の$p\in\sizen$に対して$\sigma_p(0,0)=1$としている。

	漸化式\eqref{eq:シグマの漸化式}を縦軸に$n-k$、横軸に$k$をとった次の
	格子上の経路で書き表す。
	\begin{equation*}\xymatrix@R=3ex@C=3ex{
		\sigma_p(0,0)\ar[r]^{1} & \sigma_p(1,1)\ar[r]^{1}\ar[d]^{p+1}
			& \sigma_p(2,2)\ar[r]^{1}\ar[d]^{2p+2} 
			& \sigma_p(3,3)\ar[r]^{1}\ar[d]^{3p+3} & \cdots \\
		& \sigma_p(2,1)\ar[r]^{1}\ar[d]^{2p+1} 
			& \sigma_p(3,2)\ar[r]^{1}\ar[d]^{3p+2} 
			& \sigma_p(4,3)\ar[r]^{1}\ar[d]^{4p+3} & \cdots \\
		& \sigma_p(3,1)\ar[r]^{1}\ar[d]^{3p+1} 
			& \sigma_p(4,2)\ar[r]^{1}\ar[d]^{4p+2} 
			& \sigma_p(5,3)\ar[r]^{1}\ar[d]^{5p+3} & \cdots \\
		& \vdots & \vdots & \vdots & \cdots \\
	}\end{equation*}
	この図で、$\sigma_p$の計算に対して右方向への移動
	$\sigma_p(n,k)\mapsto\sigma_p(n+1,k+1)$からの寄与はなく、
	下方向への移動$\sigma_p(n,k)\mapsto\sigma_p(n+1,k)$のみ寄与が
	あることがわかる。

	\begin{todo}[ここまで]\label{todo:ここまで} %{
		$\ket{m,n}=\ket{m}\otimes\ket{n}$、$\bra{m,n}=\bra{m}\otimes\bra{n}$
		と書くと、図の経路の和は次のように書くことができる。
		\begin{equation*}\begin{split} %{
			\sigma_p(n,k) = m_\myspace\bra{np,k} 
			\bigl(l_p\otimes\myid + a_+^p\otimes a_+\bigr)^n \ket{0,0}
		\end{split}\end{equation*} %}
	\end{todo} %todo:ここまで}
	

	第二種スターリング数$S_p(n,k)$\eqref{eq:第二種拡張スターリング数}を
	求める前に、小さいべき$n$について$l_p^n$の様子を見てみる。
	\begin{equation*}\begin{split} %{
		l_p^2 &= (p+1)l_{2p} + l_p^{*2} \\
		l_p^3 &= (p+1)(2p+1)l_{3p} + 3(p+1)l_{2p}*l_p + 2l_p^{*3} \\
		l_p^4 &= (p+1)(2p+1)(3p+1)l_{4p} + 4(p+1)(2p+1)l_{3p}*l_p \\
		&+ 3(p+1)^2l_{2p}^{*2} + 12(p+1)l_{2p}*l_p^{*2} + 6l_p^{*4} \\
	\end{split}\end{equation*} %}
	この式から、一般の$n\in\sizen$に対して、ある写像
	$\sigma:\bunkatu_+^+\to\sizen$ によって、$l_p^n$は次のように書けること
	が予想される\footnote{
		数の合成$\sosei_k^n$を使わないで数の合成$\bunkatu_k^n$だけを使って
		次のように予想するのが自然だろうが、組み合わせの数の扱いが煩雑なので、
		数の合成を使って本文のような形で予想しておく。
		\begin{equation*}\begin{split} %{
			l_p^n &= \sum_{w\in\bunkatu_+^n}(\sigma w)(\lambda_pw) \\
		\end{split}\end{equation*} %}
	}。
	\begin{equation}\label{eq:lのべき乗の形の予想}\begin{split} %{
		l_p^n &= \sum_{w\in\sosei_+^n}(\sigma w)(\lambda_pw) \\
	\end{split}\end{equation} %}
	ここで、$\lambda_p$は数の分割$\bunkatu_+^+$から$A\mycal{H}$への写像で、
	任意の数の分割$(n_1,n_2,\dots,n_k)\in\bunkatu_k^n$に対して次のように
	定義される。
	\begin{equation*}\begin{split} %{
		\lambda_p(n_1,n_2,\dots,n_k)
		&= (\lambda_p^{n_1}\lambda_p^{n_2}\cdots\lambda_p^{n_k})
		(l_{n_1p}*l_{n_2p}*\cdots*l_{n_kp}) \\
	\end{split}\end{equation*} %}
	ここで、$\lambda_p^n\in\sizen$は任意の$p,n\in\sizen_+$に対して
	次のように定義される。
	\begin{equation}\label{eq:数の組成のスケール因子}\begin{split} %{
		\lambda_p^n &= p^n\frac{\Gamma(n+\frac{1}{p})}{\Gamma(1+\frac{1}{p})}
		= \left\{\begin{split}
			2\le n &\implies  \bigl(p+1\bigr)\bigl(2p+1\bigr)
				\cdots\bigl((n-1)p+1\bigr) \\
			\text{else} &\implies 1 \\
		\end{split}\right. \\ %\}
	\end{split}\end{equation} %}
	\begin{proof} %{
		帰納法を使って証明する。ある$n$で式\eqref{eq:lのべき乗の形の予想}が
		成り立つとすると、次の式が成り立つ。
		\begin{equation*}\begin{split} %{
			l_p^{n+1} &= \sum_{k=1}^n\sum_{w\in\sosei_k^n}
				(\sigma w)l_p(\lambda_pw) \\
		\end{split}\end{equation*} %}
		ここで、任意の$(n_1,n_2,\dots,n_k)\in\sosei_k^n$に対して
		次の式が成り立つから、
		\begin{equation*}\begin{split} %{
			l_p\bigl(\lambda_p(n_1,n_2,\dots,n_k)\bigr)
			&= (\lambda_p^{n_1}\lambda_p^{n_2}\cdots\lambda_p^{n_k}\lambda_p^1)
				(l_{n_1p}*l_{n_2p}*\cdots*l_{n_kp}*l_p) \\
			&+ (\lambda_p^{n_1+1}\lambda_p^{n_2}\cdots\lambda_p^{n_k})
				(l_{(n_1+1)p}*l_{n_2p}*\cdots*l_{n_kp}) \\
			&+ (\lambda_p^{n_1}\lambda_p^{n_2+1}\cdots\lambda_p^{n_k})
				(l_{n_1p}*l_{(n_2+1)p}*\cdots*l_{n_kp}) \\
			&+\cdots \\
			&+ (\lambda_p^{n_1}\lambda_p^{n_2}\cdots\lambda_p^{n_k+1})
				(l_{n_1p}*l_{n_2p}*\cdots*l_{(n_k+1)p}) \\
		\end{split}\end{equation*} %}
		線形写像$\grow:\fukuso\sosei_+^+\to\fukuso\sosei_+^+$を任意の
		$(n_1,n_2,\dots,n_k)\in\sosei_k^n$に対して次のように定義して、
		\begin{equation}\label{eq:数の組成の自然な成長}\begin{split} %{
			\grow(n_1,n_2,\dots,n_k) &= (n_1,n_2,\dots,n_k,1) \\
			&+ (n_1+1,n_2,\dots,n_k) \\
			&+ (n_1,n_2+1,\dots,n_k) \\
			&+\cdots \\
			&+ (n_1,n_2,\dots,n_k+1)
		\end{split}\end{equation} %}
		$\lambda_p$を線形写像に拡張すると、次のように書くことができる。
		\begin{equation*}\begin{split} %{
			l_p\bigl(\lambda_p(n_1,n_2,\dots,n_k)\bigr)
			&= \lambda_p\grow(n_1,n_2,\dots,n_k)
		\end{split}\end{equation*} %}
		したがって、$l_p^{n+1}$は\grow{}を使って次のように書ける。
		\begin{equation*}\begin{split} %{
			l_p^{n+1} &= \sum_{k=1}^n\sum_{w\in\sosei_k^n}
				(\sigma w)(\lambda_p\grow w) \\
		\end{split}\end{equation*} %}
		そこで、$\gamma$を次のように行列で表すと、
		\begin{equation*}\begin{split} %{
			\grow w = \sum_{x\in\sosei_+^{n+1}}\grow_w^x x
			\quad\text{for all }w\in\sosei_+^n
		\end{split}\end{equation*} %}
		$l_p^{n+1}$は次のように書ける。
		\begin{equation*}\begin{split} %{
			l_p^{n+1} &= \sum_{w\in\sosei_+^n}\sum_{x\in\sosei_+^{n+1}}
				\grow_w^x(\sigma w)(\lambda_px) \\
		\end{split}\end{equation*} %}
		よって、$\sigma$を線形写像に拡張して、$\sosei_+^{n+1}$の元に対して
		次のように定義すれば、
		\begin{equation*}\begin{split} %{
			\sigma x = \sum_{w\in\sosei_+^n}\grow_w^x(\sigma w)
			\quad\text{for all }x\in\sosei_+^{n+1}
		\end{split}\end{equation*} %}
		任意の$w\in\sosei_+^n,\;x\in\sosei_+^{n+1}$に対して
		$\grow_w^x\in\set{0,1}$だから、
		$\sosei_+^{n+1}$に対して予想が成り立つことがわかる。
	\end{proof} %}

	この証明から写像\grow{}\eqref{eq:数の組成の自然な成長}によって、
	第二種スターリング数$S_p(n,k)$を計算することができることがわかる。
	この写像を改めて定義しておく。

	\begin{definition}[数の組成の自然な成長]
	\label{def:数の組成の自然な成長} %{
		式\eqref{eq:数の組成の自然な成長}で定義された線形写像
		$\grow:\sizen\sosei_+^+\to\sizen\sosei_+^+$を数の組成の自然な成長
		ということにする。
	\end{definition} %def:数の組成の自然な成長}

	自然な成長は論文\cite{Connes:1998qv}で定義された自然な成長の特別な場合に
	なっている。論文\cite{Connes:1998qv}では根付き平面木に対して自然な成長が
	定義されているが、ここで定義した数の組成に対する自然な成長
	\eqref{eq:数の組成の自然な成長}は根以外がリスト状になっている特別な
	根付き平面木に対する自然な成長となっている。
	\begin{equation*}\begin{array}{ccccccc} %{
		l_{2p}*l_p &\xmapsto{l_p-}& l_{2p}*l_p*l_p &+& (2p+1)l_{3p}*l_p 
			&+& (p+1)l_{2p}*l_{2p} \\
		\mytree{
			& \bullet\ar@{-}[dl]\ar@{-}[dr] \\
			\circ\ar@{-}[d] & & \circ \\
			\circ
		} &\xmapsto{\grow}& \mytree{
			& \bullet\ar@{-}[dl]\ar@{-}[d]\ar@{-}[dr] \\
			\circ\ar@{-}[d] & \circ & \circ \\
			\circ
		} &+& \mytree{
			& \bullet\ar@{-}[dl]\ar@{-}[dr] \\
			\circ\ar@{-}[d] & & \circ \\
			\circ\ar@{-}[d] \\
			\circ
		} &+& \mytree{
			& \bullet\ar@{-}[dl]\ar@{-}[dr] \\
			\circ\ar@{-}[d] & & \circ\ar@{-}[d] \\
			\circ & & \circ
		}
	\end{array}\end{equation*} %}
	\begin{todo}[ここまで]\label{todo:ここまで} %{
	\begin{equation*}\begin{split} %{
		S_p = (\sigma1)\grow(1) = \begin{pmatrix}
			1 & 1
		\end{pmatrix}\begin{pmatrix}
			2 \\ 11
		\end{pmatrix}\\
		\grow\begin{pmatrix}
			2 \\ 11
		\end{pmatrix} = \begin{pmatrix}
			1 & 1 & 0 & 0 \\
			0 & 1 & 1 & 1 \\
		\end{pmatrix}\begin{pmatrix}
			3 \\ 21 \\ 12 \\ 111
		\end{pmatrix} \\
		\grow\begin{pmatrix}
			3 \\ 21 \\ 12 \\ 111
		\end{pmatrix} = \begin{pmatrix}
			1 & 1 & 0 & 0 & 0 & 0 & 0 & 0 \\
			0 & 1 & 1 & 1 & 0 & 0 & 0 & 0 \\
			0 & 0 & 1 & 1 & 1 & 0 & 0 & 0 \\
			0 & 0 & 0 & 1 & 0 & 1 & 1 & 1 \\
		\end{pmatrix}\begin{pmatrix}
			4 \\ 31 \\ 22 \\ 211 \\ 13 \\ 121 \\ 112 \\ 1111
		\end{pmatrix} \\
	\end{split}\end{equation*} %}
4(3),31(3,21),22(21,12),211(21,111),13(12),121(12,111),112(111),1111(111)
5(4),41(4,31),32(31,22),311(31,211),23(22),221(22,211,121),212(211),2111(211,1111),23(13),14(13),131(13,121),122(121,112),1211(121),212(112),113(112),1121(112,1111),1211(1111),1112(1111),11111(1111)
	\end{todo} %todo:ここまで}
	ここで、次の式が成り立つから、
	\begin{equation*}\begin{array}{rcll} %{
		l_p(l_{kp}*l_p^{*(n-k)})
		&=& (l_pl_{kp})*l_p^{*(n-k)} + l_{kp}*(l_pl_p^{*(n-k)})
		&\lcomment{Leibniz} \\
		&=& (l_pl_{kp})*l_p^{*(n-k)}  \\
		&+& \jump{1\le n-k}(n-k)l_{kp}*(l_p^2)*l_p^{*(n-k-1)}
		&\lcomment{Leibniz} \\
		&=& \biggl((kp+1)l_{(k+1)p} + l_{kp}*l_p\biggr)*l_p^{*(n-k)} \\
		&+& \jump{1\le n-k}(n-k)l_{kp}*\biggl((p+1)l_{2p} + l_p^{*2}\biggr)
			*l_p^{*(n-k-1)}
		&\lcomment{Virasoro} \\
		&=& (kp+1)l_{(k+1)p} \\
		&+& \biggl((kp+1) + \jump{1\le n-k}(n-k)\biggr)l_{kp}*l_p^{*(n-k+1)} \\
		&+& \jump{1\le n-k}(n-k)(p+1)l_{kp}*l_{2p}*l_p^{*(n-k-1)} \\
	\end{array}\end{equation*} %}
	$c_p(n+1,k)$を次のようにおくと、
	\begin{equation*}\begin{split} %{
		c_p(n+1,)
	\end{split}\end{equation*} %}

	\begin{todo}[作用素積展開]\label{todo:作用素積展開} %{
		これらの記号を用いると、
		第二種スターリング数の現れる式\ref{eq:第二種拡張スターリング数}は
		次のようになる。
		\begin{equation*}\begin{split} %{
			l_m^n = \sum_{k\in1..n} S_{m+1}(n,k) a_+^{mn}l_0^{*k}
			\quad\text{for all }m,n\in\sizen
		\end{split}\end{equation*} %}

		エネルギーテンソル$t_z$も同様に定義しておく。
		\begin{equation*}\begin{split} %{
			t_z &= z^{-2}\sum_{m\in\sizen}l_mz^{-m} \\
		\end{split}\end{equation*} %}
		ここで定義した$t_z$は共形場と異なり、振動モードを独立に扱っていない
		ので、$l_m$の添字の範囲は自然数となり、中心拡大も存在しない。

		共形場で$T_z=z^{-2}\sum_{m\in\sei}L_mz^{-m}$として、次の散乱振幅を計算
		することを考える。
		\begin{equation*}\begin{split} %{
			f(t,w,z) &= \bra{w}e^{tv_T}\ket{z} \\
			v_T &= \frac{1}{2\pi i}\oint_0 dz v_zT_z \\
		\end{split}\end{equation*} %}
	\end{todo} %todo:作用素積展開}
%s1:スターリング数}
\endgroup %}
