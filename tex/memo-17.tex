\begingroup %{
\newcommand{\T}{\mycal{T}}
\newcommand{\W}{\mycal{W}}
\newcommand{\Pow}{\mycal{P}}
\newcommand{\End}{\myop{End}}
\newcommand{\Map}{\myop{Map}}
\newcommand{\Lin}{\mathcal{L}}
\newcommand{\Etaol}{\mathcal{H}}
\newcommand{\Aut}{\myop{Aut}}
\newcommand{\Mat}{\myop{Mat}}
\newcommand{\Etaom}{\myop{Hom}}
\newcommand{\Eta}{\mycal{H}}
%
\newcommand{\id}{\myop{id}}
\newcommand{\tran}{\mathbf{t}}
\newcommand{\dfn}{\,\myop{def}\,}
\newcommand{\xiff}[2][]{\xLongleftrightarrow[#1]{#2}}
\newcommand{\tr}{\myop{tr}}
%
\newcommand{\mvec}[2]{\begin{matrix}{#1}\\{#2}\end{matrix}}
\newcommand{\pvec}[2]{\begin{pmatrix}{#1}\\{#2}\end{pmatrix}}
\newcommand{\bvec}[2]{\begin{bmatrix}{#1}\\{#2}\end{bmatrix}}
\newcommand{\what}{\widehat}
\newcommand{\frk}[1]{\ensuremath{\mathfrak{#1}}}
\newcommand{\ad}{\myop{ad}}
\newcommand{\Ad}{\myop{Ad}}
%
\newcommand{\smallxy}[1]{\vcenter{\xymatrix@R=4pt@C=4pt@M=1pt@W=1pt{#1}}}
\newcommand{\hen}{\ar@{-}}
\newcommand{\sbt}{\vcenter{\hbox{\tiny$\bullet$}}}
%
{\setlength\arraycolsep{2pt}
%
\section{代数的Chomsky-Schutzenbergerの定理}\label{s1:代数的Chomsky-Schutzenbergerの定理} %{
	Wikipeidaに代数的Chomsky-Schutzenbergerの定理の項目\footnote{
		英語のWikipediaで次の文字列で検索すればよい。
		\begin{itemize}\setlength{\itemsep}{-1mm} %{
			\item Chomsky-Schutzenberger
		\end{itemize} %}
	}があったので記録しておく。

	$\sizen[[x]]$を自然数を係数とする形式級数全体のつくる集合、
	$\bun(x):=\bun[x,x^{-1}]$を有理数を係数とする$x$と$x^{-1}$を変数とする
	多項式全体のつくる集合とする。

	$f\in\sizen[[x]]$が次の性質を満たす時、$f$を$\bun(x)$上で代数的という。
	\begin{itemize}\setlength{\itemsep}{-1mm} %{
		\item ある有限個の$p_0,\dots,p_n\in\bun(x)$が存在して、
		\begin{equation*}\begin{split}
			p_0 + p_1\cdot f +\cdots+ p_n\cdot f^n = 0
		\end{split}\end{equation*}
		となる。ここで、積$\cdot$は次のように定義する。
		\begin{equation*}\begin{split}
			\plr{\phi\cdot\psi|x} := \plr{\phi|x}\plr{\psi|x}
		\end{split}\end{equation*}
	\end{itemize} %}

	\begin{proposition}[代数的Chomsky-Schutzenberの定理]\label{prop:代数的Chomsky-Schutzenberの定理} %{
		$A$を有限集合、$L\subseteq A^*$を曖昧さのない文脈自由文法、
		$L_n:=\set{w\in L\bou |w|=n}$とする。
		このとき、$\sum_{n\in\sizen}|L_n|t^n$は$\bun(t)$上で代数的となる。
	\end{proposition} %prop:代数的Chomsky-Schutzenberの定理}

	$A$を文字集合、$f\in\sizen\braket{A,x}$として、
	文法$x=\plr{f|x}$に曖昧さがなければ、ある言語$L\subseteq A^*$があって、
	$x=\sum_{w\in L}w$と摂動展開が書ける。
	したがって、写像$\chi_t:L\to\sizen[t]$を$\chi w=t^{|w|}$と定義すると、
	$\chi_tx=\sum_{n\in\sizen}|L_n|t^n$となる。$A$が有限集合だから、
	任意の$n\in\sizen$で$L_n\le|A^n|=|A|^n$となり、少なくとも$|t|<|A|$では
	$\chi_tx$は収束する。

	$\chi_t$を文法に作用させると、$\chi_tx_1=\chi_t\plr{f|x_1}$という
	形式級数が得られる。例を使って考えてみよう。
	文字集合$A:=\set{p,m,l,r,v}$から生成される次の文法を考える。
	\begin{equation*}\begin{split}
		X &= Y + YpX + YmX,\quad Y = v^* + lXr 
	\end{split}\end{equation*}
	任意の$\alpha\in A$に対して$\chi_t\alpha=t$だから、文法に$\chi_t$を作用
	させると、次のようになる。
	\begin{equation*}\begin{split}
		\left\{\begin{split}
			X_t &= Y_t + 2tY_tX_t \\
			Y_t &= t^* + t^2X_t
		\end{split}\right. \iff \left\{\begin{split}
			X_t &= \plr{t^* + t^2X_t}\plr{1 + 2tX_t} \\
			Y_t\plr{1 - 2tY_t} &= t^*\plr{1 - 2tY_t} + t^2Y_t
		\end{split}\right.
	\end{split}\end{equation*}

	代数的Chomsky-Schutzenberの定理は自由モノイドに対する命題だが、Kontsevich
	\cite{2011arXiv1109.2469K}によって自由群に対して同様の命題が証明されて
	いる。\cite{2011arXiv1109.2469K}ではより大きな目標の一貫として自由群に
	対する代数的Chomsky-Schutzenberの定理が証明されているようだが、
	Reutenauer達\cite{Reutenauer:2012}が、代数的Chomsky-Schutzenber
	の定理の部分だけを抜き出して解説している。
%s1:代数的Chomsky-Schutzenbergerの定理}
\section{環の中心}\label{s1:環の中心} %{
	$V$を環、$C$を$V$の中心とする。$C$は$0$と$1$を含むので、空ではない。
	また、$x,y\in C$なら、任意の$v\in V$に対して次のことが成り立つので、
	\begin{itemize}\setlength{\itemsep}{-1mm} %{
		\item $(x+y)r=r(x+y)$
		\item $xyr=rxy$
	\end{itemize} %}
	$C$は$V$の部分環となる。
%s1:環の中心}
\section{Dyck単語の分割}\label{s1:Dyck単語の分割} %{
	この節では、$V$を環として、次の部分代数の系列を考える。
	\begin{equation*}\begin{split}
		V\Eta_0 \subset V\Eta_1 \subset V\Eta_2 \subset \cdots \subset V\Eta_*
	\end{split}\end{equation*}

	$V\Eta_n$への射影$I_n\in\End_V\plr{V\Eta_*}$を次のように定義すると、
	\begin{equation*}\begin{split}
		I_n := \sum_{w\in\Eta_{n+}^*}\ket{w^\flat}\bra{w}
	\end{split}\end{equation*}
	$I_n$は$\eta_{\pm1},\dots,\eta_{\pm n}$に対しては可換、
	$\eta_{\pm\plr{n+1}},\eta_{\pm\plr{n+2}},\dots$に対しては真空のように
	振る舞う。
	\begin{alignat*}{2}
		I_n\eta_{\pm i} &= \eta_{\pm i}I_n &\quad&\text{for all } i\in1..n \\
		I_n\eta_{-i} &= 0 = \eta_iI_n &\quad&\text{for all } n < i
	\end{alignat*}
	$I_n$を用いて射影$\braket{-}_n:V\Eta_*\to V\Eta_n$を次のように定義する。
	\begin{equation*}\begin{split}
		\braket{x}_n := I_nxI_n \quad\text{for all } x\in V\Eta_*
	\end{split}\end{equation*}

	$K\in\sizen$、$a,b,c\in V\Eta_K$とし、$\phi\in V\Eta_{K+1}$を次のように
	定義する。
	\begin{equation*}\begin{split}
		\phi := a + b\eta_{K+1} + \eta_{-\plr{K+1}}c
	\end{split}\end{equation*}
	Kleeneスターの射影$\braket{\phi^*}_K$を計算してみる。
	まず、次のようになるが、
	\begin{equation}\label{ea:Dyck単語の分割その一}\begin{split}
		I_K\phi^{n+2} &= a^{n+2}I_K 
			+ \sum_{r=0}^{n+1}a^rbI_K\eta_{K+1}\phi^{n+1-r} \\
	\end{split}\end{equation}
	$I_K^{1,2}$を次のように定義すると、
	\begin{equation*}\begin{split}
		I_K^{1,2} := \sum_{w\in\Eta_n^*}\ket{w^\flat}\bra{w}\otimes\bra{1}
	\end{split}\end{equation*}
	二項目の和の中は次のように書くことができる。
	\begin{equation*}\begin{split}
		I_K\eta_{K+1}\phi^{n+1-r}
		= I_K^{1,2} \plr{\id\otimes\eta_{K+1}}m_0^\flat\phi^{n+1-r}
	\end{split}\end{equation*}
	$\phi$と$m_0^\flat$の交換関係は次の形になるが、
	\begin{equation}\label{ea:Dyck単語の分割その二}\begin{split}
		m_0^\flat\phi &= \plr{\Delta_0\phi}m_0^\flat \\
		\Delta_0\phi &= \phi\otimes\id + I_0c\otimes\eta_{-\plr{K+1}}
			+ \plr{\cdots}\otimes\plr{\text{operators in $\Eta_{-K}$}}
	\end{split}\end{equation}
	生成演算子$\Eta_{K-}$がテンソル積の二項目に現れる項は
	$I_K\plr{\id\otimes\eta_{K+1}}$に作用すると$0$になるから、			
	次の式が成り立つ。
	\begin{equation}\label{ea:Dyck単語の分割その三}\begin{split}
		I_K\eta_{K+1}\phi^{n+1-r}
		&= I_K^{1,2} \plr{\id\otimes\eta_{K+1}}m_0^\flat\phi^{n+1-r} \\
		&= I_K^{1,2} \plr{\phi^{n+1-r}\otimes\eta_{K+1}} m_0^\flat \\
		&\;+ \sum_{s=0}^n I_K^{1,2} 
			\plr{\phi^sI_0c\otimes\eta_{K+1}\eta_{-\plr{K+1}}}
			m_0^\flat\phi^{n-(r+s)} \\
		&= I_K\phi^{n+1-r}\eta_{K+1} + \sum_{s=0}^n\phi^sI_0c\phi^{n-(r+s)}
	\end{split}\end{equation}
	したがって、次の式が得られる。
	\begin{equation*}\begin{split}
		I_K\phi^{n+2} &= a^{n+2}I_K 
			+ \sum_{r=0}^{n+1} a^rbI_K\phi^{n+1-r}\eta_{K+1}
			+ \sum_{r+s+t=n} a^rbI_K\phi^sI_0c\phi^t
	\end{split}\end{equation*}
	また、次の式が成り立つが、
	\begin{equation*}\begin{split}
		\braket{\phi^{n+2}}_K = a^{n+2} 
			+ \sum_{r+s+t=n} \braket{a^rb\phi^s}_KI_0\braket{c\phi^t}_K
	\end{split}\end{equation*}
	次の式を使うと、
	\begin{equation*}\begin{split}
		\braket{\phi^{n+3}}_K &= a^{n+3} + a\sum_{r+s+t=n}
			\braket{a^rb\phi^s}_KI_0\braket{c\phi^t}_K
			+ \sum_{r+s=n+1} \braket{b\phi^s}_KI_0\braket{c\phi^t}_K \\
		&= a\braket{\phi^{n+2}}_K + \sum_{s+t=n+1}
			\braket{b\phi^s}_KI_0\braket{c\phi^t}_K
	\end{split}\end{equation*}
	次の式より、
	\begin{equation*}\begin{split}
		\braket{\phi^0}_K &= 1 \\ 
		\braket{\phi^1}_K &= a\braket{\phi^0}_K \\ 
		\braket{\phi^2}_K &= a\braket{\phi^1}_K
			+ \sum_{s+t=0} \braket{b\phi^s}_KI_0\braket{c\phi^t}_K \\
		\braket{\phi^3}_K &= a\braket{\phi^2}_K 
			+ \sum_{s+t=1} \braket{b\phi^s}_KI_0\braket{c\phi^t}_K \\
		\cdots \\
	\end{split}\end{equation*}
	任意の$n\in\sizen$に対して次の式が成り立つ。
	\begin{equation*}\begin{split}
		\sum_{i=0}^{n+2} \braket{\phi^i}_K &= 1 
			+ \sum_{i=0}^{n+1} a\braket{\phi^i}_K 
			+ \sum_{s+t=n} \braket{b\phi^s}_KI_0\braket{c\phi^t}_K
	\end{split}\end{equation*}
	したがって、$t$を$V\Eta_*$の元と可換な不定元として、次の式が成り立つ。
	\begin{equation*}\begin{split}
		\Braket{\plr{t\phi}^*}_K = 1 + t\Braket{a\plr{t\phi}^*}_K 
			+ t^2\Braket{b\plr{t\phi}^*}_KI_0\Braket{c\plr{t\phi}^*}_K 
			\quad\text{up to } t^\infty
	\end{split}\end{equation*}
	この式は$t$の有限べきについて成り立つだけで、収束性
	$\braket{\phi^*}_K\in V\Eta_K$は保証していない。
	以上のことを命題の形でまとめておく。

	\begin{proposition}[Dyck単語の分割]\label{prop:Dyck単語の分割} %{
		$V$を環とする。$K\in\sizen$、$a,b,c\in V\Eta_K$として、
		$\phi\in V\Eta_{K+1}$を次のように定義すると、
		\begin{equation*}\begin{split}
			\phi := a + b\eta_{K+1} + \eta_{-\plr{K+1}}c
		\end{split}\end{equation*}
		形式級数$\braket{\plr{t\phi}^*}_K\in V\Eta_K[[t]]$について、次の式が
		成り立つ。
		\begin{alignat*}{2}
			\Braket{\plr{t\phi}^*}_K &= 1 + t\Braket{a\plr{t\phi}^*}_K 
				+ t^2\Braket{b\plr{t\phi}^*}_KI_0\Braket{c\plr{t\phi}^*}_K
				&\quad\text{up to } t^\infty \\
			&= \ggplr{ta + t^2b\Braket{\plr{t\phi}^*}_KI_0c}^* 
				&\quad\text{up to } t^\infty
		\end{alignat*}
		この式は$t$の有限べきについて成り立つだけで、
		収束性$\braket{\phi^*}_K\in V\Eta_K$は保証していない\footnote{
			無限和や無限積を含む場合は、有限和や有限積の場合に成り立つ事柄が
			そのまま成り立たないことが多い。代数的Chomsky-Schutzenberの定理
			\ref{s1:代数的Chomsky-Schutzenbergerの定理}を参照すること。
		}。
	\end{proposition} %prop:Dyck単語の分割}

	この命題から、$V[[t]]$に対する次の式が導かれる。
	\begin{equation*}\begin{split}
		\phi := a + b\eta_1 + \eta_{-1}c 
		\xRightarrow{x := \Braket{\plr{t\phi}^*}} x = 1 + tax + t^2bxcx
		\quad\text{for all } a,b,c\in V
	\end{split}\end{equation*}
	この式を$x=1+\plr{f|x}x$という形の文法に拡張することを考える。

	まず、二次式の範囲で考えてみる。$a,b_1,\dots,b_N,c_1,\dots,c_N\in V\Eta_K$
	として、$\phi\in V\Eta_{K+N}$を次のように定義する。
	\begin{equation*}\begin{split}
		\phi := a + \phi_+ + \phi_-
		,\quad \phi_+ := \sum_{i=1}^N b_i\eta_{K+i}
		,\quad \phi_- := \sum_{i=1}^N \eta_{-(K+i)}c_i
	\end{split}\end{equation*}
	式\eqref{ea:Dyck単語の分割その二}と同様に、次の式が成り立つから、
	\begin{equation*}\begin{split}
		m_0^\flat\phi &= \plr{\Delta_0\phi}m_0^\flat \\
		\Delta_0\phi &= \phi\otimes\id 
			+ \sum_{i=1}^N I_0c\otimes\eta_{-\plr{K+i}}
			+ \plr{\cdots}\otimes\plr{\text{operators in $\Eta_{-K}$}}
	\end{split}\end{equation*}
	任意の$n\in\sizen$で次の式が成り立ち、
	\begin{equation*}\begin{split}
		I_K\phi^{n+2} &= a^{n+2} + \sum_{r=0}^{n+1} a^rI_K\phi_+\phi^{n+1-r} \\
		&= a^{n+2} + \sum_{r=0}^{n+1} a^rI_K\phi^{n+1-r}\phi_+
			+ \sum_{r+s=n}\sum_{i=1}^N a^r\Braket{b_i\phi^s}I_0c_i\phi^{n-(r+s)}
	\end{split}\end{equation*}
	次の式が成り立つ。
	\begin{alignat*}{2}
		\Braket{\plr{t\phi}^*}_K &= 1 + ta\Braket{\plr{t\phi}^*}_K 
			+ t^2\sum_{i=1}^N\Braket{b_i\plr{t\phi}^*}_K
			I_0\Braket{c_i\plr{t\phi}^*}_K &\quad\text{up to }t^\infty \\
		&= \ggplr{ta + t^2\sum_{i=1}^N b_i\Braket{\plr{t\phi}^*}_KI_0c_i}^* 
			&\quad\text{up to } t^\infty
	\end{alignat*}
	以上のことを命題の形でまとめておく。

	\begin{proposition}[Dyck単語の分割その二]\label{prop:Dyck単語の分割その二} %{
		$V$を環とする。$K,N\in\sizen$、
		$a,b_1,\dots,b_N,c_1,\dots,c_N\in V\Eta_K$として、
		$\phi\in V\Eta_{K+N}$を次のように定義すると、
		\begin{equation*}\begin{split}
			\phi := a + \sum_{i=1}^N \plr{b_i\eta_{K+i} + \eta_{-(K+i)}c_i}
		\end{split}\end{equation*}
		次の式が成り立つ。
		\begin{alignat*}{2}
			\Braket{\plr{t\phi}^*}_K &= 1 + ta\Braket{\plr{t\phi}^*}_K 
				+ t^2\sum_{i=1}^N\Braket{b_i\plr{t\phi}^*}_K
				I_0\Braket{c_i\plr{t\phi}^*}_K &\quad\text{up to }t^\infty \\
			&= \ggplr{ta + t^2\sum_{i=1}^N b_i\Braket{\plr{t\phi}^*}_KI_0c_i}^* 
				&\quad\text{up to } t^\infty
		\end{alignat*}
	\end{proposition} %prop:Dyck単語の分割その二}

	単項式のべきについては次の命題が成り立つ。

	\begin{proposition}[Dyck単語の分割その三]\label{prop:Dyck単語の分割その三} %{
		$V$を環とする。$K,N\in\sizen$、$a,b_1,\dots,b_{N+2}\in V\Eta_K$として、
		$\phi\in V\Eta_{K+N+1}$を次のように定義すると、
		\begin{equation*}\begin{split}
			\phi := a + b_1\eta_{K+1} + \eta_{-(K+N+1)}b_{N+2} 
				+ \sum_{r=2}^{N+1} \eta_{-(K+r-1)}b_r\eta_{K+r}
		\end{split}\end{equation*}
		形式級数$\braket{\plr{t\phi}^*}_K\in V\Eta_K[[t]]$について、次の式が
		成り立つ。
		\begin{alignat*}{2}
			\Braket{\plr{t\phi}^*}_K &= 1 + t\Braket{A}_K
				+ t^{N+2}\Braket{B_1}_KI_0\Braket{B_2}\cdots\Braket{B_{N+1}}
				I_0\Braket{B_{N+2}}_K &\quad\text{up to } t^\infty \\
			&= \ggplr{ta 
				+ t^{N+2}\Braket{B_1}_KI_0\Braket{B_2}\cdots\Braket{B_{N+1}}
				I_0b_{N+2}}^* &\quad\text{up to } t^\infty
		\end{alignat*}
		ここで、$A$と$B_i$は次のように定義した。
		\begin{equation*}\begin{split}
			A := a\plr{t\phi}^*,\quad B_1 := b_1\plr{t\phi}^*,\dots,\quad 
			B_{N+2} := b_{N+2}\plr{t\phi}^*
		\end{split}\end{equation*}
		$B_1,\dots,B_{N+1}$に対しては部分真空期待値ではなく、
		完全な真空期待値をとっていることに注意する。
	\end{proposition} %prop:Dyck単語の分割その三}
	\begin{proof} %{
		命題の$N$の帰納法によって証明する。$N=0$のとき命題が成り立つことは
		Dyck単語の分割\ref{prop:Dyck単語の分割}からわかる。
		ある$N\in\sizen$で命題が成り立つと仮定し、$\phi_{N+3}\in V\Eta_{K+N+3}$
		を次のように定義する。
		\begin{equation*}\begin{split}
			\phi_{N+3} := a + b_1\eta_{K+1} + \eta_{-(K+N+2)}b_{N+3}
				+ \sum_{r=2}^{N+2} \eta_{-(K+r-1)}b_r\eta_{K+r}
		\end{split}\end{equation*}
		Dyck単語の分割\ref{prop:Dyck単語の分割}によって、$\eta_{\pm(K+N+3)}$を
		積分してしまうと、次の式が成り立つ。
		\begin{equation*}\begin{split}
			\Braket{\plr{t\phi_{N+3}}^*}_{K+N+2} = \plr{t\phi_{N+2}}^*
		\end{split}\end{equation*}
		ここで、$\phi_{N+2}\in V\Eta_{K+N+2}$は次のように定義した。
		\begin{equation*}\begin{split}
			\phi_{N+2} &:= a + b_1\eta_{K+1} + t\eta_{-(K+N+1)}c 
				+ \sum_{r=2}^{N+1} \eta_{-(K+r-1)}b_r\eta_{K+r} \\
			c &:= b_{N+2}\plr{t\phi_{N+2}}^*I_0 b_{N+3}\plr{t\phi_{N+2}}^*
		\end{split}\end{equation*}
		$\phi_{N+2}$は次の性質を持つことに注意して、
		\begin{equation*}\begin{split}
			\Braket{f\plr{t\phi_{N+2}}^*g}_K = \Braket{f\plr{t\phi_{N+3}}^*g}_K
			\quad\text{for all } f,g\in V\Eta_K
		\end{split}\end{equation*}
		$\phi_{N+2}$に対して帰納法の仮定を適用すると、次の式が得られるが、
		\begin{equation*}\begin{split}
			\Braket{\plr{t\phi_{N+2}}^*}_K
			= 1 + t\Braket{A}_K + t^{N+3}\Braket{B_1}_KI_0
				\Braket{B_2}\cdots\Braket{B_{N+1}}I_0
				\Braket{c\plr{t\phi_{N+2}}^*}_K \\
		\end{split}\end{equation*}
		次の式から、
		\begin{equation*}\begin{split}
			I_0\Braket{c\plr{t\phi_{N+2}}^*}_K = I_0\Braket{b_{N+2}
				\plr{t\phi_{N+2}}^*} I_0\Braket{b_{N+3}\plr{t\phi_{N+2}^*}}_K
			= I_0\Braket{B_{N+2}}I_0\Braket{B_{N+3}}_K
		\end{split}\end{equation*}
		次の式が成り立つことがわかる。
		\begin{equation*}\begin{split}
			\Braket{\plr{t\psi}^*}_K = 1 + t\Braket{A}_K 
				+ t^{N+3}\Braket{B_1}_KI_0\Braket{B_2}\cdots\Braket{B_{N+2}}
				I_0\Braket{B_{N+3}}_K
		\end{split}\end{equation*}
		したがって、
		$\Braket{\plr{t\phi_{N+3}}^*}_K=\Braket{\plr{t\phi_{N+2}}^*}_K$だから、
		$N+1$でも命題が成り立つことがわかる。
	\end{proof} %}

	この命題の$\phi$は次のベクトル$H_0,\; H_\pm$と巡回的な行列$B$を用いて、
	\begin{equation*}\begin{split}
		H_0 := \begin{pmatrix}
			1 \\ 0 \\ 0 \\ \vdots \\ 0 \\ 0
		\end{pmatrix},\quad H_\pm := \begin{pmatrix}
			0 \\ \eta_{\pm(K+1)} \\ \eta_{\pm(K+2)} 
			\\ \vdots \\ \eta_{\pm(K+N)} \\ \eta_{\pm(K+N+1)}
		\end{pmatrix},\quad B := \begin{pmatrix}
			0 & b_1 & 0 & 0 & \cdots & 0 \\
			0 & 0 & b_2 & 0 & \cdots & 0 \\
			0 & 0 & 0 & b_3 & \cdots & 0 \\
			\vdots & \vdots & \vdots & \vdots & \ddots & 0 \\
			0 & 0 & 0 & 0 & \cdots & b_{N+1} \\
			b_{N+2} & 0 & 0 & 0 & \cdots & 0 \\
		\end{pmatrix}
	\end{split}\end{equation*}
	次のように書くことができる。
	\begin{equation*}\begin{split}
		\phi = a + \plr{H_0 + H_-}^\tran B \plr{H_0 + H_+}
	\end{split}\end{equation*}
	メモ\ref{note:Dyck単語の分割その三}を使いつつ、
	行列の形で$\Braket{\phi^*}_K$を計算してみる。まず、Kleeneスターの摂動から
	次の式が成り立つが、
	\begin{equation*}\begin{split}
		\Braket{\phi^*}_K = 1 + \plr{a + H_0^\tran BH_0}\Braket{\phi^*}_K
			+ H_0^\tran B\Braket{H_+\phi^*}_K
	\end{split}\end{equation*}
	計算\eqref{eq:真空期待値の一次微分の計算}から、次の式が成り立つことが
	わかる。
	\begin{equation*}\begin{split}
		\Braket{H_+\phi^*}_K
		= 1 + \plr{a + H_0^\tran BB_\phi^*H_0}\Braket{\phi^*}_K
	\end{split}\end{equation*}
	ここで、$B_\phi\in(V\Eta_K)^{N+2}$は次のように定義する。
	\begin{equation*}\begin{split}
		B_\phi := \Braket{\phi^*}_KI_0\plr{1 - H_0H_0^\tran}B
	\end{split}\end{equation*}
	ここまでの計算結果は$B$の形に依らない。
	行列$B$の具体的な形を用いると、計算\eqref{eq:Bの具体形を用いた計算}から、
	命題の式が得られる。
	\begin{equation*}\begin{split}
		\Braket{\phi^*}_K
		= 1 + \plr{a + H_0^\tran BB_\phi^{N+1}H_0}\Braket{\phi^*}_K
	\end{split}\end{equation*}
	\begin{note}[計算メモ]\label{note:Dyck単語の分割その三} %{
		\begin{itemize}\setlength{\itemsep}{-1mm} %{
			\item $m_0^\flat\phi=\plr{\Delta_0\phi}m_0$とすると、次の式が
			成り立つ。
			\begin{equation*}\begin{split}
				\Delta_0\phi = \phi\otimes\id 
					+ I_0{\contraction{}{B(H_0 + H_+)}{\otimes}{H_-^\tran}
						B(H_0 + H_+)\otimes H_-^\tran}
					+ \plr{\cdots}\otimes\plr{\text{operators in $\Eta_{-K}$}}
			\end{split}\end{equation*}
			ここで、$\contraction{}{X}{\otimes}{Y^\tran}X\otimes Y^\tran$は
			次のようにテンソル積を跨いだベクトルの内積を表す。
			\begin{equation*}\begin{split}
				\contraction{}{X}{\otimes}{Y^\tran}X\otimes Y^\tran 
				:= \sum_{i=1}^{N+1}X_i\otimes Y_i
				,\quad \contraction{}{X^\tran}{\otimes}{Y}X^\tran\otimes Y
				:= \sum_{i=1}^{N+1}X_i\otimes Y_i
			\end{split}\end{equation*}
			%
			\item $\Braket{H_+\phi^*}_K$の計算
			\begin{equation*}\begin{split}
				I_KH_+\phi^* & = I_K^{1,2}(\id\otimes H_+)m_0^\flat\phi^* \\
				& = I_K^{1,2}(\phi^*\otimes H_+)m_0^\flat
					+ I_K^{1,2}(\phi^*\otimes H_+)
					\plr{I_0{\contraction{}{B(H_0 + H_+)}{\otimes}{H_-^\tran}
							B(H_0 + H_+)\otimes H_-^\tran}}
					m_0^\flat\phi^* \\
				&= I_K\phi^*H_+ + I_K\phi^*I_0\plr{H_+H_-^\tran}B(H_0 + H_+)\phi^*
			\end{split}\end{equation*}
			より、縮約に対する次の式と、
			\begin{equation*}\begin{split}
				\plr{\contraction{}{X_1^\tran}{\otimes}{Y_1} X_1^\tran\otimes Y_1}
				\plr{\contraction{}{X_2}{\otimes}{Y_2^\tran} X_2\otimes Y_2^\tran}
				&= \sum_{i,j} \plr{X_{1i}X_{2j}}\otimes\plr{Y_{1i}Y_{2j}} \\
				&= \contraction{(}{X_1}{\otimes1)(1\otimes}{Y_1}
				\contraction{(X_1\otimes1)(1\otimes Y_1}{Y_2^\tran}{)(}{X_2}
				(X_1\otimes1)(1\otimes Y_1Y_2^\tran)(X_2\otimes1)
			\end{split}\end{equation*}
			$H_\pm$に対する次の式を使って、
			\begin{equation*}\begin{split}
				H_+H_-^\tran = 1 - H_0H_0^\tran\in\text{ matrix of center of }V
			\end{split}\end{equation*}
			次の式が得られる。
			\begin{equation}\label{eq:真空期待値の一次微分の計算}\begin{split}
				\Braket{H_+\phi^*}_K 
				&= \Braket{\phi^*}_KI_0\plr{H_+H_-^\tran}BH_0\Braket{\phi^*}_K
				+ \Braket{\phi^*}_KI_0\plr{H_+H_-^\tran}B\Braket{H_+\phi^*}_K \\
				&= \ggplr{\Braket{\phi^*}_KI_0\plr{H_+H_-^\tran}B}^+H_0\Braket{\phi^*}_K
			\end{split}\end{equation}
			%
			\item $H_0^\tran BB_\phi^*H_0$の計算
			\begin{equation}\label{eq:Bの具体形を用いた計算}\begin{split}
				H_0^\tran BB_\phi^*H_0 &= \begin{pmatrix}
					0 \\ b_1 \\ 0 \\ \vdots \\ 0 \\ 0
				\end{pmatrix}^\tran\plr{\Braket{\phi^*}_KI_0\begin{pmatrix}
					0 & 0 & 0 & 0 & \cdots & 0 \\
					0 & 0 & b_2 & 0 & \cdots & 0 \\
					0 & 0 & 0 & b_3 & \cdots & 0 \\
					\vdots & \vdots & \vdots & \vdots & \ddots & 0 \\
					0 & 0 & 0 & 0 & \cdots & b_{N+1} \\
					b_{N+2} & 0 & 0 & 0 & \cdots & 0 \\
				\end{pmatrix}}^*\begin{pmatrix}
					1 \\ 0 \\ 0 \\ \vdots \\ 0 \\ 0
				\end{pmatrix} \\
				&= H_0^\tran BB_\phi^{N+1}H_0
			\end{split}\end{equation}
			$\beta_i$を次のようにおくと、
			\begin{equation*}\begin{split}
				\beta_1:=b_1,\quad
				\beta_i:=b_1\Braket{\phi^*}_KI_0b_2\cdots\Braket{\phi^*}_KI_0b_i
			\end{split}\end{equation*}
			次のようになっていることから、前記の式が成り立つことがわかる。
			\begin{equation*}\begin{split}
				H_0^\tran BB_\phi^0 &= (0\; \beta_1\; 0\; 0\; \cdots\; 0) \\
				H_0^\tran BB_\phi^1 &= (0\; 0\; \beta_2\; 0\; \cdots\; 0) \\
				\vdots \\
				H_0^\tran BB_\phi^N &= (0\; 0\; 0\; 0\; 0\; \cdots\; \beta_{N+1}) \\
				H_0^\tran BB_\phi^{N+1} H_0 &= (b_{N+2}\; 0\; 0\; 0\; \cdots\; 0) \\
				H_0^\tran BB_\phi^{N+2} H_0 &= (0\; 0\; 0\; 0\; \cdots\; 0)
			\end{split}\end{equation*}
		\end{itemize} %}
	\end{note} %note:Dyck単語の分割その三}
\subsection{観察その一}\label{s2:観察その一} %{
	この節では多項式とその根について考える。

	環$V$を係数とする多項式を$V$-多項式と書く事にする。通常は、多項式の
	係数は変数変換で変わり得るので、係数を明示することはないが、この節では
	多項式を変形していくので、係数を明示することにする。
	また、環$V$、$t$を$V$と可換な不定元として、
	\begin{itemize}\setlength{\itemsep}{-1mm} %{
		\item $V[t]$を$V$を係数とする$t$の多項式全体のつくる集合、
		\item $V[[t]]$を$V$を係数とする$t$の形式級数全体のつくる集合、
		\item $V(t):=V[t,t^{-1}]$、
		\item $V((t)):=V[[t,t^{-1}]]$、
	\end{itemize} %}
	とする。また、通常は用いられることがない記号だが、見やすさを考えて
	次の記法を使うことにする。
	\begin{equation*}\begin{split}
		[V]_t:=V[t],\quad [[V]]_t:=V[[t]],\quad (V)_t:=V[t,t^{-1}]
		,\quad ((V))_t:=V[[t,t^{-1}]]
	\end{split}\end{equation*}

	$t$を変数として次の$[\sei]_t$-多項式とその根$x_\pm\in((\fukuso))_t$
	を考える。
	\begin{equation}\label{eq:元の多項式}\begin{split}
		x = 1 + tx^2,\quad x_\pm := \frac{1 \pm \sqrt{1 - 4t}}{2t}
	\end{split}\end{equation}
	この多項式を次のプッシュダウンオートマトンを構成する過程に添って変形
	してみる。
	\begin{equation*}\begin{array}{rcll}
		\xymatrix{
			*++[o][F=]{x} \ar@(dl,ul)^{tx^2}
		} &\xmapsto{\text{insert final state}}& \xymatrix{
			*++[o][F-]{x} \ar[r]^{1+tx^2} & *++[o][F=]{y}
		} \\
		&\xmapsto{\text{eliminate 1st $x$}}& \xymatrix{
			*++[o][F-]{x} \ar[r]^{1} \ar@(dl,ul)^{t\eta_1} 
			& *++[o][F=]{y} \ar@(ru,rd)^{\eta_{-1}x}
		} \\
		&\xmapsto{\text{eliminate 2nd $x$}}& \xymatrix{
			*++[o][F-]{x} \ar@<1ex>[r]^{1} \ar@(dl,ul)^{t\eta_1} 
			& *++[o][F=]{y} \ar@(ru,rd)^{\eta_{-2}} \ar@<1ex>[l]^{\eta_{-1}\eta_2}
		}
	\end{array}\end{equation*}

	元の多項式\eqref{eq:元の多項式}の左辺で、$x=X/Y$という有理変換をして、
	次のように書き換える。
	\begin{equation*}\begin{split}
		X = xY ,\quad X = Y + tx^2Y
	\end{split}\end{equation*}
	この式は、任意の$g\neq0\in((\fukuso))_t$に対して
	$(X,Y)\mapsto\plr{gX,gY}$という変換で不変になっている。
	したがって、ある$y\neq0\in((\fukuso))_t$を用いて、$Y=y$としてゲージ固定
	すると、次の$((\fukuso))_t$-連立多項式が得られる。
	\begin{equation*}\begin{split}
		X = xY ,\quad X = Y + tx^2Y ,\quad Y = y
	\end{split}\end{equation*}
	\begin{equation*}\begin{split}
		\pvec{X}{Y} = \pvec{0}{y} + \begin{pmatrix}
			0 & 1 + tx^2 \\ 0 & 0
		\end{pmatrix}\pvec{X}{Y},\quad X = xY
	\end{split}\end{equation*}

	次の二つの$((\fukuso))_t$-連立多項式を考えると、
	それぞれの解$X^{(0\pm)}\in((\fukuso))_t^2$は唯一つ定まる
	\footnote{
		この代数式の書き換えはCole-Hopf変換に似ている。
		ポテンシャルが$x_\pm$、波動関数が$X^{(0\pm)}$に対応する。
	}。
	\begin{equation}\label{eq:多項式の線形化その零}\begin{split}
		X^{(0\pm)} = \pvec{0}{1} + \begin{pmatrix}
			0 & 1 + tx_\pm^2 \\ 0 & 0
		\end{pmatrix} X^{(0\pm)}
		\implies X^{(0\pm)} = \pvec{x_\pm}{1}
	\end{split}\end{equation}
	そして、この連立多項式を書き換えた次の二つの$[\fukuso]_t$-連立多項式も、
	それぞれの解$\what{X}^{(0\pm)}\in((\fukuso))_t^2$は唯一つ定まる。
	\begin{equation}\label{eq:多項式の線形化その零改}\begin{split}
		\what{X}^{(0\pm)} = \pvec{0}{1} + \begin{pmatrix}
			0 & 1 + t\what{X}^{(0\pm)}_1x_\pm \\ 0 & 0
		\end{pmatrix} \what{X}^{(0\pm)}
		\implies \what{X}^{(0\pm)} = \pvec{x_\pm}{1}
	\end{split}\end{equation}

	\begin{todo}[妄想あるいは]\label{todo:妄想あるいは} %{
		多項式の書き換え、\eqref{eq:多項式の線形化その零}
		から\eqref{eq:多項式の線形化その零改}、が本質的な気がしてきた。
		$x_\pm=1+tx_\pm^2$というのがゲージ変換に見えてきた。
	\end{todo} %todo:妄想あるいは}

	次の二つの$((\fukuso\Eta_1))_t$-連立多項式を考える。
	\begin{equation}\label{eq:多項式の線形化その一}\begin{split}
		X^{(1\pm)} = \pvec{0}{1} + T_{1\pm}X^{(1\pm)} 
		\quad\text{where } T_{1\pm} := \begin{pmatrix}
			t\eta_1 & 1 \\ 0 & \eta_{-1}x_\pm
		\end{pmatrix}
	\end{split}\end{equation}
	命題\ref{prop:Dyck経路の分割}から次の式が成り立ち、
	\begin{equation*}\begin{split}
		\braket{T_{1\pm}^*} = 1 + \begin{pmatrix}
			0 & 1 \\ 0 & 0
		\end{pmatrix}\braket{T_{1\pm}^*} + \begin{pmatrix}
			t & 0 \\ 0 & 0
		\end{pmatrix}\braket{T_{1\pm}^*}\begin{pmatrix}
			0 & 0 \\ 0 & x_\pm
		\end{pmatrix}\braket{T_{1\pm}^*}
	\end{split}\end{equation*}
	真空期待値は$((\fukuso))_t$-連立多項式\eqref{eq:多項式の線形化その零改}
	を再現する。
	\begin{equation*}\begin{split}
		\braket{X^{(1\pm)}} = \pvec{0}{1} + \begin{pmatrix}
			0 & 1 + t\braket{X^{(1\pm)}_1}x_\pm \\ 0 & 0
		\end{pmatrix} \braket{X^{(1\pm)}}
	\end{split}\end{equation*}
	ただし、$((\fukuso))_t$-連立多項式\eqref{eq:多項式の線形化その零改}の場合
	と異なり、遷移行列$T_{1\pm}$に$0$でない対角成分があるために、
	$\braket{T_{1\pm}^*}$が収束しない可能性がある。実際に$\braket{T_{1\pm}^*}$
	を計算すると次のようになり、
	\begin{equation*}\begin{split}
		\braket{T_{1\pm}^*} = 1 + \begin{pmatrix}
			0 & \plr{tx_\pm}^* \\ 0 & 0
		\end{pmatrix} \implies \braket{X^{(1\pm)}} = \pvec{\plr{tx_\pm}^*}{1}
	\end{split}\end{equation*}
	$\braket{T_{1\pm}^*}$が収束するためには、次の条件が必要なことがわかる。
	\begin{equation*}\begin{split}
		\plr{tx_\pm}^*\in\fukuso((t))
		\iff |tx_\pm| < 1
		\iff |1\pm\sqrt{1-4t}| < 2
	\end{split}\end{equation*}
	この条件は次の式を反映したものだから、
	\begin{equation*}\begin{split}
		x = 1 + tx^2 \iff x = \frac{1}{1 - tx}
		\implies x = \plr{tx}^* \quad\text{iff}\quad |tx| < 1
	\end{split}\end{equation*}
	この条件が満たされるときに限り、$x_\pm=\plr{tx_\pm}^*$となる。
	つまり、
	\begin{itemize}\setlength{\itemsep}{-1mm} %{
		\item 真空期待値$\braket{X^{(1\pm)}}$が収束するならば、
		$\braket{X^{(1\pm)}}=(x_\pm,\; 1)^\tran$となる
	\end{itemize} %}
	ということが言える。式で書くと次のようになる。
	\begin{equation}\label{eq:多項式の収束条件その一}\begin{split}
		\braket{X^{(1\pm)}}\in((\fukuso))_t
		\implies \braket{X^{(1\pm)}} = \pvec{x_\pm}{1}
	\end{split}\end{equation}
%	そして、$\braket{X^{(1\pm)}}\in((\fukuso))_t$となるときは、
%	$((\fukuso\Eta_1))_t$-連立多項式\eqref{eq:多項式の線形化その一}を
%	次のように、$[\fukuso\Eta_1]_t$-連立多項式に書き換えることができる。
%	\begin{equation}\label{eq:多項式の線形化その一改}\begin{split}
%		\what{X}^{(1)} = \pvec{0}{1} + \what{T}_{1}\what{X}^{(1)}
%		\quad\text{where } \what{T}_{1} := \begin{pmatrix}
%			t\eta_1 & 1 \\ 0 & \eta_{-1}\braket{\what{X}^{(1)}_1}
%		\end{pmatrix}
%	\end{split}\end{equation}
%	この連立多項式には$x_\pm$が現れないので、$t$の多項式を係数とすることに
%	注意する。

	次の$[\fukuso\Eta_2]_t$-連立多項式を考える。
	\begin{equation}\label{eq:多項式の線形化その二}\begin{split}
		X^{(2)} = \pvec{0}{1} + T_{2}X^{(2)} 
		\quad\text{where } T_{2} := \begin{pmatrix}
			t\eta_1 & 1 \\ \eta_{-1}\eta_2 & \eta_{-2}
		\end{pmatrix}
	\end{split}\end{equation}
	命題\ref{prop:Dyck経路の分割}から、部分真空期待について次の式が成り立ち、
	\begin{equation*}\begin{split}
		\braket{T_{2}^*}_1 = 1 + \begin{pmatrix}
			t\eta_1 & 1 \\ 0 & 0
		\end{pmatrix}\braket{T_{2}^*}_1 + I_1\begin{pmatrix}
			0 & 0 \\ \eta_{-1} & 0
		\end{pmatrix}T_{2}^*I_0\begin{pmatrix}
			0 & 0 \\ 0 & 1
		\end{pmatrix}T_{2}^*I_1 
	\end{split}\end{equation*}
	$((\fukuso\Eta_1))_t$-連立多項式\eqref{eq:多項式の線形化その一}に似ている
	が異なる次の$[\fukuso\Eta_1]_t$-連立多項式が得られる。
	\begin{equation*}\begin{split}
		Y^{(1)} = \pvec{0}{1} + \begin{pmatrix}
			t\eta_1 & 1 \\ 0 & \eta_{-1}Y^{(1)}_1I_0
		\end{pmatrix} Y^{(1)}
		\quad\text{where } Y^{(1)} := \braket{X^{(2)}}_1
	\end{split}\end{equation*}

	\begin{todo}[多項式の係数を明示]\label{todo:多項式の係数を明示} %{
	\end{todo} %todo:多項式の係数を明示}

	\begin{todo}[以下は怪しい]\label{todo:以下は怪しい} %{
		結果は正しいが、議論はおかしい。
	\end{todo} %todo:以下は怪しい}

	この連立多項式と連立多項式\eqref{eq:多項式の線形化その一改}は異なるが、
	真空期待値をとると同じ連立多項式となる。
	\begin{equation*}\begin{split}
		\Braket{X^{(1)}} &= \pvec{0}{1} + \Braket{\begin{pmatrix}
			t\eta_1 & 1 \\ 0 & 0
		\end{pmatrix} X^{(1)}} \\
		\Braket{\braket{X^{(2)}}_1} &= \pvec{0}{1} + \Braket{\begin{pmatrix}
			t\eta_1 & 1 \\ 0 & 0
		\end{pmatrix} \braket{X^{(2)}}_1}
	\end{split}\end{equation*}

	\begin{equation*}\begin{split}
		\braket{T_{2}^*} = 1 + \Braket{\begin{pmatrix}
			t\eta_1 & 1 \\ 0 & 0
		\end{pmatrix}T_{2}^*} + \Braket{\begin{pmatrix}
			0 & 0 \\ \eta_{-1} & 0
		\end{pmatrix}T_{2}^*}\Braket{\begin{pmatrix}
			0 & 0 \\ 0 & 1
		\end{pmatrix}T_{2}^*}
	\end{split}\end{equation*}

	部分真空期待値として$\fukuso\Eta_1((t))$上の連立代数式
	\eqref{eq:多項式の線形化その一改}が再現される。
	\begin{equation*}\begin{split}
		\braket{X^{(2)}}_1 = \pvec{0}{1} + \begin{pmatrix}
			t\eta_1 & 1 \\ 0 & \eta_{-1}\braket{X^{(2)}_1}_1
		\end{pmatrix} \braket{X^{(2)}}_1
	\end{split}\end{equation*}
	$\braket{X^{(2)}}$を具体的に計算してみたいところだが、$T_2^*$は次の
	ようになり、
	\begin{equation*}\begin{split}
		T_2^* = U_1\plr{1 + U_2}U_3^* \quad\text{where }
		U_1 = \begin{pmatrix}
			\plr{t\eta_1}^* & 0 \\ 0 & \eta_{-2}^*
		\end{pmatrix},\quad U_2 = \begin{pmatrix}
			0 & \eta_{-2}^* \\ \eta_{-1}\eta_2\plr{t\eta_1}^* & 0
		\end{pmatrix} \\
		U_3 = \begin{pmatrix}
			\eta_{-2}^*\eta_{-1}\eta_2\plr{t\eta_1}^* & 0 \\
			0 & \eta_{-1}\plr{\eta_2\plr{t\eta_1}^* + \eta_{-2}^*} \\
		\end{pmatrix}
	\end{split}\end{equation*}
	収束性を議論できるところまで持っていくことが困難である。
	したがって、$\braket{T_2^*}$の具体的な計算をせずにその収束性を考える。
	$T_2$は$t$の一次式だから、$\braket{T_2^*}$は$t$の形式級数となる。
	\begin{equation*}\begin{split}
		\braket{T_2^*}\in\fukuso[[t]]\subset\fukuso((t))
		\implies \braket{X^{(2)}}\in\fukuso[[t]]
	\end{split}\end{equation*}
	したがって、真空期待値が収束するための条件は、$\fukuso\Eta_1((t))$上の
	連立多項式の場合\eqref{eq:多項式の収束条件その一}より更に厳しく、
	次のようになる。
	\begin{equation*}\begin{split}
		\braket{X^{(2)}}\in\fukuso[[t]]
		\implies \braket{X^{(2)}} = \pvec{x_-}{1}
	\end{split}\end{equation*}
	$x_+$は$t=0$近傍で正則でないので、この連立多項式
	\eqref{eq:多項式の線形化その二}の根に含まれない。

	\begin{todo}[摂動展開の基点]\label{todo:摂動展開の基点} %{
		ここでの議論は$t=0$近傍で摂動展開しているが、任意の複素数$t_0\in\fukuso$
		を基点に摂動展開できるはずである。
	\end{todo} %todo:摂動展開の基点}
%s2:観察その一}
\subsection{観察その二}\label{s2:観察その二} %{
	前節の$\fukuso\Eta_2((t))$上の連立代数式\eqref{eq:多項式の線形化その二}
	は次のもので、
	\begin{equation*}\begin{split}
		X^{(2)} = \pvec{0}{1} + T_{2}X^{(2)} 
		\quad\text{where } T_{2} := \begin{pmatrix}
			t\eta_1 & 1 \\ \eta_{-1}\eta_2 & \eta_{-2}
		\end{pmatrix}
	\end{split}\end{equation*}
	部分真空期待は次の式を満たす。
	\begin{equation*}\begin{split}
		\braket{T_{2}^*}_1 = 1 + \begin{pmatrix}
			t\eta_1 & 1 \\ 0 & 0
		\end{pmatrix}\braket{T_{2}^*}_1 + I_1\begin{pmatrix}
			0 & 0 \\ \eta_{-1} & 0
		\end{pmatrix}T_{2}^*\ket{1}\bra{1}\begin{pmatrix}
			0 & 0 \\ 0 & 1
		\end{pmatrix}T_{2}^*I_1
	\end{split}\end{equation*}
	ここで、次の式が成り立つから、
	\begin{equation*}\begin{split}
		\begin{pmatrix}0 & 1\end{pmatrix}\bra{1}T_{2} = 0
	\end{split}\end{equation*}
	部分真空期待は次のようになる。
	\begin{equation*}\begin{split}
		\braket{T_{2}^*}_1 = 1 + \begin{pmatrix}
			t\eta_1 & 1 \\ 0 & 0
		\end{pmatrix}\braket{T_{2}^*}_1 + I_1\begin{pmatrix}
			0 & 0 \\ \eta_{-1} & 0
		\end{pmatrix}T_{2}^*\ket{1}\bra{1}\begin{pmatrix}
			0 & 0 \\ 0 & 1
		\end{pmatrix}I_1
	\end{split}\end{equation*}
%s2:観察その二}
\subsection{バックアップ}\label{s2:バックアップ} %{
	Dyck経路の分割\ref{eq:Dyck経路の分割}を行列に
	応用してみる。$V$を環、$K\in\sizen$をBrzozowski代数の階数、
	$D\in\sizen_+$を$V$上の自由加群の次元、
	\begin{itemize}\setlength{\itemsep}{-1mm} %{
		\item $X_0\in V^D$を初期値、
		\item $Y\in\plr{V\Eta_K}^D$をあるベクトル、
		\item $A,B\in\Mat\plr{V\Eta_K,D}$を行列
	\end{itemize} %}
	として、$X\in\plr{V\Eta_{K+1}}^{K+1}$を次の代数式の解とすると、
	\begin{equation*}\begin{split}
		X = X_0 + \plr{A + B\eta_{K+1} + \eta_{-\plr{K+1}}X_0Y^\tran}X
	\end{split}\end{equation*}
	Dyck経路の分割\ref{eq:Dyck経路の分割}により、
	次の$\plr{V\Eta_{K+1}}^{K}$の代数式が得られる。
	\begin{equation*}\begin{split}
		[X]_K = X_0 + A[X]_K + B[X]_K\ket{1}\bra{1}Y^\tran[X]_K
	\end{split}\end{equation*}
%s2:バックアップ}
%s1:Dyck経路の分割}
\section{木の成長とDyck経路の成長}\label{s1:木の成長とDyck経路の成長} %{
	Brzozowski代数によるDyck経路の成長$g:=b\eta\plr{\eta^\flat c}^*$を、
	$q=0$の平面二分木の成長に対応させてみる。

	一次、$\bra{1}\mapsto\bra{1}g$、は次のように、
	\begin{equation*}\begin{split}
		\sbt &\mapsto \smallxy{
			& \sbt \hen[dl] \\
			\sbt \\
		} +  \smallxy{
			& \sbt \hen[dl] \hen[dr] \\
			\sbt & & \sbt \\
		}
	\end{split}\end{equation*}
	二次、$\bra{1}g\mapsto\bra{1}g^2$、は次のように、
	\begin{equation*}\begin{split}
		\smallxy{
			& \sbt \hen[dl] \\
			\sbt \\
		} &\mapsto \smallxy{
			& & \sbt \hen[dl] \\
			& \sbt \hen[dl] \\
			\sbt \\
		} + \smallxy{
			& & \sbt \hen[dl] \\
			& \sbt \hen[dl] \hen[dr] \\
			\sbt & & \sbt \\
		} + \smallxy{
			& & \sbt \hen[dl] \hen[dr] \\
			& \sbt \hen[dl] \hen[dr] & & \sbt \\
			\sbt & & \sbt \\
		} \\
		\smallxy{
			& \sbt \hen[dl] \hen[dr] \\
			\sbt & & \sbt \\
		} &\mapsto \smallxy{
			& \sbt \hen[dl] \hen[dr] \\
			\sbt & & \sbt \hen[dl] \\
			& \sbt \\
		} + \smallxy{
			& \sbt \hen[dl] \hen[dr] \\
			\sbt & & \sbt \hen[dl] \hen[dr] \\
			& \sbt & & \sbt \\
		}
	\end{split}\end{equation*}
	三次、$\bra{1}g^2\mapsto\bra{1}g^3$、は次のようになる。
	\begin{equation*}\begin{array}{rcccccccc}
		\smallxy{
			& & \sbt \hen[dl] \\
			& \sbt \hen[dl] \\
			\sbt \\
		} &\mapsto& \smallxy{
			& & & \sbt \hen[dl] \\
			& & \sbt \hen[dl] \\
			& \sbt \hen[dl] \\
			\sbt \\
		} &+& \smallxy{
			& & & \sbt \hen[dl] \\
			& & \sbt \hen[dl] \\
			& \sbt \hen[dl] \hen[dr] \\
			\sbt & & \sbt \\
		} &+& \smallxy{
			& & & \sbt \hen[dl] \\
			& & \sbt \hen[dl] \hen[dr] \\
			& \sbt \hen[dl] \hen[dr] & & \sbt \\
			\sbt & & \sbt \\
		} &+& \smallxy{
			& & & \sbt \hen[dl] \hen[dr] \\
			& & \sbt \hen[dl] \hen[dr] & & \sbt \\
			& \sbt \hen[dl] \hen[dr] & & \sbt \\
			\sbt & & \sbt \\
		} \\
		\smallxy{
			& & \sbt \hen[dl] \\
			& \sbt \hen[dl] \hen[dr] \\
			\sbt & & \sbt \\
		} &\mapsto& \smallxy{
			& & \sbt \hen[dl] \\
			& \sbt \hen[dl] \hen[dr] \\
			\sbt & & \sbt \hen[dl] \\
			& \sbt
		} &+& \smallxy{
			& & \sbt \hen[dl] \\
			& \sbt \hen[dl] \hen[dr] \\
			\sbt & & \sbt \hen[dl] \hen[dr] \\
			& \sbt & & \sbt
		} &+& \smallxy{
			& & \sbt \hen[dl] \hen[dr] \\
			& \sbt \hen[dl] \hen[dr] & & \sbt \\
			\sbt & & \sbt \hen[dl] \hen[dr] \\
			& \sbt & & \sbt
		} \\
		\smallxy{
			& \sbt \hen[dl] \hen[dr] \\
			\sbt & & \sbt \hen[dl] \\
			& \sbt \\
		} &\mapsto& \smallxy{
			& \sbt \hen[dl] \hen[dr] \\
			\sbt & & \sbt \hen[dl] \\
			& \sbt \hen[dl] \\
			\sbt \\
		} &+& \smallxy{
			& \sbt \hen[dl] \hen[dr] \\
			\sbt & & \sbt \hen[dl] \\
			& \sbt \hen[dl] \hen[dr] \\
			\sbt & & \sbt \\
		} &+& \smallxy{
			& \sbt \hen[dl] \hen[dr] \\
			\sbt & & \sbt \hen[dl] \hen[dr] \\
			& \sbt \hen[dl] \hen[dr]  & & \sbt \\
			\sbt & & \sbt \\
		} \\
		\smallxy{
			& & \sbt \hen[dl] \hen[dr] \\
			& \sbt \hen[dl] \hen[dr] & & \sbt \\
			\sbt & & \sbt \\
		} &\mapsto& \smallxy{
			& & \sbt \hen[dl] \hen[dr] \\
			& \sbt \hen[dl] \hen[d] & & \sbt \hen[d] \\
			\sbt & \sbt & & \sbt \\
		} &+& \smallxy{
			& & \sbt \hen[dl] \hen[dr] \\
			& \sbt \hen[dl] \hen[d] & & \sbt \hen[d] \hen[dr] \\
			\sbt & \sbt & & \sbt & \sbt \\
		} \\
		\smallxy{
			& \sbt \hen[dl] \hen[dr] \\
			\sbt & & \sbt \hen[dl] \hen[dr] \\
			& \sbt & & \sbt \\
		} &\mapsto& \smallxy{
			& \sbt \hen[dl] \hen[dr] \\
			\sbt & & \sbt \hen[dl] \hen[dr] \\
			& \sbt & & \sbt \hen[dl] \\
			& & \sbt \\
		} &+& \smallxy{
			& \sbt \hen[dl] \hen[dr] \\
			\sbt & & \sbt \hen[dl] \hen[dr] \\
			& \sbt & & \sbt \hen[dl] \hen[dr] \\
			& & \sbt & & \sbt \\
		} \\
	\end{array}\end{equation*}
	成長$\bra{1}g^n\mapsto\bra{1}g^{n+1}$を平面二分木の言葉で書くと次のように
	なるだろう。
	\begin{itemize}\setlength{\itemsep}{-1mm} %{
		\item 行きがけ順で最後の葉$p$を見つける。
		\item $p$に左の子供を付け足す。
		\item 行きがけ順で$p$の後ろの頂点に右の子供を付け足す。
	\end{itemize} %}
	次のように対応する。
	\begin{equation*}\begin{array}{rcccccccc}
		\bra{1}(b\eta)(bc)b\eta(\eta^\flat c)^*
		&=& \bra{1}(b\eta)(bc)(b\eta) &+& \bra{1}(b\eta)(bc)(bc)
		&+& \bra{1}(b)(bc)(bc^2) \\
		\smallxy{
			& & \sbt \hen[dl] \\
			& \sbt \hen[dl] \hen[dr] \\
			\sbt & & \sbt \\
		} &\mapsto& \smallxy{
			& & \sbt \hen[dl] \\
			& \sbt \hen[dl] \hen[dr] \\
			\sbt & & \sbt \hen[dl] \\
			& \sbt
		} &+& \smallxy{
			& & \sbt \hen[dl] \\
			& \sbt \hen[dl] \hen[dr] \\
			\sbt & & \sbt \hen[dl] \hen[dr] \\
			& \sbt & & \sbt
		} &+& \smallxy{
			& & \sbt \hen[dl] \hen[dr] \\
			& \sbt \hen[dl] \hen[dr] & & \sbt \\
			\sbt & & \sbt \hen[dl] \hen[dr] \\
			& \sbt & & \sbt
		} \\
	\end{array}\end{equation*}
%s1:木の成長とDyck経路の成長}
%
}\endgroup %}
