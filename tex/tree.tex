\section{木} %{
	この節で用いる約束を挙げておく。この節では、これらの約束を断りなく使う。
	\section{文字列の基本事項}\label{s1:文字列の基本事項} %{
	$R=(R,+,0,\myspace,1)$を自明でない可換半環とする。

	任意の集合$A$に対して$WA=(WA,m_*,1_W)$を$A$から生成された自由モノイド
	とする。ここで、積$m_*$は文字列の連結で定義された積とし、連結積と書き、
	中置記法で$*$とも書く。また、$1_W$は文字数$0$の単語とする。
	$WA$の元を$A$の元を並べたものを括弧でくくって表すことにする。
	例えば、$a_1,a_2,\dots, a_n\in A$を並べた$WA$の元を$[a_1a_2\cdots a_n]$
	と書く。
	任意の$n\in\mybf{N}$に対して$W_nA\subseteq WA$を文字数$n$の単語の集合と
	する。つまり、$WA=\oplus_{n\in N}W_nA$となる。$RW_nA$を$W_nA$を基底とする
	$R$係数自由半モジュールとする。やはり、$RWA=\oplus_{n\in N}RW_nA$となる。
	積$m_*$は双線形写像$m_*:RW_mA\otimes RW_nA\to RW_{m+n}A$として
	みることもできる。

	任意の集合$A$に対して$RA$を$A$を基底とする$R$係数自由半モジュールとする。

	任意の集合$A$に対して、次の写像$i_W$を文字列への
	標準埋め込み(canonical injection)という。
	\begin{equation*}\begin{split} %{
		i_W: A\to WA,\quad a\mapsto [a] \quad\text{for all }a\in A
	\end{split}\end{equation*} %}
	標準埋め込みを線形写像に拡張したものも$i_W:RA\to RWA$と書く。

	任意の集合$A$に対して、次の写像$i_R$を自由半モジュールへの
	標準埋め込み(canonical injection)という。
	\begin{equation*}\begin{split} %{
		i_R: A\to RA,\quad a\mapsto a \quad\text{for all }a\in A
	\end{split}\end{equation*} %}
	標準埋め込みについて次の命題が成り立つ。

	\begin{proposition}[自由半代数の普遍性]\label{prop:自由半代数の普遍性} %{
		$A$を集合、$V$を$R$半代数とする。
		\begin{itemize}\setlength{\itemsep}{-1mm} %{
			\item $RA$から$V$への線形写像全体$\homset(RA,V)$と、
			\item $RWA=(RWA,m_*,1_W)$から$V$への$R$半代数射全体$\homset(RWA,V)$
		\end{itemize} %}
		は集合同型となり、その同型$\phi:\homset(RA,V)\to\homset(RWA,V)$は次の
		可換図によって定めることができる。
		\begin{equation*}\xymatrix{
			RA \ar[r]^{i_W} \ar[rd]_{f} & RWA \ar@{.>}[d]^{\phi f} \\
			& V \\
		}\end{equation*}
	\end{proposition} %prop:自由半代数の普遍性}
	\begin{proof} %{
		$V$の積を$m_\myspace$、単位元を$1_V$とする。
		任意の$f\in\homset(RA,V)$に対して$(\phi f):RWA\to V$を次のように
		定義すると、
		\begin{equation*}\begin{split} %{
			(\phi f)1_W &= 1_V \\
			(\phi f)[a_1a_2\cdots a_m]_W &= (fa_1)(fa_2)\cdots(fa_m)
			\quad\text{for all }a_1,a_2,\dots,a_m\in A
		\end{split}\end{equation*} %}
		$\phi f$は$R$半代数射になり命題の可換図を満たす。
		逆に、任意の$R$半代数射$g\in\homset(RWA,V)$に対して$\phi^{-1}g$を
		$(\phi^{-1}g)a=g[a]$と定義すると、命題の可換図を満たす。
	\end{proof} %}

	テンソル積同士の積を$R$半モジュールに対して定義する。
	$V$を$R$半モジュールとする。
	$V$の自己線形写像全体を$\myop{end}V$または$\homset(V,V)$と書く。
	任意の$\phi\in\myop{end}V$は次の性質を満たす。
	\begin{equation*}\begin{array}{rcll} %{
		\phi(v_1+f_2) &=& (\phi v_1) + (\phi f_2)
			&\text{for all } v_1,v_2\in V \\
		\phi(rv) &=& r(\phi v) &\text{for all }r\in R,\;f\in V \\
	\end{array}\end{equation*} %}
	$V$に双線形二項演算$\beta_\square$が定義された時、
	線形写像$\myhere\square:R\to\myop{end}V$が次のように定義できる。
	\begin{equation*}\begin{split} %{
		(v_1\square)v_2 = v_1\square v_2 \quad\text{for all } v_1,v_2\in V
	\end{split}\end{equation*} %}
	$\beta_\square$が$1_\square$を単位元とする積$m_\square$であれば、
	写像$\myhere\square$は$R$半代数射
	$(V,m_\square,1_\square)\to(\myop{end}V,m_\myspace,\myid)$となる。
	ここで、$\myop{end}V$の積$m_\myspace$は写像の合成である。
	このとき、$V\square\subseteq\myop{end}V$の計算は、
	$(V,m_\square,1_\square)$の計算に翻訳できる。このことが、積を考える
	動機の一つである。

	$\beta_\square$を$V$の双線形二項演算とする。
	テンソル積に対する双線形二項演$\beta_\square$を、
	中置記法$\myhere\square\myhere$で次のように定義する。
	\begin{equation}\begin{split} %{
		&(v_{11}\otimes v_{12}\otimes\cdots\otimes v_{1m})
			\square(v_{21}\otimes v_{22}\otimes\cdots\otimes v_{2m}) \\
		&= (v_{11}\square v_{21})\otimes (v_{12}\square v_{22})\otimes\cdots
			\otimes (v_{1m}\square v_{2m}) \\
		&\quad\text{for all }v_{11},v_{12},\dots,v_{1m},v_{21},v_{22},\dots
			,v_{2m}\in V
	\end{split}\end{equation} %}
	テンソル積をベクトルのように縦に並べて書くと、$\myhere\square\myhere$は
	次のようになる。
	\begin{equation*}\begin{split} %{
		\begin{pmatrix}
			v_{11}\\ v_{12}\\ \vdots\\ v_{1m}
		\end{pmatrix}\square \begin{pmatrix}
			v_{21}\\ v_{22}\\ \vdots\\ v_{2m}
		\end{pmatrix} = \begin{pmatrix}
			v_{11}\square v_{21}\\ v_{12}\square v_{22}\\
			\vdots\\ v_{1m}\square v_{2m}
		\end{pmatrix}
	\end{split}\end{equation*} %}
	この定義は、$V$から$\myop{end}V$への線形写像$\myhere\square$を
	次のように定義したとき、
	\begin{equation*}\begin{split} %{
		(v_1\square)v_2 = v_1\square v_2 \quad\text{for all } v_1,v_2\in V
	\end{split}\end{equation*} %}
	積$m_\square$をとることと、線形写像$\myhere\square$を作用させることが、
	次の式のように同じ形で書けるようにしている。
	\begin{equation}\begin{split} %{
		&\bigl((v_{11}\square)\otimes(v_{12}\square)\otimes\cdots
			\otimes(v_{1m}\square)\bigr)
			(v_{21}\otimes v_{22}\otimes\cdots\otimes v_{2m}) \\
		&= (v_{11}\square v_{21})\otimes (v_{12}\square v_{22})\otimes\cdots
			\otimes (v_{1m}\square v_{2m}) \\
		&=(v_{11}\otimes v_{12}\otimes\cdots\otimes v_{1m})
			\square(v_{21}\otimes v_{22}\otimes\cdots\otimes v_{2m}) \\
		&\quad\text{for all }v_{11},v_{12},\dots,v_{1m},v_{21},v_{22},\dots
			,v_{2m}\in V
	\end{split}\end{equation} %}

	'ケース文'を簡潔に書くために、デルタ関数'$\jump{\mathchar`-}$を定義して
	おく。論理値$\mybf{B}=\set{0_{\mybf{B}},1_{\mybf{B}}}$から半環
	$R=(R,+,0_R,\myspace,1_R)$への写像$\jump{\mathchar`-}:\mybf{B}\mapsto R$
	を次のように定義する。
	\begin{equation*}\begin{split} %{
		0_{\mybf{B}}\mapsto 0_R,\quad 1_{\mybf{B}}\mapsto 1_R
	\end{split}\end{equation*} %}
%s1:文字列の基本事項}

\section{内積}\label{s1:内積} %{
	この節では、係数を複素数$\mybf{C}=(\mybf{C},m_+,0,m_\myspace,1)$に
	固定する。複素共役を$c\in\mybf{C}$に対して$c^\dag$と書く。

	$A$を集合とする。$\mybf{C}A$から$\mybf{C}$への線形写像全体の
	作る空間を$(\mybf{C}A)^\dag$と書く。
	$\mybf{C}A$の基底$a\in A$に双対な$(\mybf{C}A)^\dag$の基底を$a^\dag$と
	書く。
	\begin{equation*}\begin{split} %{
		a_1^\dag a_2 = \jump{a_1=a_2} \quad\text{for all }a_1,a_2\in A
	\end{split}\end{equation*} %}
	写像$\myhere^\dag:\mybf{C}A\to(\mybf{C}A)^\dag$を次のように定義する。
	\begin{equation*}\begin{array}{rcll} %{
		(f_1+f_2)^\dag &=& f_1^\dag + f_2^\dag
			& \quad\text{for all }f_1,f_2\in \mybf{C}A \\
		(cf)^\dag &=& c^\dag f^\dag
			& \quad\text{for all }c\in \mybf{C},\;f\in \mybf{C}A \\
		a^\dag &=& a^\dag & \quad\text{for all }a\in A \\
	\end{array}\end{equation*} %}
	さらに、線形写像$\myhere^\dag:\mybf{C}A^\dag\to\mybf{C}A$を
	同様に定義して、$\myhere^\dag\myhere^\dag=\myid$となるようにする。

	内積は複素共役に対して対称になる。
	\begin{equation*}\begin{split} %{
		(f_1^\dag f_2)^\dag = f_2^\dag f_1 
		\quad\text{for all }f_1,f_2\in\mybf{C}A
	\end{split}\end{equation*} %}
	\begin{proof} %{
		$i=1,2$に対して$f_i=\sum_{a\in A}f_{ia}a\in\mybf{C}A
		,\;f_{ia}\in\mybf{C}\quad\text{for all }a\in A$とする。
		すると、$(f_1^\dag f_2)^\dag=\sum_{a\in A}(f_{1a}^\dag f_{2a})^\dag
		=f_2^\dag f_1=f_2^\dag f_1$となる。
	\end{proof} %}

	$\mybf{C}A\otimes \mybf{C}A$から$\mybf{C}$への線形写像全体の
	作る空間を$(\mybf{C}A\otimes\mybf{C}A)^\dag$と書く。
	$\mybf{C}A\otimes \mybf{C}A$の基底全体の集合を$A\otimes A$と書く。
	任意の$a_1\otimes a_2\in A\otimes A$に双対な元を
	$(a_1\otimes a_2)^\dag\in(A\otimes A)^\dag$と書く。
	\begin{equation*}\begin{split} %{
		(a_1\otimes a_2)^\dag(a_3\otimes a_4) = (a_1^\dag a_3)(a_2^\dag a_4)
		\quad\text{for all }a_1,a_2,a_3,a_4\in A
	\end{split}\end{equation*} %}
	写像$\myhere^\dag:
	\mybf{C}A\times\mybf{C}A\to(\mybf{C}A\otimes\mybf{C}A)^\dag$を
	次のように定義する。
	\begin{equation*}\begin{array}{rcll} %{
		(f_1+f_2)^\dag &=& f_1^\dag + f_2^\dag
			& \quad\text{for all }f_1,f_2\in \mybf{C}A\otimes\mybf{C}A \\
		(cf)^\dag &=& c^\dag f^\dag
			& \quad\text{for all }c\in \mybf{C},\;f\in\mybf{C}A\otimes\mybf{C}A \\
		a^\dag &=& a^\dag & \quad\text{for all }a\in A\otimes A \\
	\end{array}\end{equation*} %}
	さらに、線形写像$\myhere^\dag:
	(\mybf{C}A\times\mybf{C}A)^\dag\to\mybf{C}A\otimes\mybf{C}A$を同様に
	定義して、$\myhere^\dag\myhere^\dag=\myid$となるようにする。
	二次以上のテンソル積についても同様に定義する。

	$(\mybf{C}A\times\mybf{C}A)^\dag$と
	$(\mybf{C}A)^\dag\otimes(\mybf{C}A)^\dag$は、
	$\mybf{C}\simeq\mybf{C}\otimes\mybf{C}$という同一視で、
	次のような同一視をする。
	\begin{equation*}\begin{array}{cccl} %{
		(a_1\otimes a_2)^\dag(a_3\otimes a_4) 
			&\simeq& (a_1^\dag\otimes a_2^\dag)(a_3\otimes a_4) \\
		|| && || & \text{for all }a_1,a_2\in A \\
		(a_1^\dag a_3)(a_2^\dag a_4) 
			&\simeq& (a_1^\dag a_3)\otimes (a_2^\dag a_4) \\
	\end{array}\end{equation*} %}

	\begin{note}[無限の場合]\label{note:無限の場合} %{
		集合$A$が無限集合の場合、$\mybf{C}A$と$\mybf{C}A^\dag$が集合同型
		でなくなるそうだ。$\mybf{C}A$と$\mybf{C}A^\dag$が集合同型でないと、
		$1:1$かつ$\myop{onto}$の関係を使って使って転置を定義することができなく
		なる。ここではまず、有限の場合に成り立つことをナイーブに無限に拡張して、
		成り立つ関係式を列挙する。
	\end{note} %note:無限の場合}

	\begin{example}[無限次元で双対空間との同型対応が破綻する例]\label{eg:無限次元で双対空間との同型対応が破綻する例} %{
		有理数を係数とする多項式全体の集合を$\mybf{Q}[x]$とする。
		$\mybf{Q}[x]$の元$f=\sum_{n\in\mybf{N}}\frac{1}{n!}x^n\in\mybf{Q}[x]$
		に対して、その転置と$\mybf{Q}[x]$の元
		$\frac{1}{1-x}=\sum_{n\in\mybf{N}}x^n\in\mybf{Q}[x]$との内積を
		とると、$f^\dag\frac{1}{1-x}=\exp1\not\in\mybf{Q}$となってしまう。
		こうした事態を防ぐために、$\mybf{Q}[x]$を有限化した部分集合
		$\mybf{Q}_0[x]\subset\mybf{Q}[x]$
		\begin{equation*}\begin{split} %{
			\mybf{Q}_0[x] = \set{f\in\mybf{Q}[x]
				\bou (x^n)^\dag f\neq0 \text{ for only finitely many }n\in\mybf{N}
			}
		\end{split}\end{equation*} %}
		のみを考えるという処置がとられる。定数写像が
		$\left(\frac{1}{1-x}\right)^\dag x^n=1\text{ for all }n\in\mybf{N}$
		で与えられるように、無限和を許さないと応用上有効な理論が構築できなく
		なるので、$\mybf{Q}[x]^\dag$はそのまま使って、$\mybf{Q}[x]^\dag$と
		$\mybf{Q}_0[x]$という組で理論を組み立てることが多いようだ。
		このとき、$\myhere^\dag:\mybf{Q}_0[x]\to\mybf{Q}[x]^\dag$は
		定義できるが、$\myhere^\dag:\mybf{Q}[x]^\dag\to\mybf{Q}_0[x]$は単純には
		定義できなくなってしまう。
		\begin{equation*}\begin{split} %{
			\mybf{Q}_0[x]\xrightarrow[1:1\text{ but not }\myop{onto}]
				{\myhere^\dag}\mybf{Q}[x]^\dag
		\end{split}\end{equation*} %}
		その場合でも、次の図を可換にする埋め込み$i_0$が唯一つ存在するから、
		\begin{equation*}\xymatrix{
			\mybf{Q}[x] \ar@{<->}[r]^{\myhere^\dag} & \mybf{Q}[x]^\dag \\
			\mybf{Q}_0[x] \ar[ur]_{\myhere^\dag} \ar@{.>}[u]^{i_0} \\
		}\end{equation*}
		$\myhere^\dag\myhere^\dag=i_0:\mybf{Q}_0[x]\to\mybf{Q}[x]$という形で
		一方通行の冪等性は成り立つ。
	\end{example} %eg:無限次元で双対空間との同型対応が破綻する例}

	$\mybf{C}A=(\mybf{C}A,m,1_m)$を代数とする。積$m$の中置記号は省略する。
	次の畳み込みによって、余積$\Delta:\mybf{C}A\to \mybf{C}A\otimes \mybf{C}A$
	を定義する。
	\begin{equation*}\xymatrix{
		\mybf{C}A\otimes\mybf{C}A \ar[r]^{m} \ar@{.>}[rd]_{(\Delta f)^\dag} 
			& \mybf{C}A \ar[d]^{f^\dag} \\
		& \mybf{C}
	} \quad\text{for all } f\in\mybf{C}A
	\end{equation*}
	式で書くと次のようになる。
	\begin{equation*}\begin{split} %{
		(\Delta f)^\dag = f^\dag m \quad\text{for all }f\in\mybf{C}A
	\end{split}\end{equation*} %}
	$\Delta$が余積になることは、次の式からわかる。
	\begin{equation*}\begin{array}{cccccc} %{
		\bigl((\Delta\otimes \myid)\Delta f\bigr)^\dag
		&=& (\Delta f)^\dag(m\otimes \myid) &=& f^\dag m(m\otimes \myid) \\
		&&&& ||\quad\quad & \quad\text{for all }f\in\mybf{C}A \\
		\bigl((\myid\otimes \Delta)\Delta f\bigr)^\dag
		&=& (\Delta f)^\dag(\myid\otimes m) &=& f^\dag m(\myid\otimes m) \\
	\end{array}\end{equation*} %}

	余積$\Delta$は次のように積$m$の転置となっているから、余積$\Delta$を
	$m^\dag$とも書く。
	\begin{equation*}\begin{split} %{
		(\Delta f)^\dag(f_1\otimes f_2) = f^\dag m(f_1\otimes f_2)
		\quad\text{for all }f,f_1,f_2\in\mybf{C}A
	\end{split}\end{equation*} %}
	$\Delta$は次の式を満たす必要がある。
	\begin{equation*}\begin{split} %{
		f_1^\dag f_2 = f_1^\dag m(1_m\otimes f_2) 
		= (\Delta f_1)^\dag(1_m\otimes f_2)
		\quad\text{for all }f_1,f_2\in\mybf{C}A
	\end{split}\end{equation*} %}
	したがって、$\mybf{C}\otimes\mybf{C} \simeq \mybf{C}$の同一視で、
	次の式が成り立つ必要がある。
	\begin{equation*}\begin{split} %{
		f_2^\dag f_1 \simeq (1_m^\dag\otimes f_2^\dag)\Delta f_1
		\quad\text{for all }f_1,f_2\in\mybf{C}A
	\end{split}\end{equation*} %}
	この式は余積$\Delta$に対する余単位射$1_m^\dag$の定義に他ならない。
	したがって、積$m$から転置によって余積$\Delta$を定義したときには、
	その余単位射は単位元$1_m$の転置$1_m^\dag$になる。

	以上を定義と命題の形にまとめておく。

	\begin{definition}[積の転置]\label{def:積の転置} %{
		$\mybf{C}A=(\mybf{C}A,m,1_m)$を代数とする。
		次の式で定義された余積$\Delta$を積$m$の転置という。
		\begin{equation*}\begin{split} %{
			(\Delta f)^\dag = f^\dag m \quad\text{for all }f\in\mybf{C}A
		\end{split}\end{equation*} %}
	\end{definition} %def:積の転置}

	\begin{proposition}[単位元の転置]\label{prop:単位元の転置} %{
		$\mybf{C}A=(\mybf{C}A,m,1_m)$を代数とする。
		積$m$の転置による余積の余単位射は単位元$1_m$の転置$1_m^\dag$となる。
	\end{proposition} %prop:単位元の転置}

	基底を使って転置による余積を書き表してみる。
	$\Delta$を積$m$の転置による余積とする。
	\begin{equation*}\begin{split} %{
		a^\dag(a_1a_2) = (\Delta a)^\dag(a_1\otimes a_2)
		\quad\text{for all }a,a_1,a_2\in A
	\end{split}\end{equation*} %}
	$\Delta$が具体的に求まり、次のようになる。
	\begin{equation*}\begin{split} %{
		\Delta = \sum_{a_1,a_2\in A}(a_1\otimes a_2)(a_1a_2)^\dag
	\end{split}\end{equation*} %}

	$m\Delta$はエルミートなので対角化可能で、$\Delta$の二乗
	$m\Delta=\Delta^\dag \Delta$なので非負の固有値を持つ。
	$\Delta m$も同様である。
	$m\Delta$と$\Delta m$の固有値について次の命題が成り立つ。

	\begin{proposition}[非ゼロ固有値の対応]\label{prop:非ゼロ固有値の対応} %{
		$0$でない$f_\lambda\in\mybf{C}A$を$m\Delta$の固有値$\lambda$を持つ
		固有ベクトルとする。$\Delta f_\lambda\neq0$ならば、
		\begin{itemize}\setlength{\itemsep}{-1mm} %{
			\item $\Delta f_\lambda$は$\Delta m$の固有値$\lambda$を持ち、
			\item $0<\lambda$となる。
		\end{itemize} %}
	\end{proposition} %prop:非ゼロ固有値の対応}
	\begin{proof} %{
		任意の$f\in\mybf{C}A$に対して二乗ノルムを$\zettai{f}^2=f^\dag f$と書く。

		$\Delta m\Delta f_\lambda=\lambda\Delta f_\lambda$より、
		$\Delta f_\lambda$が$\Delta m$の固有値$\lambda$を持つことがわかる。

		$m\Delta f_\lambda=\lambda f_\lambda$より、
		$\lambda\zettai{f_\lambda}^2=\zettai{\Delta f_\lambda}^2$となるから、
		$0<\zettai{\Delta f_\lambda}^2$より、$0<\lambda$となる。
	\end{proof} %}

	\begin{proposition}[ゼロ固有値の非対応]\label{prop:ゼロ固有値の非対応} %{
		任意の$f_0\in\mybf{C}A$に対して$m\Delta f_0=0\implies\Delta f_0=0$が
		成り立つ。
	\end{proposition} %prop:ゼロ固有値の非対応}
	\begin{proof} %{
		$f_0=0$ならば命題が成り立つことはすぐわかるから、$f_0\neq0$とする。
		$\Delta f_0\neq0$ならば、命題\ref{prop:非ゼロ固有値の対応}より、
		$f_0$は$0<\lambda$となる$m\Delta$の固有値$\lambda$を持つ必要があり、
		仮定に矛盾する。
	\end{proof} %}

	$\Delta m$に対しても同様の命題が成り立つ。

	\begin{proposition}[テンソル積での固有値の対応]\label{prop:テンソル積での固有値の対応} %{
		$0$でない$f_\lambda\in\mybf{C}A\otimes\mybf{C}A$を$\Delta m$の固有値
		$\lambda$を持つ固有ベクトルとする。$\Delta f_\lambda\neq0$ならば、
		\begin{itemize}\setlength{\itemsep}{-1mm} %{
			\item $\Delta f_\lambda$は$\Delta m$の固有値$\lambda$を持ち、
			\item $0<\lambda$となる。
		\end{itemize} %}
		逆に、固有値$\lambda$が$0$ならば、$mf_\lambda=0$となる。
	\end{proposition} %prop:テンソル積での固有値の対応}
	\begin{proof} %{
		命題\ref{prop:非ゼロ固有値の対応}と命題\ref{prop:ゼロ固有値の非対応}
		と同じようにする。
	\end{proof} %}

	命題\ref{prop:非ゼロ固有値の対応}、命題\ref{prop:ゼロ固有値の非対応}
	、命題\ref{prop:テンソル積での固有値の対応}により、$m\Delta$の$0$より
	大きい固有値の分布と$\Delta m$の$0$のより大きい固有値の分布は等しくなる。
	$m\Delta$と$\Delta m$の$0$より大きい固有値の集合を$\Lambda_+$とする。
	$m\Delta$の固有値$\lambda$の固有空間を
	$(\mybf{C}A)_\lambda\subseteq\mybf{C}A$、
	$\Delta m$の固有値$\lambda$の固有空間を
	$(\mybf{C}A\otimes\mybf{C}A)_\lambda\subseteq\mybf{C}A\otimes\mybf{C}A$
	と書く。$\mybf{C}A$と$\mybf{C}A\otimes\mybf{C}A$は、固有空間で
	直和分解されて次のようになる。
	\begin{equation*}\begin{split} %{
		\mybf{C}A 
		&= \oplus_{\lambda\in\set{0}\cup\Lambda_+}(\mybf{C}A)_\lambda \\
		\mybf{C}A\otimes\mybf{C}A
		&= \oplus_{\lambda\in\set{0}\cup\Lambda_+}
			(\mybf{C}A\otimes\mybf{C}A)_\lambda
	\end{split}\end{equation*} %}
	$\lambda\in\Lambda_+$に対して、$\Delta$の定義域を$(\mybf{C}A)_\lambda$に
	制限したものを$\Delta_\lambda$と書く。同様に、$\lambda\in\Lambda_+$
	に対して、$m$の定義域を$(\mybf{C}A\times\mybf{C}A)_\lambda$に制限したもの
	を$m_\lambda$と書く。
	\begin{equation*}\begin{split} %{
		(\mybf{C}A)_\lambda 
		\overset{\Delta_\lambda}{\underset{m_\lambda}{\rightleftarrows}}
		(\mybf{C}A\otimes\mybf{C}A)_\lambda
		\quad\text{for all }\lambda\in\Lambda_+
	\end{split}\end{equation*} %}
	固有値$0$の固有空間の間に$\Delta$と$m$による対応関係はない。

	任意の$\lambda\in\Lambda_+$、$f_1,f_2\in(\mybf{C}A)_\lambda$に対して
	$\Delta f_1=\Delta f_2$ならば、$\Delta(f_1-f_2)=0$となり、
	命題\ref{prop:非ゼロ固有値の対応}より、$f_1=f_2$となる。
	したがって、$\Delta_\lambda$は$1:1$となる。また、
	任意の$\lambda\in\Lambda_+$、
	$f\neq0\in(\mybf{C}A\otimes\mybf{C}A)_\lambda$に対して
	$\Delta mf=\lambda f$となるが、$f=\Delta\left(\frac{1}{\lambda}mf\right)$
	となり、$\Delta_\lambda$は$\myop{onto}$となる。
	以上より、任意の$\lambda\in\Lambda_+$に対して$\Delta_\lambda$は
	$(\mybf{C}A)_\lambda$と$(\mybf{C}A\otimes\mybf{C}A)_\lambda$の
	ベクトル同型を与える。$m_\lambda$がベクトル同型を与えることも同様に
	示される。
	
	以上を命題のかたちでまとめる。

	\begin{proposition}[非ゼロ固有空間の同型射]\label{prop:非ゼロ固有空間の同型射} %{
		任意の$\lambda\in\Lambda_+$に対して、
		$\Delta_\lambda$は次のベクトル同型を与え、
		\begin{equation*}\begin{split} %{
			\Delta_\lambda: (\mybf{C}A)_\lambda
			\xrightarrow{\simeq} (\mybf{C}A\otimes\mybf{C}A)_\lambda
		\end{split}\end{equation*} %}
		$m_\lambda$は次のベクトル同型を与える。
		\begin{equation*}\begin{split} %{
			m_\lambda: (\mybf{C}A\otimes\mybf{C}A)_\lambda
			\xrightarrow{\simeq} (\mybf{C}A)_\lambda
		\end{split}\end{equation*} %}
	\end{proposition} %prop:非ゼロ固有空間の同型射}

	積とその転置の関係の例を見てみる。

	\begin{example}[モノイドの積の転置]\label{eg:モノイドの積の転置} %{
		$G=(G,m,1_G)$をモノイド、$\mybf{C}G=(RG,m,1_G)$を$G$を基底とし、
		$G$の積$m$を線形に拡張した自由$\mybf{C}$代数とする。
		$\Delta$を積$m$の転置とする。
		このとき、$g_1g_2\in G$となり、$\Delta$は次のようになる。
		\begin{equation*}\begin{split} %{
			\Delta g = \sum_{g_1,g_2\in G}\jump{g=g_1g_2}(g_1\otimes g_2)
			\quad\text{for all }g\in G
		\end{split}\end{equation*} %}
		また、$m\Delta$は基底$G$で対角化されていて、次のようになる。
		\begin{equation*}\begin{split} %{
			m\Delta g = (\sharp_m g)g \quad\text{for all }g\in G
		\end{split}\end{equation*} %}
		ここで、線形写像$\sharp_m:G\to \mybf{N}$を次のように定義した。
		\begin{equation*}\begin{split} %{
			\sharp_m g = \sum_{g_1,g_2\in G}\jump{g=g_1g_2}
			\quad\text{for all }g\in G
		\end{split}\end{equation*} %}

		ある$g_0\in G$が$\Delta g_0$となれば、すべての$g_1,g_2\in G$に対して
		$g_0^\dag(g_1g_2)=0$となるから、$g_0$はすべての$G$の元と直交すること
		になり、転置の定義と矛盾する。したがって、$\Delta$は固有値$0$を
		持たない。一方、$m$は次のベクトル$(g_1\otimes g_2)_0\in G\otimes G$を
		$0$固有ベクトルとして持つ。
		\begin{equation*}\begin{split} %{
			(g_1\otimes g_2)_0 = \bigl((\sharp_m g_1g_2)- \Delta m\bigr)
			(g_1\otimes g_2)\quad\text{for all }g_1,g_2\in G
		\end{split}\end{equation*} %}
		ただし、次のようになるから、
		\begin{equation*}\begin{split} %{
			\sum_{g_1,g_2\in G}\jump{g_1g_2=g}(g_1\otimes g_2)_0=0
			\quad\text{for all }g\in G
		\end{split}\end{equation*} %}
		$(g_1\otimes g_2)_0$は$\sharp_m g$個すべてが一次独立ではなく、
		$(\sharp_m g)-1$個のみが一次独立になる。力学での重心$\Delta g$
		とその周りの相対座標$(g_1\otimes g_2)_0$と対応している。
	\end{example} %eg:モノイドの積の転置}

	\begin{example}[群の積の転置]\label{eg:群の積の転置} %{
		$G=(G,m,1_G)$を群、$\mybf{C}G=(RG,m,1_G)$を$G$を基底とし、
		$G$の積$m$を線形に拡張した自由$\mybf{C}$代数とする。
		$\Delta$を積$m$の転置とする。
		このとき、$\Delta$は次のようになる。
		\begin{equation*}\begin{split} %{
			\Delta = \sum_{g_1,g_2\in G}(g_1\otimes g_1^{-1}g_2)g_2^\dag
		\end{split}\end{equation*} %}
		また、例\ref{eg:モノイドの積の転置}の写像$\sharp_m$は次のように、
		$G$の元の数$\zettai{G}$を与える定数になる。
		\begin{equation*}\begin{split} %{
			\sharp_m g = \sum_{g_1,g_2\in G}\jump{g=g_1g_2} = \zettai{G}
			\quad\text{for all }g\in G
		\end{split}\end{equation*} %}
		したがって、$m\Delta$は定数$m\Delta=\zettai{G}$になる。
		一方、$\Delta m$は次のようになる。
		\begin{equation*}\begin{split} %{
			\Delta m(g_1\otimes g_2) = \sum_{h\in G}(g_1h\otimes h^{-1}g_2)
			\quad\text{for all }g_1,g_2\in G
		\end{split}\end{equation*} %}
	\end{example} %eg:群の積の転置}

	\begin{example}[自然数の乗法の転置]\label{eg:自然数の乗法の転置} %{
		$\mybf{N}_+$を基底とする自由半モジュールで、自然数の乗法を積$m$とし、
		その転置を$\Delta=m^\dag$とする。
		\begin{equation*}\begin{split} %{
			\Delta\ket{m} = \sum_{m_1,m_2\in\mybf{N}_+}
			\jump{m=m_1\cdot m_2}\ket{m_1}\otimes\ket{m_2}
			\quad\text{for all }m\in \mybf{N}_+
		\end{split}\end{equation*} %}
		特に、素数に対しては次のようになる。
		\begin{equation*}\begin{split} %{
			\Delta\ket{p} = \ket{1}\otimes\ket{p}+\ket{p}\otimes\ket{1}
			\quad\text{for all }p\in '1\text{でない素数}'
		\end{split}\end{equation*} %}
		リー環論で、余積が$v\otimes 1+1\otimes v$となるリー環の元$v$を素な元
		というが、この'素'という形容詞は自然数の乗法とのアナロジーなのだろう。
		さらに、素数に対しては次の式が成り立つ。
		\begin{equation*}\begin{split} %{
			\Delta\ket{p^m} = \sum_{0\le k\le m}\ket{p^k}\otimes\ket{p^{m-k}}
			= (\Delta\ket{p})^{m}
			\quad\text{for all }p\in '1\text{でない素数}'
		\end{split}\end{equation*} %}
		また、次の式も成り立つ。
		\begin{equation*}\begin{split} %{
			\Delta\ket{p_1^{m_1}p_2^{m_2}\cdots p_n^{m_n}} 
			= (\Delta\ket{p_1})^{m_1}(\Delta\ket{p_2})^{m_2}
				\cdots(\Delta\ket{p_n})^{m_n} \\
			\quad\text{for all }
			p_1,p_2,\dots,p_n\in '1\text{でない互いに異なる素数}'
		\end{split}\end{equation*} %}
		よって、次の式が成り立つ。
		\begin{equation*}\begin{split} %{
			m\Delta\ket{p_1^{m_1}p_2^{m_2}\cdots p_n^{m_n}}
			= 2^{m_1+m_2+\cdots+m_n}
			\ket{p_1^{m_1}p_2^{m_2}\cdots p_n^{m_n}} \\
			\quad\text{for all }
			p_1,p_2,\dots,p_n\in '1\text{でない互いに異なる素数}'
		\end{split}\end{equation*} %}
		したがって、$m\Delta\ket{m}$は$m\in\myop{N}$に含まれる$1$でない素数
		の重複を含めた数の和を($2$の冪のかたちで)与える。
	\end{example} %eg:自然数の乗法の転置}

	\begin{example}[文字列の連結の転置]\label{eg:文字列の連結の転置} %{
		$WA$の積$m_*$の転置を$\Delta_*$する。
		例\ref{eg:モノイドの積の転置}により、$\Delta_*$は次のようになり、
		\begin{equation*}\begin{split} %{
			\Delta_*1_W &= 1_W\otimes 1_W \\
			\Delta_*[a_1a_2\cdots a_m] &= 1_W\otimes [a_1a_2\cdots a_m] \\
			&\;+ [a_1]\otimes [a_2\cdots a_m] \\
			&\;+ [a_1a_2]\otimes [a_3\cdots a_m] \\
			&\;+ \cdots \\
			&\;+ [a_1a_2\cdots a_m]\otimes 1_W \\
			&\quad\text{for all }a_1,a_2,\dots, a_m\in A
		\end{split}\end{equation*} %}
		$m_*\Delta_*$は次のようになる。
		\begin{equation*}\begin{split} %{
			m_*\Delta_*w = (\zettai{w}+1)w \quad\text{for all }w\in WA
		\end{split}\end{equation*} %}
		$\Delta_*m_*$は次の式が成り立つ。
		\begin{equation*}\begin{split} %{
			\Delta_*m_* + \myid
			= (\myid\otimes m_*)(\Delta_*\otimes \myid)
			+ (m_*\otimes \myid)(\myid\otimes \Delta_*)
		\end{split}\end{equation*} %}
	\end{example} %eg:文字列の連結の転置}
	\begin{proof} %{
		例\ref{eg:文字列の連結の転置}の最後の式を証明する。
		例\ref{eg:モノイドの積の転置}より、$\Delta_*$は次のようになる。
		\begin{equation*}\begin{split} %{
			\Delta m(w_1\otimes w_2)
			= \sum_{x_1,x_2\in WA}\jump{w_1*w_2=x_1*x_2}(x_1\otimes x_2) \\
			\quad\text{for all }w_1,w_2\in WA
		\end{split}\end{equation*} %}
		文字列の場合、任意の$w\in WA$に対して、
		\begin{itemize}\setlength{\itemsep}{-1mm} %{
			\item 文字数$p$の元$x_1\in WA$と、
			\item 文字数$\zettai{w}-p$の元$x_2\in WA$で、
		\end{itemize} %}
		$w=x_1*x_2$となる分割$x_1\otimes x_2$は、$0\le p\le \zettai{w}$
		のとき一意に定まる。この分割を$w$の$p$分割と書く。
		そして、$\Delta_*w$は、任意の$w\in WA$に対して、
		$0\le p\le \zettai{w}$となる$w$の$p$分割をすべてただ一度
		だけ列挙する。したがって、任意の$w\in WA$に対して、
		\begin{itemize}\setlength{\itemsep}{-1mm} %{
			\item $(\Delta_*w_1)*(1_W\otimes w_2)$は、$0\le p\le \zettai{w_1}$
			となる$w_1*w_2$の$p$分割を、すべてただ一度だけ列挙し、
			\item $(1_W\otimes w_1)*(\Delta_*w_2)$は、
			$\zettai{w_1}\le p\le \zettai{w_1}+\zettai{w_2}$となる$w_1*w_2$の
			$p$分割を、すべてただ一度だけ列挙する。
		\end{itemize} %}
		したがって、任意の$w_1,w_2\in WA$に対して次の式が成り立つ。
		\begin{equation*}\begin{split} %{
			&(\Delta_*w_1)*(1_W\otimes w_2)+(1_W\otimes w_1)*(\Delta_*w_2) \\
			&= \left(\sum_{0\le p\le \zettai{w_1}+\zettai{w_2}}
				w_1*w_2\text{の}p\text{分割}\right) + w_1\otimes w_2 \\
			&= \Delta_*(w_1*w_2) + w_1\otimes w_2 \\
		\end{split}\end{equation*} %}
	\end{proof} %}

	\begin{example}[写像の合成の転置]\label{eg:写像の合成の転置} %{
		$\myop{end}(\mybf{C}A)$の基底は
		$AA^\dag=\set{a_1a_2^\dag}_{a_1,a_2\in A}$ととれる。
		この基底を使って写像の合成の転置を計算すると次のようになる。
		\begin{equation*}\begin{split} %{
			\Delta(a_1a_2^\dag)
			= \sum_{a_3,a_4,a_5,a_6\in A}
				\jump{a_1a_2^\dag=a_3a_4^\dag a_5a_6^\dag}
				a_3a_4^\dag\otimes a_5a_6^\dag
			= \sum_{a\in A}a_1a^\dag\otimes aa_2^\dag \\
			\quad\text{for all }a_1,a_2\in A
		\end{split}\end{equation*} %}
		この式は、群の例\ref{eg:群の積の転置}によく似ている。
		線形写像$\mu:RA\otimes RA\to \myop{end}(RWA)$を次のように定義すると、
		\begin{equation*}\begin{split} %{
			\mu(a_1\otimes a_2) = a_1a_2^\dag
			\quad\text{for all }a_1,a_2\in A
		\end{split}\end{equation*} %} 
		群と写像の合成は次のように対比できる。
		\begin{equation*}\begin{array}{rrcl} %{
			\text{群}&
			\Delta m(g_1\otimes g_2)
				&=& \sum_{g\in G} m(g_1\otimes g)\otimes m(g^{-1}\otimes g_2) \\
			\text{写像の合成}&
			\Delta \mu(a_1\otimes a_2) 
				&=& \sum_{a\in A}\mu(a_1\otimes a)\otimes \mu(a\otimes a_2) \\
		\end{array}\end{equation*} %}
		モノイドを群になりかけているものと思えば、この類似は'そういうものか'
		と思える。
	\end{example} %eg:写像の合成の転置}
%s1:内積}

\section{積による作用素}\label{s1:積による作用素} %{
	この節では、係数を複素数$\mybf{C}=(\mybf{C},m_+,0,m_\myspace,1)$に
	固定する。

	$A$を集合、$\myop{C}A=(\myop{C}A,m_\square,1_\square)$を自由代数とする。
	$\Delta_\square$を積$m_\square$の転置とする。

	積$m_\square$によって、線形写像
	$\myhere\square:\myop{C}A\to\myop{end}(\myop{C}A)$が次のように定義できる。
	\begin{equation*}\begin{split} %{
		(a_1\square)a_2 = a_1\square a_2 \quad\text{for all } a_1,a_2\in A
	\end{split}\end{equation*} %}
	定義から写像$\myhere\square$は代数射$(\myop{C}A,m_\square,1_\square)\to
	(\myop{end}(\myop{C}A),m_\myspace,\myid)$となる。
	ここで、$\myop{end}(\myop{C}A)$の積$m_\myspace$は写像の合成とする。

	作用素$a\square$の転置$(a\square)^\dag$は次のように定義される。
	\begin{equation*}\begin{split} %{
		\bigl((a\square)a_1\bigr)a_2 = a_1^\dag\bigl((a\square)^\dag a_2\bigr)
		\quad\text{for all } a,a_1,a_2\in A
	\end{split}\end{equation*} %}
	写像$(\myhere\square)^\dag$は、積の転置$\Delta_\square=m_\square^\dag$
	を使って次のように書ける。
	\begin{equation*}\begin{split} %{
		(a\square)^\dag= m_\square(1_\square a^\dag\otimes \myid)\Delta_\square
		\quad\text{for all } a\in A
	\end{split}\end{equation*} %}
	大雑把には、$(\myhere\square)^\dag$は余積$\Delta_\square$の第二成分を
	取り出す操作になる。
	一般に、転置$\myhere^\dag:\myop{end}(\myop{C}A)\to\myop{end}(\myop{C}A)$
	は逆順代数同型となる。
	\begin{equation*}\begin{split} %{
		(\phi_1\phi_2\cdots \phi_m a_1)^\dag a_2
		= (\phi_2\cdots \phi_m a_1)^\dag\phi_1^\dag a_2
		= \cdots
		= a_1^\dag \phi_m^\dag\cdots \phi_2^\dag\phi_1^\dag a_2 \\
		\quad\text{for all }a_1,a_2\in A
			,\;\phi_1,\phi_2,\dots,\phi_m\in\myop{end}(\myop{C}A)
	\end{split}\end{equation*} %}
	したがって、写像$(\myhere\square)^\dag$は、積$m_\square$について
	逆順代数射になることがわかる。
	ここでは、写像$(\myhere\square)^\dag$の定義から計算して、
	$(\myhere\square)^\dag$が逆順代数射になっていることを確かめる。
	$\myop{C}\otimes\myop{C}A\simeq\myop{C}A\simeq\myop{C}A\otimes\myop{C}$
	の同一視を使うと、作用素の積$(\myhere\square)^\dag(\myhere\square)^\dag$
	は次のようになる。
	\begin{equation*}\begin{array}{rll} %{
		(a_1\square)^\dag(a_2\square)^\dag
		&= m_\square(1_\square a_1^\dag \otimes \myid)\Delta_\square
		 m_\square(1_\square a_2^\dag \otimes \myid)\Delta_\square \\
		&\simeq (a_2^\dag\otimes a_1^\dag \otimes \myid)
		 (\myid \otimes \Delta_\square)\Delta_\square 
		 &\quad\lcomment{$\myop{C}A\to \myop{C}\otimes\myop{C}\otimes\myop{C}A$} \\
		&\simeq \bigl((a_2\otimes a_1)^\dag \otimes \myid\bigr)
		 (\myid \otimes \Delta_\square)\Delta_\square 
		 &\quad\lcomment{$\myop{C}A\to \myop{C}\otimes\myop{C}A$} \\
		&= \bigl((a_2\otimes a_1)^\dag \otimes \myid\bigr)
		 (\Delta_\square \otimes \myid)\Delta_\square 
		 &\quad\lcomment{余積の余結合性} \\
		&= \bigl((a_2\square a_1)^\dag \otimes \myid\bigr)\Delta_\square 
		 &\quad\lcomment{$\Delta_\square$は$m_\square$の転置} \\
		&\simeq m_\square\bigl(1_W(a_2\square a_1)^\dag \otimes \myid\bigr)
			\Delta_\square 
			&\quad\lcomment{$V\to V$} \\
		&\quad\text{for all }a_1,a_2\in A
	\end{array}\end{equation*} %}
	よって、写像$(\myhere\square)^\dag$は積$m_\square$について逆順代数射となる
	ことがわかる。

	線形写像$\square\myhere:\mybf{C}A\to\myop{end}(\myop{C}A)$を次のように
	定義する。
	\begin{equation*}\begin{split} %{
		(\square a_1)a_2 = a_2\square a_1 \quad\text{for all } a_1,a_2\in A
	\end{split}\end{equation*} %}
	次の式が成り立ち、写像$\square\myhere$は同型な逆順代数射になることが
	わかる。
	\begin{equation*}\begin{split} %{
		(\square a_1)(\square a_2)a = a\square a_2\square a_1 
		=  \bigl((a_2\square a_1)\square\bigr)a
		\quad\text{for all } a,a_1,a_2\in A
	\end{split}\end{equation*} %}
	積$m_\square$の結合性により、$\mybf{C}A\square$と$\square\mybf{C}A$は
	可換になる。
	\begin{equation*}\begin{split} %{
		(a_1\square)(\square a_2)a
		= a_1\square v\square a_2
		= (a_2\square)(\square a_1)a 
		\quad\text{for all } a,a_1,a_2\in A
	\end{split}\end{equation*} %}
	また、写像$(\square\myhere)^\dag$は、積の転置
	$\Delta_\square=m_\square^\dag$を使って次のように書ける。
	\begin{equation*}\begin{split} %{
		(\square a)^\dag= m_\square(\myid\otimes 1_\square a^\dag)\Delta_\square
		\quad\text{for all } a\in A
	\end{split}\end{equation*} %}
	写像$(\square\myhere)^\dag$は積$m_\square$について正順代数射となる。

	以上を定義と命題のかたちでまとめる。

	\begin{definition}[積による作用素]\label{def:積による作用素} %{
		$\myop{C}A=(\myop{C}A,m_\square,1_\square)$を自由代数とする。
		次のよう定義された線形写像$\myhere\square$を積$m_\square$による
		左から積をとる作用素ということにする。
		\begin{equation*}\begin{split} %{
			\myhere\square:\mybf{C}A &\to \myop{end}(\myop{C}A) \\
			(a_1\square)a_2 &= a_1\square a_2 
			\quad\text{for all }a_1,a_2\in A
		\end{split}\end{equation*} %}
		次のよう定義された線形写像$\square\myhere$を積$m_\square$による
		右から積をとる作用素ということにする。
		\begin{equation*}\begin{split} %{
			\square\myhere:\mybf{C}A &\to \myop{end}(\mybf{C}A) \\
			(\square a_1)a_2 &= a_2\square a_1
			\quad\text{for all }a_1,a_2\in A
		\end{split}\end{equation*} %}
	\end{definition} %def:積による作用素}

	\begin{proposition}[積による作用素の性質]\label{prop:積による作用素の性質} %{
		$\myop{C}A=(\myop{C}A,m_\square,1_\square)$を自由代数とする。
		$\Delta_\square$を積$m_\square$の転置とする。
		このとき、積$m_\square$による作用素に関して次の事柄が成り立つ。
		\begin{itemize}\setlength{\itemsep}{-1mm} %{
			%
			\item 転置への写像$(\myhere\square)^\dag$は次のようになり、
			\begin{equation*}\begin{split} %{
				(a\square)^\dag 
				= m_\square(1_\square a^\dag\otimes \myid)\Delta_\square
				\quad\text{for all }a\in A
			\end{split}\end{equation*} %}
			転置への写像$(\square\myhere)^\dag$は次のようになる。
			\begin{equation*}\begin{split} %{
				(\square a)^\dag 
				= m_\square(\myid\otimes 1_\square a^\dag)\Delta_\square
				\quad\text{for all }a\in A
			\end{split}\end{equation*} %}
			%
			\item $\myhere\square$と$(\square\myhere)^\dag$は代数射になり、
			$\square\myhere$と$(\myhere\square)^\dag$は逆順代数射となる。
			%
			\item 像$\myop{C}A\square\subseteq\myop{end}(\myop{C}A)$と
			像$\square\myop{C}A\subseteq\myop{end}(\myop{C}A)$は可換、
			像$(\myop{C}A\square)^\dag\subseteq\myop{end}(\myop{C}A)$と
			像$(\square\myop{C}A)^\dag\subseteq\myop{end}(\myop{C}A)$は可換
			になる。
		\end{itemize} %}
	\end{proposition} %prop:積による作用素の性質}

	\begin{example}[自然数]\label{eg:自然数} %{
		自然数を基底とする自由半モジュールで、
		自然数の加法$m_+$による作用素は次のようになる。
		\begin{equation*}\begin{split} %{
			\begin{array}{rl}
				(\ket{m}+)\ket{n} &= \ket{n+m} \\
				(\ket{m}+)^\dag\ket{n} &= \jump{0\le n-m}\ket{n-m} \\
			\end{array}
			\quad\text{for all }m,n\in\mybf{N}
		\end{split}\end{equation*} %}
		$(m+)^\dag n$は$\myop{max}(m\times n)$や$\zettai{m-n}$ではなく、
		$\jump{0\le n-m}(n-m)$となる。
		自然数の乗法$m_\cdot$による作用素は次のようになる。
		\begin{equation*}\begin{split} %{
			\begin{array}{rl}
				(\ket{m}\cdot)\ket{n} &= \ket{m\cdot n} \\
				(\ket{m}\cdot)^\dag\ket{n} 
				&= \jump{\frac{n}{m}\in\mybf{N}}\ket{\frac{n}{m}} \\
			\end{array}
			\quad\text{for all }m,n\in\mybf{N}
		\end{split}\end{equation*} %}
	\end{example} %eg:自然数}
%s1:積による作用素}

\section{正規言語}\label{s1:正規言語} %{
	\begin{note}[半モジュールの係数]\label{note:半モジュールの係数} %{
		自然数$\mybf{N}$からor-andブーリアン$\mybf{B}$への写像$f$を次のように
		定義する。
		\begin{equation*}\begin{split} %{
			fm = \jump{m\neq0} \quad\text{for all }m\in\mybf{N}
		\end{split}\end{equation*} %}
		すると、次の可換図が成り立ち、$f$は半環準同型になるこことがわかる。
		\begin{equation*}\xymatrix@C+2pc{
			m_1\times m_2 \ar[r]^{m_+} \ar[d]^{f\times f} 
			& m_1+m_2 \ar[d]^{f} \\
			\jump{m_1\neq0}\times \jump{m_2\neq0} \ar[r]^{m_\lor}
			& \jump{m_1\neq0}\lor\jump{m_2\neq0}=\jump{m_1+m_2\neq0}
		}\end{equation*}
		\begin{equation*}\xymatrix@C+2pc{
			m_1\times m_2 \ar[r]^{m_\myspace} \ar[d]^{f\times f} 
			& m_1m_2 \ar[d]^{f} \\
			\jump{m_1\neq0}\times \jump{m_2\neq0} \ar[r]^{m_\land}
			& \jump{m_1\neq0}\land\jump{m_2\neq0}=\jump{m_1m_2\neq0}
		}\end{equation*}
		したがって、自然数を係数とする半モジュールで作られた理論はそのまま
		or-andブーリアンを係数とする理論に移行できると思われる。
		逆の写像$\mybf{B}\to\mybf{N}$では半環準同型は多分存在しない。
		したがって、or-andブーリアンを係数とする半モジュールで作られた理論
		は自然数を係数とする理論へ移行することはできないと思われる。
	\end{note} %note:半モジュールの係数}

	\subsection{Brzozowsky微分}\label{s2:Brzozowsky微分} %{
		文字列に対する作用素$(\myhere*)^\dag$のライプニッツ規則に相当する式を
		導く。

		$R$を標数$0$の環、$A$を集合、$WA=(WA,m_*,1_W)$を$A$から生成された
		自由モノイド、$RWA$を自由モノイド環とする。

		余積$\Delta_*$を積$m_*$の転置とすると、作用素$(WA*)^\dag$は次のように
		書ける。
		\begin{equation*}\begin{split} %{
			(w*)^\dag = m_*(1_Ww^\dag\otimes \myid)\Delta_*
			\quad\text{for all }w\in WA
		\end{split}\end{equation*} %}
		例\ref{eg:文字列の連結の転置}から、次の式が成り立つので、
		\begin{equation*}\begin{split} %{
			\Delta_*m_* = (\myid\otimes m_*)(\Delta_*\otimes \myid)
			+ (m_*\otimes \myid)(\myid\otimes \Delta_*) - \myid
		\end{split}\end{equation*} %}
		次の式が成り立つ。
		\begin{equation*}\begin{split} %{
			(w*)^\dag m_* &= m_*(1_Ww^\dag\otimes \myid)\Delta_*m_* \\
			&= m_*(1_Ww^\dag\otimes m_*)(\Delta_*\otimes \myid)
				+ m_*(1_Ww^\dag m_*\otimes \myid)(\myid\otimes \Delta_*) \\
				&\;- m_*(1_Ww^\dag\otimes \myid) \\
			&\quad\text{for all }w\in WA
		\end{split}\end{equation*} %}
		一項目は次のようになり、
		\begin{equation*}\begin{array}{ll} %{
			m_*(1_Ww^\dag\otimes m_*)(\Delta_*\otimes \myid) \\
			= m_*(m_*\otimes \myid)(1_Ww^\dag\otimes \myid\otimes \myid)
				(\Delta_*\otimes \myid) & \quad\lcomment{$m_*$の結合性} \\
			= m_*((w*)^\dag\otimes \myid) 
				& \quad\lcomment{$(\myhere_*)^\dag$の定義} \\
		\end{array}\end{equation*} %}
		二項目は次のようになる。
		\begin{equation*}\begin{array}{ll} %{
			m_*(1_Ww^\dag m_*\otimes \myid)(\myid\otimes \Delta_*) \\
			= m_*\bigl(1_W(\Delta_*w)^\dag\otimes \myid\bigr)
				(\myid\otimes \Delta_*) & \quad\lcomment{$\Delta_*$の定義} \\
			= m_*(m_*\otimes \myid)\bigl(1_W(\Delta_{*(1)}w)^\dag
				\otimes 1_W(\Delta_{*(2)}w)^\dag\otimes \myid\bigr)
				(\myid\otimes \Delta_*) \\
			= m_*(\myid\otimes m_*)\bigl(1_W(\Delta_{*(1)}w)^\dag
				\otimes 1_W(\Delta_{*(2)}w)^\dag\otimes \myid\bigr)
				(\myid\otimes \Delta_*) & \quad\lcomment{$m_*$の結合性} \\
			= m_*\biggl(1_W(\Delta_{*(1)}w)^\dag
				\otimes \bigl((\Delta_{*(2)}w)*\bigr)^\dag\biggr)
				& \quad\lcomment{$(\myhere_*)^\dag$の定義} \\
		\end{array}\end{equation*} %}
		したがって、ライプニッツ規則に似た次の式が成り立つ。
		\begin{equation*}\begin{split} %{
			(w*)^\dag m_* &= m_*(\Gamma w) \quad\text{for all }w\in WA
		\end{split}\end{equation*} %}
		ここで、$\Gamma$は線形写像$\Gamma:RWA\to
		\myop{end}(RWA)\otimes \myop{end}(RWA)$で次のように定義される。
		\begin{equation*}\begin{split} %{
			\Gamma w &= (w*)^\dag\otimes \myid
			+ 1_W(\Delta_{*(1)}w)^\dag\otimes \bigl((\Delta_{*(2)}w)*\bigr)^\dag
			- 1_Ww^\dag\otimes \myid \\
			&\quad\text{for all }w\in WA
		\end{split}\end{equation*} %}
		特に、文字$A$に対しては次のようになる。
		\begin{equation*} %{
			\begin{split}
				(a*)^\dag m_* &= m_*(\Gamma a) \\
				\Gamma a &= (a*)^\dag\otimes \myid + 1_W1_W^\dag\otimes (a*)^\dag
			\end{split} \quad\text{for all }a\in A
		\end{equation*} %}
		コンパイラ理論では、$(RA*)^\dag$のことをBrzozowsky微分(derivative)
		と言う。少なくとも係数$R$がand-orブーリアンの時はBrzozowsky微分と言う。
		係数が一般の半環の時もBrzozowsky微分というかは不明であるが、上記の
		結果は引き算を含まないので、内積が定義できれば半環に対しても成り立つ。

		以上を定義のかたちでまとめておく。

		\begin{definition}[Brzozowsky微分]\label{def:Brzozowsky微分} %{
			$R$を半環、$A$を集合、$WA=(WA,m_*,1_W)$を$A$から生成された
			自由モノイド、$RWA$を$R$係数の自由モノイド環とする。
			次のように定義された線形写像$(\myhere*)^\dag:RA\to \myop{end}(RWA)$を
			Brzozowsky微分という。
			\begin{equation*}\begin{split} %{
				(a*)^\dag &= m_*(1_W[a]^\dag\otimes \myid)\Delta_*
				\quad\text{for all }a\in A
			\end{split}\end{equation*} %}
			ここで、$\Delta_*$は積$m_*$の転置である。
		\end{definition} %def:Brzozowsky微分}

		\begin{proposition}[Brzozowsky微分の積との交換関係]\label{prop:Brzozowsky微分の積との交換関係} %{
			$R$を半環、$A$を集合、$WA=(WA,m_*,1_W)$を$A$から生成された
			自由モノイド、$RWA$を$R$係数の自由モノイド環とする。
			Brzozowsky微分$(\myhere*)^\dag$は次の式を満たす。
			\begin{equation*} %{
				\begin{split}
					(a*)^\dag m_* &= m_*(\Gamma a) \\
					\Gamma a &= (a*)^\dag\otimes \myid + 1_W1_W^\dag
						\otimes (a*)^\dag
				\end{split} \quad\text{for all }a\in A
			\end{equation*} %}
			同じことは次の式でも表される。
			\begin{equation*}\begin{split} %{
				(a*)^\dag(f_1*f_2) &= \bigl((a*)^\dag f_1\bigr)*f_2 
					+ (1_W^\dag f_1)\bigl((a*)^\dag f_2\bigr) \\
				&\quad\text{for all }a\in A,\;f_1,f_2\in RWA
			\end{split}\end{equation*} %}
		\end{proposition} %prop:Brzozowsky微分の積との交換関係}

		生成元同士、$a*$と$(a*)^\dag$の交換関係を求めておく。
		$\alpha_1=1_W1_W^\dag$、$\alpha_2=\myid$とすると、
		命題\ref{prop:Brzozowsky微分の積との交換関係}から次の式が成り立つ。
		\begin{equation*}\begin{split} %{
			(a*)^\dag m_* 
			= m_*\bigl(\alpha_1\otimes (a*)^\dag+(a*)^\dag\otimes \alpha_2\bigr)
			\quad\text{for all }a\in A
		\end{split}\end{equation*} %}
		生成元同士の交換関係は、任意の$\alpha_1,\alpha_2\in\myop{end}(RWA)$
		に対して次のようになることがわかる。
		\begin{equation*}\begin{split} %{
			(a_1*)^\dag(a_2*)w &= (a_1*)^\dag m_*([a_2]\otimes w) \\
			&= m_*\biggl((\alpha_1[a_2])\otimes \bigl((a_1*)^\dag w\bigr)
			+ 1_W(a_1^\dag a_2)\otimes (\alpha_2w)\biggr) \\
			&= (\alpha_1[a_2])*\bigl((a_1*)^\dag w\bigr)
			+ (a_1^\dag a_2)(\alpha_2w) \\
			&\quad\text{for all }a_1,a_2\in A,\;w\in WA
		\end{split}\end{equation*} %}
		$\alpha_1$と$\alpha_2$の組み合わせで次のようになる。
		\begin{itemize}\setlength{\itemsep}{-1mm} %{
			\item $\alpha_1=1_W1_W^\dag$、$\alpha_2=\myid$の時は、Brzozowsky微分
			になって、次のような交換関係となる。
			\begin{equation*}\begin{split} %{
				(a_1*)^\dag(a_2*) = (a_1^\dag a_2)
				\quad\text{for all }a_1,a_2\in A
			\end{split}\end{equation*} %}
			\item $\alpha_1=\alpha_2=\myid$の時は、正準交換関係になる。
			\begin{equation*}\begin{split} %{
				(a_1*)^\dag(a_2*) = (a_2*)(a_1*)^\dag + (a_1^\dag a_2)
				\quad\text{for all }a_1,a_2\in A
			\end{split}\end{equation*} %}
		\end{itemize} %}

		\begin{todo}[正準交換関係]\label{todo:正準交換関係} %{
			正準交換関係
			$(a_1\square)^\dag(a_2*) = (a_2*)(a_1\square)^\dag+(a_1^\dag a_2)$
			を満たす作用素$(a\square)^\dag$の元になる積$m_\square$はシャッフル積
			になると予想する。
		\end{todo} %todo:正準交換関係}
	%s2:Brzozowsky微分}
%s1:正規言語}

\section{この後}\label{s1:この後} %{
	\begin{itemize}\setlength{\itemsep}{-1mm} %{
		\item 表現
	\end{itemize} %}


	\subsection{微分形式}\label{s2:微分形式} %{
		微分幾何において微分形式とは微分の転置として定義される。
		微分とは多様体から複素数への写像空間のライプニッツ規則を満たす
		線形作用として定義される。
		多様体を$M$、$M$から$\mybf{C}$への写像全体を$M^\dag$、
		その微分全体を$TM$とすると、$TM\subseteq \myop{end}M^\dag$となる。
		ここで、$M^\dag$は$\mybf{R}^{\myop{dim}M}$を基底、$\mybf{C}$を係数
		とする自由ベクトル空間と局所的に同型となるので、$\myop{end}M^\dag$
		は局所的には$\mybf{C}\mybf{R}^{\myop{dim}M}$の線形写像全体である。
		微分形式$\omega\in T_*M$は$w:TM\to\mybf{C}$として定義される。

		\begin{table}[htbp] %{
			\begin{center}\begin{tabular}{cc} \hline
				微分幾何 & 文字列 \\ \hline
				$M$ & $RA$ \\ \hline
				$TM$ & $\myop{end}(RA)$ \\ \hline
				$T_*M$ & $(\myop{end}(RA))^\dag$ \\ \hline
			\end{tabular}\end{center}
			\caption{対応}
		\end{table} %}
	%s2:微分形式}
%s1:この後}


	\subsection{作用から積の導出}\label{s2:作用から積の導出} %{
		$A,B$を集合、$R$を半環とする。
		$R$双線形作用$\beta_\rhd:RA\otimes RB\to RB$
		と$R$双線形二項演算$\beta_\perp:RA\otimes RA\to RA$が定義されていて、
		次の性質を満たすとする。
		\begin{equation}\label{eq:結合的な作用}\begin{split} %{
			a_2\rhd a_1\rhd b &= (a_2\perp a_1)\rhd b
			\quad\text{for all }a_1,a_2\in A,\;b\in B
		\end{split}\end{equation} %}
		すると、任意の$a_1,a_2,a_3\in A,\;b\in B$に対して、内側から
		二項演算$\perp$に書き直していくと
		\begin{equation*}\begin{split} %{
			a_3\rhd a_2\rhd a_1\rhd b &= a_3\rhd(a_2\perp a_1)\rhd b \\
			&= \Bigl(a_3\perp(a_2\perp a_1)\Bigr)\rhd b
		\end{split}\end{equation*} %}
		となり、外側から二項演算$\perp$に書き直していくと
		\begin{equation*}\begin{split} %{
			a_3\rhd a_2\rhd a_1\rhd b &= (a_3\perp a_2)\rhd a_1\rhd b \\
			&= \Bigl((a_3\perp a_2)\perp a_1\Bigr)\rhd b
		\end{split}\end{equation*} %}
		となる。したがって、$A$の同値関係$\sim_\rhd$
		\begin{equation}\begin{split} %{
			a_1 \sim_\rhd a_2 \implies
			a_1\rhd b = a_2\rhd b \quad\text{for all }b\in B
		\end{split}\end{equation} %}
		を用いて、$\perp$の結合性が導かれる。
		\begin{equation*}\begin{split} %{
			a_3\perp(a_2\perp a_1) \sim_\rhd (a_3\perp a_2)\perp a_1
		\end{split}\end{equation*} %}

		$\rhd$と$\perp$の関係式\eqref{eq:結合的な作用}を可換図で書くと
		次のようになる。
		\begin{equation*}\xymatrix@C+1pc{
			RA\otimes RA\otimes RB \ar[d]^{\myid\otimes \beta_\rhd} \ar[r]^{\beta_\perp\otimes \myid} 
			& RA\otimes RB \ar[d]^{\beta_\rhd} \\
			RA\otimes RB \ar[r]^{\beta_\rhd} & RB \\
		}\end{equation*}
		余作用$\nabla:RB\to RA\otimes RB$と
		余二項演算$\gamma:RA\to RA\otimes RA$を用いて、
		この可換図の矢印の向きを反転させると次の可換図になる。
		\begin{equation*}\xymatrix@C+1pc{
			RA\otimes RA\otimes RB  
			&  RA\otimes RB \ar[l]_{\gamma\otimes \myid} \\
			RA\otimes RB \ar[u]_{\myid\otimes \nabla}
			& RB \ar[l]_{\nabla} \ar[u]_{\nabla} \\
		}\end{equation*}
		式で書くと$(\myid\otimes\nabla)\nabla=(\gamma\otimes\myid)\nabla$
		となる。この関係を$
		(\myid\otimes\myid\otimes\nabla)(\myid\otimes\nabla)\nabla
		$に対して左から適用していくと
		\begin{equation*}\begin{split} %{
			(\myid\otimes\myid\otimes\nabla)(\myid\otimes\nabla)\nabla
			&= (\myid\otimes\gamma\otimes\myid)(\myid\otimes\nabla)\nabla \\
			&= (\myid\otimes\gamma\otimes\myid)(\gamma\otimes\myid)\nabla \\
		\end{split}\end{equation*} %}
		となり、右から適用していくと
		\begin{equation*}\begin{split} %{
			(\myid\otimes\myid\otimes\nabla)(\myid\otimes\nabla)\nabla
			&= (\myid\otimes\myid\otimes\nabla)(\gamma\otimes\myid)\nabla \\
			&= (\gamma\otimes\nabla)\nabla \\
			&= (\gamma\otimes\myid\otimes\myid)(\myid\otimes\nabla)\nabla \\
			&= (\gamma\otimes\myid\otimes\myid)(\gamma\otimes\myid)\nabla \\
		\end{split}\end{equation*} %}
		となる。したがって、$\set{\nabla^{(1)}b}_{b\in B}$で張られる
		部分空間$\nabla^{(1)}A\subseteq A$内で、$\gamma$の余結合性が導かれる。
		\begin{equation*}\begin{split} %{
			(\myid\otimes\gamma)\gamma = (\gamma\otimes\myid)\gamma
			\quad\text{in }\nabla^{(1)}A
		\end{split}\end{equation*} %}

		一般には、与えられた作用$\rhd$に対して、式\eqref{eq:結合的な作用}
		を満たすような二項演算$\perp$は定義できないが、
		$\beta_\rhd:RA\otimes RB\to RB$から空間$RA$を二項演算$\perp$が定義
		できるところまで拡大していけることがある。
		\begin{equation*}\begin{split} %{
			\xymatrix{
				RA \ar[dr]_{\mybiop{\rhd}} \ar[r]^{i}
				& R\widetilde{A} \ar[d]^{\mybiop{\widetilde{\rhd}}} \\
				& \myop{end}RB \\
			} \quad
			\mybiop{\widetilde{\rhd}}(\myid\otimes\mybiop{\widetilde{\rhd}})
			=\mybiop{\widetilde{\rhd}}(\mybiop{\perp}\otimes\myid)
		\end{split}\end{equation*} %}
		空間の拡大の仕方が、文字から単語、単語から木、木からグラフといった
		データ構造を拡大していく方法が使えることがある。
	%s2:作用から積の導出}

	\subsection{木の定義}\label{s2:木の定義} %{
		木構造は様々な場面で使われ、使う場面ごとに様々な定義がある。
		例えば、通常のプログラミングで使われる木いうのは、数学では
		平面上のラベル付き根付き木\cite{arxiv:hoffman:0710.3739}という長い
		修飾子がついた木になる。この節で使ういくつかの木を定義しておく。

		\begin{todo}[定義すべきもの]\label{todo:定義すべきもの} %{
			\begin{description} %{
				\item[根付きの木]木の同値関係の観点から
				\item[平面上の木]木の同値関係の観点から
				\item[ラベル付きの木]木の同値関係の観点から
				\item[頂点の指定の仕方]三種類の頂点の指定の仕方
				\begin{itemize} %{
					\item 行きがけ順
					\item 帰りがけ順
				\end{itemize} %}
				頂点に順序をつけて表す。木の操作ごとに適切な順序を用いる。
				例えば、次のような場合は、順序関係$i\le j$は行きがけ順と帰りがけ順
				で異なる。
				\begin{equation*}\begin{split} %{
					\sum_{i\le j\in\myop{pre}t}t\lhd_ix\lhd_jy
				\end{split}\end{equation*} %}
				また、木を単語に直す際、行きがけ順と帰りがけ順では、文字の並び以外
				にも単語としての同値性も違いがでる。例えば、次の五つの木は、
				行きがけ順で単語にするとすべて同じ単語になってしまう。
				\begin{equation*}\begin{array}{rlllll} %{
					& \mytree{
						& a_0 \ar@{-}[dl]\ar@{-}[d]\ar@{-}[dr] \\
						a_1 & a_2 & a_3 \\
					},& \mytree{
						& a_0 \ar@{-}[dl]\ar@{-}[dr] \\
						a_1 \ar@{-}[d] && a_3 \\
						a_2 \\
					},& \mytree{
						& a_0 \ar@{-}[dl]\ar@{-}[dr] \\
						a_1 && a_2 \ar@{-}[d]\\
						&& a_3
					},& \mytree{
						& a_0 \ar@{-}[d] \\
						& a_1 \ar@{-}[dl]\ar@{-}[dr] \\
						a_2 && a_3
					},& \mytree{
						a_0 \ar@{-}[d] \\
						a_1 \ar@{-}[d] \\
						a_2 \ar@{-}[d] \\
						a_3
					} \\
					\text{pre} & [0123] & [0123] & [0123] & [0123] & [0123] \\
					\text{post} & [1230] & [2130] & [1320] & [2310] & [3210] \\
				\end{array}\end{equation*} %}
				逆に、次の五つの木は、帰りがけ順で単語にするとすべて同じ単語に
				なってしまう。
				\begin{equation*}\begin{array}{rlllll} %{
					& \mytree{
						& a_3 \ar@{-}[dl]\ar@{-}[d]\ar@{-}[dr] \\
						a_0 & a_1 & a_2 \\
					},& \mytree{
						& a_3 \ar@{-}[dl]\ar@{-}[dr] \\
						a_1 \ar@{-}[d] && a_2 \\
						a_0 \\
					},& \mytree{
						& a_3 \ar@{-}[dl]\ar@{-}[dr] \\
						a_0 && a_2 \ar@{-}[d]\\
						&& a_1
					},& \mytree{
						& a_3 \ar@{-}[d] \\
						& a_2 \ar@{-}[dl]\ar@{-}[dr] \\
						a_0 && a_1
					},& \mytree{
						a_3 \ar@{-}[d] \\
						a_2 \ar@{-}[d] \\
						a_1 \ar@{-}[d] \\
						a_0
					} \\
					\text{pre} & [3012] & [3102] & [3021] & [3201] & [3210] \\
					\text{post} & [0123] & [0123] & [0123] & [0123] & [0123] \\
				\end{array}\end{equation*} %}
				\item[木の書き方]子供を括弧$[]$でくくって書く。また、文献
				\cite{arxiv:hoffman:0710.3739}のBalanced Bracket Arrangement
				という書き方は、この書き方のラベルなしの場合になっている。
				\item[空の木の取り扱い]空の木をどのように取り扱うかを明記すること。
				\item[森]森を定義すること。
			\end{description} %}
		\end{todo} %todo:定義すべきもの}
	%s2:木の定義}

	\subsection{この節で扱う木}\label{s2:この節で扱う木} %{
		$A$を集合、$TA$を$A$を頂点に持つ木の集合とする。
		$TA$には空の木を含めるとする。空の木を含めない場合は$T_+A$と書く。

		$TA$の元は根が固定されているものとする(根付き木)。
		例えば、次のような同値関係とする。
		\begin{equation*}\begin{split} %{
			\mytree{
				& a_0 \ar@{-}[dl] \ar@{-}[dr] \\
				a_1 && a_2 \\
			} \neq \mytree{
				a_1 \ar@{-}[d] \\
				a_0 \ar@{-}[d] \\
				a_2 \\
			} 
		\end{split}\end{equation*} %}
		また、$TA$の同値関係は子供の頂点の並びまで含めるものする(ラベル付き木)。
		例えば、次のような同値関係とする。
		\begin{equation*}\begin{split} %{
			\mytree{
				& a_0 \ar@{-}[dl] \ar@{-}[dr] \\
				a_1 && a_2 \\
			} = \mytree{
				& a_0 \ar@{-}[dl] \ar@{-}[dr] \\
				a_2 && a_1 \\
			} \iff a_1 = a_2
		\end{split}\end{equation*} %}

		木を次の図のように子供の頂点を括弧でくくって表すことにする。
		\begin{equation*}\begin{split} %{
			a_0[a_1[a_3]a_2[]] = \mytree{
				& a_0 \ar@{-}[dl] \ar@{-}[dr] \\
				a_1 \ar@{-}[d] && a_2 \\
				a_3 \\
			}
		\end{split}\end{equation*} %}
		子供を持たない頂点に関しては、$a_0[a_1[a_3]a_2]=a_0[a_1[a_3]a_2[]]$の
		ように括弧を省略して書くこともある。
		頂点を一つも持たない木を$1_T$と書く。
		木$t$の頂点の個数を$\zettai{t}$と書く。
		例えば、$\zettai{a_0[a_1[a_3]a_2]}=3$、$\zettai{1_T}=0$となる。

		$A$から$TA$への写像$i_T$を次のように定める。
		\begin{equation*}\begin{split} %{
			i_Ta = a[]
		\end{split}\end{equation*} %}
		$i_TA=\set{i_Ta}_{a\in A}\subseteq TA$は根だけからなる木の集合となり、
		頂点数が一つの木はすべて$i_TA$に含まれる。

		$TA$から生成される自由モノイド$WTA$の元を森という。森の文字が木になる。
		$a\in A$を根とし、根の子供の部分木が$t_1,t_2,\dots,t_m$となる木
		を$a[t_1t_2\cdots t_m]$と書く。また、文字列の連結$m_*$を用いて
		$a([t_1]*[t_2\cdots t_m])=a[t_1t_2\cdots t_m]$とも書く。

		\begin{todo}[空の木と空の森]\label{todo:空の木と空の森} %{
			空の木と空の森は同一視されるべきものではないだろうか?
			単語の場合と異なり、空の木を含めない$T_+A$で考えた方がきれいに
			まとまるように思える。
		\end{todo} %todo:空の木と空の森}
	%s2:この節で扱う木}

	\subsection{根への接木}\label{s2:根への接木} %{
		木の頂点を行きがけ順に並べることで、木から文字列への写像が定義できる。
		例えば、$a_0[a_1[a_3]a_2]\mapsto[a_0a_1a_3a_2]$という写像になる。
		この写像を$\pi_W$と書く。次の可換図を満たす、$T_+A$の二項演算$\land_0$
		を定義する。
		\begin{equation}\label{eq:行きがけ順を保つ二項演算}\xymatrix{
			T_+A\times T_+A \ar@{.>}[d]^{\beta_{\land_0}} \ar[r]^{\pi_W\times\pi_W}
			& WA\times WA \ar[d]^{m_*} \\
			T_+A \ar[r]^{\pi_W} & WA \\
		}\end{equation}
		$\land_0$は一意には決まらないが、次のように定義すれば可換図を満たす。

		\begin{definition}[根への接木]\label{def:根への接木} %{
			$R$双線形二項演算$\beta_{\land_0}: RT_+A\otimes RT_+A \to RT_+A$を、
			任意の$t\in T_+A$に対して次のように、
			\begin{equation*}\begin{split} %{
				t\land_0 1_T = t \\
			\end{split}\end{equation*} %}
			任意の$t,t_1,t_2,\dots,t_m\in T_+A$に対して次のように定義する。
			\begin{equation*}\begin{split} %{
				a[t_1t_2\cdots t_m]\land_0 t &= a[t_1t_2\cdots t_mt] 
			\end{split}\end{equation*} %}
			二項演算$\land_0$を根への接木ということにする。
		\end{definition} %def:根への接木}

		$t_1\land_0 t_2$は木$t_1$の根の最右の子供に木$t_2$を付け足す操作である。
		$\land_0$は結合性を満たさない。例えば、任意の$a_1,a_2,a_3\in A$に
		対して次のようになり、
		\begin{equation*}\begin{array}{rll} %{
			a_1\land_0(a_2\land_0 a_3) &= a_1\land_0 a_2[a_3] &= a_1[a_2[a_3]] \\
			(a_1\land_0 a_2)\land_0 a_3 &= a_1[a_2]\land_0 a_3 &= a_1[a_2a_3] \\
		\end{array}\end{equation*} %}
		次のように結合性が満たされないことがわかる。
		\begin{equation*}\begin{split} %{
			a_1\land_0(a_2\land_0 a_3) \neq (a_1\land_0 a_2)\land_0 a_3
		\end{split}\end{equation*} %}
		根への接木$\land_0$は結合性を満たさないが、根だけから木
		$\set{a[]}_{a\in A}$から$T_+A$を生成することができる。
		例えば、次のようになる。
		\begin{equation*}\begin{split} %{
			&a_1\land_0 a_2 = \mytree{
				a_1 \ar@{-}[d] \\
				a_2
			} \\
			&(a_1\land_0 a_2)\land_0 a_3 = \mytree{
				& a_1 \ar@{-}[ld] \ar@{-}[rd] \\
				a_2 && a_3
			},\quad a_1\land_0(a_2\land_0 a_3) = \mytree{
				a_1 \ar@{-}[d] \\
				a_2 \ar@{-}[d] \\
				a_3
			} \\
			&\bigl((a_1\land_0 a_2)\land_0 a_3\bigr)\land_0 a_4 = \mytree{
				& a_1 \ar@{-}[ld] \ar@{-}[d] \ar@{-}[rd] \\
				a_2 & a_3 & a_4
			},\quad \bigl(a_1\land_0 (a_2\land_0 a_3)\bigr)\land_0 a_4 = \mytree{
				& a_1 \ar@{-}[ld] \ar@{-}[rd] \\
				a_2 \ar@{-}[d] && a_4 \\
				a_3
			}\\
			&(a_1\land_0 a_2)\land_0 (a_3\land_0 a_4) = \mytree{
				& a_1 \ar@{-}[ld] \ar@{-}[rd] \\
				a_2 && a_3 \ar@{-}[d] \\
				&& a_4
			} \\
			&a_1\land_0\bigl((a_2\land_0 a_3)\land_0 a_4\bigr) = \mytree{
				& a_1 \ar@{-}[d] \\
				& a_2 \ar@{-}[ld] \ar@{-}[rd] \\
				a_3 && a_4
			},\quad a_1\land_0\bigl(a_2\land_0 (a_3\land_0 a_4)\bigr) = \mytree{
				a_1 \ar@{-}[d] \\
				a_2 \ar@{-}[d] \\
				a_3 \ar@{-}[d] \\
				a_4
			} \\
			&\cdots
		\end{split}\end{equation*} %}
		このことを命題の形で証明しておく。

		\begin{proposition}[木の生成]\label{prop:木の生成} %{
			$T_+A$の任意の元$t$は、$\zettai{t}$個の$i_TA$の元を根への
			接木$\land_0$で掛け合わせたもので書くことができる。
		\end{proposition} %prop:木の生成}
		\begin{proof} %{
			木の頂点数に関する帰納法によって証明する。
			頂点数が二つの木は、$a_1,a_2\in A$に対して$a_1\land_0 a_2=a_1[a_2]$
			となるから命題が成り立つ。$n$を$2$以上の自然数とし、頂点数が$n$個の木
			に対して命題が成り立つとする。頂点数が$n+1$の任意の木$t\in T_+A$は
			ある自然数$p$があって、空でない木$t_1,t_2,\dots,t_p,t_{p+1}\in T_+X$
			と根$a\in A$によって$t=a[t_1t_2\cdots t_pt_{p+1}]$と書ける。
			木$t_p$の頂点数は$n$以下で、木$a[t_1t_2\cdots t_p]$の頂点数も$n$以下
			である。したがって、帰納法の仮定によって、木$t_p$も
			木$a[t_1t_2\cdots t_p]$も共に$T_1X$からに$\land_0$よって生成される。
			また、$t=a[t_1t_2\cdots t_p]\land_0 t_{p+1}$だから、$t$もまた$\land_0$に
			よって生成される。
			したがって、頂点数が$n+1$に場合にも命題が成り立つ。
		\end{proof} %}

		根への接木$\land_0$に前節\ref{s2:作用から積の導出}の議論を適用してみる。
		任意の$a\in A,\;w\in WT_+A$と任意の木$t_1,t_2\in WT_+A$に対して、
		$((aw)\land_0 t_1)\land_0 t_2=a(w*[t_1t_2])$となるから、
		任意の森$w_1\in WT_+A$に対して$(aw)\land_0 w_1=a(w*w_1)$と定義すると、
		$((aw)\land_0 t_1)\land_0 t_2=(aw)\land_0([t_1]*[t_2])$と書ける。
		このことを、定義と命題の形にまとめておく。

		\begin{definition}[森による根への接木]\label{def:森による根への接木} %{
			$R$双線形作用$\beta_{\land_0}: RT_+A\otimes RWT_+A \to RT_+A$を、
			任意の$t\in T_+A$に対して次のように、
			\begin{equation*}\begin{split} %{
				t\land_0 1_W = t
			\end{split}\end{equation*} %}
			任意の$a\in A,\;w_1,w_2\in WT_+A$に対して次のように定義する。
			\begin{equation*}\begin{split} %{
				(aw_1)\land_0 w_2 &= a(w_1*w_2)
			\end{split}\end{equation*} %}
			作用$\land_0$を森による根への接木または、単に、根への接木ということに
			する。
		\end{definition} %def:森による根への接木}

		\begin{proposition}[根への接木と単語の連結の関係]\label{prop:根への接木と単語の連結の関係} %{
			任意の$t\in T_+A,\;w_1,w_2\in WT_+A$に対して次の式が成り立つ。
			\begin{equation*}\begin{split} %{
				(t\land_0 w_1)\land_0 w_2 = t\land_0(w_1*w_2)
			\end{split}\end{equation*} %}
		\end{proposition} %prop:根への接木と単語の連結の関係}
		\begin{proof} %{
			任意の木$t$はある$a\in A,\;w\in WT_+A$で$t=aw$と書くことができる。
			森による根への接木$\land_0$の定義\ref{def:森による根への接木}より、
			任意の$w_1,w_2\in WT_+A$に対して$
			((aw)\land_0 w_1)\land_0 w_2=a(w*w_1*w_2)=(aw)\land_0(w_1*w_2)
			$となり命題が成り立つ。
		\end{proof} %}

		根への接木に双対となる操作を定義する。

		\begin{definition}[根での枝刈り]\label{def:根での枝刈り} %{
			$R$線形余作用$\widebar{\land}_0:RT_+A\to RT_+A\otimes RWT_+A$を
			任意の$a\in A,\;w\in WT_+A$に対して次のように定義する。
			\begin{equation*}\begin{split} %{
				\widebar{\land}_0(aw) &= (aw_{(1)})\otimes(w_{(2)})
				\quad\text{where } \Delta_*w = w_{(1)}\otimes w_{(2)} \\
			\end{split}\end{equation*} %}
		\end{definition} %def:根での枝刈り}

		\begin{proposition}[根での枝刈りと単語の連結の関係]\label{prop:根での枝刈りと単語の連結の関係} %{
			根での枝刈り$\widebar{\land}_0$と余積$\Delta_*$は次の関係が成り立つ。
			\begin{equation*}\begin{split} %{
				(\widebar{\land}_0\otimes\myid)\widebar{\land}_0
				= (\myid\otimes\Delta_*)\widebar{\land}_0
			\end{split}\end{equation*} %}
		\end{proposition} %prop:根での枝刈りと単語の連結の関係}
		\begin{proof} %{
			$\Delta_*$に対してSweedler記法を使ってテンソルの成分を計算する。
			任意の$w\in WT_+A$に対して次の式が成り立つ。
			\begin{equation*}\begin{split} %{
				(\widebar{\land}_0\otimes\myid)\widebar{\land}_0(aw) 
				&= (aw_{(11)})\otimes w_{(21)}\otimes w_{(2)}
				\quad\lcomment{$\widebar{\land}_0$の定義} \\
				&= (aw_{(1)})\otimes w_{(12)}\otimes w_{(22)}
				\quad\lcomment{$\Delta_*$の余結合性} \\
				&= (\myid\otimes \Delta_*)\Bigl((aw_{(1)})\otimes w_{(2)}\Bigr) \\
				&= (\myid\otimes \Delta_*)\widebar{\land}_0(aw)
				\quad\lcomment{$\widebar{\land}_0$の定義} \\
			\end{split}\end{equation*} %}
		\end{proof} %}

		命題\ref{prop:根への接木と単語の連結の関係}と
		命題\ref{prop:根での枝刈りと単語の連結の関係}を作用素の形で書くと
		次のようになる。
		\begin{equation*}\begin{split} %{
			\land_0(\land_0\otimes \myid) &= \land_0(\myid\otimes m) \\
			(\widebar{\land}_0\otimes \myid)\widebar{\land}_0 
			&= (\myid\otimes \Delta_*)\widebar{\land}_0 \\
		\end{split}\end{equation*} %}
		結合性と余結合性の定義によく似た形をしている。

		根に森を付け足して木にする$R$双線形な操作
		$\myop{tree}:RA\otimes RWT_+A\to RT_+A$は
		$\myop{tree}= \beta_{\land_0}(i_T\otimes\myid)$と書くことができる。

		次の例は、可換図\eqref{eq:行きがけ順を保つ二項演算}が
		二項演算$\land_0$を定めない例である。
		\begin{equation*}\begin{split} %{
			\mytree{
				& a_1 \ar@{-}[dl] \ar@{-}[d] \ar@{-}[dr] \\
				a_2 & a_3 & *+[F]{t} \\
			} \xrightarrow{\pi_W} [a_1a_2a_3] * (\pi_W t) \xleftarrow{\pi_W}
			\mytree{
				& a_1 \ar@{-}[dl] \ar@{-}[dr] \\
				a_2 && a_3 \ar@{-}[d] \\
				&& *+[F]{t}
			}
		\end{split}\end{equation*} %}
		頂点が$2$個以上の木では、行きがけ順で最後になるように頂点を付け加える
		方法は$2$通りある。一つの方法が、根の最右の子供として付け加える方法、
		もう一つの方法が、現在の行きがけ順で最後の頂点の子供として付け加える
		方法である。その両方の方法を足し合わせてしまうことを考える。

		\begin{todo}[足し合わせた場合]\label{todo:足し合わせた場合} %{
			どのような森の二項演算が導かれるか?
		\end{todo} %todo:足し合わせた場合}
	%s2:根への接木}

	\begin{todo}[ここまで]\label{todo:ここまで} %{
	\end{todo} %todo:ここまで}

	木$t\in TA$の頂点の集合を$\set{t}$と書く。
	$WTA=(TA,m_*,1_*)$を$TA$から生成された自由モノイドとする。
	$WTA$の元を$TA$の元を並べて括弧でくくって表すことにする。
	例えば、$t_1,t_2,\dots,t_m\in TA$に対して$[t_1t_2\cdots t_m]$と書く。
	$m_*$は文字列の連結で、$1_*$は空文字である。
	$WTA$の元を森と言うことにする。
	森$w\in WTA$の頂点の集合を$\set{w}$と書く。例えば、
	木$t_1,t_2,\dots,t_m\in TA$に対して$
	\set{[t_1t_2\cdots t_m]}=\set{t_1}\cup\set{t_2}\cup\cdots\cup\set{t_m}
	$となる。木から頂点を取り除いてできる森を木の森ということにする。
	例えば、$t_1,t_2,\dots,t_m\in TA$に対して、森$[t_1t_2\cdots t_m]$は、
	任意の$s\in S$を頂点とする木$s[t_1t_2\cdots t_m]$の森となる。

	写像$\myop{tree}$を次のように定義する。
	\begin{equation}\begin{split} %{
		\myop{tree}: S\times WTA &\to TA \\
		s\times w &\mapsto sw \\
	\end{split}\end{equation} %}
	写像$\myop{tree}$は集合同型となる。$\myop{tree}$の逆写像を
	$\myop{tree}^{^1}$書く。
	\begin{equation}\begin{split} %{
		\myop{tree}^{-1}: TA &\to S\times WTA \\
		sw &\mapsto s\times w \\
	\end{split}\end{equation} %}
	木から根を取り出す操作を$\myop{root}=\pi_1\myop{tree}^{-1}$、
	木から根の子供達を$\myop{forest}=\pi_2\myop{tree}^{-1}$と書く。

	$R=(R,+,0,\myspace,1)$を半環とする。
	$RS$を$S$を基底とする$R$係数半モジュール、
	$RTA$を$TA$を基底とする$R$係数半モジュール、
	$RWTA$を$WTA$を基底とする$R$係数半モジュールとする。
	文字列の連結$m_*:WTA\times WTA\to WTA$を$R$線形に$RWTA$に拡張したものを
	同一の記号$m_*$で書くことにする。積$m_*$に双対になる余積$\Delta_*$を次の
	ように定義する。
	\begin{equation}\begin{split} %{
		\Delta_*: RWTA\otimes RWTA &\to RWTA \\
			[t_1t_2\cdots t_m] &\mapsto [t_1t_2\cdots t_m]\otimes 1_* \\
				&\quad + \sum_{1\le i\le m}[t_1t_2\cdots t_m]\neg\set{i}\otimes [t_i] \\
				&\quad + \sum_{1\le i<j\le m}[t_1t_2\cdots t_m]\neg\set{i,j}\otimes [t_it_j] \\
				&\quad + \cdots \\
				&\quad + \sum_{1\le i<j\le m}[t_i]\otimes [t_1t_2\cdots t_m]\neg\set{i} \\
				&\quad + \sum_{1\le i<j\le m}1_*\otimes [t_1t_2\cdots t_m] \\
	\end{split}\end{equation} %}
	ここで、$[t_1t_2\cdots t_m]\neg{i_1,i_2,\dots,i_p}$は$[t_1t_2\cdots t_m]$
	から$i_1,i_2,\dots,i_p$番目の文字を除いた単語とする。
	例えば、次のようになる。
	\begin{equation*}\begin{split} %{
		[t_1t_2t_3]\neg\set{1} &= [t_2t_3] \\
		[t_1t_2t_3]\neg\set{1,3} &= [t_2] \\
		[t_1t_2t_3]\neg\set{1,2,3} &= [] \\
	\end{split}\end{equation*} %}
	余積$\Delta_*$に対する余単位射$\epsilon_*$は次のようになる。
	\begin{equation}\begin{split} %{
		\epsilon_*: RTW &\to R \\
		w &\mapsto \begin{cases} %{
			1, &\text{ iff }w=1_* \\
			0, &\text{ otherwise } \\
		\end{cases} %}
	\end{split}\end{equation} %}

	$RTA$に積を定義するための準備をする。

	\begin{definition}[頂点を指定した接木]\label{def:頂点を指定した接木} %{
		木$t\in TA$の頂点$i\in \set{t}$の最後の子供に木$t_1\in TA$を付け加える
		操作を$t_1$の頂点$i$への接木ということにする。
	\end{definition} %def:頂点を指定した接木}

	ラベルによらずに頂点の位置を表すために頂点を'頂点の位置:ラベル'
	という形で書いて、接木を図示すると次のようになる。
	\begin{equation*}\begin{split} %{
		\mytree{
			& 0:a_0 \ar@{-}[dl] \ar@{-}[dr] \\
			1:a_1 \ar@{-}[d] && 3:a_3 \\
			2:a_2 \\
		} \lhd_{1} \mytree{
			& r_0 \ar@{-}[dl] \ar@{-}[dr] \\
			r_1 && r_2 \\
		} &= \mytree{
			&& 0:a_0 \ar@{-}[dl] \ar@{-}[dr] \\
			& 1:a_1 \ar@{-}[dl] \ar@{-}[dr] && 3:a_3 \\
			2:a_2 && r_0 \ar@{-}[dl] \ar@{-}[dr] \\
			& r_1 && r_2 \\
		}
	\end{split}\end{equation*} %}

	すべての頂点にわたって接木をする操作を定義する。

	\begin{definition}[木への接木]\label{def:木への接木} %{
		木$t\in TA$の頂点$i\in \set{t}$への接木を$\lhd_i$と書く。
		次の$R$双線形写像$\lhd$を木への接木ということにする。
		\begin{equation}\begin{split} %{
			\lhd: RTA\otimes RTA &\to RTA \\
				t\otimes u &\mapsto \sum_{i\in \set{t}}t\lhd_{i}u
				\quad\text{for all }t,u\in TA
		\end{split}\end{equation} %}
	\end{definition} %def:木への接木}

	木への接木を図示すると次のようになる。
	\begin{equation*}\begin{split} %{
		\mytree{
			& a_0 \ar@{-}[dl] \ar@{-}[dr] \\
			a_1 \ar@{-}[d] && a_3 \\
			a_2 \\
		} \lhd \mytree{
			& r_0 \ar@{-}[dl] \ar@{-}[dr] \\
			r_1 && r_2 \\
		} &= \mytree{
			& a_0 \ar@{-}[dl] \ar@{-}[d] \ar@{-}[dr] \\
			a_1 \ar@{-}[d] & a_3 & r_0 \ar@{-}[dl] \ar@{-}[dr] \\
			a_2 & r_1 && r_2 \\
		} + \mytree{
			&& a_0 \ar@{-}[dl] \ar@{-}[dr] \\
			& a_1 \ar@{-}[dl] \ar@{-}[dr] && a_3 \\
			a_2 && r_0 \ar@{-}[dl] \ar@{-}[dr] \\
			& r_1 && r_2 \\
		} \\
		&+ \mytree{
			&& a_0 \ar@{-}[dl] \ar@{-}[dr] \\
			& a_1 \ar@{-}[d] && a_3 \\
			& a_2 \ar@{-}[d] \\
			& r_0 \ar@{-}[dl] \ar@{-}[dr] \\
			r_1 && r_2 \\
		} + \mytree{
			& a_0 \ar@{-}[dl] \ar@{-}[dr] \\
			a_1 \ar@{-}[d] && a_3 \ar@{-}[d] \\
			a_2 && r_0 \ar@{-}[dl] \ar@{-}[dr] \\
			& r_1 && r_2 \\
		}
	\end{split}\end{equation*} %}

	接木はの$R$双線形二項演算であるが、結合的ではないことに注意する。
	例えば次のようになって、結合性は満たさない。
	\begin{equation*}\begin{split} %{
		\bigl([a_1]\lhd[a_2]\bigr)\lhd[a_3] &= [a_1]\lhd\bigl([a_2]\lhd[a_3]\bigr)+[a_1[a_2][a_3]] \\
		[a_1]\lhd\bigl([a_2]\lhd[a_3]\bigr) &= [a_1[a_2[a_3]]] \\
	\end{split}\end{equation*} %}

	木への接木を森による接木に拡張する。

	\begin{definition}[森による木への接木]\label{def:森による木への接木} %{
		木$t\in TA$の頂点$i\in \set{t}$への接木を$\lhd_i$と書く。
		次の$R$双線形写像$\lhd$を森による木への接木または単に木への接木
		ということにする。
		\begin{equation}\begin{split} %{
			\lhd: RTA\otimes RWTA &\to RTA \\
				t\otimes 1_* &\mapsto t \quad\text{for all }t\in TA \\
				t\otimes [t_1t_2\cdots t_m] 
				&\mapsto \sum_{i_1,i_2,\dots,i_m\in\set{t}}t\lhd_{i_1}t_1\lhd_{i_2}t_2\cdots\lhd_{i_m}t_m \\
				&\quad\text{for all }t,t_1,t_2,\dots,t_m\in TA \\
		\end{split}\end{equation} %}
	\end{definition} %def:森による木への接木}

	木への接木の結合則からのずれが、森による木への接木によって表される。

	\begin{proposition}[接木の結合則もどき]\label{prop:接木の結合則もどき} %{
		任意の木$t_1,t_2,t_3\in TA$に対して次の式が成り立つ。
		\begin{equation}\begin{split} %{
			(t_1\lhd t_2)\lhd t_3 = t_1\lhd(t_2\lhd t_3) + t_1\lhd[t_2t_3]
		\end{split}\end{equation} %}
		ここで、左辺と右辺の第一項目の$\lhd$は木による接木で、右辺の第二項目は
		森による接木である。
	\end{proposition} %prop:接木の結合則もどき}
	\begin{proof} %{
		\begin{equation*}\begin{split} %{
			(t_1\lhd t_2)\lhd t_3 
			&= \sum_{i_3\in\set{t_1}\cup\set{t_2}}(t_1\lhd t_2)\lhd_{i_3} t_3 \\
			&= \left(\sum_{i_3\in\set{t_1}}+\sum_{i_3\in\set{t_2}}\right)(t_1\lhd t_2)\lhd_{i_3} t_3 \\
			&= \left(\sum_{i_2,i_3\in\set{t_1}}t_1\lhd_{i_2} t_2\lhd_{i_3} t_3\right)
			+ \bigl(t_1\lhd(t_2\lhd t_3)\bigr) \\
		\end{split}\end{equation*} %}
		右辺の第一項目は森による接木の定義により$t_1\lhd[t_2t_3]$となり、
		命題が成り立つ。
	\end{proof} %}

	森による木への接木は根を不変に保つので、次の式により$R$双線形二項演算
	$\perp$を定義することができる。
	\begin{equation}\label{eq:接木による森の二項演算の定義}\begin{split} %{
		(aw_1)\lhd w_2 = a(w_1\perp w_2)\quad\text{for all }a\in A,\;w_1,w_2\in WTA
	\end{split}\end{equation} %}
	$\perp$は根のラベル$a$には依存しない。つまり、
	$(aw_1)\lhd w_2=a(w_1\perp_aw_2)$、$(bw_1)\lhd w_2=b(w_1\perp_bw_2)$、
	とおいたとき、任意の$w_1,w_2\in WTA$に対して
	$(w_1\perp_aw_2)=(w_1\perp_bw_2)$となる。

	\begin{definition}[接木による森の二項演算]\label{def:接木による森の二項演算} %{
		$R$双線形写像$\perp:RTWA\otimes RTWA\to RTWA$を次のように定義する。
		\begin{equation}\begin{split} %{
			(aw_1)\lhd w_2 = a(w_1\perp w_2)\quad\text{for all }a\in A,\;w_1,w_2\in WTA
		\end{split}\end{equation} %}
	\end{definition} %def:接木による森の二項演算}

	前置記号を用いると$
	\mybiop{\lhd}=\myop{tree}(\myid\otimes\mybiop{\perp})
	(\myop{tree}^{-1}\otimes\myid)
	$となる。

	$\perp$を計算してみる。まず、任意の$a\in A,\;w\in WTA$に対して
	次の式が成り立つ。
	\begin{equation}\label{eq:接木による積の定義その一}\begin{split} %{
		(a[])\lhd w=aw &\implies 1_*\perp w=w \\
		(aw)\lhd 1_*=aw &\implies w\perp 1_*=w \\
	\end{split}\end{equation} %}
	また、任意の$a\in A,\;t_1,t_2,\dots,t_m\in TA$と任意の$1_*$でない
	$w\in WTA$に対しては次の式が成り立つ。
	\begin{equation}\begin{split} %{
		(aw)\lhd[t_1t_2\cdots t_m]
			&= \sum_{i_1,i_2,\dots,i_m\in\set{aw}}(aw)
			\lhd_{i_1}t_1\lhd_{i_2}t_2\cdots\lhd_{i_m}t_m \\
			%
			&= (aw)\lhd_{\set{a}}[t_1t_2\cdots t_m] \\
			&\;+ \sum_{1\le i\le m}(aw)
			\lhd_{\set{a}}[t_1t_2\cdots t_m]_{\neg\set{i}}\lhd_{\set{w}}[t_i] \\
			&\;+ \sum_{1\le i<j\le m}(aw)
			\lhd_{\set{a}}[t_1t_2\cdots t_m]_{\neg\set{i,j}}\lhd_{\set{w}}[t_it_j] \\
			&\;+ \cdots \\
			&\;+ (aw)\lhd_{\set{w}}[t_1t_2\cdots t_m] \\
	\end{split}\end{equation} %}
	ここで、木$t\in TA$の頂点の部分集合$v\in\set{t}$に対して
	$t\lhd_v[t_1t_2\cdots t_m]$を次のようにおいた。
	\begin{equation}\begin{split} %{
		t\lhd_v[t_1t_2\cdots t_m] 
		&= \sum_{i_1,i_2,\dots,i_m\in v}t\lhd_{i_1}t_1\lhd_{i_2}t_2\cdots\lhd_{i_m}t_m
	\end{split}\end{equation} %}
	したがって、$\perp$は任意の$t,t_1,t_2,\dots,t_m\in TA$と
	任意の$1_*$でない$w\in WTA$に対して次のようになることがわかる。
	\begin{equation}\label{eq:接木による積の定義その二}\begin{split} %{
		w\perp[t_1t_2\cdots t_m]
		&= w\lhd[t_1t_2\cdots t_m] \\
		&\;+ \sum_{1\le i\le m}(w\lhd[t_1t_2\cdots t_m]_{\neg\set{i}})*[t_i] \\
		&\;+ \sum_{1\le i<j\le m}(w\lhd[t_1t_2\cdots t_m]_{\neg\set{i,j}})*[t_it_j] \\
		&\;+ \cdots \\
		&\;+ w*[t_1t_2\cdots t_m] \\
	\end{split}\end{equation} %}
	ここで、$w_1,w_2\in WTA$に対して$w_1\lhd w_2$を$w_1$のすべての頂点に
	接木をした和とした。任意の$t\in TA$、$w\in WTA$に対して
	$[t]\lhd w=[t\lhd w]$となり、
	任意の$t,u_1,u_2,\dots,u_m\in TA$、任意の$1_*$でない$w\in WTA$に対して
	次の再帰式を満たす。
	\begin{equation*}\begin{split} %{
		&([t]*w)\lhd [u_1u_2\cdots u_m] \\
		&= ([t]\lhd 1_*)*(w\lhd [u_1u_2\cdots u_m]) \\
		&\;+ \sum_{1\le i\le m}([t]\lhd[u_i])*(w\lhd [u_1u_2\cdots u_m]_{\neg\set{i}}) \\
		&\; + \sum_{1\le i<j\le m}([t]\lhd[u_iu_j])*(w\lhd [u_1u_2\cdots u_m]_{\neg\set{i,j}}) \\
		&\; + \cdots \\
		&\; + ([t]\lhd [u_1u_2\cdots u_m])*(w\lhd 1_*) \\
	\end{split}\end{equation*} %}
	この式は、任意の$1_*$でない$w_1,w_2\in WTA$と任意の$w_3\in WTA$に対する
	次の再帰式にまとまる。
	\begin{equation}\label{eq:森同士の接木}\begin{split} %{
		(w_1*w_2)\lhd w_3 = \left(w_1\lhd (\Delta_*^{(1)}w_3)\right)*\left(w_1\lhd (\Delta_*^{(2)}w_3)\right) \\
	\end{split}\end{equation} %}
	さらに、任意の$w\in WTA$に対して$1_*\lhd w=\jump{w=1_*}1_*$と定義すると、
	式\eqref{eq:森同士の接木}は、任意の$t\in TA,\;w\in WTA$に対して
	$[t]\lhd w=[t\lhd w]$を満たす。改めて、森への接木$\lhd$を定義しておく。

	\begin{definition}[森への接木]\label{def:森への接木} %{
		木$t\in TA$の頂点$i\in \set{t}$への接木を$\lhd_i$と書く。
		次の$R$双線形写像$\lhd:RWTA\otimes RWTA\to RWTA$を森による森への接木
		または単に森への接木ということにする。
		\begin{itemize} %{
			\item 任意の$w\in WTA$に対して
			\begin{equation*}\begin{split} %{
				1_*\lhd w = \begin{cases} %{
					1_*, &\text{ iff }w=1_* \\
					0, &\text{ otherwise } \\
				\end{cases} %}
			\end{split}\end{equation*} %}
			\item 任意の$t\in TA,\;w\in WTA$に対して
			\begin{equation*}\begin{split} %{
				[t]\lhd w = [t\lhd w] 
			\end{split}\end{equation*} %}
			\item 任意の$w_1,w_2\in WTA$に対して
			\begin{equation*}\begin{split} %{
				(w_1*w_2)\lhd w 
				&= \left(w_1\lhd(\Delta_*^{(1)}w)\right) * \left(w_2\lhd(\Delta_*^{(2)}w)\right) \\
				&= m_*\bigl((w_1\otimes w_2)\lhd(\Delta_*w)\bigr) \\
			\end{split}\end{equation*} %}
		\end{itemize} %}
	\end{definition} %def:森への接木}

	$\perp$の定義\eqref{eq:接木による積の定義その一}と
	\eqref{eq:接木による積の定義その二}は、任意の森$w_1,w_2\in WTA$に対して、$
	w_1\perp w_2=\left(w_1\lhd(\Delta_*^{(2)}w_2)\right)*(\Delta_*^{(1)}w_2)
	$と書かれることがわかる。
	森への接木\ref{def:森への接木}では、空文字への接木以外は
	木への接木\ref{def:森による木への接木}を自然に拡張したものになっている。
	空文字への接木に自然な拡張がないので、式$
	w_1\perp w_2=\left(w_1\lhd(\Delta_*^{(2)}w_2)\right)*(\Delta_*^{(1)}w_2)
	$が成り立つように空文字への接木$1_*\lhd1_*=\jump{w=1_*}1_*$を定義している。
	木への接木\ref{def:木への接木}が結合的でないのと同様に、森への接木
	\ref{def:森への接木}も結合的でない。
	その一方で、森への接木は右単位元$1_*$をもつことに注意する。

	\begin{proposition}[森の二項演算と森への接木の関係]\label{prop:森の二項演算と森への接木の関係} %{
		任意の$w_1,w_2\in WTS$に対して次の式が成り立つ。
		\begin{equation}\begin{split} %{
			w_1\perp w_2 
			= \left(w_1\lhd(\Delta_*^{(2)}w_2)\right)*(\Delta_*^{(1)}w_2)
		\end{split}\end{equation} %}
	\end{proposition} %prop:森の二項演算と森への接木の関係}
	\begin{proof} %{
		命題の式が成り立つように森への接木$\lhd$を定義した。
	\end{proof} %}

	$1_*$が接木による森の二項演算$\perp$の単位元になることはすぐわかる。

	\begin{proposition}[接木による森の二項演算の単位元]\label{prop:接木による森の二項演算の単位元} %{
		空文字$1_*$は接木による森の二項演算$\perp$の単位元となる。
	\end{proposition} %prop:接木による森の二項演算の単位元}
	\begin{proof} %{
		式\eqref{eq:接木による積の定義その一}によって証明は完了するが、
		ここでは、命題\ref{prop:森の二項演算と森への接木の関係}を用いて
		証明してみる。
		$\Delta_*1_*=1_*\otimes 1_*$より、任意の$w\in WTA$に対して
		$w\perp1_*=(w\lhd1_*)*1_*$となり、森への接木の定義\ref{def:森への接木}
		より、$w\lhd1_*=w$となるから$w\perp1_*=w$となる。
		ここで、$i,j=1,2$に対して$w^{(i)}=\Delta_*^{(j)}w$とする。
		$1_*\perp w=(1_*\lhd w^{(2)})*w^{(1)}$となり、
		森への接木の定義\ref{def:森への接木}より、$1_*w=(\epsilon_*w)1_*$
		となるから、$1_*\perp w=(\epsilon_*w^{(2)})w^{(1)}$となる。
	\end{proof} %}

	接木の結合則もどきの命題\ref{prop:接木の結合則もどき}を拡張することを
	考える。森の二項演算$\perp$の定義\ref{def:接木による森の二項演算}は、
	任意の$a\in A,\;w_1,w_2\in WTA$に対して次のように書き換えることができる。
	\begin{equation*}\begin{split} %{
		([a]\lhd w_1)\lhd w_2 = [a]\lhd(w_1\perp w_2)
	\end{split}\end{equation*} %}
	根だけの木$[a]$から一般の森に変更することで次の命題が得られる。

	\begin{proposition}[森への接木の結合則もどき]\label{prop:森への接木の結合則もどき} %{
		任意の森$w_1,w_2,w_3\in TA$に対して次の式が成り立つ。
		\begin{equation}\begin{split} %{
			(w_1\lhd w_2)\lhd w_3 = w_1\lhd(w_2\perp w_3)
		\end{split}\end{equation} %}
	\end{proposition} %prop:森への接木の結合則もどき}
	\begin{proof} %{
		$\perp$の定義\ref{eq:接木による森の二項演算の定義}から
		命題\ref{prop:森の二項演算と森への接木の関係}を導いた手順を繰り返す
		ことで証明が得られる。まず、任意の$w_1,w_2\in WTA$に対して、$1_*$が
		$\perp$の単位元だから次の式が成り立つ。
		\begin{equation*}\begin{split} %{
			(w_1\lhd w_2)\lhd 1_*=w_1\lhd w_2=w_1\lhd(w_2\perp 1_*)
		\end{split}\end{equation*} %}
		次に、任意の$w_1,w_2\in WTA$と$t_1,t_2,\dots,t_m\in TA$に対して、
		次の式が成り立つ。
		\begin{equation*}\begin{split} %{
			&(w_1\lhd w_2)\lhd[t_1t_2\cdots t_m] \\
			&= \sum_{i_1,i_2,\dots,i_m\in\set{w_1}\cup\set{w_1}}(w_1\lhd w_2)
			\lhd_{i_1}t_1\lhd_{i_2}t_2\cdots\lhd_{i_m}t_m \\
			&= (w_1\lhd w_2)\lhd_{\set{w_1}}[t_1t_2\cdots t_m] \\
			&\;+ \sum_{1\le i\le m}(w_1\lhd w_2)
			\lhd_{\set{w_1}}[t_1t_2\cdots t_m]_{\neg\set{i}}\lhd_{\set{w_2}}[t_i] \\
			&\;+ \sum_{1\le i<j\le m}(w_1\lhd w_2)
			\lhd_{\set{w_1}}[t_1t_2\cdots t_m]_{\neg\set{i,j}}\lhd_{\set{w_2}}[t_it_j] \\
			&\;+ \cdots \\
			&\;+ (w_1\lhd w_2)\lhd_{\set{w_2}}[t_1t_2\cdots t_m] \\
			&= w_1\lhd(w_2*[t_1t_2\cdots t_m]) \\
			&\;+ \sum_{1\le i\le m}w_1
			\lhd\bigl((w_2\lhd[t_i])*[t_1t_2\cdots t_m]_{\neg\set{i}}\bigr) \\
			&\;+ \sum_{1\le i<j\le m}w_1
			\lhd\bigl((w_2\lhd[t_it_j])*[t_1t_2\cdots t_m]_{\neg\set{i,j}}\bigr) \\
			&\;+ \cdots \\
			&\;+ w_1\lhd(w_2\lhd[t_1t_2\cdots t_m]) \\
			&= w_1\lhd\Bigl(\bigl(w_1\lhd(\Delta_*^{(2)}[t_1t_2\cdots t_m])\bigr)*(\Delta_*^{(1)}[t_1t_2\cdots t_m])\Bigr) \\
			&= w_1\lhd(w_1\perp[t_1t_2\cdots t_m]) \\
		\end{split}\end{equation*} %}
		したがって、命題が証明された。
	\end{proof} %}

	$\Delta_*w=w\otimes 1_*+1_*\otimes w+r$とおくと、
	命題\ref{prop:森への接木の結合則もどき}から、$r$に関する$0$次近似が
	木への接木の場合\ref{prop:接木の結合則もどき}と同じ形になる次の式
	が得られる。
	\begin{equation*}\begin{split} %{
		(w_1\lhd w_2)\lhd w_3 = w_1\lhd(w_2\lhd w_3) + w_1\lhd(w_2*w_3) + (\text{order}\;r)
	\end{split}\end{equation*} %}

	\begin{proposition}[接木による森の二項演算は積]\label{prop:接木による森の二項演算は積} %{
		接木による森の二項演算$\perp$は積である。
		\begin{equation}\begin{split} %{
			(w_1\perp w_2)\perp w_3 = w_1\perp(w_2\perp w_3) \quad\text{for all }w_1,w_2,w_3\in WTA
		\end{split}\end{equation} %}
	\end{proposition} %prop:接木による森の二項演算は積}
	\begin{proof} %{
		任意の$a\in A,w,w_1,w_2\in WTA$に対して次の式が成り立つ。
		\begin{equation*}\begin{split} %{
			([aw]\lhd w_2)\lhd w_3 
			&= ((aw)\lhd w_2)\lhd w_3 \quad \lcomment{$\lhd$の定義} \\
			&= (a(w\perp w_2))\lhd w_3 \quad \lcomment{$\perp$の定義} \\
			&= a((w\perp w_2)\perp w_3) \quad \lcomment{$\perp$の定義} \\
			%
			([aw])\lhd(w_2\perp w_3)
			&= (aw)\lhd(w_2\perp w_3) \quad \lcomment{$\lhd$の定義} \\
			&= a(w\perp(w_2\perp w_3)) \quad \lcomment{$\perp$の定義} \\
		\end{split}\end{equation*} %}
		この結果を結合律もどき(命題\ref{prop:森への接木の結合則もどき})
		\begin{equation*}\begin{split} %{
			(w_1\lhd w_2)\lhd w_3 = w_1\lhd(w_2\perp w_3) \quad\text{for all }w_1,w_2,w_3\in WTA
		\end{split}\end{equation*} %}
		に当てはめると、$\perp$の結合律が導かれる。
		\begin{equation*}\begin{split} %{
			(w\perp w_2)\perp w_3 = w\perp(w_2\perp w_3) \quad\text{for all }w,w_1,w_2\in WTA
		\end{split}\end{equation*} %}
	\end{proof} %}

	\subsection{考察}\label{s2:考察} %{
		ここまでの議論を振り返ってみる。木ということを忘れて文字列で考えてみる。
		$A$を集合、$R$を半環とする。
		$WA$を自由モノイドから生成された自由モノイド、$RWA$を
		$WA$を基底とする$R$係数自由半モジュールとする。
		$RWA$の$n$文字だけからなる部分空間を$RW_nA$とする。$RWA$は直和分解できて
		$RWA=\oplus_{n=0}^\infty RW_nA$となる。$WA$の文字列の連結を$m_*$、
		その双対で$W_1A$の元をプリミティブとする余積を$\Delta_*$とする。

		$X$を集合、$RX$を$X$を基底とする$R$係数自由半モジュールとする。
		$RWA$の$RX$への$R$双線形作用$\lhd:RA\otimes RWX\to RX$と
		$RWA$の$R$双線形二項演算$\perp$が定義されていて、次の性質を持つとする。
		\begin{description} %{
			\item[単位性]ある$1_\perp$が存在して、任意の$x\in X$に対して
			次の式が成り立つ。
			\begin{equation*}\begin{split} %{
				x\lhd 1_\perp x = x 
			\end{split}\end{equation*} %}
			\item[結合性]任意の$w_1,w_2\in WA,\; x\in X$に対して次の式が成り立つ。
			\begin{equation*}\begin{split} %{
				x\lhd w_1\lhd w_2 = x\lhd (w_1\perp w_2)
			\end{split}\end{equation*} %}
			\item[キャンセル可能性]任意の$w_1,w_2\in WA,\; x\in X$に対して
			次の式が成り立つ。
			\begin{equation*}\begin{split} %{
				x\lhd w_1 = x\lhd w_2 \implies w_1=w_2
			\end{split}\end{equation*} %}
		\end{description} %}
		
		任意の$w_1,w_2,w_3\in WA,\; x\in X$に対して、内側から'結合性'を適用
		していくことによって、次の式が得られる。
		\begin{equation*}\begin{split} %{
			x\lhd w_1\lhd w_2\lhd w_3
			&= x\lhd (w_1\perp w_2)\lhd w_3 \\
			&= x\lhd \bigl((w_1\perp w_2)\perp w_3)\bigr) \\
		\end{split}\end{equation*} %}
		外側から'結合性'を適用していくことによって、次の式が得られる。
		\begin{equation*}\begin{split} %{
			x\lhd w_1\lhd w_2\lhd w_3
			&= x\lhd w_1\lhd (w_2\perp w_3) \\
			&= \bigl(w_1\perp (w_2\perp w_3)\bigr) \\
		\end{split}\end{equation*} %}
		'キャンセル可能性'を適用すると、$\perp$の結合性が導かれる。
		\begin{equation*}\begin{split} %{
			(w_1\perp w_2)\perp w_3 = w_1\perp (w_2\perp w_3) 
		\end{split}\end{equation*} %}
		また、任意の$w\in WA,\; x\in X$に対して、'単位性'と'結合性'を適用
		すると、次の式が導かれ、
		\begin{equation*}\begin{split} %{
			x\lhd w\lhd 1_\perp
			&= x\lhd w \quad\lcomment{'単位性'} \\
			&= x\lhd (w\perp 1_\perp) x \quad\lcomment{'結合性'} \\
		\end{split}\end{equation*} %}
		'キャンセル可能性'を適用すると、$1_\perp$が$\perp$の右単位元になることが
		導かれる。同様に、次の式が導かれ、
		\begin{equation*}\begin{split} %{
			x\lhd 1_\perp\lhd w
			&= x\lhd w \quad\lcomment{'単位性'} \\
			&= x\lhd (1_\perp\perp w) \quad\lcomment{'結合性'} \\
		\end{split}\end{equation*} %}
		'キャンセル可能性'を適用すると、$1_\perp$が$\perp$の左単位元になることが
		導かれる。以上より、作用$\rhd$から半代数$(RWA,\perp,1_\perp)$の$RX$への
		表現が得られることがわかる。

		\subsubsection{木の場合の考察}\label{s3:木の場合の考察} %{
			$X$を集合、$R$を半環とする。
			$TX$を$X$を頂点とする木の集合、$T_{\bullet}X$を$\bullet$を根、
			$X$を根以外の頂点とする木の集合とする。
			$X$の$RT_{\bullet}X$への作用$
			\beta_\lhd:RT_\bullet X\otimes RX\to RT_\bullet X
			$を次のように定義する。
			\begin{equation}\label{eq:頂点の作用の定義}\begin{split} %{
				t\lhd x &= \sum_{i\in\set{t}}t\lhd_i x \\
				t\lhd_i x &= \text{$t$の頂点$i$の最右の子供として$x$を付け足す}
			\end{split}\end{equation} %}
			例えば、$x_1,x_2,\dots\in X$として、次のようになる。
			\begin{equation}\label{eq:頂点による接木の例}\begin{split} %{
				\bullet\lhd x_1 &= \bullet[x_1] \\
				\bullet[x_1]\lhd x_2 &= \bullet[x_1x_2] + \bullet[x_1[x_2]] \\
				\bullet[x_1x_2]\lhd x_3 &= \bullet[x_1x_2x_3] + \bullet[x_1[x_3]x_2] + \bullet[x_1x_2[x_3]] \\
				\bullet[x_1[x_2]]\lhd x_3 &= \bullet[x_1[x_2]x_3] + \bullet[x_1[x_2x_3]] + \bullet[x_1[x_2[x_3]]] \\
				\dots \\
			\end{split}\end{equation} %}

			'結合性'\;$t\lhd x_1\lhd x_2=t\lhd(x_1\perp x_2)$から積$\perp$を
			定義することを考える。
			\begin{equation}\label{eq:接木のフェチその一}\begin{split} %{
				\bullet\lhd x_1 &= \bullet[x_1] \\
				\bullet\lhd(x_1\perp x_2) 
				&\sim \bullet\lhd\Bigl([x_1x_2] + [x_1[x_2]]\Bigr) \\
				\bullet(x_1\perp x_2\perp x_3) 
				&\sim \bullet\lhd\Bigl([x_1x_2x_3]+[x_1[x_3]x_2]+[x_1x_2[x_3]] \\
				&\;+ [x_1[x_2]x_3]+[x_1[x_2x_3]]+[x_1[x_2[x_3]]]\Bigr) \\
				\dots \\
			\end{split}\end{equation} %}
			$RX$では二項演算$\perp$は定義できないが、
			式\eqref{eq:接木のフェチその一}の右辺が木の単語の和になっているので、
			$RWTX$なら$\perp$が定義できそうである。したがって、
			$X$からへ$TX$の入射$i_T:x\mapsto[x]$と$TX$からへ$WTX$の入射
			$i_W:t\mapsto[t]$を用いて、次の可換図によりそれぞれの$RT_\bullet X$
			への作用を定義することを考える。
			\begin{equation*}\xymatrix{
				RX \ar[rd]_{\beta_\lhd} \ar[r]^{i_T} 
				& RTX \ar@{.>}[d]^{\beta_\lhd} \ar[r]^{i_W}
				& RWTX \ar@{.>}[ld]^{\beta_\lhd} \\
				& \myop{end}RT_\bullet X
			}\end{equation*}

			まず、作用$\beta_\lhd:RT_\bullet X\otimes RTX\to RT_\bullet X$を
			定義することを考える。
			$X$の$RT_{\bullet}X$への作用の定義\eqref{eq:頂点の作用の定義}で
			付け足す頂点を付け足す木に拡張して、$TX$の$RT_{\bullet}X$への作用$
			\beta_\lhd:RT_\bullet X\otimes RTX\to RT_\bullet X$を次のように
			定義する。
			\begin{equation}\label{eq:木の作用の定義}\begin{split} %{
				t\lhd u &= \sum_{i\in\set{t}}t\lhd_i u 
				\quad\text{for all }t\in T_\bullet X,\; u\in TX \\
				t\lhd_i u &= \text{$t$の頂点$i$の最右の子供として$u$を付け足す}
			\end{split}\end{equation} %}
			例えば、$t_1,t_2,\dots\in TX$として、次のようになる。
			\begin{equation}\label{eq:木による接木の例}\begin{split} %{
				\bullet\lhd t_1 &= \bullet[t_1] \\
				\bullet[t_1]\lhd t_2 &= \bullet[t_1t_2] + \bullet[t_1\unlhd t_2] \\
				\bullet[t_1t_2]\lhd t_3 &= \bullet[t_1t_2t_3] + \bullet[(t_1\unlhd t_3)t_2] + \bullet[t_1(t_2\unlhd t_3)] \\
				\bullet[t_1\unlhd t_2]\lhd t_3 &= \bullet[(t_1\unlhd t_2)t_3] + \bullet[(t_1\unlhd t_2)\unlhd t_3] \\
				\dots \\
			\end{split}\end{equation} %}
			ここで、$R$双線形写像$\beta_\unlhd:RTX\otimes RTX\to RTX$を次のように
			定義した。
			\begin{equation}\label{eq:木の二項演算の定義}\begin{split} %{
				t_1\unlhd t_2 &= \sum_{i\in\set{t_1}}t_1\unlhd_i t_2
				\quad\text{for all }t_1,t_2\in TX \\
				t_1\unlhd_i t_2 &= \text{$t_1$の頂点$i$の最右の子供として$t_2$を付け足す}
			\end{split}\end{equation} %}
			$\lhd$の定義\ref{eq:木の作用の定義}と$\unlhd$の定義\ref{eq:木の二項演算の定義}
			は作用か二項演算かの違いだけで、行う操作はまったく同じである。
			したがって、ここで定義した$\unlhd$も同じ記号$\lhd$で書くことにする。
			ただし、$\lhd$を$RTX\otimes RTX\to RTX$の意味で使う場合は演算の順序
			が問題となる。演算順序の指定をしないで書いた場合は、左から演算する
			ものとする。例えば、$t_1\lhd t_2\lhd t_3:=(t_1\lhd t_2)\lhd t_3$
			である。それ以外の場合は、$t_1\lhd(t_2\lhd t_3)$のように括弧に
			よって演算順序を明示的に指定する。

			次に、作用$\beta_\lhd:RT_\bullet X\otimes RWTX\to RT_\bullet X$を
			定義することを考える。$WTX$の元を森ということにする。
			木の場合の例\eqref{eq:木による接木の例}を森の作用に拡張すると
			次のようになる。
			\begin{equation}\label{eq:森による接木の例その一}\begin{split} %{
				\bullet\lhd[t_1] &= \bullet[t_1] \\
				\bullet[t_1]\lhd[t_2] &= \bullet[t_1t_2] + \bullet[t_1\lhd t_2] \\
				\bullet[t_1t_2]\lhd[t_3] &= \bullet[t_1t_2t_3] + \bullet[(t_1\lhd t_3)t_2] + \bullet[t_1(t_2\lhd t_3)] \\
				\bullet[t_1\lhd t_2]\lhd[t_3] &= \bullet[(t_1\lhd t_2)t_3] + \bullet[t_1\lhd t_2\lhd t_3] \\
				\dots \\
			\end{split}\end{equation} %}
			さらに、次のようにを定義すると
			\begin{equation*}\begin{split} %{
				\bullet\lhd1_* &= \bullet \\
				\bullet\lhd[t_1t_2\cdots t_m] &= \bullet[t_1t_2\cdots t_m] \\
			\end{split}\end{equation*} %}
			式\eqref{eq:森による接木の例その一}は次のようになる。
			\begin{equation*}\begin{split} %{
				\bullet\lhd[t_1]\lhd[t_2] 
				&= \bullet\lhd\Bigl([t_1t_2]+[t_1\lhd t_2]\Bigr) \\
				\bullet\lhd[t_1t_2]\lhd[t_3] 
				&= \bullet\lhd\Bigl([t_1t_2t_3]+[(t_1\lhd t_3)t_2]+[t_1(t_2\lhd t_3)]\Bigr) \\
				\bullet\lhd[t_1\lhd t_2]\lhd[t_3] 
				&= \bullet\lhd\Bigl([(t_1\lhd t_2)t_3]+[t_1\lhd t_2\lhd t_3]\Bigr) \\
				\dots \\
			\end{split}\end{equation*} %}
			任意の$w_1,w_2\in WTX$に対して$
			\bullet\lhd w_1=\bullet\lhd w_2 \implies w_1=w_2
			$が成り立つから、次のように$R$双線形二項演算$
			\beta_\perp:RWTX\otimes RWTX\to RWTX
			$が定義される。
			\begin{equation*}\begin{split} %{
				[t_1]\perp[t_2] &= [t_1t_2]+[t_1\lhd t_2] \\
				[t_1t_2]\perp[t_3] 
				&= [t_1t_2t_3]+[(t_1\lhd t_3)t_2]+[t_1(t_2\lhd t_3)] \\
				[t_1\lhd t_2]\perp[t_3]
				&= [(t_1\lhd t_2)t_3]+[t_1\lhd t_2\lhd t_3] \\
				\dots \\
			\end{split}\end{equation*} %}
			さらに、$\lhd$の定義\eqref{eq:木の二項演算の定義}より
			\begin{equation*}\begin{split} %{
				t_1\lhd t_2\lhd t_3 
				&= \sum_{i\in\set{t_1}\cup\set{t_2}}t_1\lhd t_2\lhd_{i}t_3 \\
				&= \Bigl(\sum_{i\in\set{t_1}}+\sum_{i\in\set{t_2}}\Bigr)t_1\lhd t_2\lhd_{i}t_3 \\
				&= \Bigl(\sum_{i\in\set{t_1}}t_1\lhd t_2\lhd_{i}t_3\Bigr) + t_1\lhd(t_2\lhd t_3) \\
			\end{split}\end{equation*} %}
			となるから、$
			t_1\lhd[t_2t_3] = \sum_{i\in\set{t_1}}t_1\lhd t_2\lhd_{i}t_3
			$と定義すれば、
			\begin{equation*}\begin{split} %{
				t_1\lhd t_2\lhd t_3 &= t_1\lhd[t_2t_3] + t_1\lhd(t_2\lhd t_3)
			\end{split}\end{equation*} %}
			となる。


			here

			until

			を用いて$\lhd$を$
			\lhd:RT_\bullet X\otimes RWTX \to RT_\bullet X
			$に拡張して、$
			t\lhd([x_1]\perp[x_2])=t\lhd[x_1]\lhd[x_2]
			$により$RTX$の積$m_\perp$を定義する。例えば次のようになる。
			\begin{equation*}\begin{split} %{
				\bullet\lhd([x_1]\perp[x_2])
				&= \bullet[x_1x_2] + \bullet[x_1[x_2]] \\
				%
				\bullet\lhd([x_1]\perp[x_2]\perp[x_3]) 
				&= \bullet[x_1x_2x_3] + \bullet[x_1[x_3]x_2] + \bullet[x_1x_2[x_3]] \\
				&\;+ \bullet[x_1[x_2]x_3] + \bullet[x_1[x_2x_3]] + \bullet[x_1[x_2[x_3]]] \\
				\dots \\
			\end{split}\end{equation*} %}
			可換図で書くと、次の図を可換にする$\beta_\lhd$を求めることになる。

			$\bullet\in T_\bullet X$に対する$TX$の作用$\lhd$を次のように定義すれば、
			\begin{equation*}\begin{split} %{
				\bullet\lhd1_* &= \bullet \\
				\bullet\lhd[t_1t_2\cdots t_m]&= \bullet[t_1t_2\cdots t_m] \\
			\end{split}\end{equation*} %}
			任意の$w\in WTX$に対して$\bullet\lhd w\neq0$だから、次のようになる。
			\begin{equation*}\begin{split} %{
				[x_1]\perp[x_2] &= [x_1x_2] + [x_1[x_2]] \\
				[x_1]\perp[x_2]\perp[x_3]
				&= [x_1x_2x_3]+[x_1[x_3]x_2]+[x_1x_2[x_3]] \\
				&\;+ [x_1[x_2]x_3]+[x_1[x_2x_3]]+[x_1[x_2[x_3]]] \\
				\dots \\
			\end{split}\end{equation*} %}

until

			$T_{\bullet}X$へ$WX$の作用$\lhd$を次のように定義する。
			\begin{equation*}\begin{split} %{
				\bullet\lhd1_* &= \bullet \\
				\bullet\lhd[x_1] &= \bullet[x_1] \\
				\bullet[x_1]\lhd[x_2] &= \bullet[x_1x_2] + \bullet[x_1[x_2]] \\
				\bullet[x_1x_2]\lhd[x_3] &= \bullet[x_1x_2x_3] + \bullet[x_1[x_3]x_2] + \bullet[x_1x_2[x_3]] \\
				\bullet[x_1[x_2]]\lhd[x_3] &= \bullet[x_1[x_2]x_3] + \bullet[x_1[x_2x_3]] + \bullet[x_1[x_2[x_3]]] \\
				\dots \\
				\bullet\lhd[x_1x_2\cdots x_m] &= \bullet[x_1x_2\cdots x_m] \\
			\end{split}\end{equation*} %}
			'結合性'$t\lhd w_1\lhd w_2=t\lhd(w_1\perp w_2)$は$RWTX$に対して
			次の積を導く。
			\begin{equation*}\begin{split} %{
				[x_1]\perp[x_2] &= [x_1x_2] + [x_1[x_2]] \\
				[x_1x_2]\perp[x_3] &= [x_1x_2x_3] + [x_1[x_3]x_2] + [x_1x_2[x_3]] \\
				[x_1[x_2]]\perp[x_3] &= [x_1[x_2]x_3] + [x_1[x_2x_3]] + [x_1[x_2[x_3]]] \\
				\dots \\
			\end{split}\end{equation*} %}
			また、$\perp$の結合性から次の関係が得られる。
			\begin{equation*}\begin{split} %{
				[x_1]\perp[x_2]\perp[x_3] &= [x_1x_2]\perp[x_3] + [x_1[x_2]]\perp[x_3] \\
				&\Downarrow \quad\lcomment{十分条件} \\
				[x_1]\perp[x_2x_3] &= [x_1x_2x_3] + [x_1[x_3]x_2] + [x_1[x_2]x_3] + [x_1[x_2x_3]] \\
				[x_1]\perp[x_2[x_3]] &= [x_1x_2[x_3]] + [x_1[x_2[x_3]]] \\
				\dots \\
			\end{split}\end{equation*} %}
			$[x]\mapsto[x]\otimes1_*+1_*\otimes[x]$となる$\perp$に双対な余積
			$\Delta$は次のように与えられる。
			\begin{equation*}\begin{split} %{
				\Delta([x_1]\otimes[x_2]) 
				&= (\Delta[x_1])\perp(\Delta[x_2]) \\
				&\Downarrow \quad\lcomment{十分条件} \\
				\Delta[x_1x_2] &= \begin{pmatrix}
				[x_1x_2] \\
				1_*
				\end{pmatrix} + \begin{pmatrix}
				[x_1] \\
				[x_2]
				\end{pmatrix} + \begin{pmatrix}
				[x_2] \\
				[x_1]
				\end{pmatrix} + \begin{pmatrix}
				[x_1x_2] \\
				1_*
				\end{pmatrix} \\
				\Delta[x_1[x_2]] &= \begin{pmatrix}
				[x_1[x_2]] \\
				1_*
				\end{pmatrix} + \begin{pmatrix}
				1_* \\
				[x_1[x_2]]
				\end{pmatrix} \\
			\end{split}\end{equation*} %}
			余積$\Delta$は余積$\Delta_*$に他ならない。

			積$m_\perp$は積$m_*+\cdots$となっている。$m_*$の残りの部分を
			$\beta_\dashv$とする。
			\begin{equation*}\begin{split} %{
				m_\perp &= m_* + \beta_\dashv \\
			\end{split}\end{equation*} %}
			$\beta_\dashv$は次のようになる。
			\begin{equation*}\begin{split} %{
				1_*\dashv1_* &= 0 \\
				1_*\dashv[x_1] &= 0 \\
				[x_1]\dashv1_* &= 0 \\
				[x_1]\dashv[x_2] &= [x_1[x_2]] \\
				[x_1x_2]\dashv[x_3] &= [x_1[x_3]x_2] + [x_1x_2[x_3]] \\
				[x_1[x_2]]\dashv[x_3] &= [x_1[x_2x_3]] + [x_1[x_2[x_3]]] \\
				[x_1]\dashv[x_2x_3] &= [x_1[x_3]x_2] + [x_1[x_2]x_3] + [x_1[x_2x_3]] \\
				[x_1]\dashv[x_2[x_3]] &= [x_1[x_2[x_3]]] \\
				\dots \\
			\end{split}\end{equation*} %}
			結合性$m_\perp(m_\perp\otimes\myid)=m_\perp(\myid\otimes m_\perp)$
			から次の式が導かれる。
			\begin{equation*}\begin{split} %{
				\beta_\dashv(\beta_\dashv\otimes\myid) 
				+ m_*(\beta_\dashv\otimes\myid) 
				+ \beta_\dashv(m_*\otimes\myid) \\
				= \beta_\dashv(\myid\otimes\beta_\dashv)
				+ m_*(\myid\otimes\beta_\dashv)
				+ \beta_\dashv(\myid\otimes m_*)
			\end{split}\end{equation*} %}
			双対性
			\begin{equation*}\begin{split} %{
				\Delta_*m_\perp 
				&= (m_\perp\otimes m_\perp)\sigma_{23}(\Delta_*\otimes \Delta_*) \\
				\Delta_*m_*
				&= (m_*\otimes m_*)\sigma_{23}(\Delta_*\otimes \Delta_*) \\
			\end{split}\end{equation*} %}
			から次の式が導かれる。
			\begin{equation*}\begin{split} %{
				\Delta_*\beta_\dashv 
				= (\beta_\dashv\otimes \beta_\dashv 
				+ m_*\otimes \beta_\dashv
				+ \beta_\dashv\otimes m_*)\sigma_{23}(\Delta_*\otimes\Delta_*)
			\end{split}\end{equation*} %}
		%s3:木の場合の考察}
	%s2:考察}
%}
