\section{シャッフル積}\label{s1:シャッフル積} %{
	$R$を半環、$m$を積とする。積$m$の定義域はある$R$係数半モジュールとする。
	積$m_t$がパラメータ$t\in R$で次のように与えられたとする。
	\begin{equation*}\begin{split} %{
		m_t &= \beta_0 + t\beta_1 + t^2\beta_2 + \cdots \\
		\beta_0 &= m \\
	\end{split}\end{equation*} %}
	このとき、$m_t$の結合性$m_t(m_t\otimes \myid)=m_t(\myid\otimes m_t)$
	を$t$のべきについて展開すると、任意の$t$に対しての結合性が成り立つための
	必要十分条件(少なくとも十分条件)が次の式が与えられる。
	\begin{equation*}\begin{split} %{
		\beta_1(\beta_0\otimes \myid) + \beta_0(\beta_1\otimes \myid)
		&= \beta_1(\myid\otimes \beta_0) + \beta_0(\myid\otimes \beta_1)
		\\
		\beta_2(\beta_0\otimes \myid) + \beta_1(\beta_1\otimes \myid)
		+ \beta_0(\beta_2\otimes \myid)
		&= \beta_2(\myid\otimes \beta_0) + \beta_1(\myid\otimes \beta_1)
		+ \beta_0(\myid\otimes \beta_2)
		\\
		\cdots \\
		\sum_{0\le i\le n}\beta_{n-i}(\beta_i\otimes \myid)
		&= \sum_{0\le i\le n}\beta_{n-i}(\myid\otimes \beta_i)
		\cdots \\
	\end{split}\end{equation*} %}

	集合$A$から生成された自由モノイド$WA$で、文字列の連結$m_*$からの変形
	\begin{equation}\label{eq:積の変形その一}\begin{split} %{
		m_t([a_1]\otimes[a_2]) &= [a_1a_2] + t[a_2a_1] \\
	\end{split}\end{equation} %}
	を計算してみる。$m_t$の結合性の条件は、$1$以上の自然数$n$に対して
	\begin{equation*}\begin{split} %{
		\beta_L^n = \sum_{0\le i\le n}\beta_{n-i}(\beta_i\otimes \myid)
		,\quad
		\beta_R^n = \sum_{0\le i\le n}\beta_{n-i}(\myid\otimes \beta_i)
	\end{split}\end{equation*} %}
	とおくと任意の$w_1,w_2,w_3\in WA$に対して
	\begin{equation*}\begin{split} %{
		\beta_L^n(w_1\otimes w_2\otimes w_3)
		=\beta_R^n(w_1\otimes w_2\otimes w_3)
	\end{split}\end{equation*} %}
	となる。文字列の並べ替えを計算するだけなので、
	$[a_1a_2\cdots a_m]=(12\cdots m)$と並びの順序を括弧$()$でくくって書き、
	テンソル積の記号は省略する。例えば、$[a_1a_2]\otimes[a_3a_4]$は$(12)(34)$
	と書く。

	$(1)(2)(3)$に対する結合性の条件をの$t^n$の項ごとに計算する。
	$0=\beta_2(1)(2)=\beta_3(1)(2)=\cdots$だから、$(1)(2)(3)$に対する計算
	では、任意の$n\in\mybf{N}+$で、次のように$\beta_L^n$と$\beta_R^n$は
	二項だけの和になる。
	\begin{equation*}\begin{split} %{
		\beta_L^n(1)(2)(3) &= \Bigl(\beta_n(\beta_0\otimes \myid)
		+\beta_{n-1}(\beta_1\otimes \myid)\Bigr)(1)(2)(3) \\
		&= \beta_n(12)(3) + \beta_{n-1}(21)(3) \\
		\beta_R^n(1)(2)(3) &= \Bigl(\beta_n(\myid\otimes \beta_0)
		+\beta_{n-1}(\myid\otimes \beta_1)\Bigr)(1)(2)(3) \\
		&= \beta_n(1)(23) + \beta_{n-1}(1)(32) \\
	\end{split}\end{equation*} %}
	となる。つまり、$t^n$の項で$m_t(1)(2)(3)$の結合性が成り立つための条件が
	\begin{equation*}\begin{split} %{
		\beta_n(12)(3) + \beta_{n-1}(21)(3)
		=\beta_n(1)(23) + \beta_{n-1}(1)(32)
	\end{split}\end{equation*} %}
	となる。したがって、ある$n$で$\beta_n(12)(3)=\beta_n(1)(23)=0$とできれば、
	$n$以上のすべての$p$で$\beta_p(12)(3)=\beta_p(1)(23)=0$とできる。
	以上を注意して計算する。

	$(1)(2)(3)$に対する一次の条件は
	$\beta_1(12)(3) + (213)=\beta_1(1)(23) + (132)$となり、
	$\beta_1(12)(3) = (132),\quad \beta_1(1)(23) = (213)$とおける。 
	二次の条件は$\beta_2(12)(3)+(231)=\beta_2(1)(23)+(312)$となり、
	$\beta_2(12)(3)=(312),\quad\beta_2(1)(23)=(231)$とおける。
	三次の条件は$\beta_3(12)(3)+(321)= \beta_3(1)(23)+(321)$となり、
	$\beta_3(12)(3)=\beta_3(1)(23)=0$とおける。したがって、
	\begin{equation*}\begin{split} %{
		m_t(12)(3) &= (123) + t(132) + t^2(312) \\
		m_t(1)(23) &= (123) + t(213) + t^2(231) \\
	\end{split}\end{equation*} %}
	となることがわかる。

	$(12)(3)(4)$に対する$m_t$の結合性のための条件は
	\begin{equation*}\begin{split} %{
		\beta_1(123)(4)+(1324)=\beta_1(12)(34)+(1243)
	\end{split}\end{equation*} %}
	および、$n\in\mybf{N}$に対して
	\begin{equation*}\begin{split} %{
		&\beta_{n+2}(123)(4) +\beta_{n+1}(132)(4) +\beta_{n}(312)(4) \\
		&= \beta_{n+2}(12)(34)+\beta_{n+1}(12)(43)
	\end{split}\end{equation*} %}
	となり、$(12)(3)$に対する場合と同様に、$\beta_n$の次数$n$について順に
	計算していくと、
	\begin{equation*}\begin{split} %{
		m_t(123)(4) &= (1234)+ t(1243)+ t^2(1423)+ t^3(4123) \\
		m_t(12)(34) &= (1234)+ t(1324)+ t^2\Bigl((1342)+ t^2(3124)\Bigr)
		+ t^3(3142)+ t^4(3412)
	\end{split}\end{equation*} %}
	となることがわかる。

	$\beta_n$の次数$n$について低次の項を二つ計算してみたが、$m_t$の規則の
	予想がつく。$m_t([a_1a_2\cdots a_m]\otimes[b_1b_2\cdots b_n])$に対して、
	一項目の文字の集合を$(a)=\set{a_1,a_2,\dots,a_n}$、
	二項目の文字の集合を$(b)=\set{b_1,b_2}$とすると、
	\begin{itemize} %{
		\item 文字列$[a_1a_2\cdots a_mb_1b_2\cdots b_n]$から始めて、
		\item 一度に一組だけ隣り合った$(a)$と$(b)$の文字の順序を入れ替えて
		新たな文字列を作り出す。
		ただし、$(a)$を右へ$(b)$を左へ動かす入れ替えしか許さないものと
		する。例えば、$[a_1a_2b_1a_3b_2]$という文字列から$[a_1b_1a_2a_3b_2]$
		または$[a_1a_2b_1b_2a_3]$へ文字の順序を入れ替える。
		\item 文字を入れ替えた時、因子$t$を文字列に掛ける。
		\item 文字を入れ替えたとき、既に同一の文字があったらそれ以上の
		文字の入れ替えを中止する。
		\item 文字列$[b_1b_2\cdots b_na_1a_2\cdots a_m]$に到達したとき
		終了する。
	\end{itemize} %}
	例えば、$m_t([a_1a_2]\otimes[b_1b_2])$の計算では、次のような規則で
	項を列挙していく。
	\begin{equation*}\xymatrix{
		& [a_1a_2b_1b_2] \ar[d] \\
		& [a_1b_1a_2b_2] \ar[dl] \ar[dr] \\
		[b_1a_1a_2b_2] \ar[dr] && [a_1b_1b_2a_2] \ar[dl] \\
		& [b_1a_1b_2a_2] \ar[d] \\
		& [b_1b_2a_1a_2] \\
	}\end{equation*}
	\begin{equation*}\begin{split} %{
	\end{split}\end{equation*} %}
	始点と終点を唯一つだけもつDAG(分岐と合流だけでループを持たないグラフ)
	で表現される。始点からの経路の長さが$t$のべきになる。
	また、次のような木の操作で表現してもよいだろう。
	\begin{equation*}\begin{split} %{
		\mytree{
		& \bullet \ar@{-}[dl] \ar@{-}[dr] \\
		a_1 && a_2 \\
		} \lhd [b_1b_2] &= \mytree{
		&& \bullet \ar@{-}[dll] \ar@{-}[dl]\ar@{-}[dr] \ar@{-}[drr]\\
		a_1 & a_2 && b_1 & b_2 \\
		}+ \mytree{
		& \bullet \ar@{-}[dl] \ar@{-}[dr] \\
		a_1 && a_2 \ar@{-}[dl] \ar@{-}[dr] \\
		& b_1 && b_2 \\
		}+ \mytree{
		& \bullet \ar@{-}[dl] \ar@{-}[d] \ar@{-}[dr]\\
		a_1 & a_2 \ar@{-}[d]  & b_2\\
		& b_1 \\
		} \\
		&\;+ \mytree{
		&& \bullet \ar@{-}[dl] \ar@{-}[dr] \\
		& a_1 \ar@{-}[dl] \ar@{-}[dr] && a_2 \\
		b_1 && b_2 \\
		} + \mytree{
		& \bullet \ar@{-}[dl] \ar@{-}[dr] \\
		a_1 \ar@{-}[d] && a_2 \ar@{-}[d] \\
		b_1 && b_2 \\
		} + \mytree{
		& \bullet \ar@{-}[dl] \ar@{-}[d] \ar@{-}[dr] \\
		a_1 \ar@{-}[d] & a_2 & b_2\\
		b_1 \\
		} \\
		&= \sum_{1\le i_1\le i_2\le 3}\left(\mytree{
		& \bullet:3 \ar@{-}[dl] \ar@{-}[dr] \\
		a_1:1 && a_2:2 \\
		}\lhd_{i_1}b_1\right)\lhd_{i_2}b_2
	\end{split}\end{equation*} %}
	ここで、$x\lhd_i b$は葉$b$を木$x$の$i$番目の頂点の最右の子供として
	付け加えるという操作である。頂点の番号は帰りがけ順につけられる。
	こうして作られた木を帰りがけ順に並べて単語にして、根$\bullet$を取り除く
	と、積$m_t([a_1a_2]\otimes[b_1b_2])$の項の和が列挙される。この方法では
	$t$のべきが明確でないが、DAGを用いた計算で必要となる重複のチェックを
	必要としないことが利点となる。

	\begin{todo}[シャッフル積の性質]\label{todo:シャッフル積の性質} %{
		任意の$a_1,a_2\in A,\;w_{1},w_{2}\in WA$に対して次の式が成り立つ。
		\begin{equation*}\begin{split} %{
			(a_1w_1)\sqcup(a_2w_2)
			&= a_1\bigl(w_1\sqcup(a_2w_2)\bigr)
			 + a_2\bigl((a_1w_1)\sqcup w_2\bigr)
		\end{split}\end{equation*} %}
		ここで、$a\in A,\;w\in WA$に対して$aw:=[a]*w$とおいた。
		低次の項について確かめてみる。
		\begin{equation*}\begin{array}{rcrcr} %{
			[a_1]\sqcup[a_2] &=& a_1[a_2] + a_2[a_1] &=& [a_1a_2] + [a_2a_1] \\
			\bakko{a_1a_2}\sqcup[a_3] &=& a_1([a_2]\sqcup[a_3]) + a_3[a_1a_2]
			&=& [a_3a_1a_2] + a_1([a_2]\sqcup[a_3]) \\
			\bakko{a_1}\sqcup[a_2a_3] &=& a_1[a_2a_3] + a_2([a_1]\sqcup[a_2])
			&=& [a_1a_2a_3] + a_2([a_1]\sqcup[a_2]) \\
		\end{array}\end{equation*} %}
		確かに成り立っているようだ。むしろ、通常の組み合わせ論では
		シャッフル積$\sqcup$はこの式と単位元に対する式
		\begin{equation*}\begin{split} %{
			1_W\sqcup w = w = w\sqcup 1_W \quad\text{for all }w\in WA
		\end{split}\end{equation*} %}
		の二つでシャッフル積を定義する。
	\end{todo} %todo:シャッフル積の性質}
	\begin{todo}[接木によるシャッフル積の導出]\label{todo:接木によるシャッフル積の導出} %{
		$\beta_\lhd:RT_+A\otimes RWA\to RT_+A$を、任意の木$t\in T_+A$に対して
		\begin{equation*}\begin{split} %{
			t\lhd 1_W = t
		\end{split}\end{equation*} %}
		、任意の木$t\in T_+A,\;a_1,a_2,\dots,a_m\in A$に対して
		\begin{equation*}\begin{split} %{
			t\lhd [a_1a_2\cdots a_m] = 
			\sum_{i_1\le i_2\le \cdots\le i_m\in \myop{post}t}
			\Bigl(\cdots\bigl((t\lhd_{i_1}a_1)\lhd_{i_2}a_2\bigr)\cdots\Bigr)\lhd_{i_m}a_m
		\end{split}\end{equation*} %}
		と定義する。すると、
		\begin{equation*}\begin{split} %{
			t\lhd[a_1]\lhd[a_2]
			&= (\sum_{i_1<i_2\in \myop{post}t}+\sum_{i_2<i_1\in \myop{post}t})
			t\lhd_{i_1}a_1\lhd_{i_2}a_2 \\
			&\; + \sum_{i\in \myop{post}t}(t\lhd_{i}a_1)\lhd_{i}a_2
		\end{split}\end{equation*} %}
		となるが、二項目は$a_1$と$a_2$が同じの頂点$i$の部分木になる場合で、
		次の二通りの場合がある。
		\begin{equation*}\begin{split} %{
			\mytree{
				& b:i \ar@{-}[dl]\ar@{-}[d]\ar@{-}[dr] \\
				*+[F]{u} & a_1 & a_2 \\
			},\quad \mytree{
				& b:i \ar@{-}[dl]\ar@{-}[dr] \\
				*+[F]{u} && a_1 \ar@{-}[d] \\
				&& a_2 \\
			}
		\end{split}\end{equation*} %}
		ここで、$u$は操作前に$t$に存在する$i$の部分木で、$b\in A$は頂点$i$
		のラベルとする。二つ目の場合は、帰りがけ順で単語に射影
		$\pi_{\myop{post}}$すると次の木と同一の単語
		$[\pi_{\myop{post}}u]*[a_2a_1b]$を与える。
		\begin{equation*}\begin{split} %{
			\mytree{
				& b:i \ar@{-}[dl]\ar@{-}[d]\ar@{-}[dr] \\
				*+[F]{u} & a_2 & a_1 \\
			}
		\end{split}\end{equation*} %}
		したがって、$
			\pi_{\myop{post}}\Bigl(t\lhd[a_1]\lhd[a_2]\Bigr) 
			= \pi_{\myop{post}}\Bigl(t\lhd\bigl([a_1a_2] + [a_2a_1]\bigr)\Bigr)
		$となって、シャッフル積$[a_1]\sqcup[a_2]=[a_1a_2]+[a_2a_1]$を与える。
		三文字以上の場合は、場合分けが複雑になるので工夫が必要になるが、
		木の操作からシャッフル積を導出できるかもしれない。
		そもそもシャッフル積そのものが複雑である。
	\end{todo} %todo:接木によるシャッフル積の導出}

	\begin{todo}[余積からシャッフル積を導出]\label{todo:余積からシャッフル積を導出} %{
		シャッフル積を計算する手立てを余積で与えることを考える。
		余積$\Delta_\sqcup:RWA\to RWA\otimes RWA$を、
		\begin{equation*}\begin{split} %{
			\Delta_\sqcup1_W = 1_W\otimes 1_W
		\end{split}\end{equation*} %}
		任意の$a_1,a_2,\dots,a_m\in A$に対して
		\begin{equation*}\begin{split} %{
			\Delta_\sqcup [a_1a_2a_3\cdots a_m]
			& = 1_W\otimes [a_1a_2a_3\cdots a_m] \\
			&\; + [a_1]\otimes [a_2a_3\cdots a_m] \\
			&\; + [a_1a_2]\otimes [\cdots a_m] \\
			&\; + \cdots \\
			&\; + [a_1a_2a_3\cdots a_m]\otimes 1_W
		\end{split}\end{equation*} %}
		と定義する。次の可換図を満たす$R$双線形二項写像$\beta$を考える。
		\begin{equation*}\xymatrix@C+2pc{
			RWA^{\otimes 2} 
			\ar[r]^{(\Delta_\sqcup\otimes \Delta_\sqcup)\sigma_{23}}
			\ar[d]^{\beta}
			& RWA^{\otimes 4} \ar[d] 
			\ar[d]^{\beta\otimes \beta}
			\\
			RWA 
			& RWA^{\otimes 2} \ar[l]_{m_*} \\
		}\end{equation*}
	\end{todo} %todo:余積からシャッフル積を導出}

	\begin{todo}[課題]\label{todo:課題} %{
		\begin{itemize} %{
			\item 任意の元に対する積$m_t$を結合性の条件のみから定めることが
			できるか?
			\item できるとするならばなぜ?
			\item できるとするならば、線形代数で任意の元に対する積$m_t$を
			を求められないか?
			\item 双対な余積の変形
			\item 余積$
			[a_1a_2\cdots a_m]\mapsto
			1_W\otimes[a_1a_2a_3\cdots a_m]+[a_1]\otimes[a_2a_3\cdots a_m]
			+[a_1a_2]\otimes[\cdots a_m]+\cdots+[a_1a_2a_3\cdots a_m]\otimes1_W
			$に双対で$m([a_1]\otimes[a_2])=[a_1a_2]+[a_2a_1]$となる積は
			唯一定まるか?
			\item
		\end{itemize} %}
	\end{todo} %todo:課題}
%s1:シャッフル積}
